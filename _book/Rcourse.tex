\documentclass[]{book}
\usepackage{lmodern}
\usepackage{amssymb,amsmath}
\usepackage{ifxetex,ifluatex}
\usepackage{fixltx2e} % provides \textsubscript
\ifnum 0\ifxetex 1\fi\ifluatex 1\fi=0 % if pdftex
  \usepackage[T1]{fontenc}
  \usepackage[utf8]{inputenc}
\else % if luatex or xelatex
  \ifxetex
    \usepackage{mathspec}
  \else
    \usepackage{fontspec}
  \fi
  \defaultfontfeatures{Ligatures=TeX,Scale=MatchLowercase}
\fi
% use upquote if available, for straight quotes in verbatim environments
\IfFileExists{upquote.sty}{\usepackage{upquote}}{}
% use microtype if available
\IfFileExists{microtype.sty}{%
\usepackage[]{microtype}
\UseMicrotypeSet[protrusion]{basicmath} % disable protrusion for tt fonts
}{}
\PassOptionsToPackage{hyphens}{url} % url is loaded by hyperref
\usepackage[unicode=true]{hyperref}
\hypersetup{
            pdftitle={R (BGU course)},
            pdfauthor={Jonathan D. Rosenblatt},
            pdfkeywords={Rstats, Statistics},
            pdfborder={0 0 0},
            breaklinks=true}
\urlstyle{same}  % don't use monospace font for urls
\usepackage{natbib}
\bibliographystyle{apalike}
\usepackage{color}
\usepackage{fancyvrb}
\newcommand{\VerbBar}{|}
\newcommand{\VERB}{\Verb[commandchars=\\\{\}]}
\DefineVerbatimEnvironment{Highlighting}{Verbatim}{commandchars=\\\{\}}
% Add ',fontsize=\small' for more characters per line
\usepackage{framed}
\definecolor{shadecolor}{RGB}{248,248,248}
\newenvironment{Shaded}{\begin{snugshade}}{\end{snugshade}}
\newcommand{\KeywordTok}[1]{\textcolor[rgb]{0.13,0.29,0.53}{\textbf{#1}}}
\newcommand{\DataTypeTok}[1]{\textcolor[rgb]{0.13,0.29,0.53}{#1}}
\newcommand{\DecValTok}[1]{\textcolor[rgb]{0.00,0.00,0.81}{#1}}
\newcommand{\BaseNTok}[1]{\textcolor[rgb]{0.00,0.00,0.81}{#1}}
\newcommand{\FloatTok}[1]{\textcolor[rgb]{0.00,0.00,0.81}{#1}}
\newcommand{\ConstantTok}[1]{\textcolor[rgb]{0.00,0.00,0.00}{#1}}
\newcommand{\CharTok}[1]{\textcolor[rgb]{0.31,0.60,0.02}{#1}}
\newcommand{\SpecialCharTok}[1]{\textcolor[rgb]{0.00,0.00,0.00}{#1}}
\newcommand{\StringTok}[1]{\textcolor[rgb]{0.31,0.60,0.02}{#1}}
\newcommand{\VerbatimStringTok}[1]{\textcolor[rgb]{0.31,0.60,0.02}{#1}}
\newcommand{\SpecialStringTok}[1]{\textcolor[rgb]{0.31,0.60,0.02}{#1}}
\newcommand{\ImportTok}[1]{#1}
\newcommand{\CommentTok}[1]{\textcolor[rgb]{0.56,0.35,0.01}{\textit{#1}}}
\newcommand{\DocumentationTok}[1]{\textcolor[rgb]{0.56,0.35,0.01}{\textbf{\textit{#1}}}}
\newcommand{\AnnotationTok}[1]{\textcolor[rgb]{0.56,0.35,0.01}{\textbf{\textit{#1}}}}
\newcommand{\CommentVarTok}[1]{\textcolor[rgb]{0.56,0.35,0.01}{\textbf{\textit{#1}}}}
\newcommand{\OtherTok}[1]{\textcolor[rgb]{0.56,0.35,0.01}{#1}}
\newcommand{\FunctionTok}[1]{\textcolor[rgb]{0.00,0.00,0.00}{#1}}
\newcommand{\VariableTok}[1]{\textcolor[rgb]{0.00,0.00,0.00}{#1}}
\newcommand{\ControlFlowTok}[1]{\textcolor[rgb]{0.13,0.29,0.53}{\textbf{#1}}}
\newcommand{\OperatorTok}[1]{\textcolor[rgb]{0.81,0.36,0.00}{\textbf{#1}}}
\newcommand{\BuiltInTok}[1]{#1}
\newcommand{\ExtensionTok}[1]{#1}
\newcommand{\PreprocessorTok}[1]{\textcolor[rgb]{0.56,0.35,0.01}{\textit{#1}}}
\newcommand{\AttributeTok}[1]{\textcolor[rgb]{0.77,0.63,0.00}{#1}}
\newcommand{\RegionMarkerTok}[1]{#1}
\newcommand{\InformationTok}[1]{\textcolor[rgb]{0.56,0.35,0.01}{\textbf{\textit{#1}}}}
\newcommand{\WarningTok}[1]{\textcolor[rgb]{0.56,0.35,0.01}{\textbf{\textit{#1}}}}
\newcommand{\AlertTok}[1]{\textcolor[rgb]{0.94,0.16,0.16}{#1}}
\newcommand{\ErrorTok}[1]{\textcolor[rgb]{0.64,0.00,0.00}{\textbf{#1}}}
\newcommand{\NormalTok}[1]{#1}
\usepackage{longtable,booktabs}
% Fix footnotes in tables (requires footnote package)
\IfFileExists{footnote.sty}{\usepackage{footnote}\makesavenoteenv{long table}}{}
\usepackage{graphicx,grffile}
\makeatletter
\def\maxwidth{\ifdim\Gin@nat@width>\linewidth\linewidth\else\Gin@nat@width\fi}
\def\maxheight{\ifdim\Gin@nat@height>\textheight\textheight\else\Gin@nat@height\fi}
\makeatother
% Scale images if necessary, so that they will not overflow the page
% margins by default, and it is still possible to overwrite the defaults
% using explicit options in \includegraphics[width, height, ...]{}
\setkeys{Gin}{width=\maxwidth,height=\maxheight,keepaspectratio}
\IfFileExists{parskip.sty}{%
\usepackage{parskip}
}{% else
\setlength{\parindent}{0pt}
\setlength{\parskip}{6pt plus 2pt minus 1pt}
}
\setlength{\emergencystretch}{3em}  % prevent overfull lines
\providecommand{\tightlist}{%
  \setlength{\itemsep}{0pt}\setlength{\parskip}{0pt}}
\setcounter{secnumdepth}{5}
% Redefines (sub)paragraphs to behave more like sections
\ifx\paragraph\undefined\else
\let\oldparagraph\paragraph
\renewcommand{\paragraph}[1]{\oldparagraph{#1}\mbox{}}
\fi
\ifx\subparagraph\undefined\else
\let\oldsubparagraph\subparagraph
\renewcommand{\subparagraph}[1]{\oldsubparagraph{#1}\mbox{}}
\fi

% set default figure placement to htbp
\makeatletter
\def\fps@figure{htbp}
\makeatother

\usepackage{booktabs}
\usepackage{amsthm}

\usepackage[margin=0.7in]{geometry}
\usepackage{fancyhdr}
\pagestyle{fancy}


\makeatletter
\def\thm@space@setup{%
  \thm@preskip=8pt plus 2pt minus 4pt
  \thm@postskip=\thm@preskip
}
\makeatother

\title{R (BGU course)}
\author{Jonathan D. Rosenblatt}
\date{2019-06-12}

\usepackage{amsthm}
\newtheorem{theorem}{Theorem}[chapter]
\newtheorem{lemma}{Lemma}[chapter]
\newtheorem{corollary}{Corollary}[chapter]
\newtheorem{proposition}{Proposition}[chapter]
\newtheorem{conjecture}{Conjecture}[chapter]
\theoremstyle{definition}
\newtheorem{definition}{Definition}[chapter]
\theoremstyle{definition}
\newtheorem{example}{Example}[chapter]
\theoremstyle{definition}
\newtheorem{exercise}{Exercise}[chapter]
\theoremstyle{remark}
\newtheorem*{remark}{Remark}
\newtheorem*{solution}{Solution}
\let\BeginKnitrBlock\begin \let\EndKnitrBlock\end
\begin{document}
\maketitle

{
\setcounter{tocdepth}{1}
\tableofcontents
}
\section{Notation Conventions}\label{notation-conventions}

In this text we use the following conventions: Lower case \(x\) may be a
vector or a scalar, random of fixed, as implied by the context. Upper
case \(A\) will stand for matrices. Equality \(=\) is an equality, and
\(:=\) is a definition. Norm functions are denoted with
\(\Vert x \Vert\) for vector norms, and \(\Vert A \Vert\) for matrix
norms. The type of norm is indicated in the subscript; e.g.
\(\Vert x \Vert_2\) for the Euclidean (\(l_2\)) norm. Tag, \(x'\) is a
transpose. The distribution of a random vector is \(\sim\).

\section{Acknowledgements}\label{acknowledgements}

I have consulted many people during the writing of this text. I would
like to thank \href{https://kesslerlab.wordpress.com/}{Yoav Kessler},
\href{http://fohs.bgu.ac.il/research/profileBrief.aspx?id=VeeMVried}{Lena
Novack}, Efrat Vilenski, Ron Sarafian, and Liad Shekel in particular,
for their valuable inputs.

\chapter{Introduction}\label{intro}

\section{What is R?}\label{what-r}

R was not designed to be a bona-fide programming language. It is an
evolution of the S language, developed at Bell labs (later Lucent) as a
wrapper for the endless collection of statistical libraries they wrote
in Fortran.

As of 2011, half of R's libraries are
\href{https://wrathematics.github.io/2011/08/27/how-much-of-r-is-written-in-r/}{actually
written in C}.

\hypertarget{ecosystem}{\section{The R Ecosystem}\label{ecosystem}}

A large part of R's success is due to the ease in which a user, or a
firm, can augment it. This led to a large community of users,
developers, and protagonists. Some of the most important parts of R's
ecosystem include:

\begin{itemize}
\item
  \href{https://cran.r-project.org/}{CRAN}: a repository for R packages,
  mirrored worldwide.
\item
  \href{https://www.r-project.org/mail.html}{R-help}: an immensely
  active mailing list. Noways being replaced by StackExchange meta-site.
  Look for the R tags in the
  \href{http://stackoverflow.com/}{StackOverflow} and
  \href{http://stats.stackexchange.com/}{CrossValidated} sites.
\item
  \href{https://cran.r-project.org/web/views/}{Task Views}: part of CRAN
  that collects packages per topic.
\item
  \href{https://www.bioconductor.org/}{Bioconductor}: A CRAN-like
  repository dedicated to the life sciences.
\item
  \href{https://www.neuroconductor.org/}{Neuroconductor}: A CRAN-like
  repository dedicated to neuroscience, and neuroimaging.
\item
  \href{https://www.r-project.org/doc/bib/R-books.html}{Books}: An
  insane amount of books written on the language. Some are free, some
  are not.
\item
  \href{https://groups.google.com/forum/\#!forum/israel-r-user-group}{The
  Israeli-R-user-group}: just like the name suggests.
\item
  Commercial R: being open source and lacking support may seem like a
  problem that would prohibit R from being adopted for commercial
  applications. This void is filled by several very successful
  commercial versions such as
  \href{https://mran.microsoft.com/open/}{Microsoft R}, with its
  accompanying CRAN equivalent called
  \href{https://mran.microsoft.com/}{MRAN},
  \href{http://spotfire.tibco.com/discover-spotfire/what-does-spotfire-do/predictive-analytics/tibco-enterprise-runtime-for-r-terr}{Tibco's
  Spotfire}, and
  \href{https://en.wikipedia.org/wiki/R_(programming_language)\#Commercial_support_for_R}{others}.
\item
  \href{https://www.rstudio.com/products/rstudio/download-server/}{RStudio}:
  since its earliest days R came equipped with a minimal text editor. It
  later received plugins for major integrated development environments
  (IDEs) such as Eclipse, WinEdit and even
  \href{https://www.visualstudio.com/vs/rtvs/}{VisualStudio}. None of
  these, however, had the impact of the RStudio IDE. Written completely
  in JavaScript, the RStudio IDE allows the seamless integration of
  cutting edge web-design technologies, remote access, and other killer
  features, making it today's most popular IDE for R.
\item
  \href{https://www.rstudio.com/resources/cheatsheets/}{CheatSheets}
  Rstudio curates a list of CheatSheets. Very useful to print some, and
  have them around when coding.
\item
  \href{https://github.com/rstudio/RStartHere/blob/master/README.md\#import}{RStartHere}:
  a curated list of useful packages.
\end{itemize}

\section{Bibliographic Notes}\label{bibliographic-notes}

For more on the history of R see
\href{http://www.research.att.com/articles/featured_stories/2013_09/201309_SandR.html?fbid=Yxy4qyQzmMa}{AT\&T's
site}, John Chamber's talk at
\href{https://www.youtube.com/watch?v=_hcpuRB5nGs}{UserR!2014}, Nick
Thieme's
\href{https://rss.onlinelibrary.wiley.com/doi/10.1111/j.1740-9713.2018.01169.x}{recent
report} in Significance, or
\href{https://blog.revolutionanalytics.com/2017/10/updated-history-of-r.html}{Revolution
Analytics'} blog.

You can also consult the Introduction chapter of the MASS book
\citep{venables2013modern}.

\chapter{R Basics}\label{basics}

We now start with the basics of R. If you have any experience at all
with R, you can probably skip this section.

First, make sure you work with the RStudio IDE. Some useful pointers for
this IDE include:

\begin{itemize}
\tightlist
\item
  Ctrl+Return(Enter) to run lines from editor.
\item
  Alt+Shift+k for RStudio keyboard shortcuts.
\item
  Ctrl+r to browse the command history.
\item
  Alt+Shift+j to navigate between code sections
\item
  tab for auto-completion
\item
  Ctrl+1 to skip to editor.
\item
  Ctrl+2 to skip to console.
\item
  Ctrl+8 to skip to the environment list.
\item
  Ctrl + Alt + Shift + M to select all instances of the selection (for
  refactoring).
\item
  Code Folding:

  \begin{itemize}
  \tightlist
  \item
    Alt+l collapse chunk.
  \item
    Alt+Shift+l unfold chunk.
  \item
    Alt+o collapse all.
  \item
    Alt+Shift+o unfold all.
  \end{itemize}
\item
  Alt+``-'' for the assignment operator \texttt{\textless{}-}.
\end{itemize}

\subsection{Other IDEs}\label{other-ides}

Currently, I recommend RStudio, but here are some other IDEs:

\begin{enumerate}
\def\labelenumi{\arabic{enumi}.}
\item
  Jupyter Lab: a very promising IDE, originally designed for Python,
  that also supports R. At the time of writing, it seems that RStudio is
  more convenient for R, but it is definitely an IDE to follow closely.
  See \href{http://minimaxir.com/2017/06/r-notebooks/}{Max Woolf's}
  review.
\item
  Eclipse: If you are a Java programmer, you are probably familiar with
  Eclipse, which does have an R plugin:
  \href{http://www.walware.de/goto/statet}{StatEt}.
\item
  Emacs: If you are an Emacs fan, you can find an R plugin:
  \href{http://ess.r-project.org/}{ESS}.
\item
  Vim: \href{https://github.com/vim-scripts/Vim-R-plugin}{Vim-R}.
\item
  Visual Studio also
  \href{https://www.visualstudio.com/vs/features/rtvs/}{supports R}. If
  you need R for commercial purposes, it may be worthwhile trying
  Microsoft's R, instead of the usual R. See
  \href{https://mran.microsoft.com/documents/rro/installation}{here} for
  installation instructions.
\item
  Online version (currently alpha): \href{https://rstudio.cloud}{R
  Studio Cloud}.
\end{enumerate}

\section{File types}\label{file-types}

The file types you need to know when using R are the following:

\begin{itemize}
\tightlist
\item
  \textbf{.R}: An ASCII text file containing R scripts only.
\item
  \textbf{.Rmd}: An ASCII text file. If opened in RStudio can be run as
  an R-Notebook or compiled using knitr, bookdown, etc.
\end{itemize}

\section{Simple calculator}\label{simple-calculator}

R can be used as a simple calculator. Create a new R Notebook (.Rmd
file) within RStudio using File-\textgreater{} New -\textgreater{} R
Notebook, and run the following commands.

\begin{Shaded}
\begin{Highlighting}[]
\DecValTok{10}\OperatorTok{+}\DecValTok{5} 
\end{Highlighting}
\end{Shaded}

\begin{verbatim}
## [1] 15
\end{verbatim}

\begin{Shaded}
\begin{Highlighting}[]
\DecValTok{70}\OperatorTok{*}\DecValTok{81}
\end{Highlighting}
\end{Shaded}

\begin{verbatim}
## [1] 5670
\end{verbatim}

\begin{Shaded}
\begin{Highlighting}[]
\DecValTok{2}\OperatorTok{**}\DecValTok{4}
\end{Highlighting}
\end{Shaded}

\begin{verbatim}
## [1] 16
\end{verbatim}

\begin{Shaded}
\begin{Highlighting}[]
\DecValTok{2}\OperatorTok{^}\DecValTok{4}
\end{Highlighting}
\end{Shaded}

\begin{verbatim}
## [1] 16
\end{verbatim}

\begin{Shaded}
\begin{Highlighting}[]
\KeywordTok{log}\NormalTok{(}\DecValTok{10}\NormalTok{)                         }
\end{Highlighting}
\end{Shaded}

\begin{verbatim}
## [1] 2.302585
\end{verbatim}

\begin{Shaded}
\begin{Highlighting}[]
\KeywordTok{log}\NormalTok{(}\DecValTok{16}\NormalTok{, }\DecValTok{2}\NormalTok{)                      }
\end{Highlighting}
\end{Shaded}

\begin{verbatim}
## [1] 4
\end{verbatim}

\begin{Shaded}
\begin{Highlighting}[]
\KeywordTok{log}\NormalTok{(}\DecValTok{1000}\NormalTok{, }\DecValTok{10}\NormalTok{)                   }
\end{Highlighting}
\end{Shaded}

\begin{verbatim}
## [1] 3
\end{verbatim}

\section{Probability calculator}\label{probability-calculator}

R can be used as a probability calculator. You probably wish you knew
this when you did your Intro To Probability classes.

The Binomial distribution function:

\begin{Shaded}
\begin{Highlighting}[]
\KeywordTok{dbinom}\NormalTok{(}\DataTypeTok{x=}\DecValTok{3}\NormalTok{, }\DataTypeTok{size=}\DecValTok{10}\NormalTok{, }\DataTypeTok{prob=}\FloatTok{0.5}\NormalTok{)  }\CommentTok{# Compute P(X=3) for X~B(n=10, p=0.5) }
\end{Highlighting}
\end{Shaded}

\begin{verbatim}
## [1] 0.1171875
\end{verbatim}

Notice that arguments do not need to be named explicitly

\begin{Shaded}
\begin{Highlighting}[]
\KeywordTok{dbinom}\NormalTok{(}\DecValTok{3}\NormalTok{, }\DecValTok{10}\NormalTok{, }\FloatTok{0.5}\NormalTok{)}
\end{Highlighting}
\end{Shaded}

\begin{verbatim}
## [1] 0.1171875
\end{verbatim}

The Binomial cumulative distribution function (CDF):

\begin{Shaded}
\begin{Highlighting}[]
\KeywordTok{pbinom}\NormalTok{(}\DataTypeTok{q=}\DecValTok{3}\NormalTok{, }\DataTypeTok{size=}\DecValTok{10}\NormalTok{, }\DataTypeTok{prob=}\FloatTok{0.5}\NormalTok{) }\CommentTok{# Compute P(X<=3) for X~B(n=10, p=0.5)   }
\end{Highlighting}
\end{Shaded}

\begin{verbatim}
## [1] 0.171875
\end{verbatim}

The Binomial quantile function:

\begin{Shaded}
\begin{Highlighting}[]
\KeywordTok{qbinom}\NormalTok{(}\DataTypeTok{p=}\FloatTok{0.1718}\NormalTok{, }\DataTypeTok{size=}\DecValTok{10}\NormalTok{, }\DataTypeTok{prob=}\FloatTok{0.5}\NormalTok{) }\CommentTok{# For X~B(n=10, p=0.5) returns k such that P(X<=k)=0.1718}
\end{Highlighting}
\end{Shaded}

\begin{verbatim}
## [1] 3
\end{verbatim}

Generate random variables:

\begin{Shaded}
\begin{Highlighting}[]
\KeywordTok{rbinom}\NormalTok{(}\DataTypeTok{n=}\DecValTok{10}\NormalTok{, }\DataTypeTok{size=}\DecValTok{10}\NormalTok{, }\DataTypeTok{prob=}\FloatTok{0.5}\NormalTok{)}
\end{Highlighting}
\end{Shaded}

\begin{verbatim}
##  [1] 4 4 5 7 4 7 7 6 6 3
\end{verbatim}

R has many built-in distributions. Their names may change, but the
prefixes do not:

\begin{itemize}
\tightlist
\item
  \textbf{d} prefix for the \emph{distribution} function.
\item
  \textbf{p} prefix for the \emph{cummulative distribution} function
  (CDF).
\item
  \textbf{q} prefix for the \emph{quantile} function (i.e., the inverse
  CDF).
\item
  \textbf{r} prefix to generate random samples.
\end{itemize}

Demonstrating this idea, using the CDF of several popular distributions:

\begin{itemize}
\tightlist
\item
  \texttt{pbinom()} for the Binomial CDF.
\item
  \texttt{ppois()} for the Poisson CDF.
\item
  \texttt{pnorm()} for the Gaussian CDF.
\item
  \texttt{pexp()} for the Exponential CDF.
\end{itemize}

For more information see \texttt{?distributions}.

\section{Getting Help}\label{getting-help}

One of the most important parts of working with a language, is to know
where to find help. R has several in-line facilities, besides the
various help resources in the R
\protect\hyperlink{ecosystem}{ecosystem}.

Get help for a particular function.

\begin{Shaded}
\begin{Highlighting}[]
\NormalTok{?dbinom }
\KeywordTok{help}\NormalTok{(dbinom)}
\end{Highlighting}
\end{Shaded}

If you don't know the name of the function you are looking for, search
local help files for a particular string:

\begin{Shaded}
\begin{Highlighting}[]
\NormalTok{??binomial}
\KeywordTok{help.search}\NormalTok{(}\StringTok{'dbinom'}\NormalTok{) }
\end{Highlighting}
\end{Shaded}

Or load a menu where you can navigate local help in a web-based fashion:

\begin{Shaded}
\begin{Highlighting}[]
\KeywordTok{help.start}\NormalTok{() }
\end{Highlighting}
\end{Shaded}

\section{Variable Asignment}\label{variable-asignment}

Assignment of some output into an object named ``x'':

\begin{Shaded}
\begin{Highlighting}[]
\NormalTok{x =}\StringTok{ }\KeywordTok{rbinom}\NormalTok{(}\DataTypeTok{n=}\DecValTok{10}\NormalTok{, }\DataTypeTok{size=}\DecValTok{10}\NormalTok{, }\DataTypeTok{prob=}\FloatTok{0.5}\NormalTok{) }\CommentTok{# Works. Bad style.}
\NormalTok{x <-}\StringTok{ }\KeywordTok{rbinom}\NormalTok{(}\DataTypeTok{n=}\DecValTok{10}\NormalTok{, }\DataTypeTok{size=}\DecValTok{10}\NormalTok{, }\DataTypeTok{prob=}\FloatTok{0.5}\NormalTok{) }
\end{Highlighting}
\end{Shaded}

If you are familiar with other programming languages you may prefer the
\texttt{=} assignment rather than the \texttt{\textless{}-} assignment.
We recommend you make the effort to change your preferences. This is
because thinking with \texttt{\textless{}-} helps to read your code,
distinguishes between assignments and function arguments: think of
\texttt{function(argument=value)} versus
\texttt{function(argument\textless{}-value)}. It also helps understand
special assignment operators such as \texttt{\textless{}\textless{}-}
and \texttt{-\textgreater{}}.

\BeginKnitrBlock{remark}
\iffalse{} {Remark. } \fi{}\textbf{Style}: We do not discuss style
guidelines in this text, but merely remind the reader that good style is
extremely important. When you write code, think of other readers, but
also think of future self. See
\href{http://adv-r.had.co.nz/Style.html}{Hadley's style guide} for more.
\EndKnitrBlock{remark}

To print the contents of an object just type its name

\begin{Shaded}
\begin{Highlighting}[]
\NormalTok{x}
\end{Highlighting}
\end{Shaded}

\begin{verbatim}
##  [1] 7 4 6 3 4 5 2 5 7 4
\end{verbatim}

which is an implicit call to

\begin{Shaded}
\begin{Highlighting}[]
\KeywordTok{print}\NormalTok{(x)  }
\end{Highlighting}
\end{Shaded}

\begin{verbatim}
##  [1] 7 4 6 3 4 5 2 5 7 4
\end{verbatim}

Alternatively, you can assign and print simultaneously using
parenthesis.

\begin{Shaded}
\begin{Highlighting}[]
\NormalTok{(x <-}\StringTok{ }\KeywordTok{rbinom}\NormalTok{(}\DataTypeTok{n=}\DecValTok{10}\NormalTok{, }\DataTypeTok{size=}\DecValTok{10}\NormalTok{, }\DataTypeTok{prob=}\FloatTok{0.5}\NormalTok{))  }\CommentTok{# Assign and print.}
\end{Highlighting}
\end{Shaded}

\begin{verbatim}
##  [1] 5 5 5 4 6 6 6 3 6 5
\end{verbatim}

Operate on the object

\begin{Shaded}
\begin{Highlighting}[]
\KeywordTok{mean}\NormalTok{(x)  }\CommentTok{# compute mean}
\end{Highlighting}
\end{Shaded}

\begin{verbatim}
## [1] 5.1
\end{verbatim}

\begin{Shaded}
\begin{Highlighting}[]
\KeywordTok{var}\NormalTok{(x)  }\CommentTok{# compute variance}
\end{Highlighting}
\end{Shaded}

\begin{verbatim}
## [1] 0.9888889
\end{verbatim}

\begin{Shaded}
\begin{Highlighting}[]
\KeywordTok{hist}\NormalTok{(x) }\CommentTok{# plot histogram}
\end{Highlighting}
\end{Shaded}

\includegraphics[width=0.5\linewidth]{Rcourse_files/figure-latex/unnamed-chunk-15-1}

R saves every object you create in RAM\footnote{S and S-Plus used to
  save objects on disk. Working from RAM has advantages and
  disadvantages. More on this in Chapter \ref{memory}.}. The collection
of all such objects is the \textbf{workspace} which you can inspect with

\begin{Shaded}
\begin{Highlighting}[]
\KeywordTok{ls}\NormalTok{()}
\end{Highlighting}
\end{Shaded}

\begin{verbatim}
## [1] "x"
\end{verbatim}

or with Ctrl+8 in RStudio.

If you lost your object, you can use \texttt{ls} with a text pattern to
search for it

\begin{Shaded}
\begin{Highlighting}[]
\KeywordTok{ls}\NormalTok{(}\DataTypeTok{pattern=}\StringTok{'x'}\NormalTok{)}
\end{Highlighting}
\end{Shaded}

\begin{verbatim}
## [1] "x"
\end{verbatim}

To remove objects from the workspace:

\begin{Shaded}
\begin{Highlighting}[]
\KeywordTok{rm}\NormalTok{(x) }\CommentTok{# remove variable}
\KeywordTok{ls}\NormalTok{() }\CommentTok{# verify}
\end{Highlighting}
\end{Shaded}

\begin{verbatim}
## character(0)
\end{verbatim}

You may think that if an object is removed then its memory is freed.
This is almost true, and depends on a negotiation mechanism between R
and the operating system. R's memory management is discussed in Chapter
\ref{memory}.

\section{Missing}\label{missing}

Unlike typically programming, when working with real life data, you may
have \textbf{missing} values: measurements that were simply not
recorded/stored/etc. \emph{R} has rather sophisticated mechanisms to
deal with missing values. It distinguishes between the following types:

\begin{enumerate}
\def\labelenumi{\arabic{enumi}.}
\tightlist
\item
  \texttt{NA}: Not Available entries.
\item
  \texttt{NaN}: Not a number.
\end{enumerate}

\emph{R} tries to defend the analyst, and return an error, or
\texttt{NA} when the presence of missing values invalidates the
calculation:

\begin{Shaded}
\begin{Highlighting}[]
\NormalTok{missing.example <-}\StringTok{ }\KeywordTok{c}\NormalTok{(}\DecValTok{10}\NormalTok{,}\DecValTok{11}\NormalTok{,}\DecValTok{12}\NormalTok{,}\OtherTok{NA}\NormalTok{)}
\KeywordTok{mean}\NormalTok{(missing.example)}
\end{Highlighting}
\end{Shaded}

\begin{verbatim}
## [1] NA
\end{verbatim}

Most functions will typically have an inner mechanism to deal with
these. In the \texttt{mean} function, there is an \texttt{na.rm}
argument, telling \emph{R} how to Remove \texttt{NA}s.

\begin{Shaded}
\begin{Highlighting}[]
\KeywordTok{mean}\NormalTok{(missing.example, }\DataTypeTok{na.rm =} \OtherTok{TRUE}\NormalTok{)}
\end{Highlighting}
\end{Shaded}

\begin{verbatim}
## [1] 11
\end{verbatim}

A more general mechanism is removing these manually:

\begin{Shaded}
\begin{Highlighting}[]
\NormalTok{clean.example <-}\StringTok{ }\KeywordTok{na.omit}\NormalTok{(missing.example)}
\KeywordTok{mean}\NormalTok{(clean.example)}
\end{Highlighting}
\end{Shaded}

\begin{verbatim}
## [1] 11
\end{verbatim}

\section{Piping}\label{piping}

Because R originates in Unix and Linux environments, it inherits much of
its flavor.
\href{http://ryanstutorials.net/linuxtutorial/piping.php}{Piping} is an
idea taken from the Linux shell which allows to use the output of one
expression as the input to another. Piping thus makes code easier to
read and write.

\BeginKnitrBlock{remark}
\iffalse{} {Remark. } \fi{}Volleyball fans may be confused with the idea
of spiking a ball from the 3-meter line, also called
\href{https://www.youtube.com/watch?v=DEaj4X_JhSY}{piping}. So: (a)
These are very different things. (b) If you can pipe,
\href{http://in.bgu.ac.il/sport/Pages/asa.aspx}{ASA-BGU} is looking for
you!
\EndKnitrBlock{remark}

Prerequisites:

\begin{Shaded}
\begin{Highlighting}[]
\KeywordTok{library}\NormalTok{(magrittr) }\CommentTok{# load the piping functions}
\NormalTok{x <-}\StringTok{ }\KeywordTok{rbinom}\NormalTok{(}\DataTypeTok{n=}\DecValTok{1000}\NormalTok{, }\DataTypeTok{size=}\DecValTok{10}\NormalTok{, }\DataTypeTok{prob=}\FloatTok{0.5}\NormalTok{) }\CommentTok{# generate some toy data}
\end{Highlighting}
\end{Shaded}

Examples

\begin{Shaded}
\begin{Highlighting}[]
\NormalTok{x }\OperatorTok\StringTok{ }\KeywordTok{var}\NormalTok{() }\CommentTok{# Instead of var(x)}
\NormalTok{x }\OperatorTok\StringTok{ }\KeywordTok{hist}\NormalTok{()  }\CommentTok{# Instead of hist(x)}
\NormalTok{x }\OperatorTok\StringTok{ }\KeywordTok{mean}\NormalTok{() }\OperatorTok\StringTok{ }\KeywordTok{round}\NormalTok{(}\DecValTok{2}\NormalTok{) }\OperatorTok\StringTok{ }\KeywordTok{add}\NormalTok{(}\DecValTok{10}\NormalTok{) }
\end{Highlighting}
\end{Shaded}

The next example\footnote{Taken from
  \url{http://cran.r-project.org/web/packages/magrittr/vignettes/magrittr.html}}
demonstrates the benefits of piping. The next two chunks of code do the
same thing. Try parsing them in your mind:

\begin{Shaded}
\begin{Highlighting}[]
\CommentTok{# Functional (onion) style}
\NormalTok{car_data <-}\StringTok{ }
\StringTok{  }\KeywordTok{transform}\NormalTok{(}\KeywordTok{aggregate}\NormalTok{(. }\OperatorTok{~}\StringTok{ }\NormalTok{cyl, }
                      \DataTypeTok{data =} \KeywordTok{subset}\NormalTok{(mtcars, hp }\OperatorTok{>}\StringTok{ }\DecValTok{100}\NormalTok{), }
                      \DataTypeTok{FUN =} \ControlFlowTok{function}\NormalTok{(x) }\KeywordTok{round}\NormalTok{(}\KeywordTok{mean}\NormalTok{(x, }\DecValTok{2}\NormalTok{))), }
            \DataTypeTok{kpl =}\NormalTok{ mpg}\OperatorTok{*}\FloatTok{0.4251}\NormalTok{)}
\end{Highlighting}
\end{Shaded}

\begin{Shaded}
\begin{Highlighting}[]
\CommentTok{# Piping (magrittr) style}
\NormalTok{car_data <-}\StringTok{ }
\StringTok{  }\NormalTok{mtcars }\OperatorTok
\StringTok{  }\KeywordTok{subset}\NormalTok{(hp }\OperatorTok{>}\StringTok{ }\DecValTok{100}\NormalTok{) }\OperatorTok
\StringTok{  }\KeywordTok{aggregate}\NormalTok{(. }\OperatorTok{~}\StringTok{ }\NormalTok{cyl, }\DataTypeTok{data =}\NormalTok{ ., }\DataTypeTok{FUN =}\NormalTok{ . }\OperatorTok\StringTok{ }\NormalTok{mean }\OperatorTok\StringTok{ }\KeywordTok{round}\NormalTok{(}\DecValTok{2}\NormalTok{)) }\OperatorTok
\StringTok{  }\KeywordTok{transform}\NormalTok{(}\DataTypeTok{kpl =}\NormalTok{ mpg }\OperatorTok\StringTok{ }\KeywordTok{multiply_by}\NormalTok{(}\FloatTok{0.4251}\NormalTok{)) }\OperatorTok
\StringTok{  }\NormalTok{print}
\end{Highlighting}
\end{Shaded}

Tip: RStudio has a keyboard shortcut for the \texttt{\%\textgreater{}\%}
operator. Try Ctrl+Shift+m.

\section{Vector Creation and
Manipulation}\label{vector-creation-and-manipulation}

The most basic building block in R is the \textbf{vector}. We will now
see how to create them, and access their elements (i.e.~subsetting).
Here are three ways to create the same arbitrary vector:

\begin{Shaded}
\begin{Highlighting}[]
\KeywordTok{c}\NormalTok{(}\DecValTok{10}\NormalTok{, }\DecValTok{11}\NormalTok{, }\DecValTok{12}\NormalTok{, }\DecValTok{13}\NormalTok{, }\DecValTok{14}\NormalTok{, }\DecValTok{15}\NormalTok{, }\DecValTok{16}\NormalTok{, }\DecValTok{17}\NormalTok{, }\DecValTok{18}\NormalTok{, }\DecValTok{19}\NormalTok{, }\DecValTok{20}\NormalTok{, }\DecValTok{21}\NormalTok{) }\CommentTok{# manually}
\DecValTok{10}\OperatorTok{:}\DecValTok{21} \CommentTok{# the `:` operator                            }
\KeywordTok{seq}\NormalTok{(}\DataTypeTok{from=}\DecValTok{10}\NormalTok{, }\DataTypeTok{to=}\DecValTok{21}\NormalTok{, }\DataTypeTok{by=}\DecValTok{1}\NormalTok{) }\CommentTok{# the seq() function}
\end{Highlighting}
\end{Shaded}

Let's assign it to the object named ``x'':

\begin{Shaded}
\begin{Highlighting}[]
\NormalTok{x <-}\StringTok{ }\KeywordTok{c}\NormalTok{(}\DecValTok{10}\NormalTok{, }\DecValTok{11}\NormalTok{, }\DecValTok{12}\NormalTok{, }\DecValTok{13}\NormalTok{, }\DecValTok{14}\NormalTok{, }\DecValTok{15}\NormalTok{, }\DecValTok{16}\NormalTok{, }\DecValTok{17}\NormalTok{, }\DecValTok{18}\NormalTok{, }\DecValTok{19}\NormalTok{, }\DecValTok{20}\NormalTok{, }\DecValTok{21}\NormalTok{)  }
\end{Highlighting}
\end{Shaded}

Operations usually work element-wise:

\begin{Shaded}
\begin{Highlighting}[]
\NormalTok{x}\OperatorTok{+}\DecValTok{2}
\end{Highlighting}
\end{Shaded}

\begin{verbatim}
##  [1] 12 13 14 15 16 17 18 19 20 21 22 23
\end{verbatim}

\begin{Shaded}
\begin{Highlighting}[]
\NormalTok{x}\OperatorTok{*}\DecValTok{2}    
\end{Highlighting}
\end{Shaded}

\begin{verbatim}
##  [1] 20 22 24 26 28 30 32 34 36 38 40 42
\end{verbatim}

\begin{Shaded}
\begin{Highlighting}[]
\NormalTok{x}\OperatorTok{^}\DecValTok{2}    
\end{Highlighting}
\end{Shaded}

\begin{verbatim}
##  [1] 100 121 144 169 196 225 256 289 324 361 400 441
\end{verbatim}

\begin{Shaded}
\begin{Highlighting}[]
\KeywordTok{sqrt}\NormalTok{(x)  }
\end{Highlighting}
\end{Shaded}

\begin{verbatim}
##  [1] 3.162278 3.316625 3.464102 3.605551 3.741657 3.872983 4.000000
##  [8] 4.123106 4.242641 4.358899 4.472136 4.582576
\end{verbatim}

\begin{Shaded}
\begin{Highlighting}[]
\KeywordTok{log}\NormalTok{(x)   }
\end{Highlighting}
\end{Shaded}

\begin{verbatim}
##  [1] 2.302585 2.397895 2.484907 2.564949 2.639057 2.708050 2.772589
##  [8] 2.833213 2.890372 2.944439 2.995732 3.044522
\end{verbatim}

\section{Search Paths and Packages}\label{search-paths-and-packages}

R can be easily extended with packages, which are merely a set of
documented functions, which can be loaded or unloaded conveniently.
Let's look at the function \texttt{read.csv}. We can see its contents by
calling it without arguments:

\begin{Shaded}
\begin{Highlighting}[]
\NormalTok{read.csv}
\end{Highlighting}
\end{Shaded}

\begin{verbatim}
## function (file, header = TRUE, sep = ",", quote = "\"", dec = ".", 
##     fill = TRUE, comment.char = "", ...) 
## read.table(file = file, header = header, sep = sep, quote = quote, 
##     dec = dec, fill = fill, comment.char = comment.char, ...)
## <bytecode: 0x3e49070>
## <environment: namespace:utils>
\end{verbatim}

Never mind what the function does. Note the
\texttt{environment:\ namespace:utils} line at the end. It tells us that
this function is part of the \textbf{utils} package. We did not need to
know this because it is loaded by default. Here are some packages that I
have currently loaded:

\begin{Shaded}
\begin{Highlighting}[]
\KeywordTok{search}\NormalTok{()}
\end{Highlighting}
\end{Shaded}

\begin{verbatim}
##  [1] ".GlobalEnv"           "package:nycflights13" "package:doSNOW"      
##  [4] "package:snow"         "package:doParallel"   "package:parallel"    
##  [7] "package:iterators"    "package:biganalytics" "package:bigmemory"   
## [10] "package:dplyr"        "package:biglm"        "package:DBI"         
## [13] "package:MatrixModels" "package:plotly"       "package:kernlab"     
## [16] "package:scales"       "package:plyr"         "package:class"       
## [19] "package:rpart"        "package:nnet"         "package:e1071"       
## [22] "package:glmnet"       "package:foreach"      "package:ellipse"     
## [25] "package:nlme"         "package:lattice"      "package:lme4"        
## [28] "package:Matrix"       "package:multcomp"     "package:TH.data"     
## [31] "package:survival"     "package:mvtnorm"      "package:MASS"        
## [34] "package:ggalluvial"   "package:ggplot2"      "package:hexbin"      
## [37] "package:data.table"   "package:magrittr"     "tools:rstudio"       
## [40] "package:stats"        "package:graphics"     "package:grDevices"   
## [43] "package:utils"        "package:datasets"     "package:methods"     
## [46] "Autoloads"            "package:base"
\end{verbatim}

Other packages can be loaded via the \texttt{library} function, or
downloaded from the internet using the \texttt{install.packages}
function before loading with \texttt{library}. R's package import
mechanism is quite powerful, and is one of the reasons for R's success.

\section{Simple Plotting}\label{simple-plotting}

R has many plotting facilities as we will further detail in the Plotting
Chapter \ref{plotting}. We start with the simplest facilities, namely,
the \texttt{plot} function from the \textbf{graphics} package, which is
loaded by default.

\begin{Shaded}
\begin{Highlighting}[]
\NormalTok{x<-}\StringTok{ }\DecValTok{1}\OperatorTok{:}\DecValTok{100}
\NormalTok{y<-}\StringTok{ }\DecValTok{3}\OperatorTok{+}\KeywordTok{sin}\NormalTok{(x) }
\KeywordTok{plot}\NormalTok{(}\DataTypeTok{x =}\NormalTok{ x, }\DataTypeTok{y =}\NormalTok{ y) }\CommentTok{# x,y syntax                         }
\end{Highlighting}
\end{Shaded}

\includegraphics[width=0.5\linewidth]{Rcourse_files/figure-latex/basic-scatter-plot-1}

Given an \texttt{x} argument and a \texttt{y} argument, \texttt{plot}
tries to present a scatter plot. We call this the \texttt{x,y} syntax. R
has another unique syntax to state functional relations. We call
\texttt{y\textasciitilde{}x} the ``tilde'' syntax, which originates in
works of \citet{wilkinson1973symbolic} and was adopted in the early days
of S.

\begin{Shaded}
\begin{Highlighting}[]
\KeywordTok{plot}\NormalTok{(y }\OperatorTok{~}\StringTok{ }\NormalTok{x, }\DataTypeTok{type=}\StringTok{'l'}\NormalTok{) }\CommentTok{# y~x syntax }
\end{Highlighting}
\end{Shaded}

\includegraphics[width=0.5\linewidth]{Rcourse_files/figure-latex/unnamed-chunk-32-1}

The syntax \texttt{y\textasciitilde{}x} is read as ``y is a function of
x''. We will prefer the \texttt{y\textasciitilde{}x} syntax over the
\texttt{x,y} syntax since it is easier to read, and will be very useful
when we discuss more complicated models.

Here are some arguments that control the plot's appearance. We use
\texttt{type} to control the plot type, \texttt{main} to control the
main title.

\begin{Shaded}
\begin{Highlighting}[]
\KeywordTok{plot}\NormalTok{(y}\OperatorTok{~}\NormalTok{x, }\DataTypeTok{type=}\StringTok{'l'}\NormalTok{, }\DataTypeTok{main=}\StringTok{'Plotting a connected line'}\NormalTok{) }
\end{Highlighting}
\end{Shaded}

\includegraphics[width=0.5\linewidth]{Rcourse_files/figure-latex/unnamed-chunk-33-1}

We use \texttt{xlab} for the x-axis label, \texttt{ylab} for the y-axis.

\begin{Shaded}
\begin{Highlighting}[]
\KeywordTok{plot}\NormalTok{(y}\OperatorTok{~}\NormalTok{x, }\DataTypeTok{type=}\StringTok{'h'}\NormalTok{, }\DataTypeTok{main=}\StringTok{'Sticks plot'}\NormalTok{, }\DataTypeTok{xlab=}\StringTok{'Insert x axis label'}\NormalTok{, }\DataTypeTok{ylab=}\StringTok{'Insert y axis label'}\NormalTok{) }
\end{Highlighting}
\end{Shaded}

\includegraphics[width=0.5\linewidth]{Rcourse_files/figure-latex/axis-labels-1}

We use \texttt{pch} to control the point type (pch is acronym for
Plotting CHaracter).

\begin{Shaded}
\begin{Highlighting}[]
\KeywordTok{plot}\NormalTok{(y}\OperatorTok{~}\NormalTok{x, }\DataTypeTok{pch=}\DecValTok{5}\NormalTok{) }\CommentTok{# Point type with pcf}
\end{Highlighting}
\end{Shaded}

\includegraphics[width=0.5\linewidth]{Rcourse_files/figure-latex/unnamed-chunk-34-1}

We use \texttt{col} to control the color, \texttt{cex} (Character
EXpansion) for the point size, and \texttt{abline} (y=Bx+A) to add a
straight line.

\begin{Shaded}
\begin{Highlighting}[]
\KeywordTok{plot}\NormalTok{(y}\OperatorTok{~}\NormalTok{x, }\DataTypeTok{pch=}\DecValTok{10}\NormalTok{, }\DataTypeTok{type=}\StringTok{'p'}\NormalTok{, }\DataTypeTok{col=}\StringTok{'blue'}\NormalTok{, }\DataTypeTok{cex=}\DecValTok{4}\NormalTok{) }
\KeywordTok{abline}\NormalTok{(}\DecValTok{3}\NormalTok{, }\FloatTok{0.002}\NormalTok{) }
\end{Highlighting}
\end{Shaded}

\includegraphics[width=0.5\linewidth]{Rcourse_files/figure-latex/unnamed-chunk-35-1}

For more plotting options run these

\begin{Shaded}
\begin{Highlighting}[]
\KeywordTok{example}\NormalTok{(plot)}
\KeywordTok{example}\NormalTok{(points)}
\NormalTok{?plot}
\KeywordTok{help}\NormalTok{(}\DataTypeTok{package=}\StringTok{'graphics'}\NormalTok{)}
\end{Highlighting}
\end{Shaded}

When your plotting gets serious, go to Chapter \ref{plotting}.

\section{Object Types}\label{object-types}

We already saw that the basic building block of R objects is the vector.
Vectors can be of the following types:

\begin{itemize}
\tightlist
\item
  \textbf{character} Where each element is a string, i.e., a sequence of
  alphanumeric symbols.
\item
  \textbf{numeric} Where each element is a
  \href{https://en.wikipedia.org/wiki/Real_number}{real number} in
  \href{https://en.wikipedia.org/wiki/Double-precision_floating-point_format}{double
  precision} floating point format.
\item
  \textbf{integer} Where each element is an
  \href{https://en.wikipedia.org/wiki/Integer}{integer}.
\item
  \textbf{logical} Where each element is either TRUE, FALSE, or
  NA\footnote{R uses a
    \href{https://en.wikipedia.org/wiki/Three-valued_logic}{\textbf{three}
    valued logic} where a missing value (NA) is neither TRUE, nor FALSE.}
\item
  \textbf{complex} Where each element is a complex number.
\item
  \textbf{list} Where each element is an arbitrary R object.
\item
  \textbf{factor} Factors are not actually vector objects, but they feel
  like such. They are used to encode any finite set of values. This will
  be very useful when fitting linear model because they include
  information on contrasts, i.e., on the encoding of the factors levels.
  You should always be alert and recall when you are dealing with a
  factor or with a character vector. They have different behaviors.
\end{itemize}

Vectors can be combined into larger objects. A \texttt{matrix} can be
thought of as the binding of several vectors of the same type. In
reality, a matrix is merely a vector with a dimension attribute, that
tells R to read it as a matrix and not a vector.

If vectors of different types (but same length) are binded, we get a
\texttt{data.frame} which is the most fundamental object in R for data
analysis. Data frames are brilliant, but a lot has been learned since
their invention. They have thus been extended in recent years with the
\texttt{tbl} class, pronounced {[}Tibble{]}
(\url{https://cran.r-project.org/web/packages/tibble/vignettes/tibble.html}),
and the \texttt{data.table} class.\\
The latter is discussed in Chapter \ref{datatable}, and is strongly
recommended.

\section{Data Frames}\label{data-frames}

Creating a simple data frame:

\begin{Shaded}
\begin{Highlighting}[]
\NormalTok{x<-}\StringTok{ }\DecValTok{1}\OperatorTok{:}\DecValTok{10}
\NormalTok{y<-}\StringTok{ }\DecValTok{3} \OperatorTok{+}\StringTok{ }\KeywordTok{sin}\NormalTok{(x) }
\NormalTok{frame1 <-}\StringTok{ }\KeywordTok{data.frame}\NormalTok{(}\DataTypeTok{x=}\NormalTok{x, }\DataTypeTok{sin=}\NormalTok{y)    }
\end{Highlighting}
\end{Shaded}

Let's inspect our data frame:

\begin{Shaded}
\begin{Highlighting}[]
\KeywordTok{head}\NormalTok{(frame1)}
\end{Highlighting}
\end{Shaded}

\begin{verbatim}
##   x      sin
## 1 1 3.841471
## 2 2 3.909297
## 3 3 3.141120
## 4 4 2.243198
## 5 5 2.041076
## 6 6 2.720585
\end{verbatim}

Now using the RStudio Excel-like viewer:

\begin{Shaded}
\begin{Highlighting}[]
\KeywordTok{View}\NormalTok{(frame1) }
\end{Highlighting}
\end{Shaded}

We highly advise against editing the data this way since there will be
no documentation of the changes you made. Always transform your data
using scripts, so that everything is documented.

Verifying this is a data frame:

\begin{Shaded}
\begin{Highlighting}[]
\KeywordTok{class}\NormalTok{(frame1) }\CommentTok{# the object is of type data.frame}
\end{Highlighting}
\end{Shaded}

\begin{verbatim}
## [1] "data.frame"
\end{verbatim}

Check the dimension of the data

\begin{Shaded}
\begin{Highlighting}[]
\KeywordTok{dim}\NormalTok{(frame1)                             }
\end{Highlighting}
\end{Shaded}

\begin{verbatim}
## [1] 10  2
\end{verbatim}

Note that checking the dimension of a vector is different than checking
the dimension of a data frame.

\begin{Shaded}
\begin{Highlighting}[]
\KeywordTok{length}\NormalTok{(x)}
\end{Highlighting}
\end{Shaded}

\begin{verbatim}
## [1] 10
\end{verbatim}

The length of a \texttt{data.frame} is merely the number of columns.

\begin{Shaded}
\begin{Highlighting}[]
\KeywordTok{length}\NormalTok{(frame1) }
\end{Highlighting}
\end{Shaded}

\begin{verbatim}
## [1] 2
\end{verbatim}

\section{Exctraction}\label{exctraction}

R provides many ways to subset and extract elements from vectors and
other objects. The basics are fairly simple, but not paying attention to
the ``personality'' of each extraction mechanism may cause you a lot of
headache.

For starters, extraction is done with the \texttt{{[}} operator. The
operator can take vectors of many types.

Extracting element with by integer index:

\begin{Shaded}
\begin{Highlighting}[]
\NormalTok{frame1[}\DecValTok{1}\NormalTok{, }\DecValTok{2}\NormalTok{]  }\CommentTok{# exctract the element in the 1st row and 2nd column.}
\end{Highlighting}
\end{Shaded}

\begin{verbatim}
## [1] 3.841471
\end{verbatim}

Extract \textbf{column} by index:

\begin{Shaded}
\begin{Highlighting}[]
\NormalTok{frame1[,}\DecValTok{1}\NormalTok{]                              }
\end{Highlighting}
\end{Shaded}

\begin{verbatim}
##  [1]  1  2  3  4  5  6  7  8  9 10
\end{verbatim}

Extract column by name:

\begin{Shaded}
\begin{Highlighting}[]
\NormalTok{frame1[, }\StringTok{'sin'}\NormalTok{]}
\end{Highlighting}
\end{Shaded}

\begin{verbatim}
##  [1] 3.841471 3.909297 3.141120 2.243198 2.041076 2.720585 3.656987
##  [8] 3.989358 3.412118 2.455979
\end{verbatim}

As a general rule, extraction with \texttt{{[}} will conserve the class
of the parent object. There are, however, exceptions. Notice the
extraction mechanism and the class of the output in the following
examples.

\begin{Shaded}
\begin{Highlighting}[]
\KeywordTok{class}\NormalTok{(frame1[, }\StringTok{'sin'}\NormalTok{])  }\CommentTok{# extracts a column vector}
\end{Highlighting}
\end{Shaded}

\begin{verbatim}
## [1] "numeric"
\end{verbatim}

\begin{Shaded}
\begin{Highlighting}[]
\KeywordTok{class}\NormalTok{(frame1[}\StringTok{'sin'}\NormalTok{])  }\CommentTok{# extracts a data frame}
\end{Highlighting}
\end{Shaded}

\begin{verbatim}
## [1] "data.frame"
\end{verbatim}

\begin{Shaded}
\begin{Highlighting}[]
\KeywordTok{class}\NormalTok{(frame1[,}\DecValTok{1}\OperatorTok{:}\DecValTok{2}\NormalTok{])  }\CommentTok{# extracts a data frame}
\end{Highlighting}
\end{Shaded}

\begin{verbatim}
## [1] "data.frame"
\end{verbatim}

\begin{Shaded}
\begin{Highlighting}[]
\KeywordTok{class}\NormalTok{(frame1[}\DecValTok{2}\NormalTok{])  }\CommentTok{# extracts a data frame}
\end{Highlighting}
\end{Shaded}

\begin{verbatim}
## [1] "data.frame"
\end{verbatim}

\begin{Shaded}
\begin{Highlighting}[]
\KeywordTok{class}\NormalTok{(frame1[}\DecValTok{2}\NormalTok{, ])  }\CommentTok{# extract a data frame}
\end{Highlighting}
\end{Shaded}

\begin{verbatim}
## [1] "data.frame"
\end{verbatim}

\begin{Shaded}
\begin{Highlighting}[]
\KeywordTok{class}\NormalTok{(frame1}\OperatorTok{$}\NormalTok{sin)  }\CommentTok{# extracts a column vector}
\end{Highlighting}
\end{Shaded}

\begin{verbatim}
## [1] "numeric"
\end{verbatim}

The \texttt{subset()} function does the same

\begin{Shaded}
\begin{Highlighting}[]
\KeywordTok{subset}\NormalTok{(frame1, }\DataTypeTok{select=}\NormalTok{sin) }
\KeywordTok{subset}\NormalTok{(frame1, }\DataTypeTok{select=}\DecValTok{2}\NormalTok{)}
\KeywordTok{subset}\NormalTok{(frame1, }\DataTypeTok{select=} \KeywordTok{c}\NormalTok{(}\DecValTok{2}\NormalTok{,}\DecValTok{0}\NormalTok{))}
\end{Highlighting}
\end{Shaded}

If you want to force the stripping of the class attribute when
extracting, try the \texttt{{[}{[}} mechanism instead of \texttt{{[}}.

\begin{Shaded}
\begin{Highlighting}[]
\NormalTok{a <-}\StringTok{ }\NormalTok{frame1[}\DecValTok{1}\NormalTok{] }\CommentTok{# [ extraction}
\NormalTok{b <-}\StringTok{ }\NormalTok{frame1[[}\DecValTok{1}\NormalTok{]] }\CommentTok{# [[ extraction}
\KeywordTok{class}\NormalTok{(a)}\OperatorTok{==}\KeywordTok{class}\NormalTok{(b) }\CommentTok{# objects have differing classes}
\end{Highlighting}
\end{Shaded}

\begin{verbatim}
## [1] FALSE
\end{verbatim}

\begin{Shaded}
\begin{Highlighting}[]
\NormalTok{a}\OperatorTok{==}\NormalTok{b }\CommentTok{# objects are element-wise identical }
\end{Highlighting}
\end{Shaded}

\begin{verbatim}
##          x
##  [1,] TRUE
##  [2,] TRUE
##  [3,] TRUE
##  [4,] TRUE
##  [5,] TRUE
##  [6,] TRUE
##  [7,] TRUE
##  [8,] TRUE
##  [9,] TRUE
## [10,] TRUE
\end{verbatim}

The different types of output classes cause different behaviors. Compare
the behavior of \texttt{{[}} on seemingly identical objects.

\begin{Shaded}
\begin{Highlighting}[]
\NormalTok{frame1[}\DecValTok{1}\NormalTok{][}\DecValTok{1}\NormalTok{]}
\end{Highlighting}
\end{Shaded}

\begin{verbatim}
##     x
## 1   1
## 2   2
## 3   3
## 4   4
## 5   5
## 6   6
## 7   7
## 8   8
## 9   9
## 10 10
\end{verbatim}

\begin{Shaded}
\begin{Highlighting}[]
\NormalTok{frame1[[}\DecValTok{1}\NormalTok{]][}\DecValTok{1}\NormalTok{]}
\end{Highlighting}
\end{Shaded}

\begin{verbatim}
## [1] 1
\end{verbatim}

If you want to learn more about subsetting see
\href{http://adv-r.had.co.nz/Subsetting.html}{Hadley's guide}.

\section{Augmentations of the data.frame
class}\label{augmentations-of-the-data.frame-class}

As previously mentioned, the \texttt{data.frame} class has been extended
in recent years. The best known extensions are the \texttt{data.table}
and the \texttt{tbl}. For beginners, it is important to know R's basics,
so we keep focusing on data frames. For more advanced users, I recommend
learning the (amazing) \texttt{data.table} syntax.

\section{Data Import and Export}\label{data-import-and-export}

For any practical purpose, you will not be generating your data
manually. R comes with many importing and exporting mechanisms which we
now present. If, however, you do a lot of data ``munging'', make sure to
see Hadley-verse Chapter \ref{hadley}. If you work with MASSIVE data
sets, read the Memory Efficiency Chapter \ref{memory}.

\subsection{Import from WEB}\label{import-from-web}

The \texttt{read.table} function is the main importing workhorse. It can
import directly from the web.

\begin{Shaded}
\begin{Highlighting}[]
\NormalTok{URL <-}\StringTok{ 'http://statweb.stanford.edu/~tibs/ElemStatLearn/datasets/bone.data'}
\NormalTok{tirgul1 <-}\StringTok{ }\KeywordTok{read.table}\NormalTok{(URL)}
\end{Highlighting}
\end{Shaded}

Always look at the imported result!

\begin{Shaded}
\begin{Highlighting}[]
\KeywordTok{head}\NormalTok{(tirgul1)}
\end{Highlighting}
\end{Shaded}

\begin{verbatim}
##      V1    V2     V3          V4
## 1 idnum   age gender      spnbmd
## 2     1  11.7   male  0.01808067
## 3     1  12.7   male  0.06010929
## 4     1 13.75   male 0.005857545
## 5     2 13.25   male  0.01026393
## 6     2  14.3   male   0.2105263
\end{verbatim}

Oh dear. \texttt{read.,table} tried to guess the structure of the input,
but failed to recognize the header row. Set it manually with
\texttt{header=TRUE}:

\begin{Shaded}
\begin{Highlighting}[]
\NormalTok{tirgul1 <-}\StringTok{ }\KeywordTok{read.table}\NormalTok{(}\StringTok{'data/bone.data'}\NormalTok{, }\DataTypeTok{header =} \OtherTok{TRUE}\NormalTok{) }
\KeywordTok{head}\NormalTok{(tirgul1)}
\end{Highlighting}
\end{Shaded}

\subsection{Import From Clipboard}\label{import-from-clipboard}

TODO:\href{https://github.com/MilesMcBain/datapasta}{datapasta}

\subsection{Export as CSV}\label{export-as-csv}

Let's write a simple file so that we have something to import

\begin{Shaded}
\begin{Highlighting}[]
\KeywordTok{head}\NormalTok{(airquality) }\CommentTok{#  examine the data to export}
\end{Highlighting}
\end{Shaded}

\begin{verbatim}
##   Ozone Solar.R Wind Temp Month Day
## 1    41     190  7.4   67     5   1
## 2    36     118  8.0   72     5   2
## 3    12     149 12.6   74     5   3
## 4    18     313 11.5   62     5   4
## 5    NA      NA 14.3   56     5   5
## 6    28      NA 14.9   66     5   6
\end{verbatim}

\begin{Shaded}
\begin{Highlighting}[]
\NormalTok{temp.file.name <-}\StringTok{ }\KeywordTok{tempfile}\NormalTok{() }\CommentTok{# get some arbitrary file name}
\KeywordTok{write.csv}\NormalTok{(}\DataTypeTok{x =}\NormalTok{ airquality, }\DataTypeTok{file =}\NormalTok{ temp.file.name) }\CommentTok{# export}
\end{Highlighting}
\end{Shaded}

Now let's import the exported file. Being a .csv file, I can use
\texttt{read.csv} instead of \texttt{read.table}.

\begin{Shaded}
\begin{Highlighting}[]
\NormalTok{my.data<-}\StringTok{ }\KeywordTok{read.csv}\NormalTok{(}\DataTypeTok{file=}\NormalTok{temp.file.name) }\CommentTok{# import}
\KeywordTok{head}\NormalTok{(my.data) }\CommentTok{# verify import}
\end{Highlighting}
\end{Shaded}

\begin{verbatim}
##   X Ozone Solar.R Wind Temp Month Day
## 1 1    41     190  7.4   67     5   1
## 2 2    36     118  8.0   72     5   2
## 3 3    12     149 12.6   74     5   3
## 4 4    18     313 11.5   62     5   4
## 5 5    NA      NA 14.3   56     5   5
## 6 6    28      NA 14.9   66     5   6
\end{verbatim}

\BeginKnitrBlock{remark}
\iffalse{} {Remark. } \fi{}Windows users may need to use
``\textbackslash{}'' instead of ``/''.
\EndKnitrBlock{remark}

\subsection{Export non-CSV files}\label{export-non-csv-files}

You can export your R objects in endlessly many ways: If instead of the
comma delimiter in .csv you want other column delimiters, look into
\texttt{?write.table}. If you are exporting only for R users, you can
consider exporting as binary objects with \texttt{saveRDS},
\texttt{feather::write\_feather}, or \texttt{fst::write.fst}. See
(\url{http://www.fstpackage.org/}) for a comparison.

\subsection{Reading From Text Files}\label{reading-from-text-files}

Some general notes on importing text files via the \texttt{read.table}
function. But first, we need to know what is the active directory. Here
is how to get and set R's active directory:

\begin{Shaded}
\begin{Highlighting}[]
\KeywordTok{getwd}\NormalTok{() }\CommentTok{#What is the working directory?}
\KeywordTok{setwd}\NormalTok{() }\CommentTok{#Setting the working directory in Linux}
\end{Highlighting}
\end{Shaded}

We can now call the \texttt{read.table} function to import text files.
If you care about your sanity, see \texttt{?read.table} before starting
imports. Some notable properties of the function:

\begin{itemize}
\tightlist
\item
  \texttt{read.table} will try to guess column separators (tab, comma,
  etc.)
\item
  \texttt{read.table} will try to guess if a header row is present.
\item
  \texttt{read.table} will convert character vectors to factors unless
  told not to using the \texttt{stringsAsFactors=FALSE} argument.
\item
  The output of \texttt{read.table} needs to be explicitly assigned to
  an object for it to be saved.
\end{itemize}

\subsection{Writing Data to Text
Files}\label{writing-data-to-text-files}

The function \texttt{write.table} is the exporting counterpart of
\texttt{read.table}.

\subsection{.XLS(X) files}\label{xlsx-files}

Strongly recommended to convert to .csv in Excel, and then import as
csv. If you still insist see the \textbf{xlsx} package.

\subsection{Massive files}\label{massive-files}

The above importing and exporting mechanisms were not designed for
massive files. See the section on the \textbf{data.table} package
(\ref{datatable}), Sparse Representation (\ref{sparse}), and Out-of-Ram
Algorithms (\ref{memory}) for more on working with massive data files.

\subsection{Databases}\label{databases}

R does not need to read from text files; it can read directly from a
database. This is very useful since it allows the filtering, selecting
and joining operations to rely on the database's optimized algorithms.
Then again, if you will only be analyzing your data with R, you are
probably better of by working from a file, without the databases'
overhead. See Chapter \ref{memory} for more on this matter.

\section{Functions}\label{functions}

One of the most basic building blocks of programming is the ability of
writing your own functions. A function in R, like everything else, is an
object accessible using its name. We first define a simple function that
sums its two arguments

\begin{Shaded}
\begin{Highlighting}[]
\NormalTok{my.sum <-}\StringTok{ }\ControlFlowTok{function}\NormalTok{(x,y) \{}
  \KeywordTok{return}\NormalTok{(x}\OperatorTok{+}\NormalTok{y)}
\NormalTok{\}}
\KeywordTok{my.sum}\NormalTok{(}\DecValTok{10}\NormalTok{,}\DecValTok{2}\NormalTok{)}
\end{Highlighting}
\end{Shaded}

\begin{verbatim}
## [1] 12
\end{verbatim}

From this example you may notice that:

\begin{itemize}
\item
  The function \texttt{function} tells R to construct a function object.
\item
  Unlike some programming languages, a period (\texttt{.}) is allowed as
  part of an object's name.
\item
  The arguments of the \texttt{function}, i.e. \texttt{(x,y)}, need to
  be named but we are not required to specify their class. This makes
  writing functions very easy, but it is also the source of many bugs,
  and slowness of R compared to type declaring languages (C,
  Fortran,Java,\ldots{}).
\item
  A typical R function does not change objects\footnote{This is a
    classical \emph{functional programming} paradigm. If you want an
    object oriented flavor of R programming, see Hadley's
    \href{http://adv-r.had.co.nz/OO-essentials.html}{Advanced R book}.}
  but rather creates new ones. To save the output of \texttt{my.sum} we
  will need to assign it using the \texttt{\textless{}-} operator.
\end{itemize}

Here is a (slightly) more advanced function:

\begin{Shaded}
\begin{Highlighting}[]
\NormalTok{my.sum.}\DecValTok{2}\NormalTok{ <-}\StringTok{ }\ControlFlowTok{function}\NormalTok{(x, y , }\DataTypeTok{absolute=}\OtherTok{FALSE}\NormalTok{) \{}
  \ControlFlowTok{if}\NormalTok{(absolute}\OperatorTok{==}\OtherTok{TRUE}\NormalTok{) \{}
\NormalTok{    result <-}\StringTok{ }\KeywordTok{abs}\NormalTok{(x}\OperatorTok{+}\NormalTok{y)}
\NormalTok{  \}}
  \ControlFlowTok{else}\NormalTok{\{}
\NormalTok{    result <-}\StringTok{ }\NormalTok{x}\OperatorTok{+}\NormalTok{y}
\NormalTok{  \} }
\NormalTok{  result}
\NormalTok{\}}
\KeywordTok{my.sum.2}\NormalTok{(}\OperatorTok{-}\DecValTok{10}\NormalTok{,}\DecValTok{2}\NormalTok{,}\OtherTok{TRUE}\NormalTok{)}
\end{Highlighting}
\end{Shaded}

\begin{verbatim}
## [1] 8
\end{verbatim}

Things to note:

\begin{itemize}
\item
  \texttt{if(condition)\{expression1\}\ else\{expression2\}} does just
  what the name suggests.
\item
  The function will output its last evaluated expression. You don't need
  to use the \texttt{return} function explicitly.
\item
  Using \texttt{absolute=FALSE} sets the default value of
  \texttt{absolute} to \texttt{FALSE}. This is overridden if
  \texttt{absolute} is stated explicitly in the function call.
\end{itemize}

An important behavior of R is the \emph{scoping rules}. This refers to
the way R seeks for variables used in functions. As a rule of thumb, R
will first look for variables inside the function and if not found, will
search for the variable values in outer environments\footnote{More
  formally, this is called
  \href{https://darrenjw.wordpress.com/2011/11/23/lexical-scope-and-function-closures-in-r/}{Lexical
  Scoping}.}. Think of the next example.

\begin{Shaded}
\begin{Highlighting}[]
\NormalTok{a <-}\StringTok{ }\DecValTok{1}
\NormalTok{b <-}\StringTok{ }\DecValTok{2}
\NormalTok{x <-}\StringTok{ }\DecValTok{3}
\NormalTok{scoping <-}\StringTok{ }\ControlFlowTok{function}\NormalTok{(a,b)\{}
\NormalTok{  a}\OperatorTok{+}\NormalTok{b}\OperatorTok{+}\NormalTok{x}
\NormalTok{\}}
\KeywordTok{scoping}\NormalTok{(}\DecValTok{10}\NormalTok{,}\DecValTok{11}\NormalTok{)}
\end{Highlighting}
\end{Shaded}

\begin{verbatim}
## [1] 24
\end{verbatim}

\section{Looping}\label{looping}

The real power of scripting is when repeated operations are done by
iteration. R supports the usual \texttt{for}, \texttt{while}, and
\texttt{repated} loops. Here is an embarrassingly simple example

\begin{Shaded}
\begin{Highlighting}[]
\ControlFlowTok{for}\NormalTok{ (i }\ControlFlowTok{in} \DecValTok{1}\OperatorTok{:}\DecValTok{5}\NormalTok{)\{}
    \KeywordTok{print}\NormalTok{(i)}
\NormalTok{    \}}
\end{Highlighting}
\end{Shaded}

\begin{verbatim}
## [1] 1
## [1] 2
## [1] 3
## [1] 4
## [1] 5
\end{verbatim}

A slightly more advanced example, is vector multiplication

\begin{Shaded}
\begin{Highlighting}[]
\NormalTok{result <-}\StringTok{ }\DecValTok{0}
\NormalTok{n <-}\StringTok{ }\FloatTok{1e3}
\NormalTok{x <-}\StringTok{ }\DecValTok{1}\OperatorTok{:}\NormalTok{n}
\NormalTok{y <-}\StringTok{ }\NormalTok{(}\DecValTok{1}\OperatorTok{:}\NormalTok{n)}\OperatorTok{/}\NormalTok{n}
\ControlFlowTok{for}\NormalTok{(i }\ControlFlowTok{in} \DecValTok{1}\OperatorTok{:}\NormalTok{n)\{}
\NormalTok{  result <-}\StringTok{ }\NormalTok{result}\OperatorTok{+}\StringTok{ }\NormalTok{x[i]}\OperatorTok{*}\NormalTok{y[i]}
\NormalTok{\}}
\end{Highlighting}
\end{Shaded}

\BeginKnitrBlock{remark}
\iffalse{} {Remark. } \fi{}\textbf{Vector Operations}: You should NEVER
write your own vector and matrix products like in the previous example.
Only use existing facilities such as \texttt{\%*\%}, \texttt{sum()},
etc.
\EndKnitrBlock{remark}

\BeginKnitrBlock{remark}
\iffalse{} {Remark. } \fi{}\textbf{Parallel Operations}: If you already
know that you will be needing to parallelize your work, get used to
working with \texttt{foreach} loops in the \textbf{foreach} package,
rather then regular \texttt{for} loops.
\EndKnitrBlock{remark}

\section{Apply}\label{apply}

For applying the same function to a set of elements, there is no need to
write an explicit loop. This is such an elementary operation that every
programming language will provide some facility to \textbf{apply}, or
\textbf{map} the function to all elements of a set. R provides several
facilities to perform this. The most basic of which is \texttt{lapply}
which applies a function over all elements of a list, and return a list
of outputs:

\begin{Shaded}
\begin{Highlighting}[]
\NormalTok{the.list <-}\StringTok{ }\KeywordTok{list}\NormalTok{(}\DecValTok{1}\NormalTok{,}\StringTok{'a'}\NormalTok{,mean) }\CommentTok{# a list of 3 elements from different classes}
\KeywordTok{lapply}\NormalTok{(}\DataTypeTok{X =}\NormalTok{ the.list, }\DataTypeTok{FUN =}\NormalTok{ class) }\CommentTok{# apply the function `class` to each elements}
\end{Highlighting}
\end{Shaded}

\begin{verbatim}
## [[1]]
## [1] "numeric"
## 
## [[2]]
## [1] "character"
## 
## [[3]]
## [1] "standardGeneric"
## attr(,"package")
## [1] "methods"
\end{verbatim}

\begin{Shaded}
\begin{Highlighting}[]
\KeywordTok{sapply}\NormalTok{(}\DataTypeTok{X =}\NormalTok{ the.list, }\DataTypeTok{FUN =}\NormalTok{ class) }\CommentTok{# lapply with cleaned output}
\end{Highlighting}
\end{Shaded}

\begin{verbatim}
## [1] "numeric"         "character"       "standardGeneric"
\end{verbatim}

What is the function you are using requires some arguments? One useful
trick is to create your own function that takes only one argument:

\begin{Shaded}
\begin{Highlighting}[]
\NormalTok{quantile.}\DecValTok{25}\NormalTok{ <-}\StringTok{ }\ControlFlowTok{function}\NormalTok{(x) }\KeywordTok{quantile}\NormalTok{(x,}\FloatTok{0.25}\NormalTok{)}
\KeywordTok{sapply}\NormalTok{(USArrests, quantile.}\DecValTok{25}\NormalTok{)}
\end{Highlighting}
\end{Shaded}

\begin{verbatim}
##   Murder.25%  Assault.25% UrbanPop.25%     Rape.25% 
##        4.075      109.000       54.500       15.075
\end{verbatim}

What if you are applying the same function with \textbf{two} lists of
arguments? Use \textbf{mapply}. The following will compute a different
quantile to each column in the data:

\begin{Shaded}
\begin{Highlighting}[]
\NormalTok{quantiles <-}\StringTok{ }\KeywordTok{c}\NormalTok{(}\FloatTok{0.1}\NormalTok{, }\FloatTok{0.5}\NormalTok{, }\FloatTok{0.3}\NormalTok{, }\FloatTok{0.2}\NormalTok{)}
\KeywordTok{mapply}\NormalTok{(quantile, USArrests, quantiles)}
\end{Highlighting}
\end{Shaded}

\begin{verbatim}
##   Murder.10%  Assault.50% UrbanPop.30%     Rape.20% 
##         2.56       159.00        57.70        13.92
\end{verbatim}

R provides many variations on \texttt{lapply} to facilitate programming.
Here is a partial list:

\begin{itemize}
\tightlist
\item
  \texttt{sapply}: The same as \texttt{lapply} but tries to arrange
  output in a vector or matrix, and not an unstructured list.
\item
  \texttt{vapply}: A safer version of \texttt{sapply}, where the output
  class is pre-specified.
\item
  \texttt{apply}: For applying over the rows or columns of matrices.
\item
  \texttt{mapply}: For applying functions with more than a single input.
\item
  \texttt{tapply}: For splitting vectors and applying functions on
  subsets.
\item
  \texttt{rapply}: A recursive version of \texttt{lapply}.
\item
  \texttt{eapply}: Like \texttt{lapply}, only operates on
  \texttt{environments} instead of lists.
\item
  \texttt{Map}+\texttt{Reduce}: For a
  \href{https://en.wikipedia.org/wiki/Common_Lisp}{Common Lisp} look and
  feel of \texttt{lapply}.
\item
  \texttt{parallel::parLapply}: A parallel version of \texttt{lapply}
  from the package \textbf{parallel}.
\item
  \texttt{parallel::parLBapply}: A parallel version of \texttt{lapply},
  with load balancing from the package \textbf{parallel}.
\end{itemize}

\section{Recursion}\label{recursion}

The R compiler is really not designed for recursion, and you will rarely
need to do so.\\
See the RCpp Chapter \ref{rcpp} for linking C code, which is better
suited for recursion. If you really insist to write recursions in R,
make sure to use the \texttt{Recall} function, which, as the name
suggests, recalls the function in which it is place. Here is a
demonstration with the Fibonacci series.

\begin{Shaded}
\begin{Highlighting}[]
\NormalTok{fib<-}\ControlFlowTok{function}\NormalTok{(n) \{}
    \ControlFlowTok{if}\NormalTok{ (n }\OperatorTok{<=}\StringTok{ }\DecValTok{2}\NormalTok{) fn<-}\DecValTok{1} 
    \ControlFlowTok{else}\NormalTok{ fn <-}\StringTok{ }\KeywordTok{Recall}\NormalTok{(n }\OperatorTok{-}\StringTok{ }\DecValTok{1}\NormalTok{) }\OperatorTok{+}\StringTok{ }\KeywordTok{Recall}\NormalTok{(n }\OperatorTok{-}\StringTok{ }\DecValTok{2}\NormalTok{) }
    \KeywordTok{return}\NormalTok{(fn)}
\NormalTok{\} }
\KeywordTok{fib}\NormalTok{(}\DecValTok{5}\NormalTok{)}
\end{Highlighting}
\end{Shaded}

\begin{verbatim}
## [1] 5
\end{verbatim}

\section{Strings}\label{strings}

Note: this section is courtesy of Ron Sarafian.

Strings may appear as character vectors,files names, paths
(directories), graphing elements, and more.

Strings can be concatenated with the super useful \texttt{paste}
function.

\begin{Shaded}
\begin{Highlighting}[]
\NormalTok{a <-}\StringTok{ "good"}
\NormalTok{b <-}\StringTok{ "morning"}
\KeywordTok{is.character}\NormalTok{(a)}
\end{Highlighting}
\end{Shaded}

\begin{verbatim}
## [1] TRUE
\end{verbatim}

\begin{Shaded}
\begin{Highlighting}[]
\KeywordTok{paste}\NormalTok{(a,b)}
\end{Highlighting}
\end{Shaded}

\begin{verbatim}
## [1] "good morning"
\end{verbatim}

\begin{Shaded}
\begin{Highlighting}[]
\NormalTok{(c <-}\StringTok{ }\KeywordTok{paste}\NormalTok{(a,b, }\DataTypeTok{sep =} \StringTok{"."}\NormalTok{))}
\end{Highlighting}
\end{Shaded}

\begin{verbatim}
## [1] "good.morning"
\end{verbatim}

\begin{Shaded}
\begin{Highlighting}[]
\KeywordTok{paste}\NormalTok{(a,b,}\DecValTok{1}\OperatorTok{:}\DecValTok{3}\NormalTok{, }\DataTypeTok{paste=}\StringTok{'@@@'}\NormalTok{, }\DataTypeTok{collapse =} \StringTok{'^^^^'}\NormalTok{)}
\end{Highlighting}
\end{Shaded}

\begin{verbatim}
## [1] "good morning 1 @@@^^^^good morning 2 @@@^^^^good morning 3 @@@"
\end{verbatim}

Things to note:

\begin{itemize}
\tightlist
\item
  \texttt{sep} is used to separate strings.
\item
  \texttt{collapse} is used to separate results.
\end{itemize}

The \texttt{substr} function extract or replace substrings in a
character vector:

\begin{Shaded}
\begin{Highlighting}[]
\KeywordTok{substr}\NormalTok{(c, }\DataTypeTok{start=}\DecValTok{2}\NormalTok{, }\DataTypeTok{stop=}\DecValTok{4}\NormalTok{)}
\end{Highlighting}
\end{Shaded}

\begin{verbatim}
## [1] "ood"
\end{verbatim}

\begin{Shaded}
\begin{Highlighting}[]
\KeywordTok{substr}\NormalTok{(c, }\DataTypeTok{start=}\DecValTok{6}\NormalTok{, }\DataTypeTok{stop=}\DecValTok{12}\NormalTok{) <-}\StringTok{ "evening"}
\end{Highlighting}
\end{Shaded}

The \texttt{grep} function is a very powerful tool to search for
patterns in text. These patterns are called
\href{https://en.wikipedia.org/wiki/Regular_expression}{regular
expressions}

\begin{Shaded}
\begin{Highlighting}[]
\NormalTok{(d <-}\StringTok{ }\KeywordTok{c}\NormalTok{(a,b,c))}
\end{Highlighting}
\end{Shaded}

\begin{verbatim}
## [1] "good"         "morning"      "good.evening"
\end{verbatim}

\begin{Shaded}
\begin{Highlighting}[]
\KeywordTok{grep}\NormalTok{(}\DataTypeTok{pattern =} \StringTok{"good"}\NormalTok{,}\DataTypeTok{x =}\NormalTok{ d) }
\end{Highlighting}
\end{Shaded}

\begin{verbatim}
## [1] 1 3
\end{verbatim}

\begin{Shaded}
\begin{Highlighting}[]
\KeywordTok{grep}\NormalTok{(}\StringTok{"good"}\NormalTok{,d, }\DataTypeTok{value=}\OtherTok{TRUE}\NormalTok{, }\DataTypeTok{ignore.case=}\OtherTok{TRUE}\NormalTok{) }
\end{Highlighting}
\end{Shaded}

\begin{verbatim}
## [1] "good"         "good.evening"
\end{verbatim}

\begin{Shaded}
\begin{Highlighting}[]
\KeywordTok{grep}\NormalTok{(}\StringTok{"([a-zA-Z]+)}\CharTok{\textbackslash{}\textbackslash{}}\StringTok{1"}\NormalTok{,d, }\DataTypeTok{value=}\OtherTok{TRUE}\NormalTok{, }\DataTypeTok{perl=}\OtherTok{TRUE}\NormalTok{) }
\end{Highlighting}
\end{Shaded}

\begin{verbatim}
## [1] "good"         "good.evening"
\end{verbatim}

Things to note:

\begin{itemize}
\tightlist
\item
  Use \texttt{value=TRUE} to return the string itself, instead of its
  index.
\item
  \texttt{({[}a-zA-Z{]}+)\textbackslash{}\textbackslash{}1} is a regular
  expression to find repeating characters. \texttt{perl=TRUE} to
  activate the \href{https://en.wikipedia.org/wiki/Perl}{Perl}
  ``flavored'' regular expressions.
\end{itemize}

Use \texttt{gsub} to replace characters in a string object:

\begin{Shaded}
\begin{Highlighting}[]
\KeywordTok{gsub}\NormalTok{(}\StringTok{"o"}\NormalTok{, }\StringTok{"q"}\NormalTok{, d) }\CommentTok{# replace the letter "o" with "q".}
\end{Highlighting}
\end{Shaded}

\begin{verbatim}
## [1] "gqqd"         "mqrning"      "gqqd.evening"
\end{verbatim}

\begin{Shaded}
\begin{Highlighting}[]
\KeywordTok{gsub}\NormalTok{(}\StringTok{"([a-zA-Z]+)}\CharTok{\textbackslash{}\textbackslash{}}\StringTok{1"}\NormalTok{, }\StringTok{"q"}\NormalTok{, d, }\DataTypeTok{perl=}\OtherTok{TRUE}\NormalTok{) }\CommentTok{# replace repeating characters with "q".}
\end{Highlighting}
\end{Shaded}

\begin{verbatim}
## [1] "gqd"         "morning"     "gqd.evening"
\end{verbatim}

The \texttt{strsplit} allows to split string vectors to list:

\begin{Shaded}
\begin{Highlighting}[]
\NormalTok{(x <-}\StringTok{ }\KeywordTok{c}\NormalTok{(}\DataTypeTok{a =} \StringTok{"thiszis"}\NormalTok{, }\DataTypeTok{b =} \StringTok{"justzan"}\NormalTok{, }\DataTypeTok{c =} \StringTok{"example"}\NormalTok{))}
\end{Highlighting}
\end{Shaded}

\begin{verbatim}
##         a         b         c 
## "thiszis" "justzan" "example"
\end{verbatim}

\begin{Shaded}
\begin{Highlighting}[]
\KeywordTok{strsplit}\NormalTok{(x, }\StringTok{"z"}\NormalTok{) }\CommentTok{# split x on the letter z}
\end{Highlighting}
\end{Shaded}

\begin{verbatim}
## $a
## [1] "this" "is"  
## 
## $b
## [1] "just" "an"  
## 
## $c
## [1] "example"
\end{verbatim}

Some more examples:

\begin{Shaded}
\begin{Highlighting}[]
\KeywordTok{nchar}\NormalTok{(x) }\CommentTok{#  count the nuber of characters in every element of a string vector.}
\end{Highlighting}
\end{Shaded}

\begin{verbatim}
## a b c 
## 7 7 7
\end{verbatim}

\begin{Shaded}
\begin{Highlighting}[]
\KeywordTok{toupper}\NormalTok{(x) }\CommentTok{# translate characters in character vectors to upper case}
\end{Highlighting}
\end{Shaded}

\begin{verbatim}
##         a         b         c 
## "THISZIS" "JUSTZAN" "EXAMPLE"
\end{verbatim}

\begin{Shaded}
\begin{Highlighting}[]
\KeywordTok{tolower}\NormalTok{(}\KeywordTok{toupper}\NormalTok{(x)) }\CommentTok{# vice verca}
\end{Highlighting}
\end{Shaded}

\begin{verbatim}
##         a         b         c 
## "thiszis" "justzan" "example"
\end{verbatim}

\begin{Shaded}
\begin{Highlighting}[]
\NormalTok{letters[}\DecValTok{1}\OperatorTok{:}\DecValTok{10}\NormalTok{] }\CommentTok{# lower case letters vector}
\end{Highlighting}
\end{Shaded}

\begin{verbatim}
##  [1] "a" "b" "c" "d" "e" "f" "g" "h" "i" "j"
\end{verbatim}

\begin{Shaded}
\begin{Highlighting}[]
\NormalTok{LETTERS[}\DecValTok{1}\OperatorTok{:}\DecValTok{10}\NormalTok{] }\CommentTok{# upper case letters vector}
\end{Highlighting}
\end{Shaded}

\begin{verbatim}
##  [1] "A" "B" "C" "D" "E" "F" "G" "H" "I" "J"
\end{verbatim}

\begin{Shaded}
\begin{Highlighting}[]
\KeywordTok{cat}\NormalTok{(}\StringTok{"the sum of"}\NormalTok{, }\DecValTok{1}\NormalTok{, }\StringTok{"and"}\NormalTok{, }\DecValTok{2}\NormalTok{, }\StringTok{"is"}\NormalTok{, }\DecValTok{1}\OperatorTok{+}\DecValTok{2}\NormalTok{) }\CommentTok{#  concatenate and print strings and values}
\end{Highlighting}
\end{Shaded}

\begin{verbatim}
## the sum of 1 and 2 is 3
\end{verbatim}

If you need more than this, look for the
\href{https://r4ds.had.co.nz/strings.html}{stringr} package that
provides a set of internally consistent tools.

\section{Dates and Times}\label{dates-and-times}

Note: This Section is courtesy of
\href{https://www.linkedin.com/in/ron-sarafian-4a5a95110/}{Ron
Sarafian}.

\subsection{Dates}\label{dates}

R provides several packages for dealing with date and date/time data. We
start with the \texttt{base} package.

R needs to be informed explicitly that an object holds dates. The
\texttt{as.Date} function convert values to dates. You can pass it a
\texttt{character}, a \texttt{numeric}, or a \texttt{POSIXct} (we'll
soon explain what it is).

\begin{Shaded}
\begin{Highlighting}[]
\NormalTok{start <-}\StringTok{ "1948-05-14"}
\KeywordTok{class}\NormalTok{(start)}
\end{Highlighting}
\end{Shaded}

\begin{verbatim}
## [1] "character"
\end{verbatim}

\begin{Shaded}
\begin{Highlighting}[]
\NormalTok{start <-}\StringTok{ }\KeywordTok{as.Date}\NormalTok{(start)}
\KeywordTok{class}\NormalTok{(start)}
\end{Highlighting}
\end{Shaded}

\begin{verbatim}
## [1] "Date"
\end{verbatim}

But what if our date is not in the yyyy-mm-dd format? We can tell R what
is the character date's format.

\begin{Shaded}
\begin{Highlighting}[]
\KeywordTok{as.Date}\NormalTok{(}\StringTok{"14/5/1948"}\NormalTok{, }\DataTypeTok{format=}\StringTok{"%d/%m/%Y"}\NormalTok{)}
\end{Highlighting}
\end{Shaded}

\begin{verbatim}
## [1] "1948-05-14"
\end{verbatim}

\begin{Shaded}
\begin{Highlighting}[]
\KeywordTok{as.Date}\NormalTok{(}\StringTok{"14may1948"}\NormalTok{, }\DataTypeTok{format=}\StringTok{"%d%b%Y"}\NormalTok{)}
\end{Highlighting}
\end{Shaded}

\begin{verbatim}
## [1] "1948-05-14"
\end{verbatim}

Things to note:

\begin{itemize}
\tightlist
\item
  The format of the date is specified with the \texttt{format=}
  argument. \texttt{\%d} for day of the month, \texttt{/} for
  separation, \texttt{\%m} for month, and \texttt{\%Y} for year in four
  digits. See \texttt{?strptime} for more available formatting.
\item
  If it returns NA, then use the command
  \texttt{Sys.setlocale("LC\_TIME","C")}
\end{itemize}

Many functions are content aware, and adapt their behavior when dealing
with dates:

\begin{Shaded}
\begin{Highlighting}[]
\NormalTok{(today <-}\StringTok{ }\KeywordTok{Sys.Date}\NormalTok{()) }\CommentTok{# the current date}
\end{Highlighting}
\end{Shaded}

\begin{verbatim}
## [1] "2019-03-31"
\end{verbatim}

\begin{Shaded}
\begin{Highlighting}[]
\NormalTok{today }\OperatorTok{+}\StringTok{ }\DecValTok{1} \CommentTok{# Add one day}
\end{Highlighting}
\end{Shaded}

\begin{verbatim}
## [1] "2019-04-01"
\end{verbatim}

\begin{Shaded}
\begin{Highlighting}[]
\NormalTok{today }\OperatorTok{-}\StringTok{ }\NormalTok{start }\CommentTok{# Diffenrece between dates}
\end{Highlighting}
\end{Shaded}

\begin{verbatim}
## Time difference of 25888 days
\end{verbatim}

\begin{Shaded}
\begin{Highlighting}[]
\KeywordTok{min}\NormalTok{(start,today)}
\end{Highlighting}
\end{Shaded}

\begin{verbatim}
## [1] "1948-05-14"
\end{verbatim}

\subsection{Times}\label{times}

Specifying times is similar to dates, only that more formatting
parameters are required. The \texttt{POSIXct} is the object class for
times. It expects strings to be in the format YYYY-MM-DD HH:MM:SS. With
\texttt{POSIXct} you can also specify the timezone, e.g.,
\texttt{"Asia/Jerusalem"}.

\begin{Shaded}
\begin{Highlighting}[]
\NormalTok{time1 <-}\StringTok{ }\KeywordTok{Sys.time}\NormalTok{()}
\KeywordTok{class}\NormalTok{(time1)}
\end{Highlighting}
\end{Shaded}

\begin{verbatim}
## [1] "POSIXct" "POSIXt"
\end{verbatim}

\begin{Shaded}
\begin{Highlighting}[]
\NormalTok{time2 <-}\StringTok{ }\NormalTok{time1 }\OperatorTok{+}\StringTok{ }\DecValTok{72}\OperatorTok{*}\DecValTok{60}\OperatorTok{*}\DecValTok{60} \CommentTok{# add 72 hours}
\NormalTok{time2}\OperatorTok{-}\NormalTok{time1}
\end{Highlighting}
\end{Shaded}

\begin{verbatim}
## Time difference of 3 days
\end{verbatim}

\begin{Shaded}
\begin{Highlighting}[]
\KeywordTok{class}\NormalTok{(time2}\OperatorTok{-}\NormalTok{time1)}
\end{Highlighting}
\end{Shaded}

\begin{verbatim}
## [1] "difftime"
\end{verbatim}

Things to note:

\begin{itemize}
\tightlist
\item
  Be careful about DST, because
  \texttt{as.POSIXct("2019-03-29\ 01:30")+3600} will not add 1 hour, but
  2 with the result: \texttt{{[}1{]}\ "2019-03-29\ 03:30:00\ IDT"}
\end{itemize}

Compute differences in your unit of choice:

\begin{Shaded}
\begin{Highlighting}[]
\KeywordTok{difftime}\NormalTok{(time2,time1, }\DataTypeTok{units =}  \StringTok{"hour"}\NormalTok{)}
\end{Highlighting}
\end{Shaded}

\begin{verbatim}
## Time difference of 72 hours
\end{verbatim}

\begin{Shaded}
\begin{Highlighting}[]
\KeywordTok{difftime}\NormalTok{(time2,time1, }\DataTypeTok{units =}  \StringTok{"week"}\NormalTok{)}
\end{Highlighting}
\end{Shaded}

\begin{verbatim}
## Time difference of 0.4285714 weeks
\end{verbatim}

Generate sequences:

\begin{Shaded}
\begin{Highlighting}[]
\KeywordTok{seq}\NormalTok{(}\DataTypeTok{from =}\NormalTok{ time1, }\DataTypeTok{to =}\NormalTok{ time2, }\DataTypeTok{by =} \StringTok{"day"}\NormalTok{) }
\end{Highlighting}
\end{Shaded}

\begin{verbatim}
## [1] "2019-03-27 22:25:09 IST" "2019-03-28 22:25:09 IST"
## [3] "2019-03-29 23:25:09 IDT" "2019-03-30 23:25:09 IDT"
\end{verbatim}

\begin{Shaded}
\begin{Highlighting}[]
\KeywordTok{seq}\NormalTok{(time1, }\DataTypeTok{by =} \StringTok{"month"}\NormalTok{, }\DataTypeTok{length.out =} \DecValTok{12}\NormalTok{)}
\end{Highlighting}
\end{Shaded}

\begin{verbatim}
##  [1] "2019-03-27 22:25:09 IST" "2019-04-27 22:25:09 IDT"
##  [3] "2019-05-27 22:25:09 IDT" "2019-06-27 22:25:09 IDT"
##  [5] "2019-07-27 22:25:09 IDT" "2019-08-27 22:25:09 IDT"
##  [7] "2019-09-27 22:25:09 IDT" "2019-10-27 22:25:09 IST"
##  [9] "2019-11-27 22:25:09 IST" "2019-12-27 22:25:09 IST"
## [11] "2020-01-27 22:25:09 IST" "2020-02-27 22:25:09 IST"
\end{verbatim}

\subsection{lubridate Package}\label{lubridate-package}

The \textbf{lubridate} package replaces many of the \textbf{base}
package functionality, with a more consistent interface. You only need
to specify the order of arguments, not their format:

\begin{Shaded}
\begin{Highlighting}[]
\KeywordTok{library}\NormalTok{(lubridate)}
\KeywordTok{ymd}\NormalTok{(}\StringTok{"2017/01/31"}\NormalTok{)}
\end{Highlighting}
\end{Shaded}

\begin{verbatim}
## [1] "2017-01-31"
\end{verbatim}

\begin{Shaded}
\begin{Highlighting}[]
\KeywordTok{mdy}\NormalTok{(}\StringTok{"January 31st, 2017"}\NormalTok{)}
\end{Highlighting}
\end{Shaded}

\begin{verbatim}
## [1] "2017-01-31"
\end{verbatim}

\begin{Shaded}
\begin{Highlighting}[]
\KeywordTok{dmy}\NormalTok{(}\StringTok{"31-Jan-2017"}\NormalTok{)}
\end{Highlighting}
\end{Shaded}

\begin{verbatim}
## [1] "2017-01-31"
\end{verbatim}

\begin{Shaded}
\begin{Highlighting}[]
\KeywordTok{ymd_hms}\NormalTok{(}\StringTok{"2000-01-01 00:00:01"}\NormalTok{)}
\end{Highlighting}
\end{Shaded}

\begin{verbatim}
## [1] "2000-01-01 00:00:01 UTC"
\end{verbatim}

\begin{Shaded}
\begin{Highlighting}[]
\KeywordTok{ymd_hms}\NormalTok{(}\StringTok{"20000101000001"}\NormalTok{)}
\end{Highlighting}
\end{Shaded}

\begin{verbatim}
## [1] "2000-01-01 00:00:01 UTC"
\end{verbatim}

Another nice thing in \textbf{lubridate}, is that periods can be created
with a number of friendly constructor functions that you can combine
time objects. E.g.:

\begin{Shaded}
\begin{Highlighting}[]
\KeywordTok{seconds}\NormalTok{(}\DecValTok{1}\NormalTok{)}
\end{Highlighting}
\end{Shaded}

\begin{verbatim}
## [1] "1S"
\end{verbatim}

\begin{Shaded}
\begin{Highlighting}[]
\KeywordTok{minutes}\NormalTok{(}\KeywordTok{c}\NormalTok{(}\DecValTok{2}\NormalTok{,}\DecValTok{3}\NormalTok{))}
\end{Highlighting}
\end{Shaded}

\begin{verbatim}
## [1] "2M 0S" "3M 0S"
\end{verbatim}

\begin{Shaded}
\begin{Highlighting}[]
\KeywordTok{hours}\NormalTok{(}\DecValTok{4}\NormalTok{)}
\end{Highlighting}
\end{Shaded}

\begin{verbatim}
## [1] "4H 0M 0S"
\end{verbatim}

\begin{Shaded}
\begin{Highlighting}[]
\KeywordTok{days}\NormalTok{(}\DecValTok{5}\NormalTok{)}
\end{Highlighting}
\end{Shaded}

\begin{verbatim}
## [1] "5d 0H 0M 0S"
\end{verbatim}

\begin{Shaded}
\begin{Highlighting}[]
\KeywordTok{months}\NormalTok{(}\KeywordTok{c}\NormalTok{(}\DecValTok{6}\NormalTok{,}\DecValTok{7}\NormalTok{,}\DecValTok{8}\NormalTok{))}
\end{Highlighting}
\end{Shaded}

\begin{verbatim}
## [1] "6m 0d 0H 0M 0S" "7m 0d 0H 0M 0S" "8m 0d 0H 0M 0S"
\end{verbatim}

\begin{Shaded}
\begin{Highlighting}[]
\KeywordTok{weeks}\NormalTok{(}\DecValTok{9}\NormalTok{)}
\end{Highlighting}
\end{Shaded}

\begin{verbatim}
## [1] "63d 0H 0M 0S"
\end{verbatim}

\begin{Shaded}
\begin{Highlighting}[]
\KeywordTok{years}\NormalTok{(}\DecValTok{10}\NormalTok{)}
\end{Highlighting}
\end{Shaded}

\begin{verbatim}
## [1] "10y 0m 0d 0H 0M 0S"
\end{verbatim}

\begin{Shaded}
\begin{Highlighting}[]
\NormalTok{(t <-}\StringTok{ }\KeywordTok{ymd_hms}\NormalTok{(}\StringTok{"20000101000001"}\NormalTok{))}
\end{Highlighting}
\end{Shaded}

\begin{verbatim}
## [1] "2000-01-01 00:00:01 UTC"
\end{verbatim}

\begin{Shaded}
\begin{Highlighting}[]
\NormalTok{t }\OperatorTok{+}\StringTok{ }\KeywordTok{seconds}\NormalTok{(}\DecValTok{1}\NormalTok{)}
\end{Highlighting}
\end{Shaded}

\begin{verbatim}
## [1] "2000-01-01 00:00:02 UTC"
\end{verbatim}

\begin{Shaded}
\begin{Highlighting}[]
\NormalTok{t }\OperatorTok{+}\StringTok{ }\KeywordTok{minutes}\NormalTok{(}\KeywordTok{c}\NormalTok{(}\DecValTok{2}\NormalTok{,}\DecValTok{3}\NormalTok{)) }\OperatorTok{+}\StringTok{ }\KeywordTok{years}\NormalTok{(}\DecValTok{10}\NormalTok{)}
\end{Highlighting}
\end{Shaded}

\begin{verbatim}
## [1] "2010-01-01 00:02:01 UTC" "2010-01-01 00:03:01 UTC"
\end{verbatim}

And you can also extract and assign the time components:

\begin{Shaded}
\begin{Highlighting}[]
\NormalTok{t}
\end{Highlighting}
\end{Shaded}

\begin{verbatim}
## [1] "2000-01-01 00:00:01 UTC"
\end{verbatim}

\begin{Shaded}
\begin{Highlighting}[]
\KeywordTok{second}\NormalTok{(t)}
\end{Highlighting}
\end{Shaded}

\begin{verbatim}
## [1] 1
\end{verbatim}

\begin{Shaded}
\begin{Highlighting}[]
\KeywordTok{second}\NormalTok{(t) <-}\StringTok{ }\DecValTok{26}
\NormalTok{t}
\end{Highlighting}
\end{Shaded}

\begin{verbatim}
## [1] "2000-01-01 00:00:26 UTC"
\end{verbatim}

Analyzing temporal data is different than actually storing it. If you
are interested in time-series analysis, try the \textbf{tseries},
\textbf{forecast} and \textbf{zoo} packages.

\section{Complex Objects}\label{complex-objects}

Say you have a list with many elements, and you want to inspect this
list. You can do it using the \emph{Environment} pane in RStudio
(Ctrl+8), or using the \textbf{str} function:

\begin{Shaded}
\begin{Highlighting}[]
\NormalTok{complex.object <-}\StringTok{ }\KeywordTok{list}\NormalTok{(}\DecValTok{7}\NormalTok{, }\StringTok{'hello'}\NormalTok{, }\KeywordTok{list}\NormalTok{(}\DataTypeTok{a=}\DecValTok{7}\NormalTok{,}\DataTypeTok{b=}\DecValTok{8}\NormalTok{,}\DataTypeTok{c=}\DecValTok{9}\NormalTok{), }\DataTypeTok{FOO=}\NormalTok{read.csv)}
\KeywordTok{str}\NormalTok{(complex.object)}
\end{Highlighting}
\end{Shaded}

\begin{verbatim}
## List of 4
##  $    : num 7
##  $    : chr "hello"
##  $    :List of 3
##   ..$ a: num 7
##   ..$ b: num 8
##   ..$ c: num 9
##  $ FOO:function (file, header = TRUE, sep = ",", quote = "\"", dec = ".", 
##     fill = TRUE, comment.char = "", ...)
\end{verbatim}

Some (very) advanced users may want a deeper look into object. Try the
\href{https://github.com/r-lib/lobstr/blob/master/README.md}{lobstr}
package, or the \textbf{.Internal(inspect(\ldots{}))} function described
\href{https://www.brodieg.com/2019/02/18/an-unofficial-reference-for-internal-inspect/}{here}.

\begin{Shaded}
\begin{Highlighting}[]
\NormalTok{x <-}\StringTok{ }\KeywordTok{c}\NormalTok{(}\DecValTok{7}\NormalTok{,}\DecValTok{10}\NormalTok{)}
\KeywordTok{.Internal}\NormalTok{(}\KeywordTok{inspect}\NormalTok{(x))}
\end{Highlighting}
\end{Shaded}

\begin{verbatim}
## @1f076148 14 REALSXP g0c2 [NAM(3)] (len=2, tl=0) 7,10
\end{verbatim}

\section{Vectors and Matrix Products}\label{vectors-and-matrix-products}

This section is courtesy of Ron Sarafian.

If you are operating with numeric vectors, or matrices, you may want to
compute products. You can easily write your own R loops, but it is much
more efficient to use the built-in operations.

\BeginKnitrBlock{definition}[Matrix Product]
\protect\hypertarget{def:matrix-product}{}{\label{def:matrix-product}
\iffalse (Matrix Product) \fi{} }The matrix-product between matrix
\(n \times m\) matrix \(A\), and \(m \times p\) matrix \(B\), is a
\(n \times p\) matrix \(C\), where:
\[c_{i,j}:=\sum_{k=1}^m a_{i,k} b_{k,j}\]
\EndKnitrBlock{definition}

Vectors can be seen as single row/column matrices. We can thus use
matrix products to define the following:

\BeginKnitrBlock{definition}[Dot Product]
\protect\hypertarget{def:dot-product}{}{\label{def:dot-product}
\iffalse (Dot Product) \fi{} }The dot-product, a.k.a. scalar-product, or
inner-product, between row-vectors \(x:=(x_1,\dots,x_n)\) and
\(y:=(y_1,\dots,y_n)\) is defined as the matrix product between the
\(1 \times n\) matrix \(x'\), and the \(n \times 1\) matrix y:
\[x'y:= \sum_i x_i y_i\]
\EndKnitrBlock{definition}

\BeginKnitrBlock{definition}[Outer Product]
\protect\hypertarget{def:outer-product}{}{\label{def:outer-product}
\iffalse (Outer Product) \fi{} }The outer product between row-vectors
\(x:=(x_1,\dots,x_n)\) and \(y:=(y_1,\dots,y_n)\) is defined as the
matrix product between the \(n \times 1\) matrix \(x\), and the
\(1 \times n\) matrix \(y'\): \[(xy')_{i,j}:=x_i \, y_j\]
\EndKnitrBlock{definition}

Matrix products are computed with the \texttt{\%*\%} operator:

\begin{Shaded}
\begin{Highlighting}[]
\NormalTok{x <-}\StringTok{ }\KeywordTok{rnorm}\NormalTok{(}\DecValTok{4}\NormalTok{) }
\NormalTok{y <-}\StringTok{ }\KeywordTok{exp}\NormalTok{(}\OperatorTok{-}\NormalTok{x) }
\KeywordTok{t}\NormalTok{(x) }\OperatorTok\StringTok{ }\NormalTok{y }\CommentTok{# Dot product.}
\end{Highlighting}
\end{Shaded}

\begin{verbatim}
##           [,1]
## [1,] -3.298627
\end{verbatim}

\begin{Shaded}
\begin{Highlighting}[]
\NormalTok{x }\OperatorTok\StringTok{ }\NormalTok{y }\CommentTok{# Dot product.}
\end{Highlighting}
\end{Shaded}

\begin{verbatim}
##           [,1]
## [1,] -3.298627
\end{verbatim}

\begin{Shaded}
\begin{Highlighting}[]
\KeywordTok{crossprod}\NormalTok{(x,y) }\CommentTok{# Dot product.}
\end{Highlighting}
\end{Shaded}

\begin{verbatim}
##           [,1]
## [1,] -3.298627
\end{verbatim}

\begin{Shaded}
\begin{Highlighting}[]
\KeywordTok{crossprod}\NormalTok{(}\KeywordTok{t}\NormalTok{(x),y) }\CommentTok{# Outer product.}
\end{Highlighting}
\end{Shaded}

\begin{verbatim}
##            [,1]       [,2]       [,3]       [,4]
## [1,] -1.5412664 -0.5513476 -1.7862644 -0.5988587
## [2,]  0.6075926  0.2173503  0.7041748  0.2360800
## [3,] -1.8496379 -0.6616595 -2.1436542 -0.7186764
## [4,]  0.4348046  0.1555399  0.5039206  0.1689432
\end{verbatim}

\begin{Shaded}
\begin{Highlighting}[]
\KeywordTok{crossprod}\NormalTok{(}\KeywordTok{t}\NormalTok{(x),}\KeywordTok{t}\NormalTok{(y)) }\CommentTok{# Outer product.}
\end{Highlighting}
\end{Shaded}

\begin{verbatim}
##            [,1]       [,2]       [,3]       [,4]
## [1,] -1.5412664 -0.5513476 -1.7862644 -0.5988587
## [2,]  0.6075926  0.2173503  0.7041748  0.2360800
## [3,] -1.8496379 -0.6616595 -2.1436542 -0.7186764
## [4,]  0.4348046  0.1555399  0.5039206  0.1689432
\end{verbatim}

\begin{Shaded}
\begin{Highlighting}[]
\NormalTok{x }\OperatorTok\StringTok{ }\KeywordTok{t}\NormalTok{(y) }\CommentTok{# Outer product}
\end{Highlighting}
\end{Shaded}

\begin{verbatim}
##            [,1]       [,2]       [,3]       [,4]
## [1,] -1.5412664 -0.5513476 -1.7862644 -0.5988587
## [2,]  0.6075926  0.2173503  0.7041748  0.2360800
## [3,] -1.8496379 -0.6616595 -2.1436542 -0.7186764
## [4,]  0.4348046  0.1555399  0.5039206  0.1689432
\end{verbatim}

\begin{Shaded}
\begin{Highlighting}[]
\NormalTok{x }\OperatorTok\StringTok{ }\NormalTok{y }\CommentTok{# Outer product}
\end{Highlighting}
\end{Shaded}

\begin{verbatim}
##            [,1]       [,2]       [,3]       [,4]
## [1,] -1.5412664 -0.5513476 -1.7862644 -0.5988587
## [2,]  0.6075926  0.2173503  0.7041748  0.2360800
## [3,] -1.8496379 -0.6616595 -2.1436542 -0.7186764
## [4,]  0.4348046  0.1555399  0.5039206  0.1689432
\end{verbatim}

\begin{Shaded}
\begin{Highlighting}[]
\KeywordTok{outer}\NormalTok{(x,y) }\CommentTok{# Outer product}
\end{Highlighting}
\end{Shaded}

\begin{verbatim}
##            [,1]       [,2]       [,3]       [,4]
## [1,] -1.5412664 -0.5513476 -1.7862644 -0.5988587
## [2,]  0.6075926  0.2173503  0.7041748  0.2360800
## [3,] -1.8496379 -0.6616595 -2.1436542 -0.7186764
## [4,]  0.4348046  0.1555399  0.5039206  0.1689432
\end{verbatim}

Things to note:

\begin{itemize}
\tightlist
\item
  The definition of the matrix product has to do with the view of a
  matrix as a linear operator, and not only a table with numbers. Pick
  up any linear algebra book to understand why it is defined this way.
\item
  Vectors are matrices. The dot product, is a matrix product where
  \(m=1\).
\item
  \texttt{*} is an element-wise product, whereas \texttt{\%*\%} is a dot
  product.
\item
  While not specifying whether the vectors are horizontal or vertical, R
  treats the operation as \((1 \times n) * (n \times 1)\).
\item
  \texttt{t()} is the vector/ matrix transpose.
\end{itemize}

Now for matrix multiplication:

\begin{Shaded}
\begin{Highlighting}[]
\NormalTok{(x <-}\StringTok{ }\KeywordTok{rep}\NormalTok{(}\DecValTok{1}\NormalTok{,}\DecValTok{5}\NormalTok{))}
\end{Highlighting}
\end{Shaded}

\begin{verbatim}
## [1] 1 1 1 1 1
\end{verbatim}

\begin{Shaded}
\begin{Highlighting}[]
\NormalTok{(A <-}\StringTok{ }\KeywordTok{matrix}\NormalTok{(}\DataTypeTok{data =} \KeywordTok{rep}\NormalTok{(}\DecValTok{1}\OperatorTok{:}\DecValTok{5}\NormalTok{,}\DecValTok{5}\NormalTok{), }\DataTypeTok{nrow =} \DecValTok{5}\NormalTok{, }\DataTypeTok{ncol =} \DecValTok{5}\NormalTok{, }\DataTypeTok{byrow =} \OtherTok{TRUE}\NormalTok{)) }\CommentTok{# }
\end{Highlighting}
\end{Shaded}

\begin{verbatim}
##      [,1] [,2] [,3] [,4] [,5]
## [1,]    1    2    3    4    5
## [2,]    1    2    3    4    5
## [3,]    1    2    3    4    5
## [4,]    1    2    3    4    5
## [5,]    1    2    3    4    5
\end{verbatim}

\begin{Shaded}
\begin{Highlighting}[]
\NormalTok{x }\OperatorTok\StringTok{ }\NormalTok{A }\CommentTok{# (1X5) * (5X5) => (1X5)}
\end{Highlighting}
\end{Shaded}

\begin{verbatim}
##      [,1] [,2] [,3] [,4] [,5]
## [1,]    5   10   15   20   25
\end{verbatim}

\begin{Shaded}
\begin{Highlighting}[]
\NormalTok{A }\OperatorTok\StringTok{ }\NormalTok{x }\CommentTok{# (5X5) * (5X1) => (1X5)}
\end{Highlighting}
\end{Shaded}

\begin{verbatim}
##      [,1]
## [1,]   15
## [2,]   15
## [3,]   15
## [4,]   15
## [5,]   15
\end{verbatim}

\begin{Shaded}
\begin{Highlighting}[]
\FloatTok{0.5} \OperatorTok{*}\StringTok{ }\NormalTok{A }
\end{Highlighting}
\end{Shaded}

\begin{verbatim}
##      [,1] [,2] [,3] [,4] [,5]
## [1,]  0.5    1  1.5    2  2.5
## [2,]  0.5    1  1.5    2  2.5
## [3,]  0.5    1  1.5    2  2.5
## [4,]  0.5    1  1.5    2  2.5
## [5,]  0.5    1  1.5    2  2.5
\end{verbatim}

\begin{Shaded}
\begin{Highlighting}[]
\NormalTok{A }\OperatorTok\StringTok{ }\KeywordTok{t}\NormalTok{(A) }\CommentTok{# Gram matrix}
\end{Highlighting}
\end{Shaded}

\begin{verbatim}
##      [,1] [,2] [,3] [,4] [,5]
## [1,]   55   55   55   55   55
## [2,]   55   55   55   55   55
## [3,]   55   55   55   55   55
## [4,]   55   55   55   55   55
## [5,]   55   55   55   55   55
\end{verbatim}

\begin{Shaded}
\begin{Highlighting}[]
\KeywordTok{t}\NormalTok{(x) }\OperatorTok\StringTok{ }\NormalTok{A }\OperatorTok\StringTok{ }\NormalTok{x }\CommentTok{# Quadratic form}
\end{Highlighting}
\end{Shaded}

\begin{verbatim}
##      [,1]
## [1,]   75
\end{verbatim}

Can I write these functions myself? Yes! But a pure-R implementation
will be much slower than \texttt{\%*\%}:

\begin{Shaded}
\begin{Highlighting}[]
\NormalTok{my.crossprod <-}\StringTok{ }\ControlFlowTok{function}\NormalTok{(x,y)\{}
\NormalTok{  result <-}\StringTok{ }\DecValTok{0}
  \ControlFlowTok{for}\NormalTok{(i }\ControlFlowTok{in} \DecValTok{1}\OperatorTok{:}\KeywordTok{length}\NormalTok{(x)) result <-}\StringTok{ }\NormalTok{result }\OperatorTok{+}\StringTok{ }\NormalTok{x[i]}\OperatorTok{*}\NormalTok{y[i]}
\NormalTok{  result}
\NormalTok{\}}
\NormalTok{x <-}\StringTok{ }\KeywordTok{rnorm}\NormalTok{(}\FloatTok{1e8}\NormalTok{)}
\NormalTok{y <-}\StringTok{ }\KeywordTok{rnorm}\NormalTok{(}\FloatTok{1e8}\NormalTok{)}
\KeywordTok{system.time}\NormalTok{(a1 <-}\StringTok{ }\KeywordTok{my.crossprod}\NormalTok{(x,y))}
\end{Highlighting}
\end{Shaded}

\begin{verbatim}
##    user  system elapsed 
##   6.445   0.007   6.452
\end{verbatim}

\begin{Shaded}
\begin{Highlighting}[]
\KeywordTok{system.time}\NormalTok{(a2 <-}\StringTok{ }\KeywordTok{sum}\NormalTok{(x}\OperatorTok{*}\NormalTok{y))}
\end{Highlighting}
\end{Shaded}

\begin{verbatim}
##    user  system elapsed 
##    0.22    0.14    0.36
\end{verbatim}

\begin{Shaded}
\begin{Highlighting}[]
\KeywordTok{system.time}\NormalTok{(a3 <-}\StringTok{ }\KeywordTok{c}\NormalTok{(x}\OperatorTok\NormalTok{y))}
\end{Highlighting}
\end{Shaded}

\begin{verbatim}
##    user  system elapsed 
##   0.349   0.000   0.349
\end{verbatim}

\begin{Shaded}
\begin{Highlighting}[]
\KeywordTok{all.equal}\NormalTok{(a1,a2)}
\end{Highlighting}
\end{Shaded}

\begin{verbatim}
## [1] TRUE
\end{verbatim}

\begin{Shaded}
\begin{Highlighting}[]
\KeywordTok{all.equal}\NormalTok{(a1,a3)}
\end{Highlighting}
\end{Shaded}

\begin{verbatim}
## [1] TRUE
\end{verbatim}

\begin{Shaded}
\begin{Highlighting}[]
\KeywordTok{all.equal}\NormalTok{(a2,a3)}
\end{Highlighting}
\end{Shaded}

\begin{verbatim}
## [1] TRUE
\end{verbatim}

\section{Bibliographic Notes}\label{bibliographic-notes-1}

There are endlessly many introductory texts on R. For a list of free
resources see
\href{http://stats.stackexchange.com/questions/138/free-resources-for-learning-r}{CrossValidated}.
I personally recommend the official introduction
\citet{venables2004introduction},
\href{https://cran.r-project.org/doc/manuals/r-release/R-intro.pdf}{available
online}, or anything else Bill Venables writes.

For Importing and Exporting see
(\url{https://cran.r-project.org/doc/manuals/r-release/R-data.html}).
For working with databases see
(\url{https://rforanalytics.wordpress.com/useful-links-for-r/odbc-databases-for-r/}).
For a little intro on time-series objects in R see
\href{http://www.christophsax.com/2018/05/15/tsbox/}{Cristoph Sax's
blog}. For working with strings see
\href{http://www.gastonsanchez.com/r4strings/}{Gaston Sanchez's book}.
For advanced R programming see \citet{wickham2014advanced},
\href{http://adv-r.had.co.nz/}{available online}, or anything else
Hadley Wickham writes. For a curated list of recommended packages see
\href{https://github.com/rstudio/RStartHere/blob/master/README.md}{here}.

\section{Practice Yourself}\label{practice-yourself}

\begin{enumerate}
\def\labelenumi{\arabic{enumi}.}
\item
  Load the package \textbf{MASS}. That was easy. Now load
  \textbf{ggplot2}, after looking into \texttt{install.pacakges()}.
\item
  Save the numbers 1 to 1,000,000 (\texttt{1e6}) into an object named
  \texttt{object}.
\item
  Write a function that computes the mean of its input. Write a version
  that uses \texttt{sum()}, and another that uses a \texttt{for} loop
  and the summation \texttt{+}. Try checking which is faster using
  \texttt{system.time}. Is the difference considerable? Ask me about it
  in class.
\item
  Write a function that returns \texttt{TRUE} if a number is divisible
  by 13, \texttt{FALSE} if not, and a nice warning to the user if the
  input is not an integer number.
\item
  Apply the previous function to all the numbers in \texttt{object}. Try
  using a \texttt{for} loop, but also a mapping/apply function.
\item
  Make a matrix of random numbers using
  \texttt{A\ \textless{}-\ matrix(rnorm(40),\ ncol=10,\ nrow=4)}.
  Compute the mean of each column. Do it using your own loop and then do
  the same with \texttt{lapply} or \texttt{apply}.
\item
  Make a data frame (\texttt{dataA}) with three columns, and 100 rows.
  The first column with 100 numbers generated from the
  \(\mathcal{N}(10,1)\) distribution, second column with samples from
  \(Poiss(\lambda=4)\). The third column contains only \texttt{1}.\\
  Make another data frame (\texttt{dataB}) with three columns and 100
  rows. Now with \(\mathcal{N}(10,0.5^2)\), \(Poiss(\lambda=4)\) and
  \texttt{2}. Combine the two data frames into an object named
  \texttt{dataAB} with \texttt{rbind}. Make a scatter plot of
  \texttt{dataAB} where the x-axes is the first column, the y-axes is
  the second and define the shape of the points to be the third column.
\item
  In a sample generated of 1,000 observations from the
  \(\mathcal{N}(10,1)\) distribution:

  \begin{enumerate}
  \def\labelenumii{\arabic{enumii}.}
  \tightlist
  \item
    What is the proportion of samples smaller than \(12.4\) ?
  \item
    What is the \(0.23\) percentile of the sample?
  \end{enumerate}
\item
  Nothing like cleaning a dataset, to practice your R basics. Have a
  look at
  \href{https://makingnoiseandhearingthings.com/2018/04/19/datasets-for-data-cleaning-practice/}{RACHAEL
  TATMAN} collected several datasets which BADLY need some cleansing.
\end{enumerate}

You can also self practice with DataCamp's
\href{https://www.datacamp.com/courses/free-introduction-to-r}{Intoroduction
to R} course, or go directly to exercising with
\href{https://www.r-exercises.com/start-here-to-learn-r/}{R-exercises}.

\chapter{data.table}\label{datatable}

\texttt{data.table} is an excellent extension of the \texttt{data.frame}
class. If used as a \texttt{data.frame} it will look and feel like a
data frame. If, however, it is used with it's unique capabilities, it
will prove faster and easier to manipulate. This is because
\texttt{data.frame}s, like most of R objects, make a copy of themselves
when modified. This is known as
\href{https://stackoverflow.com/questions/373419/whats-the-difference-between-passing-by-reference-vs-passing-by-value}{passing
by value}, and it is done to ensure that object are not corrupted if an
operation fails (if your computer shuts down before the operation is
completed, for instance). Making copies of large objects is clearly time
and memory consuming. A \texttt{data.table} can make changes in place.
This is known as
\href{https://stackoverflow.com/questions/373419/whats-the-difference-between-passing-by-reference-vs-passing-by-value}{passing
by reference}, which is considerably faster than passing by value.

Let's start with importing some freely available car sales data from
\href{https://www.kaggle.com/orgesleka/used-cars-database}{Kaggle}.

\begin{Shaded}
\begin{Highlighting}[]
\KeywordTok{library}\NormalTok{(data.table)}
\KeywordTok{library}\NormalTok{(magrittr)}
\NormalTok{auto <-}\StringTok{ }\KeywordTok{fread}\NormalTok{(}\StringTok{'data/autos.csv'}\NormalTok{)}
\end{Highlighting}
\end{Shaded}

\begin{verbatim}
## Warning in fread("data/autos.csv"): Found and resolved improper
## quoting out-of-sample. First healed line 5263: <<2016-03-29
## 16:46:46,"_SPARDOSE"______Polo_1_4___6N1___60PS___5Tuerer____FESTPREIS,privat,Angebot,
## 500,control,limousine,1999,manuell,60,polo,150000,12,benzin,volkswagen,ja,
## 2016-03-25 00:00:00,0,59581,2016-03-30 11:46:58>>. If the fields are
## not quoted (e.g. field separator does not appear within any field), try
## quote="" to avoid this warning.
\end{verbatim}

\begin{Shaded}
\begin{Highlighting}[]
\KeywordTok{View}\NormalTok{(auto)}
\end{Highlighting}
\end{Shaded}

\begin{Shaded}
\begin{Highlighting}[]
\KeywordTok{dim}\NormalTok{(auto) }\CommentTok{#  Rows and columns}
\end{Highlighting}
\end{Shaded}

\begin{verbatim}
## [1] 371824     20
\end{verbatim}

\begin{Shaded}
\begin{Highlighting}[]
\KeywordTok{names}\NormalTok{(auto) }\CommentTok{# Variable names}
\end{Highlighting}
\end{Shaded}

\begin{verbatim}
##  [1] "dateCrawled"         "name"                "seller"             
##  [4] "offerType"           "price"               "abtest"             
##  [7] "vehicleType"         "yearOfRegistration"  "gearbox"            
## [10] "powerPS"             "model"               "kilometer"          
## [13] "monthOfRegistration" "fuelType"            "brand"              
## [16] "notRepairedDamage"   "dateCreated"         "nrOfPictures"       
## [19] "postalCode"          "lastSeen"
\end{verbatim}

\begin{Shaded}
\begin{Highlighting}[]
\KeywordTok{class}\NormalTok{(auto) }\CommentTok{# Object class}
\end{Highlighting}
\end{Shaded}

\begin{verbatim}
## [1] "data.table" "data.frame"
\end{verbatim}

\begin{Shaded}
\begin{Highlighting}[]
\KeywordTok{file.info}\NormalTok{(}\StringTok{'data/autos.csv'}\NormalTok{) }\CommentTok{# File info on disk}
\end{Highlighting}
\end{Shaded}

\begin{verbatim}
##                    size isdir mode               mtime               ctime
## data/autos.csv 68439217 FALSE  644 2019-02-24 23:52:04 2019-02-24 23:52:04
##                              atime  uid  gid   uname  grname
## data/autos.csv 2019-03-27 22:15:55 1000 1000 johnros johnros
\end{verbatim}

\begin{Shaded}
\begin{Highlighting}[]
\NormalTok{gdata}\OperatorTok{::}\KeywordTok{humanReadable}\NormalTok{(}\DecValTok{68439217}\NormalTok{)}
\end{Highlighting}
\end{Shaded}

\begin{verbatim}
## [1] "65.3 MiB"
\end{verbatim}

\begin{Shaded}
\begin{Highlighting}[]
\KeywordTok{object.size}\NormalTok{(auto) }\OperatorTok\StringTok{ }\KeywordTok{print}\NormalTok{(}\DataTypeTok{units =} \StringTok{'auto'}\NormalTok{) }\CommentTok{# File size in memory}
\end{Highlighting}
\end{Shaded}

\begin{verbatim}
## 103.3 Mb
\end{verbatim}

Things to note:

\begin{itemize}
\tightlist
\item
  The import has been done with \texttt{fread} instead of
  \texttt{read.csv}. This is more efficient, and directly creates a
  \texttt{data.table} object.
\item
  The import is very fast.
\item
  The data after import is slightly larger than when stored on disk (in
  this case). The extra data allows faster operation of this object, and
  the rule of thumb is to have 3 to 5 times more
  \href{https://en.wikipedia.org/wiki/Random-access_memory}{RAM} than
  file size (e.g.: 4GB RAM for 1GB file)
\item
  \texttt{auto} has two classes. It means that everything that expects a
  \texttt{data.frame} we can feed it a \texttt{data.table} and it will
  work.
\end{itemize}

Let's start with verifying that it behaves like a \texttt{data.frame}
when expected.

\begin{Shaded}
\begin{Highlighting}[]
\NormalTok{auto[,}\DecValTok{2}\NormalTok{] }\OperatorTok\StringTok{ }\NormalTok{head}
\end{Highlighting}
\end{Shaded}

\begin{verbatim}
##                                                  name
## 1:                                         Golf_3_1.6
## 2:                               A5_Sportback_2.7_Tdi
## 3:                     Jeep_Grand_Cherokee_"Overland"
## 4:                              GOLF_4_1_4__3T\xdcRER
## 5:                     Skoda_Fabia_1.4_TDI_PD_Classic
## 6: BMW_316i___e36_Limousine___Bastlerfahrzeug__Export
\end{verbatim}

\begin{Shaded}
\begin{Highlighting}[]
\NormalTok{auto[[}\DecValTok{2}\NormalTok{]] }\OperatorTok\StringTok{ }\NormalTok{head}
\end{Highlighting}
\end{Shaded}

\begin{verbatim}
## [1] "Golf_3_1.6"                                        
## [2] "A5_Sportback_2.7_Tdi"                              
## [3] "Jeep_Grand_Cherokee_\"Overland\""                  
## [4] "GOLF_4_1_4__3T\xdcRER"                             
## [5] "Skoda_Fabia_1.4_TDI_PD_Classic"                    
## [6] "BMW_316i___e36_Limousine___Bastlerfahrzeug__Export"
\end{verbatim}

\begin{Shaded}
\begin{Highlighting}[]
\NormalTok{auto[}\DecValTok{1}\NormalTok{,}\DecValTok{2}\NormalTok{] }\OperatorTok\StringTok{ }\NormalTok{head}
\end{Highlighting}
\end{Shaded}

\begin{verbatim}
##          name
## 1: Golf_3_1.6
\end{verbatim}

But notice the difference between \texttt{data.frame} and
\texttt{data.table} when subsetting multiple rows. Uhh!

\begin{Shaded}
\begin{Highlighting}[]
\NormalTok{auto[}\DecValTok{1}\OperatorTok{:}\DecValTok{3}\NormalTok{] }\OperatorTok\StringTok{ }\NormalTok{dim }\CommentTok{# data.table will exctract *rows*}
\end{Highlighting}
\end{Shaded}

\begin{verbatim}
## [1]  3 20
\end{verbatim}

\begin{Shaded}
\begin{Highlighting}[]
\KeywordTok{as.data.frame}\NormalTok{(auto)[}\DecValTok{1}\OperatorTok{:}\DecValTok{3}\NormalTok{] }\OperatorTok\StringTok{ }\NormalTok{dim }\CommentTok{# data.frame will exctract *columns*}
\end{Highlighting}
\end{Shaded}

\begin{verbatim}
## [1] 371824      3
\end{verbatim}

Just use columns (\texttt{,}) and be explicit regarding the dimension
you are extracting\ldots{}

Now let's do some \texttt{data.table} specific operations. The general
syntax has the form \texttt{DT{[}i,j,by{]}}. SQL users may think of
\texttt{i} as \texttt{WHERE}, \texttt{j} as \texttt{SELECT}, and
\texttt{by} as \texttt{GROUP\ BY}. We don't need to name the arguments
explicitly. Also, the \texttt{Tab} key will typically help you to fill
in column names.

\begin{Shaded}
\begin{Highlighting}[]
\NormalTok{auto[,vehicleType,] }\OperatorTok\StringTok{ }\NormalTok{table }\CommentTok{# Exctract column and tabulate}
\end{Highlighting}
\end{Shaded}

\begin{verbatim}
## .
##                andere        bus     cabrio      coupe kleinwagen 
##      37899       3362      30220      22914      19026      80098 
##      kombi  limousine        suv 
##      67626      95963      14716
\end{verbatim}

\begin{Shaded}
\begin{Highlighting}[]
\NormalTok{auto[vehicleType}\OperatorTok{==}\StringTok{'coupe'}\NormalTok{,,] }\OperatorTok\StringTok{ }\NormalTok{dim }\CommentTok{# Exctract rows }
\end{Highlighting}
\end{Shaded}

\begin{verbatim}
## [1] 19026    20
\end{verbatim}

\begin{Shaded}
\begin{Highlighting}[]
\NormalTok{auto[,gearbox}\OperatorTok{:}\NormalTok{model,] }\OperatorTok\StringTok{ }\NormalTok{head }\CommentTok{# exctract column range}
\end{Highlighting}
\end{Shaded}

\begin{verbatim}
##      gearbox powerPS model
## 1:   manuell       0  golf
## 2:   manuell     190      
## 3: automatik     163 grand
## 4:   manuell      75  golf
## 5:   manuell      69 fabia
## 6:   manuell     102   3er
\end{verbatim}

\begin{Shaded}
\begin{Highlighting}[]
\NormalTok{auto[,gearbox,] }\OperatorTok\StringTok{ }\NormalTok{table}
\end{Highlighting}
\end{Shaded}

\begin{verbatim}
## .
##           automatik   manuell 
##     20223     77169    274432
\end{verbatim}

\begin{Shaded}
\begin{Highlighting}[]
\NormalTok{auto[vehicleType}\OperatorTok{==}\StringTok{'coupe'} \OperatorTok{&}\StringTok{ }\NormalTok{gearbox}\OperatorTok{==}\StringTok{'automatik'}\NormalTok{,,] }\OperatorTok\StringTok{ }\NormalTok{dim }\CommentTok{# intersect conditions}
\end{Highlighting}
\end{Shaded}

\begin{verbatim}
## [1] 6008   20
\end{verbatim}

\begin{Shaded}
\begin{Highlighting}[]
\NormalTok{auto[,}\KeywordTok{table}\NormalTok{(vehicleType),] }\CommentTok{# uhh? why would this even work?!?}
\end{Highlighting}
\end{Shaded}

\begin{verbatim}
## vehicleType
##                andere        bus     cabrio      coupe kleinwagen 
##      37899       3362      30220      22914      19026      80098 
##      kombi  limousine        suv 
##      67626      95963      14716
\end{verbatim}

\begin{Shaded}
\begin{Highlighting}[]
\NormalTok{auto[, }\KeywordTok{mean}\NormalTok{(price), by=vehicleType] }\CommentTok{# average price by car group}
\end{Highlighting}
\end{Shaded}

\begin{verbatim}
## Warning in gmean(price): The sum of an integer column for a group was more
## than type 'integer' can hold so the result has been coerced to 'numeric'
## automatically for convenience.
\end{verbatim}

\begin{verbatim}
##    vehicleType         V1
## 1:              20124.688
## 2:       coupe  25951.506
## 3:         suv  13252.392
## 4:  kleinwagen   5691.167
## 5:   limousine  11111.107
## 6:      cabrio  15072.998
## 7:         bus  10300.686
## 8:       kombi   7739.518
## 9:      andere 676327.100
\end{verbatim}

The \texttt{.N} operator is very useful if you need to count the length
of the result. Notice where I use it:

\begin{Shaded}
\begin{Highlighting}[]
\NormalTok{auto[.N,,] }\CommentTok{# will exctract the *last* row}
\end{Highlighting}
\end{Shaded}

\begin{verbatim}
##            dateCrawled                                         name seller
## 1: 2016-03-07 19:39:19 BMW_M135i_vollausgestattet_NP_52.720____Euro privat
##    offerType price  abtest vehicleType yearOfRegistration gearbox powerPS
## 1:   Angebot 28990 control   limousine               2013 manuell     320
##      model kilometer monthOfRegistration fuelType brand notRepairedDamage
## 1: m_reihe     50000                   8   benzin   bmw              nein
##            dateCreated nrOfPictures postalCode            lastSeen
## 1: 2016-03-07 00:00:00            0      73326 2016-03-22 03:17:10
\end{verbatim}

\begin{Shaded}
\begin{Highlighting}[]
\NormalTok{auto[,.N] }\CommentTok{# will count rows}
\end{Highlighting}
\end{Shaded}

\begin{verbatim}
## [1] 371824
\end{verbatim}

\begin{Shaded}
\begin{Highlighting}[]
\NormalTok{auto[,.N, vehicleType] }\CommentTok{# will count rows by type}
\end{Highlighting}
\end{Shaded}

\begin{verbatim}
##    vehicleType     N
## 1:             37899
## 2:       coupe 19026
## 3:         suv 14716
## 4:  kleinwagen 80098
## 5:   limousine 95963
## 6:      cabrio 22914
## 7:         bus 30220
## 8:       kombi 67626
## 9:      andere  3362
\end{verbatim}

You may concatenate results into a vector:

\begin{Shaded}
\begin{Highlighting}[]
\NormalTok{auto[,}\KeywordTok{c}\NormalTok{(}\KeywordTok{mean}\NormalTok{(price), }\KeywordTok{mean}\NormalTok{(powerPS)),]}
\end{Highlighting}
\end{Shaded}

\begin{verbatim}
## [1] 17286.2996   115.5414
\end{verbatim}

This \texttt{c()} syntax no longer behaves well if splitting:

\begin{Shaded}
\begin{Highlighting}[]
\NormalTok{auto[,}\KeywordTok{c}\NormalTok{(}\KeywordTok{mean}\NormalTok{(price), }\KeywordTok{mean}\NormalTok{(powerPS)), by=vehicleType]}
\end{Highlighting}
\end{Shaded}

\begin{verbatim}
##     vehicleType           V1
##  1:              20124.68801
##  2:                 71.23249
##  3:       coupe  25951.50589
##  4:       coupe    172.97614
##  5:         suv  13252.39182
##  6:         suv    166.01903
##  7:  kleinwagen   5691.16738
##  8:  kleinwagen     68.75733
##  9:   limousine  11111.10661
## 10:   limousine    132.26936
## 11:      cabrio  15072.99782
## 12:      cabrio    145.17684
## 13:         bus  10300.68561
## 14:         bus    113.58137
## 15:       kombi   7739.51760
## 16:       kombi    136.40654
## 17:      andere 676327.09964
## 18:      andere    102.11154
\end{verbatim}

Use a \texttt{list()} instead of \texttt{c()}, within
\texttt{data.table} commands:

\begin{Shaded}
\begin{Highlighting}[]
\NormalTok{auto[,}\KeywordTok{list}\NormalTok{(}\KeywordTok{mean}\NormalTok{(price), }\KeywordTok{mean}\NormalTok{(powerPS)), by=vehicleType]}
\end{Highlighting}
\end{Shaded}

\begin{verbatim}
## Warning in gmean(price): The sum of an integer column for a group was more
## than type 'integer' can hold so the result has been coerced to 'numeric'
## automatically for convenience.
\end{verbatim}

\begin{verbatim}
##    vehicleType         V1        V2
## 1:              20124.688  71.23249
## 2:       coupe  25951.506 172.97614
## 3:         suv  13252.392 166.01903
## 4:  kleinwagen   5691.167  68.75733
## 5:   limousine  11111.107 132.26936
## 6:      cabrio  15072.998 145.17684
## 7:         bus  10300.686 113.58137
## 8:       kombi   7739.518 136.40654
## 9:      andere 676327.100 102.11154
\end{verbatim}

You can add names to your new variables:

\begin{Shaded}
\begin{Highlighting}[]
\NormalTok{auto[,}\KeywordTok{list}\NormalTok{(}\DataTypeTok{Price=}\KeywordTok{mean}\NormalTok{(price), }\DataTypeTok{Power=}\KeywordTok{mean}\NormalTok{(powerPS)), by=vehicleType]}
\end{Highlighting}
\end{Shaded}

\begin{verbatim}
## Warning in gmean(price): The sum of an integer column for a group was more
## than type 'integer' can hold so the result has been coerced to 'numeric'
## automatically for convenience.
\end{verbatim}

\begin{verbatim}
##    vehicleType      Price     Power
## 1:              20124.688  71.23249
## 2:       coupe  25951.506 172.97614
## 3:         suv  13252.392 166.01903
## 4:  kleinwagen   5691.167  68.75733
## 5:   limousine  11111.107 132.26936
## 6:      cabrio  15072.998 145.17684
## 7:         bus  10300.686 113.58137
## 8:       kombi   7739.518 136.40654
## 9:      andere 676327.100 102.11154
\end{verbatim}

You can use \texttt{.()} to replace the longer \texttt{list()} command:

\begin{Shaded}
\begin{Highlighting}[]
\NormalTok{auto[,.(}\DataTypeTok{Price=}\KeywordTok{mean}\NormalTok{(price), }\DataTypeTok{Power=}\KeywordTok{mean}\NormalTok{(powerPS)), by=vehicleType]}
\end{Highlighting}
\end{Shaded}

\begin{verbatim}
## Warning in gmean(price): The sum of an integer column for a group was more
## than type 'integer' can hold so the result has been coerced to 'numeric'
## automatically for convenience.
\end{verbatim}

\begin{verbatim}
##    vehicleType      Price     Power
## 1:              20124.688  71.23249
## 2:       coupe  25951.506 172.97614
## 3:         suv  13252.392 166.01903
## 4:  kleinwagen   5691.167  68.75733
## 5:   limousine  11111.107 132.26936
## 6:      cabrio  15072.998 145.17684
## 7:         bus  10300.686 113.58137
## 8:       kombi   7739.518 136.40654
## 9:      andere 676327.100 102.11154
\end{verbatim}

And split by multiple variables:

\begin{Shaded}
\begin{Highlighting}[]
\NormalTok{auto[,.(}\DataTypeTok{Price=}\KeywordTok{mean}\NormalTok{(price), }\DataTypeTok{Power=}\KeywordTok{mean}\NormalTok{(powerPS)), by=.(vehicleType,fuelType)] }\OperatorTok\StringTok{ }\NormalTok{head}
\end{Highlighting}
\end{Shaded}

\begin{verbatim}
## Warning in gmean(price): The sum of an integer column for a group was more
## than type 'integer' can hold so the result has been coerced to 'numeric'
## automatically for convenience.
\end{verbatim}

\begin{verbatim}
##    vehicleType fuelType     Price     Power
## 1:               benzin 11820.443  70.14477
## 2:       coupe   diesel 51170.248 179.48704
## 3:         suv   diesel 15549.369 168.16115
## 4:  kleinwagen   benzin  5786.514  68.74309
## 5:  kleinwagen   diesel  4295.550  76.83666
## 6:   limousine   benzin  6974.360 127.87025
\end{verbatim}

Compute with variables created on the fly:

\begin{Shaded}
\begin{Highlighting}[]
\NormalTok{auto[,}\KeywordTok{sum}\NormalTok{(price}\OperatorTok{<}\FloatTok{1e4}\NormalTok{),] }\CommentTok{# Count prices lower than 10,000}
\end{Highlighting}
\end{Shaded}

\begin{verbatim}
## [1] 310497
\end{verbatim}

\begin{Shaded}
\begin{Highlighting}[]
\NormalTok{auto[,}\KeywordTok{mean}\NormalTok{(price}\OperatorTok{<}\FloatTok{1e4}\NormalTok{),] }\CommentTok{# Proportion of prices lower than 10,000}
\end{Highlighting}
\end{Shaded}

\begin{verbatim}
## [1] 0.8350644
\end{verbatim}

\begin{Shaded}
\begin{Highlighting}[]
\NormalTok{auto[,.(}\DataTypeTok{Power=}\KeywordTok{mean}\NormalTok{(powerPS)), by=.(}\DataTypeTok{PriceRange=}\NormalTok{price}\OperatorTok{>}\FloatTok{1e4}\NormalTok{)] }
\end{Highlighting}
\end{Shaded}

\begin{verbatim}
##    PriceRange    Power
## 1:      FALSE 101.8838
## 2:       TRUE 185.9029
\end{verbatim}

Things to note:

\begin{itemize}
\tightlist
\item
  The term \texttt{price\textless{}1e4} creates \emph{on the fly} a
  binary vector of TRUE=1 / FALSE=0 for prices less than 10k and then
  sums/means this vector, hence \texttt{sum} is actually a count, and
  \texttt{mean} is proportion=count/total
\item
  Summing all prices lower than 10k is done with the command
  \texttt{auto{[}price\textless{}1e4,sum(price),{]}}
\end{itemize}

You may sort along one or more columns

\begin{Shaded}
\begin{Highlighting}[]
\NormalTok{auto[}\KeywordTok{order}\NormalTok{(}\OperatorTok{-}\NormalTok{price), price,] }\OperatorTok\StringTok{ }\NormalTok{head }\CommentTok{# Order along price. Descending}
\end{Highlighting}
\end{Shaded}

\begin{verbatim}
## [1] 2147483647   99999999   99999999   99999999   99999999   99999999
\end{verbatim}

\begin{Shaded}
\begin{Highlighting}[]
\NormalTok{auto[}\KeywordTok{order}\NormalTok{(price, }\OperatorTok{-}\NormalTok{lastSeen), price,] }\OperatorTok\StringTok{ }\NormalTok{head}\CommentTok{# Order along price and last seen . Ascending and descending.}
\end{Highlighting}
\end{Shaded}

\begin{verbatim}
## [1] 0 0 0 0 0 0
\end{verbatim}

You may apply a function to ALL columns using a Subset of the Data using
\texttt{.SD}

\begin{Shaded}
\begin{Highlighting}[]
\NormalTok{count.uniques <-}\StringTok{ }\ControlFlowTok{function}\NormalTok{(x) }\KeywordTok{length}\NormalTok{(}\KeywordTok{unique}\NormalTok{(x))}
\NormalTok{auto[,}\KeywordTok{lapply}\NormalTok{(.SD, count.uniques), vehicleType]}
\end{Highlighting}
\end{Shaded}

\begin{verbatim}
##    vehicleType dateCrawled  name seller offerType price abtest
## 1:                   36714 32891      1         2  1378      2
## 2:       coupe       18745 13182      1         2  1994      2
## 3:         suv       14549  9707      1         1  1667      2
## 4:  kleinwagen       75591 49302      2         2  1927      2
## 5:   limousine       89352 58581      2         1  2986      2
## 6:      cabrio       22497 13411      1         1  2014      2
## 7:         bus       29559 19651      1         2  1784      2
## 8:       kombi       64415 41976      2         1  2529      2
## 9:      andere        3352  3185      1         1   562      2
##    yearOfRegistration gearbox powerPS model kilometer monthOfRegistration
## 1:                101       3     374   244        13                  13
## 2:                 75       3     414   117        13                  13
## 3:                 73       3     342   122        13                  13
## 4:                 75       3     317   163        13                  13
## 5:                 83       3     506   210        13                  13
## 6:                 88       3     363    95        13                  13
## 7:                 65       3     251   106        13                  13
## 8:                 64       3     393   177        13                  13
## 9:                 81       3     230   162        13                  13
##    fuelType brand notRepairedDamage dateCreated nrOfPictures postalCode
## 1:        8    40                 3          65            1       6304
## 2:        8    35                 3          51            1       5159
## 3:        8    37                 3          61            1       4932
## 4:        8    38                 3          68            1       7343
## 5:        8    39                 3          82            1       7513
## 6:        7    38                 3          70            1       5524
## 7:        8    33                 3          63            1       6112
## 8:        8    38                 3          75            1       7337
## 9:        8    38                 3          41            1       2220
##    lastSeen
## 1:    32813
## 2:    16568
## 3:    13367
## 4:    59354
## 5:    65813
## 6:    19125
## 7:    26094
## 8:    50668
## 9:     3294
\end{verbatim}

Things to note:

\begin{itemize}
\tightlist
\item
  \texttt{.SD} is the data subset after splitting along the \texttt{by}
  argument.
\item
  Recall that \texttt{lapply} applies the same function to all elements
  of a list. In this example, to all columns of \texttt{.SD}.
\end{itemize}

If you want to apply a function only to a subset of columns, use the
\texttt{.SDcols} argument

\begin{Shaded}
\begin{Highlighting}[]
\NormalTok{auto[,}\KeywordTok{lapply}\NormalTok{(.SD, count.uniques), by=vehicleType, .SDcols=price}\OperatorTok{:}\NormalTok{gearbox]}
\end{Highlighting}
\end{Shaded}

\begin{verbatim}
##    vehicleType price abtest vehicleType yearOfRegistration gearbox
## 1:              1378      2           1                101       3
## 2:       coupe  1994      2           1                 75       3
## 3:         suv  1667      2           1                 73       3
## 4:  kleinwagen  1927      2           1                 75       3
## 5:   limousine  2986      2           1                 83       3
## 6:      cabrio  2014      2           1                 88       3
## 7:         bus  1784      2           1                 65       3
## 8:       kombi  2529      2           1                 64       3
## 9:      andere   562      2           1                 81       3
\end{verbatim}

\section{Make your own variables}\label{make-your-own-variables}

It is very easy to compute new variables

\begin{Shaded}
\begin{Highlighting}[]
\NormalTok{auto[,}\KeywordTok{log}\NormalTok{(price}\OperatorTok{/}\NormalTok{powerPS),] }\OperatorTok\StringTok{ }\NormalTok{head }\CommentTok{# This makes no sense}
\end{Highlighting}
\end{Shaded}

\begin{verbatim}
## [1]      Inf 4.567632 4.096387 2.995732 3.954583 1.852000
\end{verbatim}

And if you want to store the result in a new variable, use the
\texttt{:=} operator

\begin{Shaded}
\begin{Highlighting}[]
\NormalTok{auto[,newVar}\OperatorTok{:}\ErrorTok{=}\KeywordTok{log}\NormalTok{(price}\OperatorTok{/}\NormalTok{powerPS),]}
\end{Highlighting}
\end{Shaded}

Or create multiple variables at once. The syntax
\texttt{c("A","B"):=.(expression1,expression2)}is read ``save the
\textbf{list} of results from expression1 and expression2 using the
\textbf{vector} of names A, and B''.

\begin{Shaded}
\begin{Highlighting}[]
\NormalTok{auto[,}\KeywordTok{c}\NormalTok{(}\StringTok{'newVar'}\NormalTok{,}\StringTok{'newVar2'}\NormalTok{)}\OperatorTok{:}\ErrorTok{=}\NormalTok{.(}\KeywordTok{log}\NormalTok{(price}\OperatorTok{/}\NormalTok{powerPS),price}\OperatorTok{^}\DecValTok{2}\OperatorTok{/}\NormalTok{powerPS),]}
\end{Highlighting}
\end{Shaded}

\section{Join}\label{join}

\textbf{data.table} can be used for joining. A \emph{join} is the
operation of aligning two (or more) data frames/tables along some index.
The index can be a single variable, or a combination thereof.

Here is a simple example of aligning age and gender from two different
data tables:

\begin{Shaded}
\begin{Highlighting}[]
\NormalTok{DT1 <-}\StringTok{ }\KeywordTok{data.table}\NormalTok{(}\DataTypeTok{Names=}\KeywordTok{c}\NormalTok{(}\StringTok{"Alice"}\NormalTok{,}\StringTok{"Bob"}\NormalTok{), }\DataTypeTok{Age=}\KeywordTok{c}\NormalTok{(}\DecValTok{29}\NormalTok{,}\DecValTok{31}\NormalTok{))}
\NormalTok{DT2 <-}\StringTok{ }\KeywordTok{data.table}\NormalTok{(}\DataTypeTok{Names=}\KeywordTok{c}\NormalTok{(}\StringTok{"Alice"}\NormalTok{,}\StringTok{"Bob"}\NormalTok{,}\StringTok{"Carl"}\NormalTok{), }\DataTypeTok{Gender=}\KeywordTok{c}\NormalTok{(}\StringTok{"F"}\NormalTok{,}\StringTok{"M"}\NormalTok{,}\StringTok{"M"}\NormalTok{))}
\KeywordTok{setkey}\NormalTok{(DT1, Names)}
\KeywordTok{setkey}\NormalTok{(DT2, Names)}
\NormalTok{DT1[DT2,,] }
\end{Highlighting}
\end{Shaded}

\begin{verbatim}
##    Names Age Gender
## 1: Alice  29      F
## 2:   Bob  31      M
## 3:  Carl  NA      M
\end{verbatim}

\begin{Shaded}
\begin{Highlighting}[]
\NormalTok{DT2[DT1,,] }
\end{Highlighting}
\end{Shaded}

\begin{verbatim}
##    Names Gender Age
## 1: Alice      F  29
## 2:   Bob      M  31
\end{verbatim}

Things to note:

\begin{itemize}
\tightlist
\item
  A join with \texttt{data.tables} is performed by indexing one
  \texttt{data.table} with another. Which is the outer and which is the
  inner will affect the result.
\item
  The indexing variable needs to be set using the \texttt{setkey}
  function.
\end{itemize}

There are several types of joins:

\begin{itemize}
\tightlist
\item
  \textbf{Inner join}: Returns the rows along the intersection of keys,
  i.e., rows that appear in \textbf{all} data sets.
\item
  \textbf{Outer join}: Returns the rows along the union of keys, i.e.,
  rows that appear in \textbf{any} of the data sets.
\item
  \textbf{Left join}: Returns the rows along the index of the ``left''
  data set.
\item
  \textbf{Right join}: Returns the rows along the index of the ``right''
  data set.
\end{itemize}

Assuming \texttt{DT1} is the ``left'' data set, we see that
\texttt{DT1{[}DT2,,{]}} is a right join, and \texttt{DT2{[}DT1,,{]}} is
a left join. For an inner join use the \texttt{nomath=0} argument:

\begin{Shaded}
\begin{Highlighting}[]
\NormalTok{DT1[DT2,,,nomatch=}\DecValTok{0}\NormalTok{]}
\end{Highlighting}
\end{Shaded}

\begin{verbatim}
##    Names Age Gender
## 1: Alice  29      F
## 2:   Bob  31      M
\end{verbatim}

\begin{Shaded}
\begin{Highlighting}[]
\NormalTok{DT2[DT1,,,nomatch=}\DecValTok{0}\NormalTok{]}
\end{Highlighting}
\end{Shaded}

\begin{verbatim}
##    Names Gender Age
## 1: Alice      F  29
## 2:   Bob      M  31
\end{verbatim}

\section{Reshaping data}\label{reshaping-data}

Data sets (i.e.~frames or tables) may arrive in a ``wide'' form or a
``long'' form. The difference is best illustrated with an example. The
\texttt{ChickWeight} data encodes the weight of various chicks. It is
``long'' in that a variable encodes the time of measurement, making the
data, well, simply long:

\begin{Shaded}
\begin{Highlighting}[]
\NormalTok{ChickWeight }\OperatorTok\StringTok{  }\NormalTok{head}
\end{Highlighting}
\end{Shaded}

\begin{verbatim}
## Grouped Data: weight ~ Time | Chick
##   weight Time Chick Diet
## 1     42    0     1    1
## 2     51    2     1    1
## 3     59    4     1    1
## 4     64    6     1    1
## 5     76    8     1    1
## 6     93   10     1    1
\end{verbatim}

The \texttt{mtcars} data encodes 11 characteristics of 32 types of
automobiles. It is ``wide'' since the various characteristics are
encoded in different variables, making the data, well, simply wide.

\begin{Shaded}
\begin{Highlighting}[]
\NormalTok{mtcars }\OperatorTok\StringTok{ }\NormalTok{head}
\end{Highlighting}
\end{Shaded}

\begin{verbatim}
##                    mpg cyl disp  hp drat    wt  qsec vs am gear carb
## Mazda RX4         21.0   6  160 110 3.90 2.620 16.46  0  1    4    4
## Mazda RX4 Wag     21.0   6  160 110 3.90 2.875 17.02  0  1    4    4
## Datsun 710        22.8   4  108  93 3.85 2.320 18.61  1  1    4    1
## Hornet 4 Drive    21.4   6  258 110 3.08 3.215 19.44  1  0    3    1
## Hornet Sportabout 18.7   8  360 175 3.15 3.440 17.02  0  0    3    2
## Valiant           18.1   6  225 105 2.76 3.460 20.22  1  0    3    1
\end{verbatim}

Most of \emph{R}'s functions, with exceptions, will prefer data in the
long format. There are thus various facilities to convert from one
format to another. We will focus on the \texttt{melt} and \texttt{dcast}
functions to convert from one format to another.

\subsection{Wide to long}\label{wide-to-long}

\texttt{melt} will convert from wide to long.

\begin{Shaded}
\begin{Highlighting}[]
\KeywordTok{dimnames}\NormalTok{(mtcars)}
\end{Highlighting}
\end{Shaded}

\begin{verbatim}
## [[1]]
##  [1] "Mazda RX4"           "Mazda RX4 Wag"       "Datsun 710"         
##  [4] "Hornet 4 Drive"      "Hornet Sportabout"   "Valiant"            
##  [7] "Duster 360"          "Merc 240D"           "Merc 230"           
## [10] "Merc 280"            "Merc 280C"           "Merc 450SE"         
## [13] "Merc 450SL"          "Merc 450SLC"         "Cadillac Fleetwood" 
## [16] "Lincoln Continental" "Chrysler Imperial"   "Fiat 128"           
## [19] "Honda Civic"         "Toyota Corolla"      "Toyota Corona"      
## [22] "Dodge Challenger"    "AMC Javelin"         "Camaro Z28"         
## [25] "Pontiac Firebird"    "Fiat X1-9"           "Porsche 914-2"      
## [28] "Lotus Europa"        "Ford Pantera L"      "Ferrari Dino"       
## [31] "Maserati Bora"       "Volvo 142E"         
## 
## [[2]]
##  [1] "mpg"  "cyl"  "disp" "hp"   "drat" "wt"   "qsec" "vs"   "am"   "gear"
## [11] "carb"
\end{verbatim}

\begin{Shaded}
\begin{Highlighting}[]
\NormalTok{mtcars}\OperatorTok{$}\NormalTok{type <-}\StringTok{ }\KeywordTok{rownames}\NormalTok{(mtcars)}
\KeywordTok{melt}\NormalTok{(mtcars, }\DataTypeTok{id.vars=}\KeywordTok{c}\NormalTok{(}\StringTok{"type"}\NormalTok{)) }\OperatorTok\StringTok{ }\NormalTok{head}
\end{Highlighting}
\end{Shaded}

\begin{verbatim}
##                type variable value
## 1         Mazda RX4      mpg  21.0
## 2     Mazda RX4 Wag      mpg  21.0
## 3        Datsun 710      mpg  22.8
## 4    Hornet 4 Drive      mpg  21.4
## 5 Hornet Sportabout      mpg  18.7
## 6           Valiant      mpg  18.1
\end{verbatim}

Things to note:

\begin{itemize}
\tightlist
\item
  The car type was originally encoded in the rows' names, and not as a
  variable. We thus created an explicit variable with the cars' type
  using the \texttt{rownames} function.
\item
  The \texttt{id.vars} of the \texttt{melt} function names the variables
  that will be used as identifiers. All other variables are assumed to
  be measurements. These can have been specified using their index
  instead of their name.
\item
  If not all variables are measurements, we could have names measurement
  variables explicitly using the \texttt{measure.vars} argument of the
  \texttt{melt} function. These can have been specified using their
  index instead of their name.
\item
  By default, the molten columns are automatically named
  \texttt{variable} and \texttt{value}.
\end{itemize}

We can replace the automatic namings using \texttt{variable.name} and
\texttt{value.name}:

\begin{Shaded}
\begin{Highlighting}[]
\KeywordTok{melt}\NormalTok{(mtcars, }\DataTypeTok{id.vars=}\KeywordTok{c}\NormalTok{(}\StringTok{"type"}\NormalTok{), }\DataTypeTok{variable.name=}\StringTok{"Charachteristic"}\NormalTok{, }\DataTypeTok{value.name=}\StringTok{"Measurement"}\NormalTok{) }\OperatorTok\StringTok{ }\NormalTok{head}
\end{Highlighting}
\end{Shaded}

\begin{verbatim}
##                type Charachteristic Measurement
## 1         Mazda RX4             mpg        21.0
## 2     Mazda RX4 Wag             mpg        21.0
## 3        Datsun 710             mpg        22.8
## 4    Hornet 4 Drive             mpg        21.4
## 5 Hornet Sportabout             mpg        18.7
## 6           Valiant             mpg        18.1
\end{verbatim}

\subsection{Long to wide}\label{long-to-wide}

\texttt{dcast} will convert from long to wide:

\begin{Shaded}
\begin{Highlighting}[]
\KeywordTok{dcast}\NormalTok{(ChickWeight, Chick}\OperatorTok{~}\NormalTok{Time, }\DataTypeTok{value.var=}\StringTok{"weight"}\NormalTok{)}
\end{Highlighting}
\end{Shaded}

\begin{verbatim}
##    Chick  0  2  4  6   8  10  12  14  16  18  20  21
## 1     18 39 35 NA NA  NA  NA  NA  NA  NA  NA  NA  NA
## 2     16 41 45 49 51  57  51  54  NA  NA  NA  NA  NA
## 3     15 41 49 56 64  68  68  67  68  NA  NA  NA  NA
## 4     13 41 48 53 60  65  67  71  70  71  81  91  96
## 5      9 42 51 59 68  85  96  90  92  93 100 100  98
## 6     20 41 47 54 58  65  73  77  89  98 107 115 117
## 7     10 41 44 52 63  74  81  89  96 101 112 120 124
## 8      8 42 50 61 71  84  93 110 116 126 134 125  NA
## 9     17 42 51 61 72  83  89  98 103 113 123 133 142
## 10    19 43 48 55 62  65  71  82  88 106 120 144 157
## 11     4 42 49 56 67  74  87 102 108 136 154 160 157
## 12     6 41 49 59 74  97 124 141 148 155 160 160 157
## 13    11 43 51 63 84 112 139 168 177 182 184 181 175
## 14     3 43 39 55 67  84  99 115 138 163 187 198 202
## 15     1 42 51 59 64  76  93 106 125 149 171 199 205
## 16    12 41 49 56 62  72  88 119 135 162 185 195 205
## 17     2 40 49 58 72  84 103 122 138 162 187 209 215
## 18     5 41 42 48 60  79 106 141 164 197 199 220 223
## 19    14 41 49 62 79 101 128 164 192 227 248 259 266
## 20     7 41 49 57 71  89 112 146 174 218 250 288 305
## 21    24 42 52 58 74  66  68  70  71  72  72  76  74
## 22    30 42 48 59 72  85  98 115 122 143 151 157 150
## 23    22 41 55 64 77  90  95 108 111 131 148 164 167
## 24    23 43 52 61 73  90 103 127 135 145 163 170 175
## 25    27 39 46 58 73  87 100 115 123 144 163 185 192
## 26    28 39 46 58 73  92 114 145 156 184 207 212 233
## 27    26 42 48 57 74  93 114 136 147 169 205 236 251
## 28    25 40 49 62 78 102 124 146 164 197 231 259 265
## 29    29 39 48 59 74  87 106 134 150 187 230 279 309
## 30    21 40 50 62 86 125 163 217 240 275 307 318 331
## 31    33 39 50 63 77  96 111 137 144 151 146 156 147
## 32    37 41 48 56 68  80  83 103 112 135 157 169 178
## 33    36 39 48 61 76  98 116 145 166 198 227 225 220
## 34    31 42 53 62 73  85 102 123 138 170 204 235 256
## 35    39 42 50 61 78  89 109 130 146 170 214 250 272
## 36    38 41 49 61 74  98 109 128 154 192 232 280 290
## 37    32 41 49 65 82 107 129 159 179 221 263 291 305
## 38    40 41 55 66 79 101 120 154 182 215 262 295 321
## 39    34 41 49 63 85 107 134 164 186 235 294 327 341
## 40    35 41 53 64 87 123 158 201 238 287 332 361 373
## 41    44 42 51 65 86 103 118 127 138 145 146  NA  NA
## 42    45 41 50 61 78  98 117 135 141 147 174 197 196
## 43    43 42 55 69 96 131 157 184 188 197 198 199 200
## 44    41 42 51 66 85 103 124 155 153 175 184 199 204
## 45    47 41 53 66 79 100 123 148 157 168 185 210 205
## 46    49 40 53 64 85 108 128 152 166 184 203 233 237
## 47    46 40 52 62 82 101 120 144 156 173 210 231 238
## 48    50 41 54 67 84 105 122 155 175 205 234 264 264
## 49    42 42 49 63 84 103 126 160 174 204 234 269 281
## 50    48 39 50 62 80 104 125 154 170 222 261 303 322
\end{verbatim}

Things to note:

\begin{itemize}
\tightlist
\item
  \texttt{dcast} uses a formula interface (\texttt{\textasciitilde{}})
  to specify the row identifier and the variables. The LHS is the row
  identifier, and the RHS for the variables to be created.
\item
  The measurement of each LHS at each RHS, is specified using the
  \texttt{value.var} argument.
\end{itemize}

\section{Bibliographic Notes}\label{bibliographic-notes-2}

\texttt{data.table} has excellent online documentation. See
\href{https://cran.r-project.org/web/packages/data.table/vignettes/datatable-intro.html}{here}.
See
\href{https://rstudio-pubs-static.s3.amazonaws.com/52230_5ae0d25125b544caab32f75f0360e775.html}{here}
for \textbf{joining}. See
\href{https://cran.r-project.org/web/packages/data.table/vignettes/datatable-reshape.html}{here}
for more on \textbf{reshaping}. See
\href{https://www.r-bloggers.com/intro-to-the-data-table-package/}{here}
for a comparison of the \texttt{data.frame} way, versus the
\texttt{data.table} way. For some advanced tips and tricks see
\href{http://brooksandrew.github.io/simpleblog/articles/advanced-data-table/}{Andrew
Brooks' blog}.

\section{Practice Yourself}\label{practice-yourself-1}

\begin{enumerate}
\def\labelenumi{\arabic{enumi}.}
\tightlist
\item
  Create a matrix of ones with \texttt{1e5} rows and \texttt{1e2}
  columns. Create a \texttt{data.table} using this matrix.

  \begin{enumerate}
  \def\labelenumii{\arabic{enumii}.}
  \tightlist
  \item
    Replace the first column of each, with the sequence \(1,2,3,\dots\).
  \item
    Create a column which is the sum of all columns, and a
    \(\mathcal{N}(0,1)\) random variable.
  \end{enumerate}
\item
  Use the cars dataset used in this chapter from kaggle
  \href{https://www.kaggle.com/orgesleka/used-cars-database}{Kaggle}.

  \begin{enumerate}
  \def\labelenumii{\arabic{enumii}.}
  \tightlist
  \item
    Import the data using the function \texttt{fread}. What is the class
    of your object?
  \item
    Use \texttt{system.time()} to measure the time to sort along
    ``seller''. Do the same after converting the data to
    \texttt{data.frame}. Are data tables faster?
  \end{enumerate}
\end{enumerate}

Also, see DataCamp's
\href{https://www.datacamp.com/courses/data-manipulation-in-r-with-datatable}{Data
Manipulation in R with data.table}, by Matt Dowle, the author of
\emph{data.table} for more self practice.

\chapter{Exploratory Data Analysis}\label{eda}

Exploratory Data Analysis (EDA) is a term coined by
\href{https://en.wikipedia.org/wiki/John_Tukey}{John W. Tukey} in his
seminal book \citep{tukey1977exploratory}. It is also (arguably) known
as \emph{Visual Analytics}, or \emph{Descriptive Statistics}. It is the
practice of inspecting, and exploring your data, before stating
hypotheses, fitting predictors, and other more ambitious inferential
goals. It typically includes the computation of simple \emph{summary
statistics} which capture some property of interest in the data, and
\emph{visualization}. EDA can be thought of as an assumption free,
purely algorithmic practice.

In this text we present EDA techniques along the following lines:

\begin{itemize}
\tightlist
\item
  How we explore: with summary-statistics, or visually?
\item
  How many variables analyzed simultaneously: univariate, bivariate, or
  multivariate?
\item
  What type of variable: categorical or continuous?
\end{itemize}

\section{Summary Statistics}\label{summary-statistics}

\subsection{Categorical Data}\label{categorical-data}

Categorical variables do not admit any mathematical operations on them.
We cannot sum them, or even sort them. We can only \textbf{count} them.
As such, summaries of categorical variables will always start with the
counting of the frequency of each category.

\subsubsection{Summary of Univariate Categorical
Data}\label{summary-of-univariate-categorical-data}

\begin{Shaded}
\begin{Highlighting}[]
\CommentTok{# Make some data}
\NormalTok{gender <-}\StringTok{ }\KeywordTok{c}\NormalTok{(}\KeywordTok{rep}\NormalTok{(}\StringTok{'Boy'}\NormalTok{, }\DecValTok{10}\NormalTok{), }\KeywordTok{rep}\NormalTok{(}\StringTok{'Girl'}\NormalTok{, }\DecValTok{12}\NormalTok{))}
\NormalTok{drink <-}\StringTok{ }\KeywordTok{c}\NormalTok{(}\KeywordTok{rep}\NormalTok{(}\StringTok{'Coke'}\NormalTok{, }\DecValTok{5}\NormalTok{), }\KeywordTok{rep}\NormalTok{(}\StringTok{'Sprite'}\NormalTok{, }\DecValTok{3}\NormalTok{), }\KeywordTok{rep}\NormalTok{(}\StringTok{'Coffee'}\NormalTok{, }\DecValTok{6}\NormalTok{), }\KeywordTok{rep}\NormalTok{(}\StringTok{'Tea'}\NormalTok{, }\DecValTok{7}\NormalTok{), }\KeywordTok{rep}\NormalTok{(}\StringTok{'Water'}\NormalTok{, }\DecValTok{1}\NormalTok{))  }
\NormalTok{age <-}\StringTok{  }\KeywordTok{sample}\NormalTok{(}\KeywordTok{c}\NormalTok{(}\StringTok{'Young'}\NormalTok{, }\StringTok{'Old'}\NormalTok{), }\DataTypeTok{size =} \KeywordTok{length}\NormalTok{(gender), }\DataTypeTok{replace =} \OtherTok{TRUE}\NormalTok{)}
\CommentTok{# Count frequencies}
\KeywordTok{table}\NormalTok{(gender)}
\end{Highlighting}
\end{Shaded}

\begin{verbatim}
## gender
##  Boy Girl 
##   10   12
\end{verbatim}

\begin{Shaded}
\begin{Highlighting}[]
\KeywordTok{table}\NormalTok{(drink)}
\end{Highlighting}
\end{Shaded}

\begin{verbatim}
## drink
## Coffee   Coke Sprite    Tea  Water 
##      6      5      3      7      1
\end{verbatim}

\begin{Shaded}
\begin{Highlighting}[]
\KeywordTok{table}\NormalTok{(age)}
\end{Highlighting}
\end{Shaded}

\begin{verbatim}
## age
##   Old Young 
##    10    12
\end{verbatim}

If instead of the level counts you want the proportions, you can use
\texttt{prop.table}

\begin{Shaded}
\begin{Highlighting}[]
\KeywordTok{prop.table}\NormalTok{(}\KeywordTok{table}\NormalTok{(gender))}
\end{Highlighting}
\end{Shaded}

\begin{verbatim}
## gender
##       Boy      Girl 
## 0.4545455 0.5454545
\end{verbatim}

\subsubsection{Summary of Bivariate Categorical
Data}\label{summary-of-bivariate-categorical-data}

\begin{Shaded}
\begin{Highlighting}[]
\KeywordTok{library}\NormalTok{(magrittr)}
\KeywordTok{cbind}\NormalTok{(gender, drink) }\OperatorTok\StringTok{ }\NormalTok{head }\CommentTok{# bind vectors into matrix and inspect (`c` for column)}
\end{Highlighting}
\end{Shaded}

\begin{verbatim}
##      gender drink   
## [1,] "Boy"  "Coke"  
## [2,] "Boy"  "Coke"  
## [3,] "Boy"  "Coke"  
## [4,] "Boy"  "Coke"  
## [5,] "Boy"  "Coke"  
## [6,] "Boy"  "Sprite"
\end{verbatim}

\begin{Shaded}
\begin{Highlighting}[]
\NormalTok{table1 <-}\StringTok{ }\KeywordTok{table}\NormalTok{(gender, drink) }\CommentTok{# count frequencies of bivariate combinations}
\NormalTok{table1                                      }
\end{Highlighting}
\end{Shaded}

\begin{verbatim}
##       drink
## gender Coffee Coke Sprite Tea Water
##   Boy       2    5      3   0     0
##   Girl      4    0      0   7     1
\end{verbatim}

\subsubsection{Summary of Multivariate Categorical
Data}\label{summary-of-multivariate-categorical-data}

You may be wondering how does R handle tables with more than two
dimensions. It is indeed not trivial to report this in a human-readable
way. R offers several solutions: \texttt{table} is easier to compute
with, and \texttt{ftable} is human readable.

\begin{Shaded}
\begin{Highlighting}[]
\NormalTok{table2.}\DecValTok{1}\NormalTok{ <-}\StringTok{ }\KeywordTok{table}\NormalTok{(gender, drink, age) }\CommentTok{# A machine readable table. }
\NormalTok{table2.}\DecValTok{1}
\end{Highlighting}
\end{Shaded}

\begin{verbatim}
## , , age = Old
## 
##       drink
## gender Coffee Coke Sprite Tea Water
##   Boy       0    3      1   0     0
##   Girl      1    0      0   5     0
## 
## , , age = Young
## 
##       drink
## gender Coffee Coke Sprite Tea Water
##   Boy       2    2      2   0     0
##   Girl      3    0      0   2     1
\end{verbatim}

\begin{Shaded}
\begin{Highlighting}[]
\NormalTok{table.}\FloatTok{2.2}\NormalTok{ <-}\StringTok{ }\KeywordTok{ftable}\NormalTok{(gender, drink, age) }\CommentTok{# A human readable table (`f` for Flat).}
\NormalTok{table.}\FloatTok{2.2}
\end{Highlighting}
\end{Shaded}

\begin{verbatim}
##               age Old Young
## gender drink               
## Boy    Coffee       0     2
##        Coke         3     2
##        Sprite       1     2
##        Tea          0     0
##        Water        0     0
## Girl   Coffee       1     3
##        Coke         0     0
##        Sprite       0     0
##        Tea          5     2
##        Water        0     1
\end{verbatim}

If you want proportions instead of counts, you need to specify the
denominator, i.e., the margins. Think: what is the margin in each of the
following outputs?

\begin{Shaded}
\begin{Highlighting}[]
\KeywordTok{prop.table}\NormalTok{(table1, }\DataTypeTok{margin =} \DecValTok{1}\NormalTok{) }\CommentTok{# every *row* sums to to 1}
\end{Highlighting}
\end{Shaded}

\begin{verbatim}
##       drink
## gender     Coffee       Coke     Sprite        Tea      Water
##   Boy  0.20000000 0.50000000 0.30000000 0.00000000 0.00000000
##   Girl 0.33333333 0.00000000 0.00000000 0.58333333 0.08333333
\end{verbatim}

\begin{Shaded}
\begin{Highlighting}[]
\KeywordTok{prop.table}\NormalTok{(table1, }\DataTypeTok{margin =} \DecValTok{2}\NormalTok{) }\CommentTok{# every *column* sums to 1}
\end{Highlighting}
\end{Shaded}

\begin{verbatim}
##       drink
## gender    Coffee      Coke    Sprite       Tea     Water
##   Boy  0.3333333 1.0000000 1.0000000 0.0000000 0.0000000
##   Girl 0.6666667 0.0000000 0.0000000 1.0000000 1.0000000
\end{verbatim}

\subsection{Continous Data}\label{continous-data}

Continuous variables admit many more operations than categorical. We can
compute sums, means, quantiles, and more.

\subsubsection{Summary of Univariate Continuous
Data}\label{summary-of-univariate-continuous-data}

We distinguish between several types of summaries, each capturing a
different property of the data.

\subsubsection{Summary of Location}\label{summary-of-location}

Capture the ``location'' of the data. These include:

\BeginKnitrBlock{definition}[Average]
\protect\hypertarget{def:unnamed-chunk-113}{}{\label{def:unnamed-chunk-113}
\iffalse (Average) \fi{} }The mean, or average, of a sample
\(x:=(x_1,\dots,x_n)\), denoted \(\bar x\) is defined as
\[ \bar x := n^{-1} \sum x_i. \]
\EndKnitrBlock{definition}

The sample mean is \textbf{non robust}. A single large observation may
inflate the mean indefinitely. For this reason, we define several other
summaries of location, which are more robust, i.e., less affected by
``contaminations'' of the data.

We start by defining the sample quantiles, themselves \textbf{not} a
summary of location.

\BeginKnitrBlock{definition}[Quantiles]
\protect\hypertarget{def:unnamed-chunk-114}{}{\label{def:unnamed-chunk-114}
\iffalse (Quantiles) \fi{} }The \(\alpha\) quantile of a sample \(x\),
denoted \(x_\alpha\), is (non uniquely) defined as a value above
\(100 \alpha \%\) of the sample, and below \(100 (1-\alpha) \%\).
\EndKnitrBlock{definition}

We emphasize that sample quantiles are non-uniquely defined. See
\texttt{?quantile} for the 9(!) different definitions that R provides.

Using the sample quantiles, we can now define another summary of
location, the \textbf{median}.

\BeginKnitrBlock{definition}[Median]
\protect\hypertarget{def:unnamed-chunk-115}{}{\label{def:unnamed-chunk-115}
\iffalse (Median) \fi{} }The median of a sample \(x\), denoted
\(x_{0.5}\) is the \(\alpha=0.5\) quantile of the sample.
\EndKnitrBlock{definition}

A whole family of summaries of locations is the \textbf{alpha trimmed
mean}.

\BeginKnitrBlock{definition}[Alpha Trimmed Mean]
\protect\hypertarget{def:unnamed-chunk-116}{}{\label{def:unnamed-chunk-116}
\iffalse (Alpha Trimmed Mean) \fi{} }The \(\alpha\) trimmed mean of a
sample \(x\), denoted \(\bar x_\alpha\) is the average of the sample
after removing the \(\alpha\) proportion of largest and \(\alpha\)
proportion of smallest observations.
\EndKnitrBlock{definition}

The simple mean and median are instances of the alpha trimmed mean:
\(\bar x_0\) and \(\bar x_{0.5}\) respectively.

Here are the R implementations:

\begin{Shaded}
\begin{Highlighting}[]
\NormalTok{x <-}\StringTok{ }\KeywordTok{rexp}\NormalTok{(}\DecValTok{100}\NormalTok{) }\CommentTok{# generate some (assymetric) random data}
\KeywordTok{mean}\NormalTok{(x) }\CommentTok{# simple mean}
\end{Highlighting}
\end{Shaded}

\begin{verbatim}
## [1] 1.017118
\end{verbatim}

\begin{Shaded}
\begin{Highlighting}[]
\KeywordTok{median}\NormalTok{(x) }\CommentTok{# median}
\end{Highlighting}
\end{Shaded}

\begin{verbatim}
## [1] 0.5805804
\end{verbatim}

\begin{Shaded}
\begin{Highlighting}[]
\KeywordTok{mean}\NormalTok{(x, }\DataTypeTok{trim =} \FloatTok{0.2}\NormalTok{) }\CommentTok{# alpha trimmed mean with alpha=0.2}
\end{Highlighting}
\end{Shaded}

\begin{verbatim}
## [1] 0.7711528
\end{verbatim}

\subsubsection{Summary of Scale}\label{summary-of-scale}

The \emph{scale} of the data, sometimes known as \emph{spread}, can be
thought of its variability.

\BeginKnitrBlock{definition}[Standard Deviation]
\protect\hypertarget{def:unnamed-chunk-118}{}{\label{def:unnamed-chunk-118}
\iffalse (Standard Deviation) \fi{} }The standard deviation of a sample
\(x\), denoted \(S(x)\), is defined as
\[ S(x):=\sqrt{(n-1)^{-1} \sum (x_i-\bar x)^2} . \]
\EndKnitrBlock{definition}

For reasons of robustness, we define other, more robust, measures of
scale.

\BeginKnitrBlock{definition}[MAD]
\protect\hypertarget{def:unnamed-chunk-119}{}{\label{def:unnamed-chunk-119}
\iffalse (MAD) \fi{} }The Median Absolute Deviation from the median,
denoted as \(MAD(x)\), is defined as
\[MAD(x):= c \: |x-x_{0.5}|_{0.5} . \]
\EndKnitrBlock{definition}

where \(c\) is some constant, typically set to \(c=1.4826\) so that MAD
and \(S(x)\) have the same large sample limit.

\BeginKnitrBlock{definition}[IQR]
\protect\hypertarget{def:unnamed-chunk-120}{}{\label{def:unnamed-chunk-120}
\iffalse (IQR) \fi{} }The Inter Quartile Range of a sample \(x\),
denoted as \(IQR(x)\), is defined as \[ IQR(x):= x_{0.75}-x_{0.25} .\]
\EndKnitrBlock{definition}

Here are the R implementations

\begin{Shaded}
\begin{Highlighting}[]
\KeywordTok{sd}\NormalTok{(x) }\CommentTok{# standard deviation}
\end{Highlighting}
\end{Shaded}

\begin{verbatim}
## [1] 0.9981981
\end{verbatim}

\begin{Shaded}
\begin{Highlighting}[]
\KeywordTok{mad}\NormalTok{(x) }\CommentTok{# MAD}
\end{Highlighting}
\end{Shaded}

\begin{verbatim}
## [1] 0.6835045
\end{verbatim}

\begin{Shaded}
\begin{Highlighting}[]
\KeywordTok{IQR}\NormalTok{(x) }\CommentTok{# IQR}
\end{Highlighting}
\end{Shaded}

\begin{verbatim}
## [1] 1.337731
\end{verbatim}

\subsubsection{Summary of Asymmetry}\label{summary-of-asymmetry}

Summaries of asymmetry, also known as \emph{skewness}, quantify the
departure of the \(x\) from a symmetric sample.

\BeginKnitrBlock{definition}[Yule]
\protect\hypertarget{def:unnamed-chunk-122}{}{\label{def:unnamed-chunk-122}
\iffalse (Yule) \fi{} }The Yule measure of assymetry, denoted
\(Yule(x)\) is defined as
\[Yule(x) := \frac{1/2 \: (x_{0.75}+x_{0.25}) - x_{0.5} }{1/2 \: IQR(x)} .\]
\EndKnitrBlock{definition}

Here is an R implementation

\begin{Shaded}
\begin{Highlighting}[]
\NormalTok{yule <-}\StringTok{ }\ControlFlowTok{function}\NormalTok{(x)\{}
\NormalTok{  numerator <-}\StringTok{ }\FloatTok{0.5} \OperatorTok{*}\StringTok{ }\NormalTok{(}\KeywordTok{quantile}\NormalTok{(x,}\FloatTok{0.75}\NormalTok{) }\OperatorTok{+}\StringTok{ }\KeywordTok{quantile}\NormalTok{(x,}\FloatTok{0.25}\NormalTok{))}\OperatorTok{-}\KeywordTok{median}\NormalTok{(x) }
\NormalTok{  denominator <-}\StringTok{ }\FloatTok{0.5}\OperatorTok{*}\StringTok{ }\KeywordTok{IQR}\NormalTok{(x)}
  \KeywordTok{c}\NormalTok{(numerator}\OperatorTok{/}\NormalTok{denominator, }\DataTypeTok{use.names=}\OtherTok{FALSE}\NormalTok{)}
\NormalTok{\}}
\KeywordTok{yule}\NormalTok{(x)}
\end{Highlighting}
\end{Shaded}

\begin{verbatim}
## [1] 0.2080004
\end{verbatim}

Things to note:

\begin{itemize}
\tightlist
\item
  A perfectly symmetric vector will return 0 because the median will be
  exactly on the midway.
\item
  It is bounded between -1 and 1 because of the denominator
\end{itemize}

\subsubsection{Summary of Bivariate Continuous
Data}\label{summary-of-bivariate-continuous-data}

When dealing with bivariate, or multivariate data, we can obviously
compute univariate summaries for each variable separately. This is not
the topic of this section, in which we want to summarize the association
\textbf{between} the variables, and not within them.

\BeginKnitrBlock{definition}[Covariance]
\protect\hypertarget{def:unnamed-chunk-123}{}{\label{def:unnamed-chunk-123}
\iffalse (Covariance) \fi{} }The covariance between two samples, \(x\)
and \(y\), of same length \(n\), is defined as
\[Cov(x,y):= (n-1)^{-1} \sum (x_i-\bar x)(y_i-\bar y)  \]
\EndKnitrBlock{definition}

We emphasize this is not the covariance you learned about in probability
classes, since it is not the covariance between two \emph{random
variables} but rather, between two \emph{samples}. For this reasons,
some authors call it the \emph{empirical covariance}, or \emph{sample
covariance}.

\BeginKnitrBlock{definition}[Pearson's Correlation Coefficient]
\protect\hypertarget{def:unnamed-chunk-124}{}{\label{def:unnamed-chunk-124}
\iffalse (Pearson's Correlation Coefficient) \fi{} }Peasrson's
correlation coefficient, a.k.a. Pearson's moment product correlation, or
simply, the correlation, denoted \texttt{r(x,y)}, is defined as
\[r(x,y):=\frac{Cov(x,y)}{S(x)S(y)}. \]
\EndKnitrBlock{definition}

If you find this definition enigmatic, just think of the correlation as
the covariance between \(x\) and \(y\) after transforming each to the
unitless scale of z-scores.

\BeginKnitrBlock{definition}[Z-Score]
\protect\hypertarget{def:unnamed-chunk-125}{}{\label{def:unnamed-chunk-125}
\iffalse (Z-Score) \fi{} }The z-scores of a sample \(x\) are defined as
the mean-centered, scale normalized observations:
\[z_i(x):= \frac{x_i-\bar x}{S(x)}.\]
\EndKnitrBlock{definition}

We thus have that \(r(x,y)=Cov(z(x),z(y))\).

Here are the R implementations

\begin{Shaded}
\begin{Highlighting}[]
\NormalTok{y <-}\StringTok{ }\KeywordTok{rexp}\NormalTok{(}\DecValTok{100}\NormalTok{) }\CommentTok{# generate another vector of some random data}
\KeywordTok{cov}\NormalTok{(x,y) }\CommentTok{# covariance between x and y}
\end{Highlighting}
\end{Shaded}

\begin{verbatim}
## [1] 0.0203559
\end{verbatim}

\begin{Shaded}
\begin{Highlighting}[]
\KeywordTok{cor}\NormalTok{(x,y) }\CommentTok{# correlation between x and y (default is pearson)}
\end{Highlighting}
\end{Shaded}

\begin{verbatim}
## [1] 0.02380989
\end{verbatim}

\begin{Shaded}
\begin{Highlighting}[]
\KeywordTok{scale}\NormalTok{(x) }\OperatorTok\StringTok{ }\NormalTok{head }\CommentTok{# z-score of x}
\end{Highlighting}
\end{Shaded}

\begin{verbatim}
##             [,1]
## [1,]  1.72293613
## [2,]  0.83367533
## [3,]  0.27703737
## [4,] -1.00110536
## [5,]  0.07671776
## [6,] -0.66044228
\end{verbatim}

\subsubsection{Summary of Multivariate Continuous
Data}\label{summary-of-multivariate-continuous-data}

The covariance is a simple summary of association between two variables,
but it certainly may not capture the whole ``story'' when dealing with
more than two variables. The most common summary of multivariate
relation, is the \textbf{covariance matrix}, but we warn that only the
simplest multivariate relations are fully summarized by this matrix.

\BeginKnitrBlock{definition}[Sample Covariance Matrix]
\protect\hypertarget{def:unnamed-chunk-127}{}{\label{def:unnamed-chunk-127}
\iffalse (Sample Covariance Matrix) \fi{} }Given \(n\) observations on
\(p\) variables, denote \(x_{i,j}\) the \(i\)'th observation of the
\(j\)'th variable. The \emph{sample covariance matrix}, denoted
\(\hat \Sigma\) is defined as
\[\hat \Sigma_{k,l}=(n-1)^{-1} \sum_i [(x_{i,k}-\bar x_k)(x_{i,l}-\bar x_l)],\]
where \(\bar x_k:=n^{-1} \sum_i x_{i,k}\). Put differently, the
\(k,l\)'th entry in \(\hat \Sigma\) is the sample covariance between
variables \(k\) and \(l\).
\EndKnitrBlock{definition}

\BeginKnitrBlock{remark}
\iffalse{} {Remark. } \fi{}\(\hat \Sigma\) is clearly non robust. How
would you define a robust covariance matrix?
\EndKnitrBlock{remark}

\section{Visualization}\label{visualization}

Summarizing the information in a variable to a single number clearly
conceals much of the story in the sample. This is like inspecting a
person using a caricature, instead of a picture. Visualizing the data,
when possible, is more informative.

\subsection{Categorical Data}\label{categorical-data-1}

Recalling that with categorical variables we can only count the
frequency of each level, the plotting of such variables are typically
variations on the \emph{bar plot}.

\subsubsection{Visualizing Univariate Categorical
Data}\label{visualizing-univariate-categorical-data}

\begin{Shaded}
\begin{Highlighting}[]
\KeywordTok{barplot}\NormalTok{(}\KeywordTok{table}\NormalTok{(age))}
\end{Highlighting}
\end{Shaded}

\includegraphics[width=0.5\linewidth]{Rcourse_files/figure-latex/barplot-1}

\subsubsection{Visualizing Bivariate Categorical
Data}\label{visualizing-bivariate-categorical-data}

There are several generalizations of the barplot, aimed to deal with the
visualization of bivariate categorical data. They are sometimes known as
the \emph{clustered bar plot} and the \emph{stacked bar plot}. In this
text, we advocate the use of the \emph{mosaic plot} which is also the
default in R.

\begin{Shaded}
\begin{Highlighting}[]
\KeywordTok{plot}\NormalTok{(table1, }\DataTypeTok{main=}\StringTok{'Bivariate mosaic plot'}\NormalTok{)}
\end{Highlighting}
\end{Shaded}

\includegraphics[width=0.5\linewidth]{Rcourse_files/figure-latex/unnamed-chunk-129-1}

Things to note:

\begin{itemize}
\tightlist
\item
  The proportion of each category is encoded in the width of the bars
  (more girls than boys here)
\item
  Zero observations are marked as a line.
\end{itemize}

\subsubsection{Visualizing Multivariate Categorical
Data}\label{visualizing-multivariate-categorical-data}

The \emph{mosaic plot} is not easy to generalize to more than two
variables, but it is still possible (at the cost of interpretability).

\begin{Shaded}
\begin{Highlighting}[]
\KeywordTok{plot}\NormalTok{(table2.}\DecValTok{1}\NormalTok{, }\DataTypeTok{main=}\StringTok{'Trivaraite mosaic plot'}\NormalTok{)}
\end{Highlighting}
\end{Shaded}

\includegraphics[width=0.5\linewidth]{Rcourse_files/figure-latex/unnamed-chunk-130-1}

When one of the variables is a (discrete) time variable, then the plot
has a notion dynamics in time. For this see the Alluvial plot
\ref{alluvial}.

If the variables represent a hierarchy, consider a \textbf{Sunburst
Plot}:

\begin{Shaded}
\begin{Highlighting}[]
\KeywordTok{library}\NormalTok{(sunburstR)}
\CommentTok{# read in sample visit-sequences.csv data provided in source}
\CommentTok{# https://gist.github.com/kerryrodden/7090426#file-visit-sequences-csv}
\NormalTok{sequences <-}\StringTok{ }\KeywordTok{read.csv}\NormalTok{(}
  \KeywordTok{system.file}\NormalTok{(}\StringTok{"examples/visit-sequences.csv"}\NormalTok{,}\DataTypeTok{package=}\StringTok{"sunburstR"}\NormalTok{)}
\NormalTok{  ,}\DataTypeTok{header=}\NormalTok{F}
\NormalTok{  ,}\DataTypeTok{stringsAsFactors =} \OtherTok{FALSE}
\NormalTok{)}
\KeywordTok{sunburst}\NormalTok{(sequences) }\CommentTok{# In the HTML version of the book this plot is interactive.}
\end{Highlighting}
\end{Shaded}

\includegraphics[width=0.5\linewidth]{Rcourse_files/figure-latex/sunburst-1}

\subsection{Continuous Data}\label{continuous-data}

\subsubsection{Visualizing Univariate Continuous
Data}\label{visualizing-univariate-continuous-data}

Unlike categorical variables, there are endlessly many ways to visualize
continuous variables. The simplest way is to look at the raw data via
the \texttt{stripchart}.

\begin{Shaded}
\begin{Highlighting}[]
\NormalTok{sample1 <-}\StringTok{ }\KeywordTok{rexp}\NormalTok{(}\DecValTok{10}\NormalTok{)                             }
\KeywordTok{stripchart}\NormalTok{(sample1)}
\end{Highlighting}
\end{Shaded}

\includegraphics[width=0.5\linewidth]{Rcourse_files/figure-latex/unnamed-chunk-131-1}

Clearly, if there are many observations, the \texttt{stripchart} will be
a useless line of black dots. We thus bin them together, and look at the
frequency of each bin; this is the \emph{histogram}. R's
\texttt{histogram} function has very good defaults to choose the number
of bins. Here is a histogram showing the counts of each bin.

\begin{Shaded}
\begin{Highlighting}[]
\NormalTok{sample1 <-}\StringTok{ }\KeywordTok{rexp}\NormalTok{(}\DecValTok{100}\NormalTok{)                            }
\KeywordTok{hist}\NormalTok{(sample1, }\DataTypeTok{freq=}\NormalTok{T, }\DataTypeTok{main=}\StringTok{'Counts'}\NormalTok{)        }
\end{Highlighting}
\end{Shaded}

\includegraphics[width=0.5\linewidth]{Rcourse_files/figure-latex/unnamed-chunk-132-1}

The bin counts can be replaced with the proportion of each bin using the
\texttt{freq} argument.

\begin{Shaded}
\begin{Highlighting}[]
\KeywordTok{hist}\NormalTok{(sample1, }\DataTypeTok{freq=}\NormalTok{F, }\DataTypeTok{main=}\StringTok{'Proportion'}\NormalTok{)    }
\end{Highlighting}
\end{Shaded}

\includegraphics[width=0.5\linewidth]{Rcourse_files/figure-latex/unnamed-chunk-133-1}

Things to note:

\begin{itemize}
\tightlist
\item
  The bins' proportion summary is larger than 1 because it considers
  each bin's width, which in this case has a constant width of 0.5,
  hence the total proportion sum is 1/0.5=2.
\end{itemize}

The bins of a histogram are non overlapping. We can adopt a sliding
window approach, instead of binning. This is the \emph{density plot}
which is produced with the \texttt{density} function, and added to an
existing plot with the \texttt{lines} function. The \texttt{rug}
function adds the original data points as ticks on the axes, and is
strongly recommended to detect artifacts introduced by the binning of
the histogram, or the smoothing of the density plot.

\begin{Shaded}
\begin{Highlighting}[]
\KeywordTok{hist}\NormalTok{(sample1, }\DataTypeTok{freq=}\NormalTok{F, }\DataTypeTok{main=}\StringTok{'Frequencies'}\NormalTok{)   }
\KeywordTok{lines}\NormalTok{(}\KeywordTok{density}\NormalTok{(sample1))                     }
\KeywordTok{rug}\NormalTok{(sample1)}
\end{Highlighting}
\end{Shaded}

\includegraphics[width=0.5\linewidth]{Rcourse_files/figure-latex/unnamed-chunk-134-1}

\BeginKnitrBlock{remark}
\iffalse{} {Remark. } \fi{}Why would it make no sense to make a table,
or a barplot, of continuous data?
\EndKnitrBlock{remark}

One particularly useful visualization, due to John W. Tukey, is the
\emph{boxplot}. The boxplot is designed to capture the main phenomena in
the data, and simultaneously point to outlines.

\begin{Shaded}
\begin{Highlighting}[]
\KeywordTok{boxplot}\NormalTok{(sample1)    }
\end{Highlighting}
\end{Shaded}

\includegraphics[width=0.5\linewidth]{Rcourse_files/figure-latex/unnamed-chunk-136-1}

Another way to deal with a massive amount of data points, is to
emphasize important points, and conceal non-important. This is the
purpose of \textbf{circle-packing} (example from
\href{https://www.r-graph-gallery.com/308-interactive-circle-packing/}{r-graph
gallery}):

\includegraphics[width=0.5\linewidth]{Rcourse_files/figure-latex/unnamed-chunk-137-1}

\subsubsection{Visualizing Bivariate Continuous
Data}\label{visualizing-bivariate-continuous-data}

The bivariate counterpart of the \texttt{stipchart} is the celebrated
scatter plot.

\begin{Shaded}
\begin{Highlighting}[]
\NormalTok{n <-}\StringTok{ }\DecValTok{20}
\NormalTok{x1 <-}\StringTok{ }\KeywordTok{rexp}\NormalTok{(n)}
\NormalTok{x2 <-}\StringTok{ }\DecValTok{2}\OperatorTok{*}\StringTok{ }\NormalTok{x1 }\OperatorTok{+}\StringTok{ }\DecValTok{4} \OperatorTok{+}\StringTok{ }\KeywordTok{rexp}\NormalTok{(n)}
\KeywordTok{plot}\NormalTok{(x2}\OperatorTok{~}\NormalTok{x1)}
\end{Highlighting}
\end{Shaded}

\includegraphics[width=0.5\linewidth]{Rcourse_files/figure-latex/unnamed-chunk-138-1}

A scatter-plot may be augmented with marginal univariate visualization.
See, for instance, the \emph{rug} function to add the raw data on the
margins:

\begin{Shaded}
\begin{Highlighting}[]
\KeywordTok{plot}\NormalTok{(x2}\OperatorTok{~}\NormalTok{x1)}
\KeywordTok{rug}\NormalTok{(x1,}\DataTypeTok{side =} \DecValTok{1}\NormalTok{)}
\KeywordTok{rug}\NormalTok{(x2,}\DataTypeTok{side =} \DecValTok{2}\NormalTok{)}
\end{Highlighting}
\end{Shaded}

\includegraphics[width=0.5\linewidth]{Rcourse_files/figure-latex/unnamed-chunk-139-1}

A fancier version may use a histogram on the margins:

\includegraphics[width=0.5\linewidth]{Rcourse_files/figure-latex/unnamed-chunk-140-1}

Like the univariate \texttt{stripchart}, the scatter plot will be an
uninformative mess in the presence of a lot of data. A nice bivariate
counterpart of the univariate histogram is the \emph{hexbin plot}, which
tessellates the plane with hexagons, and reports their frequencies.

\begin{Shaded}
\begin{Highlighting}[]
\KeywordTok{library}\NormalTok{(hexbin) }\CommentTok{# load required library}
\NormalTok{n <-}\StringTok{ }\FloatTok{2e5}
\NormalTok{x1 <-}\StringTok{ }\KeywordTok{rexp}\NormalTok{(n)}
\NormalTok{x2 <-}\StringTok{ }\DecValTok{2}\OperatorTok{*}\StringTok{ }\NormalTok{x1 }\OperatorTok{+}\StringTok{ }\DecValTok{4} \OperatorTok{+}\StringTok{ }\KeywordTok{rnorm}\NormalTok{(n)}
\KeywordTok{plot}\NormalTok{(}\KeywordTok{hexbin}\NormalTok{(}\DataTypeTok{x =}\NormalTok{ x1, }\DataTypeTok{y =}\NormalTok{ x2))}
\end{Highlighting}
\end{Shaded}

\includegraphics[width=0.5\linewidth]{Rcourse_files/figure-latex/unnamed-chunk-141-1}

\subsubsection{Visualizing Multivariate Continuous
Data}\label{visualizing-multivariate-continuous-data}

Visualizing multivariate data is a tremendous challenge given that we
cannot grasp \(4\) dimensional spaces, nor can the computer screen
present more than \(2\) dimensional spaces. We thus have several
options: (i) To project the data to 2D. This is discussed in the
Dimensionality Reduction Section \ref{dim-reduce}. (ii) To visualize not
the raw data, but rather its summaries, like the covariance matrix.

Our own \href{https://github.com/EfratVil/MultiNav}{Multinav} package
adopts an interactive approach. For each (multivariate) observation a
simple univariate summary may be computed and visualized. These
summaries may be compared, and the original (multivariate) observation
inspected upon demand. Contact
\href{http://efratvil.github.io/home/index.html}{Efrat} for more
details.\\
\includegraphics{art/multinav.png}

An alternative approach starts with the covariance matrix,
\(\hat \Sigma\), that can be visualized as an image. Note the use of the
\texttt{::} operator (called \emph{Double Colon Operator}, for help:
\texttt{?\textquotesingle{}::\textquotesingle{}}), which is used to call
a function from some package, without loading the whole package. We will
use the \texttt{::} operator when we want to emphasize the package of
origin of a function.

\begin{Shaded}
\begin{Highlighting}[]
\NormalTok{covariance <-}\StringTok{ }\KeywordTok{cov}\NormalTok{(longley) }\CommentTok{# The covariance of the longley dataset}
\NormalTok{correlations <-}\StringTok{ }\KeywordTok{cor}\NormalTok{(longley) }\CommentTok{# The correlations of the longley dataset}
\NormalTok{lattice}\OperatorTok{::}\KeywordTok{levelplot}\NormalTok{(correlations)}
\end{Highlighting}
\end{Shaded}

\includegraphics[width=0.5\linewidth]{Rcourse_files/figure-latex/unnamed-chunk-142-1}

If we believe the covariance has some structure, we can do better than
viewing the raw correlations. In temporal, and spatial data, we believe
correlations decay as some function of distances. We can thus view
correlations as a function of the distance between observations. This is
known as a \emph{variogram}. Note that for a variogram to be
informative, it is implied that correlations are merely a function of
distances (and not locations themselves). This is formally known as
\emph{stationary} and \emph{isotropic} correlations.

\begin{figure}
\centering
\includegraphics{art/variogram.png}
\caption{Variogram: plotting correlation as a function of spatial
distance. Courtesy of Ron Sarafian.}
\end{figure}

\hypertarget{parcoord}{\subsubsection{Parallel Coordinate
Plots}\label{parcoord}}

In a parallel coordinate plot, we plot a multivariate observation as a
function of its coordinates. In the following example, we visualize the
celebrated
\href{https://en.wikipedia.org/wiki/Iris_flower_data_set}{Iris dataset}.
In this dataset, for each of 50 iris flowers, Edgar Anderson measured 4
characteristics.

\begin{Shaded}
\begin{Highlighting}[]
\NormalTok{ir <-}\StringTok{ }\KeywordTok{rbind}\NormalTok{(iris3[,,}\DecValTok{1}\NormalTok{], iris3[,,}\DecValTok{2}\NormalTok{], iris3[,,}\DecValTok{3}\NormalTok{])}
\NormalTok{MASS}\OperatorTok{::}\KeywordTok{parcoord}\NormalTok{(}\KeywordTok{log}\NormalTok{(ir)[, }\KeywordTok{c}\NormalTok{(}\DecValTok{3}\NormalTok{, }\DecValTok{4}\NormalTok{, }\DecValTok{2}\NormalTok{, }\DecValTok{1}\NormalTok{)], }\DataTypeTok{col =} \DecValTok{1} \OperatorTok{+}\StringTok{ }\NormalTok{(}\DecValTok{0}\OperatorTok{:}\DecValTok{149}\NormalTok{)}\OperatorTok\DecValTok{50}\NormalTok{)}
\end{Highlighting}
\end{Shaded}

\includegraphics[width=0.5\linewidth]{Rcourse_files/figure-latex/unnamed-chunk-143-1}

\section{Mixed Type Data}\label{mixed-type-data}

Most real data sets will be of mixed type: both categorical and
continuous. One approach to view such data, is to visualize the
continuous variables separately, for each level of the categorical
variables. There are, however, interesting dedicated visualization for
such data.

\subsection{Alluvial Diagram}\label{alluvial}

An Alluvial plot is a type of \protect\hyperlink{parcoord}{Parallel
Coordinate Plot} for multivariate categorical data. It is particularly
interesting when the \(x\) axis is a discretized time variable, and it
is used to visualize flow.

The following example, from the \textbf{ggalluvial} package Vignette by
\href{https://cran.r-project.org/web/packages/ggalluvial/vignettes/ggalluvial.html}{Jason
Cory Brunson}, demonstrates the flow of students between different
majors, as semesters evolve.

\begin{Shaded}
\begin{Highlighting}[]
\KeywordTok{library}\NormalTok{(ggalluvial)}
\KeywordTok{data}\NormalTok{(majors)}
\NormalTok{majors}\OperatorTok{$}\NormalTok{curriculum <-}\StringTok{ }\KeywordTok{as.factor}\NormalTok{(majors}\OperatorTok{$}\NormalTok{curriculum)}
\KeywordTok{ggplot}\NormalTok{(majors,}
       \KeywordTok{aes}\NormalTok{(}\DataTypeTok{x =}\NormalTok{ semester, }\DataTypeTok{stratum =}\NormalTok{ curriculum, }\DataTypeTok{alluvium =}\NormalTok{ student,}
           \DataTypeTok{fill =}\NormalTok{ curriculum, }\DataTypeTok{label =}\NormalTok{ curriculum)) }\OperatorTok{+}
\StringTok{  }\KeywordTok{scale_fill_brewer}\NormalTok{(}\DataTypeTok{type =} \StringTok{"qual"}\NormalTok{, }\DataTypeTok{palette =} \StringTok{"Set2"}\NormalTok{) }\OperatorTok{+}
\StringTok{  }\KeywordTok{geom_flow}\NormalTok{(}\DataTypeTok{stat =} \StringTok{"alluvium"}\NormalTok{, }\DataTypeTok{lode.guidance =} \StringTok{"rightleft"}\NormalTok{,}
            \DataTypeTok{color =} \StringTok{"darkgray"}\NormalTok{) }\OperatorTok{+}
\StringTok{  }\KeywordTok{geom_stratum}\NormalTok{() }\OperatorTok{+}
\StringTok{  }\KeywordTok{theme}\NormalTok{(}\DataTypeTok{legend.position =} \StringTok{"bottom"}\NormalTok{) }\OperatorTok{+}
\StringTok{  }\KeywordTok{ggtitle}\NormalTok{(}\StringTok{"student curricula across several semesters"}\NormalTok{)}
\end{Highlighting}
\end{Shaded}

\includegraphics[width=0.5\linewidth]{Rcourse_files/figure-latex/unnamed-chunk-144-1}

Things to note:

\begin{itemize}
\tightlist
\item
  We used the \textbf{ggalluvial} package of the \textbf{ggplot2}
  ecosystem. More on \textbf{ggplot2} in the
  \protect\hyperlink{plotting}{Plotting Chapter}.
\item
  Time is on the \(x\) axis. Categories are color coded.
\end{itemize}

\BeginKnitrBlock{remark}
\iffalse{} {Remark. } \fi{}If the width of the lines encode magnitude,
the plot is also called a Sankey diagram.
\EndKnitrBlock{remark}

\section{Bibliographic Notes}\label{bibliographic-notes-3}

Like any other topic in this book, you can consult
\citet{venables2013modern}. The seminal book on EDA, written long before
R was around, is \citet{tukey1977exploratory}. For an excellent text on
robust statistics see \citet{wilcox2011introduction}.

\section{Practice Yourself}\label{practice-yourself-2}

\begin{enumerate}
\def\labelenumi{\arabic{enumi}.}
\item
  Read about the Titanic data set using \texttt{?Titanic}. Inspect it
  with the \texttt{table} and with the \texttt{ftable} commands. Which
  do you prefer?
\item
  Inspect the Titanic data with a plot. Start with
  \texttt{plot(Titanic)} Try also \texttt{lattice::dotplot}. Which is
  the passenger category with most survivors? Which plot do you prefer?
  Which scales better to more categories?
\item
  Read about the women data using \texttt{?women}.

  \begin{enumerate}
  \def\labelenumii{\arabic{enumii}.}
  \tightlist
  \item
    Compute the average of each variable. What is the average of the
    heights?
  \item
    Plot a histogram of the heights. Add ticks using \texttt{rug}.
  \item
    Plot a boxplot of the weights.
  \item
    Plot the heights and weights using a scatter plot. Add ticks using
    \texttt{rug}.
  \end{enumerate}
\item
  Choose \(\alpha\) to define a new symmetry measure:
  \(1/2(x_\alpha+x_{1-\alpha})-x_{0.5}\). Write a function that computes
  it, and apply it on women's heights data.
\item
  Compute the covariance matrix of women's heights and weights. Compute
  the correlation matrix. View the correlation matrix as an image using
  \texttt{lattice::levelplot}.
\item
  Pick a dataset with two LONG continous variables from
  \texttt{?datasets}. Plot it using \texttt{hexbin::hexbin}.
\end{enumerate}

\chapter{Linear Models}\label{lm}

\section{Problem Setup}\label{problem-setup}

\BeginKnitrBlock{example}[Bottle Cap Production]
\protect\hypertarget{exm:cap-experiment}{}{\label{exm:cap-experiment}
\iffalse (Bottle Cap Production) \fi{} }Consider a randomized experiment
designed to study the effects of temperature and pressure on the
diameter of manufactured a bottle cap.
\EndKnitrBlock{example}

\BeginKnitrBlock{example}[Rental Prices]
\protect\hypertarget{exm:rental}{}{\label{exm:rental} \iffalse (Rental
Prices) \fi{} }Consider the prediction of rental prices given an
appartment's attributes.
\EndKnitrBlock{example}

Both examples require some statistical model, but they are very
different. The first is a \emph{causal inference} problem: we want to
design an intervention so that we need to recover the causal effect of
temperature and pressure. The second is a
\href{https://en.wikipedia.org/wiki/Prediction}{prediction} problem,
a.k.a. a \href{https://en.wikipedia.org/wiki/Forecasting}{forecasting}
problem, in which we don't care about the causal effects, we just want
good predictions.

In this chapter we discuss the causal problem in Example
\ref{exm:cap-experiment}. This means that when we assume a model, we
assume it is the actual \emph{data generating process}, i.e., we assume
the \emph{sampling distribution} is well specified. The second type of
problems is discussed in the Supervised Learning Chapter
\ref{supervised}.

Here are some more examples of the types of problems we are discussing.

\BeginKnitrBlock{example}[Plant Growth]
\protect\hypertarget{exm:unnamed-chunk-146}{}{\label{exm:unnamed-chunk-146}
\iffalse (Plant Growth) \fi{} }Consider the treatment of various plants
with various fertilizers to study the fertilizer's effect on growth.
\EndKnitrBlock{example}

\BeginKnitrBlock{example}[Return to Education]
\protect\hypertarget{exm:unnamed-chunk-147}{}{\label{exm:unnamed-chunk-147}
\iffalse (Return to Education) \fi{} }Consider the study of return to
education by analyzing the incomes of individuals with different
education years.
\EndKnitrBlock{example}

\BeginKnitrBlock{example}[Drug Effect]
\protect\hypertarget{exm:unnamed-chunk-148}{}{\label{exm:unnamed-chunk-148}
\iffalse (Drug Effect) \fi{} }Consider the study of the effect of a new
drug for hemophilia, by analyzing the level of blood coagulation after
the administration of various amounts of the new drug.
\EndKnitrBlock{example}

Let's present the linear model. We assume that a response\footnote{The
  ``response'' is also known as the ``dependent'' variable in the
  statistical literature, or the ``labels'' in the machine learning
  literature.} variable is the sum of effects of some factors\footnote{The
  ``factors'' are also known as the ``independent variable'', or ``the
  design'', in the statistical literature, and the ``features'', or
  ``attributes'' in the machine learning literature.}. Denoting the
response variable by \(y\), the factors by \(x=(x_1,\dots,x_p)\), and
the effects by \(\beta:=(\beta_1,\dots,\beta_p)\) the linear model
assumption implies that the expected response is the sum of the factors
effects:

\begin{align}
  E[y]=x_1 \beta_1 + \dots + x_p \beta_p = \sum_{j=1}^p x_j \beta_j = x'\beta .
  \label{eq:linear-mean}
\end{align}

Clearly, there may be other factors that affect the the caps' diameters.
We thus introduce an error term\footnote{The ``error term'' is also
  known as the ``noise'', or the ``common causes of variability''.},
denoted by \(\varepsilon\), to capture the effects of all unmodeled
factors and measurement error\footnote{You may philosophize if the
  measurement error is a mere instance of unmodeled factors or not, but
  this has no real implication for our purposes.}. The implied
generative process of a sample of \(i=1,\dots,n\) observations it thus

\begin{align}
  y_i = x_i'\beta + \varepsilon_i = \sum_j x_{i,j} \beta_j + \varepsilon_i , i=1,\dots,n .
  \label{eq:linear-observed}
\end{align}

or in matrix notation

\begin{align}
  y = X \beta + \varepsilon .
  \label{eq:linear-matrix}
\end{align}

Let's demonstrate Eq.\eqref{eq:linear-observed}. In our bottle-caps
example {[}\ref{exm:cap-experiment}{]}, we may produce bottle caps at
various temperatures. We design an experiment where we produce
bottle-caps at varying temperatures. Let \(x_i\) be the temperature at
which bottle-cap \(i\) was manufactured. Let \(y_i\) be its measured
diameter. By the linear model assumption, the expected diameter varies
linearly with the temperature:
\(\mathbb{E}[y_i]=\beta_0 + x_i \beta_1\). This implies that \(\beta_1\)
is the (expected) change in diameter due to a unit change in
temperature.

\BeginKnitrBlock{remark}
\iffalse{} {Remark. } \fi{}In
\href{https://en.wikipedia.org/wiki/Regression_toward_the_mean}{Galton's}
classical regression problem, where we try to seek the relation between
the heights of sons and fathers then \(p=1\), \(y_i\) is the height of
the \(i\)'th father, and \(x_i\) the height of the \(i\)'th son. This is
a prediction problem, more than it is a causal-inference problem.
\EndKnitrBlock{remark}

There are many reasons linear models are very popular:

\begin{enumerate}
\def\labelenumi{\arabic{enumi}.}
\item
  Before the computer age, these were pretty much the only models that
  could actually be computed\footnote{By ``computed'' we mean what
    statisticians call ``fitted'', or ``estimated'', and computer
    scientists call ``learned''.}. The whole Analysis of Variance
  (ANOVA) literature is an instance of linear models, that relies on
  sums of squares, which do not require a computer to work with.
\item
  For purposes of prediction, where the actual data generating process
  is not of primary importance, they are popular because they simply
  work. Why is that? They are simple so that they do not require a lot
  of data to be computed. Put differently, they may be biased, but their
  variance is small enough to make them more accurate than other models.
\item
  For non continuous predictors, \textbf{any} functional relation can be
  cast as a linear model.
\item
  For the purpose of \emph{screening}, where we only want to show the
  existence of an effect, and are less interested in the magnitude of
  that effect, a linear model is enough.
\item
  If the true generative relation is not linear, but smooth enough, then
  the linear function is a good approximation via Taylor's theorem.
\end{enumerate}

There are still two matters we have to attend: (i) How to estimate
\(\beta\)? (ii) How to perform inference?

In the simplest linear models the estimation of \(\beta\) is done using
the method of least squares. A linear model with least squares
estimation is known as Ordinary Least Squares (OLS). The OLS problem:

\begin{align}
  \hat \beta:= argmin_\beta \{ \sum_i (y_i-x_i'\beta)^2 \},
  \label{eq:ols}
\end{align}

and in matrix notation

\begin{align}
  \hat \beta:= argmin_\beta \{ \Vert y-X\beta \Vert^2_2 \}.
  \label{eq:ols-matrix}
\end{align}

\BeginKnitrBlock{remark}
\iffalse{} {Remark. } \fi{}Personally, I prefer the matrix notation
because it is suggestive of the geometry of the problem. The reader is
referred to \citet{friedman2001elements}, Section 3.2, for more on the
geometry of OLS.
\EndKnitrBlock{remark}

Different software suits, and even different R packages, solve
Eq.\eqref{eq:ols} in different ways so that we skip the details of how
exactly it is solved. These are discussed in Chapters \ref{algebra} and
\ref{convex}.

The last matter we need to attend is how to do inference on
\(\hat \beta\). For that, we will need some assumptions on
\(\varepsilon\). A typical set of assumptions is the following:

\begin{enumerate}
\def\labelenumi{\arabic{enumi}.}
\tightlist
\item
  \textbf{Independence}: we assume \(\varepsilon_i\) are independent of
  everything else. Think of them as the measurement error of an
  instrument: it is independent of the measured value and of previous
  measurements.
\item
  \textbf{Centered}: we assume that \(E[\varepsilon]=0\), meaning there
  is no systematic error, sometimes it called The ``Linearity
  assumption''.
\item
  \textbf{Normality}: we will typically assume that
  \(\varepsilon \sim \mathcal{N}(0,\sigma^2)\), but we will later see
  that this is not really required.
\end{enumerate}

We emphasize that these assumptions are only needed for inference on
\(\hat \beta\) and not for the estimation itself, which is done by the
purely algorithmic framework of OLS.

Given the above assumptions, we can apply some probability theory and
linear algebra to get the distribution of the estimation error:

\begin{align}
  \hat \beta - \beta \sim \mathcal{N}(0, (X'X)^{-1} \sigma^2).
  \label{eq:ols-distribution}
\end{align}

The reason I am not too strict about the normality assumption above, is
that Eq.\eqref{eq:ols-distribution} is approximately correct even if
\(\varepsilon\) is not normal, provided that there are many more
observations than factors (\(n \gg p\)).

\section{OLS Estimation in R}\label{ols-estimation-in-r}

We are now ready to estimate some linear models with R. We will use the
\texttt{whiteside} data from the \textbf{MASS} package, recording the
outside temperature and gas consumption, before and after an apartment's
insulation.

\begin{Shaded}
\begin{Highlighting}[]
\KeywordTok{library}\NormalTok{(MASS) }\CommentTok{# load the package}
\KeywordTok{library}\NormalTok{(data.table) }\CommentTok{# for some data manipulations}
\KeywordTok{data}\NormalTok{(whiteside) }\CommentTok{# load the data}
\KeywordTok{head}\NormalTok{(whiteside) }\CommentTok{# inspect the data}
\end{Highlighting}
\end{Shaded}

\begin{verbatim}
##    Insul Temp Gas
## 1 Before -0.8 7.2
## 2 Before -0.7 6.9
## 3 Before  0.4 6.4
## 4 Before  2.5 6.0
## 5 Before  2.9 5.8
## 6 Before  3.2 5.8
\end{verbatim}

We do the OLS estimation on the pre-insulation data with \texttt{lm}
function (acronym for Linear Model), possibly the most important
function in R.

\begin{Shaded}
\begin{Highlighting}[]
\KeywordTok{library}\NormalTok{(data.table)}
\NormalTok{whiteside <-}\StringTok{ }\KeywordTok{data.table}\NormalTok{(whiteside)}
\NormalTok{lm.}\DecValTok{1}\NormalTok{ <-}\StringTok{ }\KeywordTok{lm}\NormalTok{(Gas}\OperatorTok{~}\NormalTok{Temp, }\DataTypeTok{data=}\NormalTok{whiteside[Insul}\OperatorTok{==}\StringTok{'Before'}\NormalTok{]) }\CommentTok{# OLS estimation }
\end{Highlighting}
\end{Shaded}

Things to note:

\begin{itemize}
\tightlist
\item
  We used the tilde syntax \texttt{Gas\textasciitilde{}Temp}, reading
  ``gas as linear function of temperature''.
\item
  The \texttt{data} argument tells R where to look for the variables Gas
  and Temp. We used
  \texttt{Insul==\textquotesingle{}Before\textquotesingle{}} to subset
  observations before the insulation.
\item
  The result is assigned to the object \texttt{lm.1}.
\end{itemize}

Like any other language, spoken or programmable, there are many ways to
say the same thing. Some more elegant than others\ldots{}

\begin{Shaded}
\begin{Highlighting}[]
\NormalTok{lm.}\DecValTok{1}\NormalTok{ <-}\StringTok{ }\KeywordTok{lm}\NormalTok{(}\DataTypeTok{y=}\NormalTok{Gas, }\DataTypeTok{x=}\NormalTok{Temp, }\DataTypeTok{data=}\NormalTok{whiteside[whiteside}\OperatorTok{$}\NormalTok{Insul}\OperatorTok{==}\StringTok{'Before'}\NormalTok{,]) }
\NormalTok{lm.}\DecValTok{1}\NormalTok{ <-}\StringTok{ }\KeywordTok{lm}\NormalTok{(}\DataTypeTok{y=}\NormalTok{whiteside[whiteside}\OperatorTok{$}\NormalTok{Insul}\OperatorTok{==}\StringTok{'Before'}\NormalTok{,]}\OperatorTok{$}\NormalTok{Gas,}\DataTypeTok{x=}\NormalTok{whiteside[whiteside}\OperatorTok{$}\NormalTok{Insul}\OperatorTok{==}\StringTok{'Before'}\NormalTok{,]}\OperatorTok{$}\NormalTok{Temp)  }
\NormalTok{lm.}\DecValTok{1}\NormalTok{ <-}\StringTok{ }\NormalTok{whiteside[whiteside}\OperatorTok{$}\NormalTok{Insul}\OperatorTok{==}\StringTok{'Before'}\NormalTok{,] }\OperatorTok\StringTok{ }\KeywordTok{lm}\NormalTok{(Gas}\OperatorTok{~}\NormalTok{Temp, }\DataTypeTok{data=}\NormalTok{.)}
\end{Highlighting}
\end{Shaded}

The output is an object of class \texttt{lm}.

\begin{Shaded}
\begin{Highlighting}[]
\KeywordTok{class}\NormalTok{(lm.}\DecValTok{1}\NormalTok{)}
\end{Highlighting}
\end{Shaded}

\begin{verbatim}
## [1] "lm"
\end{verbatim}

Objects of class \texttt{lm} are very complicated. They store a lot of
information which may be used for inference, plotting, etc. The
\texttt{str} function, short for ``structure'', shows us the various
elements of the object.

\begin{Shaded}
\begin{Highlighting}[]
\KeywordTok{str}\NormalTok{(lm.}\DecValTok{1}\NormalTok{)}
\end{Highlighting}
\end{Shaded}

\begin{verbatim}
## List of 12
##  $ coefficients : Named num [1:2] 6.854 -0.393
##   ..- attr(*, "names")= chr [1:2] "(Intercept)" "Temp"
##  $ residuals    : Named num [1:26] 0.0316 -0.2291 -0.2965 0.1293 0.0866 ...
##   ..- attr(*, "names")= chr [1:26] "1" "2" "3" "4" ...
##  $ effects      : Named num [1:26] -24.2203 -5.6485 -0.2541 0.1463 0.0988 ...
##   ..- attr(*, "names")= chr [1:26] "(Intercept)" "Temp" "" "" ...
##  $ rank         : int 2
##  $ fitted.values: Named num [1:26] 7.17 7.13 6.7 5.87 5.71 ...
##   ..- attr(*, "names")= chr [1:26] "1" "2" "3" "4" ...
##  $ assign       : int [1:2] 0 1
##  $ qr           :List of 5
##   ..$ qr   : num [1:26, 1:2] -5.099 0.196 0.196 0.196 0.196 ...
##   .. ..- attr(*, "dimnames")=List of 2
##   .. .. ..$ : chr [1:26] "1" "2" "3" "4" ...
##   .. .. ..$ : chr [1:2] "(Intercept)" "Temp"
##   .. ..- attr(*, "assign")= int [1:2] 0 1
##   ..$ qraux: num [1:2] 1.2 1.35
##   ..$ pivot: int [1:2] 1 2
##   ..$ tol  : num 1e-07
##   ..$ rank : int 2
##   ..- attr(*, "class")= chr "qr"
##  $ df.residual  : int 24
##  $ xlevels      : Named list()
##  $ call         : language lm(formula = Gas ~ Temp, data = whiteside[Insul == "Before"])
##  $ terms        :Classes 'terms', 'formula'  language Gas ~ Temp
##   .. ..- attr(*, "variables")= language list(Gas, Temp)
##   .. ..- attr(*, "factors")= int [1:2, 1] 0 1
##   .. .. ..- attr(*, "dimnames")=List of 2
##   .. .. .. ..$ : chr [1:2] "Gas" "Temp"
##   .. .. .. ..$ : chr "Temp"
##   .. ..- attr(*, "term.labels")= chr "Temp"
##   .. ..- attr(*, "order")= int 1
##   .. ..- attr(*, "intercept")= int 1
##   .. ..- attr(*, "response")= int 1
##   .. ..- attr(*, ".Environment")=<environment: R_GlobalEnv> 
##   .. ..- attr(*, "predvars")= language list(Gas, Temp)
##   .. ..- attr(*, "dataClasses")= Named chr [1:2] "numeric" "numeric"
##   .. .. ..- attr(*, "names")= chr [1:2] "Gas" "Temp"
##  $ model        :'data.frame':   26 obs. of  2 variables:
##   ..$ Gas : num [1:26] 7.2 6.9 6.4 6 5.8 5.8 5.6 4.7 5.8 5.2 ...
##   ..$ Temp: num [1:26] -0.8 -0.7 0.4 2.5 2.9 3.2 3.6 3.9 4.2 4.3 ...
##   ..- attr(*, "terms")=Classes 'terms', 'formula'  language Gas ~ Temp
##   .. .. ..- attr(*, "variables")= language list(Gas, Temp)
##   .. .. ..- attr(*, "factors")= int [1:2, 1] 0 1
##   .. .. .. ..- attr(*, "dimnames")=List of 2
##   .. .. .. .. ..$ : chr [1:2] "Gas" "Temp"
##   .. .. .. .. ..$ : chr "Temp"
##   .. .. ..- attr(*, "term.labels")= chr "Temp"
##   .. .. ..- attr(*, "order")= int 1
##   .. .. ..- attr(*, "intercept")= int 1
##   .. .. ..- attr(*, "response")= int 1
##   .. .. ..- attr(*, ".Environment")=<environment: R_GlobalEnv> 
##   .. .. ..- attr(*, "predvars")= language list(Gas, Temp)
##   .. .. ..- attr(*, "dataClasses")= Named chr [1:2] "numeric" "numeric"
##   .. .. .. ..- attr(*, "names")= chr [1:2] "Gas" "Temp"
##  - attr(*, "class")= chr "lm"
\end{verbatim}

In RStudio it is particularly easy to extract objects. Just write
\texttt{your.object\$} and press \texttt{tab} after the \texttt{\$} for
auto-completion.

If we only want \(\hat \beta\), it can also be extracted with the
\texttt{coef} function.

\begin{Shaded}
\begin{Highlighting}[]
\KeywordTok{coef}\NormalTok{(lm.}\DecValTok{1}\NormalTok{)}
\end{Highlighting}
\end{Shaded}

\begin{verbatim}
## (Intercept)        Temp 
##   6.8538277  -0.3932388
\end{verbatim}

Things to note:

\begin{itemize}
\item
  R automatically adds an \texttt{(Intercept)} term. This means we
  estimate \(Gas=\beta_0 + \beta_1 Temp + \varepsilon\) and not
  \(Gas=\beta_1 Temp + \varepsilon\). This makes sense because we are
  interested in the contribution of the temperature to the variability
  of the gas consumption about its \textbf{mean}, and not about zero.
\item
  The effect of temperature, i.e., \(\hat \beta_1\), is -0.39. The
  negative sign means that the higher the temperature, the less gas is
  consumed. The magnitude of the coefficient means that for a unit
  increase in the outside temperature, the gas consumption decreases by
  0.39 units.
\end{itemize}

We can use the \texttt{predict} function to make predictions, but we
emphasize that if the purpose of the model is to make predictions, and
not interpret coefficients, better skip to the Supervised Learning
Chapter \ref{supervised}.

\begin{Shaded}
\begin{Highlighting}[]
\CommentTok{# Gas predictions (b0+b1*temperature) vs. actual Gas measurements, ideal slope should be 1.}
\KeywordTok{plot}\NormalTok{(}\KeywordTok{predict}\NormalTok{(lm.}\DecValTok{1}\NormalTok{)}\OperatorTok{~}\NormalTok{whiteside[Insul}\OperatorTok{==}\StringTok{'Before'}\NormalTok{,Gas])}
\CommentTok{# plots identity line (slope 1), lty=Line Type, 2 means dashed line.}
\KeywordTok{abline}\NormalTok{(}\DecValTok{0}\NormalTok{,}\DecValTok{1}\NormalTok{, }\DataTypeTok{lty=}\DecValTok{2}\NormalTok{)}
\end{Highlighting}
\end{Shaded}

\includegraphics[width=0.5\linewidth]{Rcourse_files/figure-latex/unnamed-chunk-157-1}

The model seems to fit the data nicely. A common measure of the goodness
of fit is the \emph{coefficient of determination}, more commonly known
as the \(R^2\).



\BeginKnitrBlock{definition}[R2]
\protect\hypertarget{def:unnamed-chunk-158}{}{\label{def:unnamed-chunk-158}
\iffalse (R2) \fi{} }The coefficient of determination, denoted \(R^2\),
is defined as

\begin{align}
  R^2:= 1-\frac{\sum_i (y_i - \hat y_i)^2}{\sum_i (y_i - \bar y)^2},
\end{align}

where \(\hat y_i\) is the model's prediction,
\(\hat y_i = x_i \hat \beta\).
\EndKnitrBlock{definition}

It can be easily computed

\begin{Shaded}
\begin{Highlighting}[]
\KeywordTok{library}\NormalTok{(magrittr)}
\NormalTok{R2 <-}\StringTok{ }\ControlFlowTok{function}\NormalTok{(y, y.hat)\{}
\NormalTok{  numerator <-}\StringTok{ }\NormalTok{(y}\OperatorTok{-}\NormalTok{y.hat)}\OperatorTok{^}\DecValTok{2} \OperatorTok\StringTok{ }\NormalTok{sum}
\NormalTok{  denominator <-}\StringTok{ }\NormalTok{(y}\OperatorTok{-}\KeywordTok{mean}\NormalTok{(y))}\OperatorTok{^}\DecValTok{2} \OperatorTok\StringTok{ }\NormalTok{sum}
  \DecValTok{1}\OperatorTok{-}\NormalTok{numerator}\OperatorTok{/}\NormalTok{denominator}
\NormalTok{\}}
\KeywordTok{R2}\NormalTok{(}\DataTypeTok{y=}\NormalTok{whiteside[Insul}\OperatorTok{==}\StringTok{'Before'}\NormalTok{,Gas], }\DataTypeTok{y.hat=}\KeywordTok{predict}\NormalTok{(lm.}\DecValTok{1}\NormalTok{))}
\end{Highlighting}
\end{Shaded}

\begin{verbatim}
## [1] 0.9438081
\end{verbatim}

This is a nice result implying that about \(94\%\) of the variability in
gas consumption can be attributed to changes in the outside temperature.

Obviously, R does provide the means to compute something as basic as
\(R^2\), but I will let you find it for yourselves.

\section{Inference}\label{inference}

To perform inference on \(\hat \beta\), in order to test hypotheses and
construct confidence intervals, we need to quantify the uncertainly in
the reported \(\hat \beta\). This is exactly what
Eq.\eqref{eq:ols-distribution} gives us.

Luckily, we don't need to manipulate multivariate distributions
manually, and everything we need is already implemented. The most
important function is \texttt{summary} which gives us an overview of the
model's fit. We emphasize that fitting a model with \texttt{lm} is an
assumption free algorithmic step. Inference using \texttt{summary} is
\textbf{not} assumption free, and requires the set of assumptions
leading to Eq.\eqref{eq:ols-distribution}.

\begin{Shaded}
\begin{Highlighting}[]
\KeywordTok{summary}\NormalTok{(lm.}\DecValTok{1}\NormalTok{)}
\end{Highlighting}
\end{Shaded}

\begin{verbatim}
## 
## Call:
## lm(formula = Gas ~ Temp, data = whiteside[Insul == "Before"])
## 
## Residuals:
##      Min       1Q   Median       3Q      Max 
## -0.62020 -0.19947  0.06068  0.16770  0.59778 
## 
## Coefficients:
##             Estimate Std. Error t value Pr(>|t|)    
## (Intercept)  6.85383    0.11842   57.88   <2e-16 ***
## Temp        -0.39324    0.01959  -20.08   <2e-16 ***
## ---
## Signif. codes:  0 '***' 0.001 '**' 0.01 '*' 0.05 '.' 0.1 ' ' 1
## 
## Residual standard error: 0.2813 on 24 degrees of freedom
## Multiple R-squared:  0.9438, Adjusted R-squared:  0.9415 
## F-statistic: 403.1 on 1 and 24 DF,  p-value: < 2.2e-16
\end{verbatim}

Things to note:

\begin{itemize}
\tightlist
\item
  The estimated \(\hat \beta\) is reported in the `Coefficients' table,
  which has point estimates, standard errors, t-statistics, and the
  p-values of a two-sided hypothesis test for each coefficient
  \(H_{0,j}:\beta_j=0, j=1,\dots,p\).
\item
  The \(R^2\) is reported at the bottom. The ``Adjusted R-squared'' is a
  variation that compensates for the model's complexity.
\item
  The original call to \texttt{lm} is saved in the \texttt{Call}
  section.
\item
  Some summary statistics of the residuals (\(y_i-\hat y_i\)) in the
  \texttt{Residuals} section.
\item
  The ``residuals standard error''\footnote{Sometimes known as the Root
    Mean Squared Error (RMSE).} is
  \(\sqrt{(n-p)^{-1} \sum_i (y_i-\hat y_i)^2}\). The denominator of this
  expression is the \emph{degrees of freedom}, \(n-p\), which can be
  thought of as the hardness of the problem.
\end{itemize}

As the name suggests, \texttt{summary} is merely a summary. The full
\texttt{summary(lm.1)} object is a monstrous object. Its various
elements can be queried using \texttt{str(sumary(lm.1))}.

Can we check the assumptions required for inference? Some. Let's start
with the linearity assumption. If we were wrong, and the data is not
arranged about a linear line, the residuals will have some shape. We
thus plot the residuals as a function of the predictor to diagnose
shape.

\begin{Shaded}
\begin{Highlighting}[]
\CommentTok{# errors (epsilons) vs. temperature, should oscillate around zero.}
\KeywordTok{plot}\NormalTok{(}\KeywordTok{residuals}\NormalTok{(lm.}\DecValTok{1}\NormalTok{)}\OperatorTok{~}\NormalTok{whiteside[Insul}\OperatorTok{==}\StringTok{'Before'}\NormalTok{,Temp])}
\KeywordTok{abline}\NormalTok{(}\DecValTok{0}\NormalTok{,}\DecValTok{0}\NormalTok{, }\DataTypeTok{lty=}\DecValTok{2}\NormalTok{)}
\end{Highlighting}
\end{Shaded}

\includegraphics[width=0.5\linewidth]{Rcourse_files/figure-latex/unnamed-chunk-161-1}

I can't say I see any shape. Let's fit a \textbf{wrong} model, just to
see what ``shape'' means.

\begin{Shaded}
\begin{Highlighting}[]
\NormalTok{lm.}\FloatTok{1.1}\NormalTok{ <-}\StringTok{ }\KeywordTok{lm}\NormalTok{(Gas}\OperatorTok{~}\KeywordTok{I}\NormalTok{(Temp}\OperatorTok{^}\DecValTok{2}\NormalTok{), }\DataTypeTok{data=}\NormalTok{whiteside[Insul}\OperatorTok{==}\StringTok{'Before'}\NormalTok{,])}
\KeywordTok{plot}\NormalTok{(}\KeywordTok{residuals}\NormalTok{(lm.}\FloatTok{1.1}\NormalTok{)}\OperatorTok{~}\NormalTok{whiteside[Insul}\OperatorTok{==}\StringTok{'Before'}\NormalTok{,Temp]); }\KeywordTok{abline}\NormalTok{(}\DecValTok{0}\NormalTok{,}\DecValTok{0}\NormalTok{, }\DataTypeTok{lty=}\DecValTok{2}\NormalTok{)}
\end{Highlighting}
\end{Shaded}

\includegraphics[width=0.5\linewidth]{Rcourse_files/figure-latex/unnamed-chunk-162-1}

Things to note:

\begin{itemize}
\tightlist
\item
  We used \texttt{I(Temp\^{}2)} to specify the model
  \(Gas=\beta_0 + \beta_1 Temp^2+ \varepsilon\).
\item
  The residuals have a ``belly''. Because they are not a cloud around
  the linear trend, and we have the wrong model.
\end{itemize}

To the next assumption. We assumed \(\varepsilon_i\) are independent of
everything else. The residuals, \(y_i-\hat y_i\) can be thought of a
sample of \(\varepsilon_i\). When diagnosing the linearity assumption,
we already saw their distribution does not vary with the \(x\)'s,
\texttt{Temp} in our case. They may be correlated with themselves; a
positive departure from the model, may be followed by a series of
positive departures etc. Diagnosing these \emph{auto-correlations} is a
real art, which is not part of our course.

The last assumption we required is normality. As previously stated, if
\(n \gg p\), this assumption can be relaxed. If \(n\) is in the order of
\(p\), we need to verify this assumption. My favorite tool for this task
is the \emph{qqplot}. A qqplot compares the quantiles of the sample with
the respective quantiles of the assumed distribution. If quantiles align
along a line, the assumed distribution is OK. If quantiles depart from a
line, then the assumed distribution does not fit the sample.

\begin{Shaded}
\begin{Highlighting}[]
\KeywordTok{qqnorm}\NormalTok{(}\KeywordTok{resid}\NormalTok{(lm.}\DecValTok{1}\NormalTok{))}
\end{Highlighting}
\end{Shaded}

\includegraphics[width=0.5\linewidth]{Rcourse_files/figure-latex/unnamed-chunk-163-1}

Things to note:

\begin{itemize}
\tightlist
\item
  The \texttt{qqnorm} function plots a qqplot against a normal
  distribution. For non-normal distributions try \texttt{qqplot}.
\item
  \texttt{resid(lm.1)} extracts the residuals from the linear model,
  i.e., the vector of \(y_i-x_i'\hat \beta\).
\end{itemize}

Judging from the figure, the normality assumption is quite plausible.
Let's try the same on a non-normal sample, namely a uniformly
distributed sample, to see how that would look.

\begin{Shaded}
\begin{Highlighting}[]
\KeywordTok{qqnorm}\NormalTok{(}\KeywordTok{runif}\NormalTok{(}\DecValTok{100}\NormalTok{))}
\end{Highlighting}
\end{Shaded}

\includegraphics[width=0.5\linewidth]{Rcourse_files/figure-latex/unnamed-chunk-164-1}

\subsection{Testing a Hypothesis on a Single
Coefficient}\label{testing-a-hypothesis-on-a-single-coefficient}

The first inferential test we consider is a hypothesis test on a single
coefficient. In our gas example, we may want to test that the
temperature has no effect on the gas consumption. The answer for that is
given immediately by \texttt{summary(lm.1)}

\begin{Shaded}
\begin{Highlighting}[]
\NormalTok{summary.lm1 <-}\StringTok{ }\KeywordTok{summary}\NormalTok{(lm.}\DecValTok{1}\NormalTok{)}
\NormalTok{coefs.lm1 <-}\StringTok{ }\NormalTok{summary.lm1}\OperatorTok{$}\NormalTok{coefficients}
\NormalTok{coefs.lm1}
\end{Highlighting}
\end{Shaded}

\begin{verbatim}
##               Estimate Std. Error   t value     Pr(>|t|)
## (Intercept)  6.8538277 0.11842341  57.87561 2.717533e-27
## Temp        -0.3932388 0.01958601 -20.07754 1.640469e-16
\end{verbatim}

We see that the p-value for \(H_{0,1}: \beta_1=0\) against a two sided
alternative is effectively 0 (row 2 column 4), so that \(\beta_1\) is
unlikely to be \(0\) (the null hypothesis can be rejected).

\subsection{Constructing a Confidence Interval on a Single
Coefficient}\label{constructing-a-confidence-interval-on-a-single-coefficient}

Since the \texttt{summary} function gives us the standard errors of
\(\hat \beta\), we can immediately compute
\(\hat \beta_j \pm 2 \sqrt{Var[\hat \beta_j]}\) to get ourselves a
(roughly) \(95\%\) confidence interval. In our example the interval is

\begin{Shaded}
\begin{Highlighting}[]
\NormalTok{coefs.lm1[}\DecValTok{2}\NormalTok{,}\DecValTok{1}\NormalTok{] }\OperatorTok{+}\StringTok{ }\KeywordTok{c}\NormalTok{(}\OperatorTok{-}\DecValTok{2}\NormalTok{,}\DecValTok{2}\NormalTok{) }\OperatorTok{*}\StringTok{ }\NormalTok{coefs.lm1[}\DecValTok{2}\NormalTok{,}\DecValTok{2}\NormalTok{]}
\end{Highlighting}
\end{Shaded}

\begin{verbatim}
## [1] -0.4324108 -0.3540668
\end{verbatim}

Things to note:

\begin{itemize}
\tightlist
\item
  The function \texttt{confint(lm.1)} can calculate it. Sometimes it's
  more simple to write 20 characters of code than finding a function
  that does it for us.
\end{itemize}

\subsection{Multiple Regression}\label{multiple-regression}

\BeginKnitrBlock{remark}
\iffalse{} {Remark. } \fi{}\emph{Multiple regression} is not to be
confused with \emph{multivariate regression} discussed in Chapter
\ref{multivariate}.
\EndKnitrBlock{remark}

The \texttt{swiss} dataset encodes the fertility at each of
Switzerland's 47 French speaking provinces, along other socio-economic
indicators. Let's see if these are statistically related:

\begin{Shaded}
\begin{Highlighting}[]
\KeywordTok{head}\NormalTok{(swiss)}
\end{Highlighting}
\end{Shaded}

\begin{verbatim}
##              Fertility Agriculture Examination Education Catholic
## Courtelary        80.2        17.0          15        12     9.96
## Delemont          83.1        45.1           6         9    84.84
## Franches-Mnt      92.5        39.7           5         5    93.40
## Moutier           85.8        36.5          12         7    33.77
## Neuveville        76.9        43.5          17        15     5.16
## Porrentruy        76.1        35.3           9         7    90.57
##              Infant.Mortality
## Courtelary               22.2
## Delemont                 22.2
## Franches-Mnt             20.2
## Moutier                  20.3
## Neuveville               20.6
## Porrentruy               26.6
\end{verbatim}

\begin{Shaded}
\begin{Highlighting}[]
\NormalTok{lm.}\DecValTok{5}\NormalTok{ <-}\StringTok{ }\KeywordTok{lm}\NormalTok{(}\DataTypeTok{data=}\NormalTok{swiss, Fertility}\OperatorTok{~}\NormalTok{Agriculture}\OperatorTok{+}\NormalTok{Examination}\OperatorTok{+}\NormalTok{Education}\OperatorTok{+}\NormalTok{Catholic}\OperatorTok{+}\NormalTok{Infant.Mortality)}
\KeywordTok{summary}\NormalTok{(lm.}\DecValTok{5}\NormalTok{)}
\end{Highlighting}
\end{Shaded}

\begin{verbatim}
## 
## Call:
## lm(formula = Fertility ~ Agriculture + Examination + Education + 
##     Catholic + Infant.Mortality, data = swiss)
## 
## Residuals:
##      Min       1Q   Median       3Q      Max 
## -15.2743  -5.2617   0.5032   4.1198  15.3213 
## 
## Coefficients:
##                  Estimate Std. Error t value Pr(>|t|)    
## (Intercept)      66.91518   10.70604   6.250 1.91e-07 ***
## Agriculture      -0.17211    0.07030  -2.448  0.01873 *  
## Examination      -0.25801    0.25388  -1.016  0.31546    
## Education        -0.87094    0.18303  -4.758 2.43e-05 ***
## Catholic          0.10412    0.03526   2.953  0.00519 ** 
## Infant.Mortality  1.07705    0.38172   2.822  0.00734 ** 
## ---
## Signif. codes:  0 '***' 0.001 '**' 0.01 '*' 0.05 '.' 0.1 ' ' 1
## 
## Residual standard error: 7.165 on 41 degrees of freedom
## Multiple R-squared:  0.7067, Adjusted R-squared:  0.671 
## F-statistic: 19.76 on 5 and 41 DF,  p-value: 5.594e-10
\end{verbatim}

Things to note:

\begin{itemize}
\tightlist
\item
  The \texttt{\textasciitilde{}} syntax allows to specify various
  predictors separated by the \texttt{+} operator.
\item
  The summary of the model now reports the estimated effect, i.e., the
  regression coefficient, of each of the variables.
\end{itemize}

Clearly, naming each variable explicitly is a tedious task if there are
many. The use of \texttt{Fertility\textasciitilde{}.} in the next
example reads: ``Fertility as a function of all other variables in the
\texttt{swiss} data.frame''.

\begin{Shaded}
\begin{Highlighting}[]
\NormalTok{lm.}\DecValTok{5}\NormalTok{ <-}\StringTok{ }\KeywordTok{lm}\NormalTok{(}\DataTypeTok{data=}\NormalTok{swiss, Fertility}\OperatorTok{~}\NormalTok{.)}
\KeywordTok{summary}\NormalTok{(lm.}\DecValTok{5}\NormalTok{)}
\end{Highlighting}
\end{Shaded}

\begin{verbatim}
## 
## Call:
## lm(formula = Fertility ~ ., data = swiss)
## 
## Residuals:
##      Min       1Q   Median       3Q      Max 
## -15.2743  -5.2617   0.5032   4.1198  15.3213 
## 
## Coefficients:
##                  Estimate Std. Error t value Pr(>|t|)    
## (Intercept)      66.91518   10.70604   6.250 1.91e-07 ***
## Agriculture      -0.17211    0.07030  -2.448  0.01873 *  
## Examination      -0.25801    0.25388  -1.016  0.31546    
## Education        -0.87094    0.18303  -4.758 2.43e-05 ***
## Catholic          0.10412    0.03526   2.953  0.00519 ** 
## Infant.Mortality  1.07705    0.38172   2.822  0.00734 ** 
## ---
## Signif. codes:  0 '***' 0.001 '**' 0.01 '*' 0.05 '.' 0.1 ' ' 1
## 
## Residual standard error: 7.165 on 41 degrees of freedom
## Multiple R-squared:  0.7067, Adjusted R-squared:  0.671 
## F-statistic: 19.76 on 5 and 41 DF,  p-value: 5.594e-10
\end{verbatim}

\subsection{ANOVA (*)}\label{anova}

Our next example\footnote{The example is taken from
  \url{http://rtutorialseries.blogspot.co.il/2011/02/r-tutorial-series-two-way-anova-with.html}}
contains a hypothetical sample of \(60\) participants who are divided
into three stress reduction treatment groups (mental, physical, and
medical) and three age groups groups. The stress reduction values are
represented on a scale that ranges from 1 to 10. The values represent
how effective the treatment programs were at reducing participant's
stress levels, with larger effects indicating higher effectiveness.

\begin{Shaded}
\begin{Highlighting}[]
\NormalTok{twoWay <-}\StringTok{ }\KeywordTok{read.csv}\NormalTok{(}\StringTok{'data/dataset_anova_twoWay_comparisons.csv'}\NormalTok{)}
\KeywordTok{head}\NormalTok{(twoWay)}
\end{Highlighting}
\end{Shaded}

\begin{verbatim}
##   Treatment   Age StressReduction
## 1    mental young              10
## 2    mental young               9
## 3    mental young               8
## 4    mental   mid               7
## 5    mental   mid               6
## 6    mental   mid               5
\end{verbatim}

How many observations per group?

\begin{Shaded}
\begin{Highlighting}[]
\KeywordTok{table}\NormalTok{(twoWay}\OperatorTok{$}\NormalTok{Treatment, twoWay}\OperatorTok{$}\NormalTok{Age)}
\end{Highlighting}
\end{Shaded}

\begin{verbatim}
##           
##            mid old young
##   medical    3   3     3
##   mental     3   3     3
##   physical   3   3     3
\end{verbatim}

Since we have two factorial predictors, this multiple regression is
nothing but a \emph{two way ANOVA}. Let's fit the model and inspect it.

\begin{Shaded}
\begin{Highlighting}[]
\NormalTok{lm.}\DecValTok{2}\NormalTok{ <-}\StringTok{ }\KeywordTok{lm}\NormalTok{(StressReduction}\OperatorTok{~}\NormalTok{.,}\DataTypeTok{data=}\NormalTok{twoWay)}
\KeywordTok{summary}\NormalTok{(lm.}\DecValTok{2}\NormalTok{)}
\end{Highlighting}
\end{Shaded}

\begin{verbatim}
## 
## Call:
## lm(formula = StressReduction ~ ., data = twoWay)
## 
## Residuals:
##    Min     1Q Median     3Q    Max 
##     -1     -1      0      1      1 
## 
## Coefficients:
##                   Estimate Std. Error t value Pr(>|t|)    
## (Intercept)         4.0000     0.3892  10.276 7.34e-10 ***
## Treatmentmental     2.0000     0.4264   4.690 0.000112 ***
## Treatmentphysical   1.0000     0.4264   2.345 0.028444 *  
## Ageold             -3.0000     0.4264  -7.036 4.65e-07 ***
## Ageyoung            3.0000     0.4264   7.036 4.65e-07 ***
## ---
## Signif. codes:  0 '***' 0.001 '**' 0.01 '*' 0.05 '.' 0.1 ' ' 1
## 
## Residual standard error: 0.9045 on 22 degrees of freedom
## Multiple R-squared:  0.9091, Adjusted R-squared:  0.8926 
## F-statistic:    55 on 4 and 22 DF,  p-value: 3.855e-11
\end{verbatim}

Things to note:

\begin{itemize}
\item
  The \texttt{StressReduction\textasciitilde{}.} syntax is read as
  ``Stress reduction as a function of everything else''.
\item
  All the (main) effects and the intercept seem to be significant.
\item
  Mid age and medical treatment are missing, hence it is implied that
  they are the baseline, and this model accounts for the departure from
  this baseline.
\item
  The data has 2 factors, but the coefficients table has 4 predictors.
  This is because \texttt{lm} noticed that \texttt{Treatment} and
  \texttt{Age} are factors. Each level of each factor is thus encoded as
  a different (dummy) variable. The numerical values of the factors are
  meaningless. Instead, R has constructed a dummy variable for each
  level of each factor. The names of the effect are a concatenation of
  the factor's name, and its level. You can inspect these dummy
  variables with the \texttt{model.matrix} command.
\end{itemize}

\begin{Shaded}
\begin{Highlighting}[]
\KeywordTok{model.matrix}\NormalTok{(lm.}\DecValTok{2}\NormalTok{) }\OperatorTok\StringTok{ }\NormalTok{lattice}\OperatorTok{::}\KeywordTok{levelplot}\NormalTok{()}
\end{Highlighting}
\end{Shaded}

\includegraphics[width=0.5\linewidth]{Rcourse_files/figure-latex/unnamed-chunk-173-1}
If you don't want the default dummy coding, look at \texttt{?contrasts}.

If you are more familiar with the ANOVA literature, or that you don't
want the effects of each level separately, but rather, the effect of
\textbf{all} the levels of each factor, use the \texttt{anova} command.

\begin{Shaded}
\begin{Highlighting}[]
\KeywordTok{anova}\NormalTok{(lm.}\DecValTok{2}\NormalTok{)}
\end{Highlighting}
\end{Shaded}

\begin{verbatim}
## Analysis of Variance Table
## 
## Response: StressReduction
##           Df Sum Sq Mean Sq F value    Pr(>F)    
## Treatment  2     18   9.000      11 0.0004883 ***
## Age        2    162  81.000      99     1e-11 ***
## Residuals 22     18   0.818                      
## ---
## Signif. codes:  0 '***' 0.001 '**' 0.01 '*' 0.05 '.' 0.1 ' ' 1
\end{verbatim}

Things to note:

\begin{itemize}
\tightlist
\item
  The ANOVA table, unlike the \texttt{summary} function, tests if
  \textbf{any} of the levels of a factor has an effect, and not one
  level at a time.
\item
  The significance of each factor is computed using an F-test.
\item
  The degrees of freedom, encoding the number of levels of a factor, is
  given in the \texttt{Df} column.
\item
  The StressReduction seems to vary for different ages and treatments,
  since both factors are significant.
\end{itemize}

If you are extremely more comfortable with the ANOVA literature, you
could have replaced the \texttt{lm} command with the \texttt{aov}
command all along.

\begin{Shaded}
\begin{Highlighting}[]
\NormalTok{lm.}\FloatTok{2.2}\NormalTok{ <-}\StringTok{ }\KeywordTok{aov}\NormalTok{(StressReduction}\OperatorTok{~}\NormalTok{.,}\DataTypeTok{data=}\NormalTok{twoWay)}
\KeywordTok{class}\NormalTok{(lm.}\FloatTok{2.2}\NormalTok{)}
\end{Highlighting}
\end{Shaded}

\begin{verbatim}
## [1] "aov" "lm"
\end{verbatim}

\begin{Shaded}
\begin{Highlighting}[]
\KeywordTok{summary}\NormalTok{(lm.}\FloatTok{2.2}\NormalTok{)}
\end{Highlighting}
\end{Shaded}

\begin{verbatim}
##             Df Sum Sq Mean Sq F value   Pr(>F)    
## Treatment    2     18    9.00      11 0.000488 ***
## Age          2    162   81.00      99    1e-11 ***
## Residuals   22     18    0.82                     
## ---
## Signif. codes:  0 '***' 0.001 '**' 0.01 '*' 0.05 '.' 0.1 ' ' 1
\end{verbatim}

Things to note:

\begin{itemize}
\tightlist
\item
  The \texttt{lm} function has been replaced with an \texttt{aov}
  function.
\item
  The output of \texttt{aov} is an \texttt{aov} class object, which
  extends the \texttt{lm} class.
\item
  The summary of an \texttt{aov} is not like the summary of an
  \texttt{lm} object, but rather, like an ANOVA table.
\end{itemize}

As in any two-way ANOVA, we may want to ask if different age groups
respond differently to different treatments. In the statistical
parlance, this is called an \emph{interaction}, or more precisely, an
\emph{interaction of order 2}.

\begin{Shaded}
\begin{Highlighting}[]
\NormalTok{lm.}\DecValTok{3}\NormalTok{ <-}\StringTok{ }\KeywordTok{lm}\NormalTok{(StressReduction}\OperatorTok{~}\NormalTok{Treatment}\OperatorTok{+}\NormalTok{Age}\OperatorTok{+}\NormalTok{Treatment}\OperatorTok{:}\NormalTok{Age}\OperatorTok{-}\DecValTok{1}\NormalTok{,}\DataTypeTok{data=}\NormalTok{twoWay)}
\end{Highlighting}
\end{Shaded}

The syntax
\texttt{StressReduction\textasciitilde{}Treatment+Age+Treatment:Age-1}
tells R to include main effects of Treatment, Age, and their
interactions. The -1 removes the intercept. Here are other ways to
specify the same model.

\begin{Shaded}
\begin{Highlighting}[]
\NormalTok{lm.}\DecValTok{3}\NormalTok{ <-}\StringTok{ }\KeywordTok{lm}\NormalTok{(StressReduction }\OperatorTok{~}\StringTok{ }\NormalTok{Treatment }\OperatorTok{*}\StringTok{ }\NormalTok{Age }\OperatorTok{-}\StringTok{ }\DecValTok{1}\NormalTok{,}\DataTypeTok{data=}\NormalTok{twoWay)}
\NormalTok{lm.}\DecValTok{3}\NormalTok{ <-}\StringTok{ }\KeywordTok{lm}\NormalTok{(StressReduction}\OperatorTok{~}\NormalTok{(.)}\OperatorTok{^}\DecValTok{2} \OperatorTok{-}\StringTok{ }\DecValTok{1}\NormalTok{,}\DataTypeTok{data=}\NormalTok{twoWay)}
\end{Highlighting}
\end{Shaded}

The syntax \texttt{Treatment\ *\ Age} means ``main effects with second
order interactions''. The syntax \texttt{(.)\^{}2} means ``everything
with second order interactions'', this time we don't have I() as in the
temperature example because here we want the second order interaction
and not the square of each variable.

Let's inspect the model

\begin{Shaded}
\begin{Highlighting}[]
\KeywordTok{summary}\NormalTok{(lm.}\DecValTok{3}\NormalTok{)}
\end{Highlighting}
\end{Shaded}

\begin{verbatim}
## 
## Call:
## lm(formula = StressReduction ~ Treatment + Age + Treatment:Age - 
##     1, data = twoWay)
## 
## Residuals:
##    Min     1Q Median     3Q    Max 
##     -1     -1      0      1      1 
## 
## Coefficients:
##                              Estimate Std. Error t value Pr(>|t|)    
## Treatmentmedical            4.000e+00  5.774e-01   6.928 1.78e-06 ***
## Treatmentmental             6.000e+00  5.774e-01  10.392 4.92e-09 ***
## Treatmentphysical           5.000e+00  5.774e-01   8.660 7.78e-08 ***
## Ageold                     -3.000e+00  8.165e-01  -3.674  0.00174 ** 
## Ageyoung                    3.000e+00  8.165e-01   3.674  0.00174 ** 
## Treatmentmental:Ageold      4.246e-16  1.155e+00   0.000  1.00000    
## Treatmentphysical:Ageold    1.034e-15  1.155e+00   0.000  1.00000    
## Treatmentmental:Ageyoung   -3.126e-16  1.155e+00   0.000  1.00000    
## Treatmentphysical:Ageyoung  5.128e-16  1.155e+00   0.000  1.00000    
## ---
## Signif. codes:  0 '***' 0.001 '**' 0.01 '*' 0.05 '.' 0.1 ' ' 1
## 
## Residual standard error: 1 on 18 degrees of freedom
## Multiple R-squared:  0.9794, Adjusted R-squared:  0.9691 
## F-statistic:    95 on 9 and 18 DF,  p-value: 2.556e-13
\end{verbatim}

Things to note:

\begin{itemize}
\tightlist
\item
  There are still \(5\) main effects, but also \(4\) interactions. This
  is because when allowing a different average response for every
  \(Treatment*Age\) combination, we are effectively estimating \(3*3=9\)
  cell means, even if they are not parametrized as cell means, but
  rather as main effect and interactions.
\item
  The interactions do not seem to be significant.
\item
  The assumptions required for inference are clearly not met in this
  example, which is there just to demonstrate R's capabilities.
\end{itemize}

Asking if all the interactions are significant, is asking if the
different age groups have the same response to different treatments. Can
we answer that based on the various interactions? We might, but it is
possible that no single interaction is significant, while the
combination is. To test for all the interactions together, we can simply
check if the model without interactions is (significantly) better than a
model with interactions. I.e., compare \texttt{lm.2} to \texttt{lm.3}.
This is done with the \texttt{anova} command.

\begin{Shaded}
\begin{Highlighting}[]
\KeywordTok{anova}\NormalTok{(lm.}\DecValTok{2}\NormalTok{,lm.}\DecValTok{3}\NormalTok{, }\DataTypeTok{test=}\StringTok{'F'}\NormalTok{)}
\end{Highlighting}
\end{Shaded}

\begin{verbatim}
## Analysis of Variance Table
## 
## Model 1: StressReduction ~ Treatment + Age
## Model 2: StressReduction ~ Treatment + Age + Treatment:Age - 1
##   Res.Df RSS Df   Sum of Sq F Pr(>F)
## 1     22  18                        
## 2     18  18  4 -3.5527e-15
\end{verbatim}

We see that \texttt{lm.3} is \textbf{not} (significantly) better than
\texttt{lm.2}, so that we can conclude that there are no interactions:
different ages have the same response to different treatments.

\subsection{Testing a Hypothesis on a Single Contrast
(*)}\label{testing-a-hypothesis-on-a-single-contrast}

Returning to the model without interactions, \texttt{lm.2}.

\begin{Shaded}
\begin{Highlighting}[]
\KeywordTok{coef}\NormalTok{(}\KeywordTok{summary}\NormalTok{(lm.}\DecValTok{2}\NormalTok{))}
\end{Highlighting}
\end{Shaded}

\begin{verbatim}
##                   Estimate Std. Error   t value     Pr(>|t|)
## (Intercept)              4  0.3892495 10.276186 7.336391e-10
## Treatmentmental          2  0.4264014  4.690416 1.117774e-04
## Treatmentphysical        1  0.4264014  2.345208 2.844400e-02
## Ageold                  -3  0.4264014 -7.035624 4.647299e-07
## Ageyoung                 3  0.4264014  7.035624 4.647299e-07
\end{verbatim}

We see that the effect of the various treatments is rather similar. It
is possible that all treatments actually have the same effect. Comparing
the effects of factor levels is called a \emph{contrast}. Let's test if
the medical treatment, has in fact, the same effect as the physical
treatment.

\begin{Shaded}
\begin{Highlighting}[]
\KeywordTok{library}\NormalTok{(multcomp)}
\NormalTok{my.contrast <-}\StringTok{ }\KeywordTok{matrix}\NormalTok{(}\KeywordTok{c}\NormalTok{(}\OperatorTok{-}\DecValTok{1}\NormalTok{,}\DecValTok{0}\NormalTok{,}\DecValTok{1}\NormalTok{,}\DecValTok{0}\NormalTok{,}\DecValTok{0}\NormalTok{), }\DataTypeTok{nrow =}  \DecValTok{1}\NormalTok{)}
\NormalTok{lm.}\DecValTok{4}\NormalTok{ <-}\StringTok{ }\KeywordTok{glht}\NormalTok{(lm.}\DecValTok{2}\NormalTok{, }\DataTypeTok{linfct=}\NormalTok{my.contrast)}
\KeywordTok{summary}\NormalTok{(lm.}\DecValTok{4}\NormalTok{)}
\end{Highlighting}
\end{Shaded}

\begin{verbatim}
## 
##   Simultaneous Tests for General Linear Hypotheses
## 
## Fit: lm(formula = StressReduction ~ ., data = twoWay)
## 
## Linear Hypotheses:
##        Estimate Std. Error t value Pr(>|t|)    
## 1 == 0  -3.0000     0.7177   -4.18 0.000389 ***
## ---
## Signif. codes:  0 '***' 0.001 '**' 0.01 '*' 0.05 '.' 0.1 ' ' 1
## (Adjusted p values reported -- single-step method)
\end{verbatim}

Things to note:

\begin{itemize}
\tightlist
\item
  A contrast is a linear function of the coefficients. In our example
  \(H_0:\beta_1-\beta_3=0\), which justifies the construction of
  \texttt{my.contrast}.
\item
  We used the \texttt{glht} function (generalized linear hypothesis
  test) from the package \textbf{multcomp}.
\item
  The contrast is significant, i.e., the effect of a medical treatment,
  is different than that of a physical treatment.
\end{itemize}

\section{Bibliographic Notes}\label{bibliographic-notes-4}

Like any other topic in this book, you can consult
\citet{venables2013modern} for more on linear models. For the theory of
linear models, I like \citet{greene2003econometric}.

\section{Practice Yourself}\label{practice-yourself-3}

\begin{enumerate}
\def\labelenumi{\arabic{enumi}.}
\item
  Inspect women's heights and weights with \texttt{?women}.

  \begin{enumerate}
  \def\labelenumii{\arabic{enumii}.}
  \tightlist
  \item
    What is the change in weight per unit change in height? Use the
    \texttt{lm} function.
  \item
    Is the relation of height on weight significant? Use
    \texttt{summary}.
  \item
    Plot the residuals of the linear model with \texttt{plot} and
    \texttt{resid}.
  \item
    Plot the predictions of the model using \texttt{abline}.
  \item
    Inspect the normality of residuals using \texttt{qqnorm}.
  \item
    Inspect the design matrix using \texttt{model.matrix}.
  \end{enumerate}
\item
  Write a function that takes an \texttt{lm} class object, and returns
  the confidence interval on the first coefficient. Apply it on the
  height and weight data.
\item
  Use the \texttt{ANOVA} function to test the significance of the effect
  of height.
\item
  Read about the ``mtcars'' dataset using \texttt{?\ mtcars}. Inspect
  the dependency of the fuel consumption (mpg) in the weight (wt) and
  the 1/4 mile time (qsec).

  \begin{enumerate}
  \def\labelenumii{\arabic{enumii}.}
  \tightlist
  \item
    Make a pairs scatter plot with
    \texttt{plot(mtcars{[},c("mpg","wt","qsec"){]})} Does the connection
    look linear?
  \item
    Fit a multiple linear regression with \texttt{lm}. Call it
    \texttt{model1}.
  \item
    Try to add the transmission (am) as independent variable. Let R know
    this is a categorical variable with \texttt{factor(am)}. Call it
    \texttt{model2}.
  \item
    Compare the ``Adjusted R-squared'' measure of the two models (we
    can't use the regular R2 to compare two models with a different
    number of variables).
  \item
    Do the coefficients significant?
  \item
    Inspect the normality of residuals and the linearity assumptions.
  \item
    Now Inspect the hypothesis that the effect of weight is different
    between the transmission types with adding interaction to the model
    \texttt{wt*factor(am)}.
  \item
    According to this model, what is the addition of one unit of weight
    in a manual transmission to the fuel consumption
    (-2.973-4.141=-7.11)?
  \end{enumerate}
\end{enumerate}

\chapter{Generalized Linear Models}\label{glm}

\BeginKnitrBlock{example}
\protect\hypertarget{exm:cigarettes}{}{\label{exm:cigarettes} }Consider the
relation between cigarettes smoked, and the occurance of lung cancer. Do
we expect the probability of cancer to be linear in the number of
cigarettes? Probably not. Do we expect the variability of events to be
constant about the trend? Probably not.
\EndKnitrBlock{example}

\BeginKnitrBlock{example}
\protect\hypertarget{exm:cars}{}{\label{exm:cars} }Consider the relation
between the travel times to the distance travelled. Do you agree that
the longer the distance travelled, then not only the travel times get
longer, but they also get more variable?
\EndKnitrBlock{example}

\section{Problem Setup}\label{problem-setup-1}

In the Linear Models Chapter \ref{lm}, we assumed the generative process
to be linear in the effects of the predictors \(x\). We now write that
same linear model, slightly differently: \[
 y|x \sim \mathcal{N}(x'\beta, \sigma^2).
\]

This model not allow for the non-linear relations of Example
\ref{exm:cigarettes}, nor does it allow for the distribution of
\(\varepsilon\) to change with \(x\), as in Example \ref{exm:cars}.
\emph{Generalize linear models} (GLM), as the name suggests, are a
generalization of the linear models in Chapter \ref{lm} that allow
that\footnote{Do not confuse \emph{generalized linear models} with
  \href{https://en.wikipedia.org/wiki/Nonlinear_regression}{\emph{non-linear
  regression}}, or
  \href{https://en.wikipedia.org/wiki/Generalized_least_squares}{\emph{generalized
  least squares}}. These are different things, that we do not discuss.}.

For Example \ref{exm:cigarettes}, we would like something in the lines
of \[
 y|x \sim Binom(1,p(x))
\]

For Example \ref{exm:cars}, we would like something in the lines of \[
 y|x \sim \mathcal{N}(x'\beta,\sigma^2(x)),
\] or more generally \[
 y|x \sim \mathcal{N}(\mu(x),\sigma^2(x)),
\] or maybe not Gaussian \[
 y|x \sim Pois(\lambda(x)).
\]

Even more generally, for some distribution \(F(\theta)\), with a
parameter \(\theta\), we would like to assume that the data is generated
via

\begin{align}
  \label{eq:general}
  y|x \sim F(\theta(x))
\end{align}

Possible examples include

\begin{align}
 y|x &\sim Poisson(\lambda(x)) \\
 y|x &\sim Exp(\lambda(x)) \\
 y|x &\sim \mathcal{N}(\mu(x),\sigma^2(x)) 
\end{align}

GLMs allow models of the type of Eq.\eqref{eq:general}, while imposing
some constraints on \(F\) and on the relation \(\theta(x)\). GLMs assume
the data distribution \(F\) to be in a ``well-behaved'' family known as
the
\href{https://en.wikipedia.org/wiki/Natural_exponential_family}{\emph{Natural
Exponential Family}} of distributions. This family includes the
Gaussian, Gamma, Binomial, Poisson, and Negative Binomial distributions.
These five include as special cases the exponential, chi-squared,
Rayleigh, Weibull, Bernoulli, and geometric distributions.

GLMs also assume that the distribution's parameter, \(\theta\), is some
simple function of a linear combination of the effects. In our
cigarettes example this amounts to assuming that each cigarette has an
additive effect, but not on the probability of cancer, but rather, on
some simple function of it. Formally \[g(\theta(x))=x'\beta,\] and we
recall that \[x'\beta=\beta_0 + \sum_j x_j \beta_j.\] The function \(g\)
is called the \emph{link} function, its inverse, \(g^{-1}\) is the
\emph{mean function}. We thus say that ``the effects of each cigarette
is linear \textbf{in link scale}''. This terminology will later be
required to understand R's output.

\section{Logistic Regression}\label{logistic-regression}

The best known of the GLM class of models is the \emph{logistic
regression} that deals with Binomial, or more precisely,
Bernoulli-distributed data. The link function in the logistic regression
is the \emph{logit function}

\begin{align}
  g(t)=log\left( \frac{t}{(1-t)} \right)
  \label{eq:logistic-link}  
\end{align}

implying that under the logistic model assumptions

\begin{align}
  y|x \sim Binom \left( 1, p=\frac{e^{x'\beta}}{1+e^{x'\beta}} \right).
  \label{eq:logistic}
\end{align}

Before we fit such a model, we try to justify this construction, in
particular, the enigmatic link function in Eq.\eqref{eq:logistic-link}.
Let's look at the simplest possible case: the comparison of two groups
indexed by \(x\): \(x=0\) for the first, and \(x=1\) for the second. We
start with some definitions.

\BeginKnitrBlock{definition}[Odds]
\protect\hypertarget{def:unnamed-chunk-182}{}{\label{def:unnamed-chunk-182}
\iffalse (Odds) \fi{} }The \emph{odds}, of a binary random variable,
\(y\), is defined as \[\frac{P(y=1)}{P(y=0)}.\]
\EndKnitrBlock{definition}

Odds are the same as probabilities, but instead of telling me there is a
\(66\%\) of success, they tell me the odds of success are ``2 to 1''. If
you ever placed a bet, the language of ``odds'' should not be unfamiliar
to you.

\BeginKnitrBlock{definition}[Odds Ratio]
\protect\hypertarget{def:unnamed-chunk-183}{}{\label{def:unnamed-chunk-183}
\iffalse (Odds Ratio) \fi{} }The \emph{odds ratio} between two binary
random variables, \(y_1\) and \(y_2\), is defined as the ratio between
their odds. Formally:
\[OR(y_1,y_2):=\frac{P(y_1=1)/P(y_1=0)}{P(y_2=1)/P(y_2=0)}.\]
\EndKnitrBlock{definition}

Odds ratios (OR) compare between the probabilities of two groups, only
that it does not compare them in probability scale, but rather in odds
scale. You can also think of ORs as a measure of distance between two
Brenoulli distributions. ORs have better mathematical properties than
other candidate distance measures, such as \(P(y_1=1)-P(y_2=1)\).

Under the logit link assumption formalized in Eq.\eqref{eq:logistic}, the
OR between two conditions indexed by \(y|x=1\) and \(y|x=0\), returns:

\begin{align}
   OR(y|x=1,y|x=0) 
   = \frac{P(y=1|x=1)/P(y=0|x=1)}{P(y=1|x=0)/P(y=0|x=0)} 
   = e^{\beta_1}.  
\end{align}

The last equality demystifies the choice of the link function in the
logistic regression: \textbf{it allows us to interpret \(\beta\) of the
logistic regression as a measure of change of binary random variables,
namely, as the (log) odds-ratios due to a unit increase in \(x\)}.

\BeginKnitrBlock{remark}
\iffalse{} {Remark. } \fi{}Another popular link function is the normal
quantile function, a.k.a., the Gaussian inverse CDF, leading to
\emph{probit regression} instead of logistic regression.
\EndKnitrBlock{remark}

\subsection{Logistic Regression with
R}\label{logistic-regression-with-r}

Let's get us some data. The \texttt{PlantGrowth} data records the weight
of plants under three conditions: control, treatment1, and treatment2.

\begin{Shaded}
\begin{Highlighting}[]
\KeywordTok{head}\NormalTok{(PlantGrowth)}
\end{Highlighting}
\end{Shaded}

\begin{verbatim}
##   weight group
## 1   4.17  ctrl
## 2   5.58  ctrl
## 3   5.18  ctrl
## 4   6.11  ctrl
## 5   4.50  ctrl
## 6   4.61  ctrl
\end{verbatim}

We will now \texttt{attach} the data so that its contents is available
in the workspace (don't forget to \texttt{detach} afterwards, or you can
expect some conflicting object names). We will also use the \texttt{cut}
function to create a binary response variable for Light, and Heavy
plants (we are doing logistic regression, so we need a two-class
response), notice also that \texttt{cut} splits according to range and
not to length. As a general rule of thumb, when we discretize continuous
variables, we lose information. For pedagogical reasons, however, we
will proceed with this bad practice.

Look at the following output and think: how many \texttt{group} effects
do we expect? What should be the sign of each effect?

\begin{Shaded}
\begin{Highlighting}[]
\KeywordTok{attach}\NormalTok{(PlantGrowth)}
\NormalTok{weight.factor<-}\StringTok{ }\KeywordTok{cut}\NormalTok{(weight, }\DecValTok{2}\NormalTok{, }\DataTypeTok{labels=}\KeywordTok{c}\NormalTok{(}\StringTok{'Light'}\NormalTok{, }\StringTok{'Heavy'}\NormalTok{)) }\CommentTok{# binarize weights}
\KeywordTok{plot}\NormalTok{(}\KeywordTok{table}\NormalTok{(group, weight.factor))}
\end{Highlighting}
\end{Shaded}

\includegraphics[width=0.5\linewidth]{Rcourse_files/figure-latex/unnamed-chunk-186-1}

Let's fit a logistic regression, and inspect the output.

\begin{Shaded}
\begin{Highlighting}[]
\NormalTok{glm.}\DecValTok{1}\NormalTok{<-}\StringTok{ }\KeywordTok{glm}\NormalTok{(weight.factor}\OperatorTok{~}\NormalTok{group, }\DataTypeTok{family=}\NormalTok{binomial)}
\KeywordTok{summary}\NormalTok{(glm.}\DecValTok{1}\NormalTok{)}
\end{Highlighting}
\end{Shaded}

\begin{verbatim}
## 
## Call:
## glm(formula = weight.factor ~ group, family = binomial)
## 
## Deviance Residuals: 
##     Min       1Q   Median       3Q      Max  
## -2.1460  -0.6681   0.4590   0.8728   1.7941  
## 
## Coefficients:
##             Estimate Std. Error z value Pr(>|z|)  
## (Intercept)   0.4055     0.6455   0.628   0.5299  
## grouptrt1    -1.7918     1.0206  -1.756   0.0792 .
## grouptrt2     1.7918     1.2360   1.450   0.1471  
## ---
## Signif. codes:  0 '***' 0.001 '**' 0.01 '*' 0.05 '.' 0.1 ' ' 1
## 
## (Dispersion parameter for binomial family taken to be 1)
## 
##     Null deviance: 41.054  on 29  degrees of freedom
## Residual deviance: 29.970  on 27  degrees of freedom
## AIC: 35.97
## 
## Number of Fisher Scoring iterations: 4
\end{verbatim}

Things to note:

\begin{itemize}
\tightlist
\item
  The \texttt{glm} function is our workhorse for all GLM models.
\item
  The \texttt{family} argument of \texttt{glm} tells R the respose
  variable is brenoulli, thus, performing a logistic regression.
\item
  The \texttt{summary} function is content aware. It gives a different
  output for \texttt{glm} class objects than for other objects, such as
  the \texttt{lm} we saw in Chapter \ref{lm}. In fact, what
  \texttt{summary} does is merely call \texttt{summary.glm}.
\item
  As usual, we get the coefficients table, but recall that they are to
  be interpreted as (log) odd-ratios, i.e., in ``link scale''. To return
  to probabilities (``response scale''), we will need to undo the
  logistic transformation.
\item
  As usual, we get the significance for the test of no-effect, versus a
  two-sided alternative. P-values are asymptotic, thus, only approximate
  (and can be very bad approximations in small samples).
\item
  The residuals of \texttt{glm} are slightly different than the
  \texttt{lm} residuals, and called \emph{Deviance Residuals}.
\item
  For help see \texttt{?glm}, \texttt{?family}, and
  \texttt{?summary.glm}.
\end{itemize}

Like in the linear models, we can use an ANOVA table to check if
treatments have any effect, and not one treatment at a time. In the case
of GLMs, this is called an \emph{analysis of deviance} table.

\begin{Shaded}
\begin{Highlighting}[]
\KeywordTok{anova}\NormalTok{(glm.}\DecValTok{1}\NormalTok{, }\DataTypeTok{test=}\StringTok{'LRT'}\NormalTok{)}
\end{Highlighting}
\end{Shaded}

\begin{verbatim}
## Analysis of Deviance Table
## 
## Model: binomial, link: logit
## 
## Response: weight.factor
## 
## Terms added sequentially (first to last)
## 
## 
##       Df Deviance Resid. Df Resid. Dev Pr(>Chi)   
## NULL                     29     41.054            
## group  2   11.084        27     29.970 0.003919 **
## ---
## Signif. codes:  0 '***' 0.001 '**' 0.01 '*' 0.05 '.' 0.1 ' ' 1
\end{verbatim}

Things to note:

\begin{itemize}
\tightlist
\item
  The \texttt{anova} function, like the \texttt{summary} function, are
  content-aware and produce a different output for the \texttt{glm}
  class than for the \texttt{lm} class. All that \texttt{anova} does is
  call \texttt{anova.glm}.
\item
  In GLMs there is no canonical test (like the F test for \texttt{lm}).
  \texttt{LRT} implies we want an approximate Likelihood Ratio Test. We
  thus specify the type of test desired with the \texttt{test} argument.
\item
  The distribution of the weights of the plants does vary with the
  treatment given, as we may see from the significance of the
  \texttt{group} factor.
\item
  Readers familiar with ANOVA tables, should know that we computed the
  GLM equivalent of a type I sum- of-squares. Run
  \texttt{drop1(glm.1,\ test=\textquotesingle{}Chisq\textquotesingle{})}
  for a GLM equivalent of a type III sum-of-squares.
\item
  For help see \texttt{?anova.glm}.
\end{itemize}

Let's predict the probability of a heavy plant for each treatment.

\begin{Shaded}
\begin{Highlighting}[]
\KeywordTok{predict}\NormalTok{(glm.}\DecValTok{1}\NormalTok{, }\DataTypeTok{type=}\StringTok{'response'}\NormalTok{)}
\end{Highlighting}
\end{Shaded}

\begin{verbatim}
##   1   2   3   4   5   6   7   8   9  10  11  12  13  14  15  16  17  18 
## 0.6 0.6 0.6 0.6 0.6 0.6 0.6 0.6 0.6 0.6 0.2 0.2 0.2 0.2 0.2 0.2 0.2 0.2 
##  19  20  21  22  23  24  25  26  27  28  29  30 
## 0.2 0.2 0.9 0.9 0.9 0.9 0.9 0.9 0.9 0.9 0.9 0.9
\end{verbatim}

Things to note:

\begin{itemize}
\tightlist
\item
  Like the \texttt{summary} and \texttt{anova} functions, the
  \texttt{predict} function is aware that its input is of \texttt{glm}
  class. All that \texttt{predict} does is call \texttt{predict.glm}.
\item
  In GLMs there are many types of predictions. The \texttt{type}
  argument controls which type is returned. Use \texttt{type=response}
  for predictions in probability scale; use `type=link' for predictions
  in log-odds scale.
\item
  How do I know we are predicting the probability of a heavy plant, and
  not a light plant? Just run \texttt{contrasts(weight.factor)} to see
  which of the categories of the factor \texttt{weight.factor} is
  encoded as 1, and which as 0.
\item
  For help see \texttt{?predict.glm}.
\end{itemize}

Let's detach the data so it is no longer in our workspace, and object
names do not collide.

\begin{Shaded}
\begin{Highlighting}[]
\KeywordTok{detach}\NormalTok{(PlantGrowth)}
\end{Highlighting}
\end{Shaded}

We gave an example with a factorial (i.e.~discrete) predictor. We can do
the same with multiple continuous predictors.

\begin{Shaded}
\begin{Highlighting}[]
\KeywordTok{data}\NormalTok{(}\StringTok{'Pima.te'}\NormalTok{, }\DataTypeTok{package=}\StringTok{'MASS'}\NormalTok{) }\CommentTok{# Loads data}
\KeywordTok{head}\NormalTok{(Pima.te)}
\end{Highlighting}
\end{Shaded}

\begin{verbatim}
##   npreg glu bp skin  bmi   ped age type
## 1     6 148 72   35 33.6 0.627  50  Yes
## 2     1  85 66   29 26.6 0.351  31   No
## 3     1  89 66   23 28.1 0.167  21   No
## 4     3  78 50   32 31.0 0.248  26  Yes
## 5     2 197 70   45 30.5 0.158  53  Yes
## 6     5 166 72   19 25.8 0.587  51  Yes
\end{verbatim}

\begin{Shaded}
\begin{Highlighting}[]
\NormalTok{glm.}\DecValTok{2}\NormalTok{<-}\StringTok{ }\KeywordTok{step}\NormalTok{(}\KeywordTok{glm}\NormalTok{(type}\OperatorTok{~}\NormalTok{., }\DataTypeTok{data=}\NormalTok{Pima.te, }\DataTypeTok{family=}\KeywordTok{binomial}\NormalTok{(}\DataTypeTok{link=}\StringTok{'probit'}\NormalTok{)))}
\end{Highlighting}
\end{Shaded}

\begin{verbatim}
## Start:  AIC=302.41
## type ~ npreg + glu + bp + skin + bmi + ped + age
## 
##         Df Deviance    AIC
## - bp     1   286.92 300.92
## - skin   1   286.94 300.94
## - age    1   287.74 301.74
## <none>       286.41 302.41
## - ped    1   291.06 305.06
## - npreg  1   292.55 306.55
## - bmi    1   294.52 308.52
## - glu    1   342.35 356.35
## 
## Step:  AIC=300.92
## type ~ npreg + glu + skin + bmi + ped + age
## 
##         Df Deviance    AIC
## - skin   1   287.50 299.50
## - age    1   287.92 299.92
## <none>       286.92 300.92
## - ped    1   291.70 303.70
## - npreg  1   293.06 305.06
## - bmi    1   294.55 306.55
## - glu    1   342.41 354.41
## 
## Step:  AIC=299.5
## type ~ npreg + glu + bmi + ped + age
## 
##         Df Deviance    AIC
## - age    1   288.47 298.47
## <none>       287.50 299.50
## - ped    1   292.41 302.41
## - npreg  1   294.21 304.21
## - bmi    1   304.37 314.37
## - glu    1   343.48 353.48
## 
## Step:  AIC=298.47
## type ~ npreg + glu + bmi + ped
## 
##         Df Deviance    AIC
## <none>       288.47 298.47
## - ped    1   293.78 301.78
## - bmi    1   305.17 313.17
## - npreg  1   305.49 313.49
## - glu    1   349.25 357.25
\end{verbatim}

\begin{Shaded}
\begin{Highlighting}[]
\KeywordTok{summary}\NormalTok{(glm.}\DecValTok{2}\NormalTok{)}
\end{Highlighting}
\end{Shaded}

\begin{verbatim}
## 
## Call:
## glm(formula = type ~ npreg + glu + bmi + ped, family = binomial(link = "probit"), 
##     data = Pima.te)
## 
## Deviance Residuals: 
##     Min       1Q   Median       3Q      Max  
## -2.9935  -0.6487  -0.3585   0.6326   2.5791  
## 
## Coefficients:
##              Estimate Std. Error z value Pr(>|z|)    
## (Intercept) -5.445143   0.569373  -9.563  < 2e-16 ***
## npreg        0.102410   0.025607   3.999 6.35e-05 ***
## glu          0.021739   0.002988   7.276 3.45e-13 ***
## bmi          0.048709   0.012291   3.963 7.40e-05 ***
## ped          0.534366   0.250584   2.132    0.033 *  
## ---
## Signif. codes:  0 '***' 0.001 '**' 0.01 '*' 0.05 '.' 0.1 ' ' 1
## 
## (Dispersion parameter for binomial family taken to be 1)
## 
##     Null deviance: 420.30  on 331  degrees of freedom
## Residual deviance: 288.47  on 327  degrees of freedom
## AIC: 298.47
## 
## Number of Fisher Scoring iterations: 5
\end{verbatim}

Things to note:

\begin{itemize}
\tightlist
\item
  We used the \texttt{\textasciitilde{}.} syntax to tell R to fit a
  model with all the available predictors.
\item
  Since we want to focus on significant predictors, we used the
  \texttt{step} function to perform a \emph{step-wise} regression,
  i.e.~sequentially remove non-significant predictors. The function
  reports each model it has checked, and the variable it has decided to
  remove at each step.
\item
  The output of \texttt{step} is a single model, with the subset of
  selected predictors.
\end{itemize}

\section{Poisson Regression}\label{poisson-regression}

Poisson regression means we fit a model assuming
\(y|x \sim Poisson(\lambda(x))\). Put differently, we assume that for
each treatment, encoded as a combinations of predictors \(x\), the
response is Poisson distributed with a rate that depends on the
predictors.

The typical link function for Poisson regression is the logarithm:
\(g(t)=log(t)\). This means that we assume
\(y|x \sim Poisson(\lambda(x) = e^{x'\beta})\). Why is this a good
choice? We again resort to the two-group case, encoded by \(x=1\) and
\(x=0\), to understand this model:
\(\lambda(x=1)=e^{\beta_0+\beta_1}=e^{\beta_0} \; e^{\beta_1}= \lambda(x=0) \; e^{\beta_1}\).
We thus see that this link function implies that a change in \(x\)
\textbf{multiples} the rate of events by \(e^{\beta_1}\).

For our example\footnote{Taken from
  \url{http://www.theanalysisfactor.com/generalized-linear-models-in-r-part-6-poisson-regression-count-variables/}}
we inspect the number of infected high-school kids, as a function of the
days since an outbreak.

\begin{Shaded}
\begin{Highlighting}[]
\NormalTok{cases <-}\StringTok{  }
\KeywordTok{structure}\NormalTok{(}\KeywordTok{list}\NormalTok{(}\DataTypeTok{Days =} \KeywordTok{c}\NormalTok{(1L, 2L, 3L, 3L, 4L, 4L, 4L, 6L, 7L, 8L, }
\NormalTok{8L, 8L, 8L, 12L, 14L, 15L, 17L, 17L, 17L, 18L, 19L, 19L, 20L, }
\NormalTok{23L, 23L, 23L, 24L, 24L, 25L, 26L, 27L, 28L, 29L, 34L, 36L, 36L, }
\NormalTok{42L, 42L, 43L, 43L, 44L, 44L, 44L, 44L, 45L, 46L, 48L, 48L, 49L, }
\NormalTok{49L, 53L, 53L, 53L, 54L, 55L, 56L, 56L, 58L, 60L, 63L, 65L, 67L, }
\NormalTok{67L, 68L, 71L, 71L, 72L, 72L, 72L, 73L, 74L, 74L, 74L, 75L, 75L, }
\NormalTok{80L, 81L, 81L, 81L, 81L, 88L, 88L, 90L, 93L, 93L, 94L, 95L, 95L, }
\NormalTok{95L, 96L, 96L, 97L, 98L, 100L, 101L, 102L, 103L, 104L, 105L, }
\NormalTok{106L, 107L, 108L, 109L, 110L, 111L, 112L, 113L, 114L, 115L), }
    \DataTypeTok{Students =} \KeywordTok{c}\NormalTok{(6L, 8L, 12L, 9L, 3L, 3L, 11L, 5L, 7L, 3L, 8L, }
\NormalTok{    4L, 6L, 8L, 3L, 6L, 3L, 2L, 2L, 6L, 3L, 7L, 7L, 2L, 2L, 8L, }
\NormalTok{    3L, 6L, 5L, 7L, 6L, 4L, 4L, 3L, 3L, 5L, 3L, 3L, 3L, 5L, 3L, }
\NormalTok{    5L, 6L, 3L, 3L, 3L, 3L, 2L, 3L, 1L, 3L, 3L, 5L, 4L, 4L, 3L, }
\NormalTok{    5L, 4L, 3L, 5L, 3L, 4L, 2L, 3L, 3L, 1L, 3L, 2L, 5L, 4L, 3L, }
\NormalTok{    0L, 3L, 3L, 4L, 0L, 3L, 3L, 4L, 0L, 2L, 2L, 1L, 1L, 2L, 0L, }
\NormalTok{    2L, 1L, 1L, 0L, 0L, 1L, 1L, 2L, 2L, 1L, 1L, 1L, 1L, 0L, 0L, }
\NormalTok{    0L, 1L, 1L, 0L, 0L, 0L, 0L, 0L)), }\DataTypeTok{.Names =} \KeywordTok{c}\NormalTok{(}\StringTok{"Days"}\NormalTok{, }\StringTok{"Students"}
\NormalTok{), }\DataTypeTok{class =} \StringTok{"data.frame"}\NormalTok{, }\DataTypeTok{row.names =} \KeywordTok{c}\NormalTok{(}\OtherTok{NA}\NormalTok{, }\OperatorTok{-}\NormalTok{109L))}
\KeywordTok{attach}\NormalTok{(cases)}
\KeywordTok{head}\NormalTok{(cases) }
\end{Highlighting}
\end{Shaded}

\begin{verbatim}
##   Days Students
## 1    1        6
## 2    2        8
## 3    3       12
## 4    3        9
## 5    4        3
## 6    4        3
\end{verbatim}

Look at the following plot and think:

\begin{itemize}
\tightlist
\item
  Can we assume that the errors have constant variance?
\item
  What is the sign of the effect of time on the number of sick students?
\item
  Can we assume a linear effect of time?
\end{itemize}

\begin{Shaded}
\begin{Highlighting}[]
\KeywordTok{plot}\NormalTok{(Days, Students, }\DataTypeTok{xlab =} \StringTok{"DAYS"}\NormalTok{, }\DataTypeTok{ylab =} \StringTok{"STUDENTS"}\NormalTok{, }\DataTypeTok{pch =} \DecValTok{16}\NormalTok{)}
\end{Highlighting}
\end{Shaded}

\includegraphics[width=0.5\linewidth]{Rcourse_files/figure-latex/unnamed-chunk-193-1}

We now fit a model to check for the change in the rate of events as a
function of the days since the outbreak.

\begin{Shaded}
\begin{Highlighting}[]
\NormalTok{glm.}\DecValTok{3}\NormalTok{ <-}\StringTok{ }\KeywordTok{glm}\NormalTok{(Students }\OperatorTok{~}\StringTok{ }\NormalTok{Days, }\DataTypeTok{family =}\NormalTok{ poisson)}
\KeywordTok{summary}\NormalTok{(glm.}\DecValTok{3}\NormalTok{)}
\end{Highlighting}
\end{Shaded}

\begin{verbatim}
## 
## Call:
## glm(formula = Students ~ Days, family = poisson)
## 
## Deviance Residuals: 
##      Min        1Q    Median        3Q       Max  
## -2.00482  -0.85719  -0.09331   0.63969   1.73696  
## 
## Coefficients:
##              Estimate Std. Error z value Pr(>|z|)    
## (Intercept)  1.990235   0.083935   23.71   <2e-16 ***
## Days        -0.017463   0.001727  -10.11   <2e-16 ***
## ---
## Signif. codes:  0 '***' 0.001 '**' 0.01 '*' 0.05 '.' 0.1 ' ' 1
## 
## (Dispersion parameter for poisson family taken to be 1)
## 
##     Null deviance: 215.36  on 108  degrees of freedom
## Residual deviance: 101.17  on 107  degrees of freedom
## AIC: 393.11
## 
## Number of Fisher Scoring iterations: 5
\end{verbatim}

Things to note:

\begin{itemize}
\tightlist
\item
  We used \texttt{family=poisson} in the \texttt{glm} function to tell R
  that we assume a Poisson distribution.
\item
  The coefficients table is there as usual. When interpreting the table,
  we need to recall that the effect, i.e.~the \(\hat \beta\), are
  \textbf{multiplicative} due to the assumed link function.
\item
  Each day \textbf{decreases} the rate of events by a factor of about
  \(e^{\beta_1}=\) 0.98.
\item
  For more information see \texttt{?glm} and \texttt{?family}.
\end{itemize}

\section{Extensions}\label{extensions}

As we already implied, GLMs are a very wide class of models. We do not
need to use the default link function,but more importantly, we are not
constrained to Binomial, or Poisson distributed response. For
exponential, gamma, and other response distributions, see \texttt{?glm}
or the references in the Bibliographic Notes section.

\section{Bibliographic Notes}\label{bibliographic-notes-5}

The ultimate reference on GLMs is \citet{mccullagh1984generalized}. For
a less technical exposition, we refer to the usual
\citet{venables2013modern}.

\section{Practice Yourself}\label{practice-glm}

\begin{enumerate}
\def\labelenumi{\arabic{enumi}.}
\item
  Try using \texttt{lm} for analyzing the plant growth data in
  \texttt{weight.factor} as a function of \texttt{group} in the
  \texttt{PlantGrowth} data.
\item
  Generate some synthetic data for a logistic regression:

  \begin{enumerate}
  \def\labelenumii{\alph{enumii}.}
  \tightlist
  \item
    Generate two predictor variables of length \(100\). They can be
    random from your favorite distribution.
  \item
    Fix \texttt{beta\textless{}-\ c(-1,2)}, and generate the response
    with:\texttt{rbinom(n=100,size=1,prob=exp(x\ \%*\%\ beta)/(1+exp(x\ \%*\%\ beta)))}.
    Think: why is this the model implied by the logistic regression?
  \item
    Fit a Logistic regression to your synthetic data using \texttt{glm}.
  \item
    Are the estimated coefficients similar to the true ones you used?
  \item
    What is the estimated probability of an event at \texttt{x=1,1}? Use
    \texttt{predict.glm} but make sure to read the documentation on the
    \texttt{type} argument.
  \end{enumerate}
\item
  Read about the \texttt{epil} dataset using \texttt{?\ MASS::epil}.
  Inspect the dependency of the number of seizures (\(y\)) in the age of
  the patient (\texttt{age}) and the treatment (\texttt{trt}).

  \begin{enumerate}
  \def\labelenumii{\arabic{enumii}.}
  \tightlist
  \item
    Fit a Poisson regression with \texttt{glm} and
    \texttt{family\ =\ "poisson"}.
  \item
    Are the coefficients significant?\\
  \item
    Does the treatment reduce the frequency of the seizures?
  \item
    According to this model, what would be the number of seizures for 20
    years old patient with progabide treatment?
  \end{enumerate}
\end{enumerate}

See DataCamp's
\href{https://www.datacamp.com/courses/generalized-linear-models-in-r}{Generalized
Linear Models in R} for more self practice.

\chapter{Linear Mixed Models}\label{lme}

\BeginKnitrBlock{example}[Dependent Samples on the Mean]
\protect\hypertarget{exm:dependence}{}{\label{exm:dependence}
\iffalse (Dependent Samples on the Mean) \fi{} }Consider inference on a
population's mean. Supposdly, more observations imply more infotmation
on the mean. This, however, is not the case if samples are completely
dependant. More observations do not add any new information. From this
example one may think that dependence is a bad thing. This is a false
intuitiont: negative correlations imply oscilations about the mean, so
they are actually more informative on the mean than independent
observations.
\EndKnitrBlock{example}

\BeginKnitrBlock{example}[Repeated Measures]
\protect\hypertarget{exm:repeated-measures}{}{\label{exm:repeated-measures}
\iffalse (Repeated Measures) \fi{} }Consider a prospective study, i.e.,
data that originates from selecting a set of subjects and making
measurements on them over time. Also assume that some subjects received
some treatment, and other did not. When we want to infer on the
population from which these subjects have been sampled, we need to
recall that some series of observations came from the same subject. If
we were to ignore the subject of origin, and treat each observation as
an independent sample point, we will think we have more information in
our data than we actually do. For a rough intuition, think of a case
where observatiosn within subject are perfectly dependent.
\EndKnitrBlock{example}

The sources of variability, i.e.~noise, are known in the statistical
literature as ``random effects''. Specifying these sources determines
the correlation structure in our measurements. In the simplest linear
models of Chapter \ref{lm}, we thought of the variability as a
measurement error, independent of anything else. This, however, is
rarely the case when time or space are involved.

The variability in our data is rarely the object of interest. It is
merely the source of uncertainty in our measurements. The effects we
want to infer on are assumingly non-random, thus known as
``fixed-effects''. A model which has several sources of variability,
i.e.~random-effects, and several deterministic effects to study,
i.e.~fixed-effects, is known as a ``mixed effects'' model. If the model
is also linear, it is known as a \emph{linear mixed model} (LMM). Here
are some examples of such models.

\BeginKnitrBlock{example}[Fixed and Random Machine Effect]
\protect\hypertarget{exm:fixed-effects}{}{\label{exm:fixed-effects}
\iffalse (Fixed and Random Machine Effect) \fi{} }Consider the problem
of testing for a change in the distribution of diamteters of
manufactured bottle caps. We want to study the (fixed) effect of time:
before versus after. Bottle caps are produced by several machines.
Clearly there is variablity in the diameters within-machine and
between-machines. Given many measurements on many bottle caps from many
machines, we could standardize measurements by removing each machine's
average. This implies the within-machine variability is the only source
of variability we care about, because the substration of the machine
effect, removed information on the between-machine variability.\\
Alternatively, we could treat the between-machine variability as another
source of noise/uncertainty when inferring on the temporal fixed effect.
\EndKnitrBlock{example}

\BeginKnitrBlock{example}[Fixed and Random Subject Effect]
\protect\hypertarget{exm:random-effects}{}{\label{exm:random-effects}
\iffalse (Fixed and Random Subject Effect) \fi{} }Consider an
experimenal design where each subject is given 2 types of diets, and his
health condition is recorded. We could standardize over subjects by
removing the subject-wise average, before comparing diets. This is what
a paired t-test does. This also implies the within-subject variability
is the only source of variability we care about. Alternatively, for
inference on the population of ``all subjects'' we need to adress the
between-subject variability, and not only the within-subject
variability.
\EndKnitrBlock{example}

The unifying theme of the above examples, is that the variability in our
data has several sources. Which are the sources of variability that need
to concern us? This is a delicate matter which depends on your goals. As
a rule of thumb, we will suggest the following view: \textbf{If
information of an effect will be available at the time of prediction,
treat it as a fixed effect. If it is not, treat it as a random-effect.}

LMMs are so fundamental, that they have earned many names:

\begin{itemize}
\item
  \textbf{Mixed Effects}: Because we may have both \emph{fixed effects}
  we want to estimate and remove, and \emph{random effects} which
  contribute to the variability to infer against.
\item
  \textbf{Variance Components}: Because as the examples show, variance
  has more than a single source (like in the Linear Models of Chapter
  \ref{lm}).
\item
  \textbf{Hirarchial Models}: Because as Example
  \ref{exm:random-effects} demonstrates, we can think of the sampling as
  hierarchical-- first sample a subject, and then sample its response.
\item
  \textbf{Multilevel Analysis}: For the same reasons it is also known as
  Hierarchical Models.
\item
  \textbf{Repeated Measures}: Because we make several measurements from
  each unit, like in Example \ref{exm:random-effects}.
\item
  \textbf{Longitudinal Data}: Because we follow units over time, like in
  Example \ref{exm:random-effects}.
\item
  \textbf{Panel Data}: Is the term typically used in econometric for
  such longitudinal data.
\item
  \textbf{MANOVA}: Many of the problems that may be solved with a
  multivariate analysis of variance (MANOVA), may be solved with an LMM
  for reasons we detail in \ref{multivariate}.
\item
  \textbf{Structured Prediction}: In the machine learning literature,
  predicting outcomes with structure, such as correlated vectors, is
  known as Structured Learning. Because LMMs merely specify
  correlations, using a LMM for making predictions may be thought of as
  an instance of structured prediction.
\end{itemize}

Whether we are aiming to infer on a generative model's parameters, or to
make predictions, there is no ``right'' nor ``wrong'' approach. Instead,
there is always some implied measure of error, and an algorithm may be
good, or bad, with respect to this measure (think of false and true
positives, for instance). This is why we care about dependencies in the
data: ignoring the dependence structure will probably yield inefficient
algorithms. Put differently, if we ignore the statistical dependence in
the data we will probably me making more errors than possible/optimal.

We now emphasize:

\begin{enumerate}
\def\labelenumi{\arabic{enumi}.}
\item
  Like in previous chapters, by ``model'' we refer to the assumed
  generative distribution, i.e., the sampling distribution.
\item
  LMMs are a way to infer against the right level of variability. Using
  a naive linear model (which assumes a single source of variability)
  instead of a mixed effects model, probably means your inference is
  overly anti-conservative. Put differently, the uncertainty in your
  estimates is higher than the linear model from Chapter \ref{lm} may
  suggest.
\item
  In a LMM we will specify the dependence structure via the hierarchy in
  the sampling scheme (e.g.~caps within machine, students within class,
  etc.). Not all dependency models can be specified in this way.
  Dependency structures that are not hierarchical include temporal
  dependencies
  (\href{https://en.wikipedia.org/wiki/Autoregressive_model}{AR},
  \href{https://en.wikipedia.org/wiki/Autoregressive_integrated_moving_average}{ARIMA},
  \href{https://en.wikipedia.org/wiki/Autoregressive_conditional_heteroskedasticity}{ARCH}
  and GARCH),
  \href{https://en.wikipedia.org/wiki/Spatial_dependence}{spatial},
  \href{https://en.wikipedia.org/wiki/Markov_chain}{Markov Chains}, and
  more. To specify dependency structures that are no hierarchical, see
  Chapter 8 in (the excellent) \citet{weiss2005modeling}.
\item
  If you are using the model merely for predictions, and not for
  inference on the fixed effects or variance components, then stating
  the generative distribution may be be useful, but not necessarily. See
  the Supervised Learning Chapter \ref{supervised} for more on
  prediction problems. Also recall that machine learning from
  non-independent observations (such as LMMs) is a delicate matter that
  is rarely treated in the literature.
\end{enumerate}

\section{Problem Setup}\label{problem-setup-2}

\begin{align}
  y|x,u = x'\beta + z'u + \varepsilon
  \label{eq:mixed-model}  
\end{align}

where \(x\) are the factors with fixed effects, \(\beta\), which we may
want to study. The factors \(z\), with effects \(u\), are the random
effects which contribute to variability. In our repeated measures
example (\ref{exm:repeated-measures}) the treatment is a fixed effect,
and the subject is a random effect. In our bottle-caps example
(\ref{exm:fixed-effects}) the time (before vs.~after) is a fixed effect,
and the machines may be either a fixed or a random effect (depending on
the purpose of inference). In our diet example
(\ref{exm:random-effects}) the diet is the fixed effect and the family
is a random effect.

Notice that we state \(y|x,z\) merely as a convenient way to do
inference on \(y|x\), instead of directly specifying \(Var[y|x]\). This
is exactly the power of LMMs: we specify the covariance not via the
matrix \(Var[y,z]\), but rather via the sampling hierarchy.

Given a sample of \(n\) observations \((y_i,x_i,z_i)\) from model
\eqref{eq:mixed-model}, we will want to estimate \((\beta,u)\). Under some
assumption on the distribution of \(\varepsilon\) and \(z\), we can use
\emph{maximum likelihood} (ML). In the context of LMMs, however, ML is
typically replaced with \emph{restricted maximum likelihood} (ReML),
because it returns unbiased estimates of \(Var[y|x]\) and ML does not.

\subsection{Non-Linear Mixed Models}\label{non-linear-mixed-models}

The idea of random-effects can also be implemented for non-linear mean
models. Formally, this means that \(y|x,z=f(x,z,\varepsilon)\) for some
non-linear \(f\). This is known as \emph{non-linead-mixed-models}, which
will not be discussed in this text.

\subsection{Generalized Linear Mixed Models
(GLMM)}\label{generalized-linear-mixed-models-glmm}

You can marry the ideas of random effects, with non-linear link
functions, and non-Gaussian distribution of the response. These are
known as
\href{https://en.wikipedia.org/wiki/Generalized_linear_mixed_model}{Generalized
Linear Mixed Models}.
\href{http://glmm.wikidot.com/pkg-comparison}{Wikidot} has a nice
comparison of several software suits for GLMMs. Also consider the
\href{https://www.jstatsoft.org/article/view/v084i04}{mcglm} R pacakge
\citep{bonat2018multiple}.

\section{Mixed Models with R}\label{mixed-models-with-r}

We will fit mixed models with the \texttt{lmer} function from the
\textbf{lme4} package, written by the mixed-models Guru
\href{http://www.stat.wisc.edu/~bates/}{Douglas Bates}. We start with a
small simulation demonstrating the importance of acknowledging your
sources of variability. Our demonstration consists of fitting a linear
model that assumes independence, when data is clearly dependent.

\begin{Shaded}
\begin{Highlighting}[]
\CommentTok{# Simulation parameters}
\NormalTok{n.groups <-}\StringTok{ }\DecValTok{4} \CommentTok{# number of groups}
\NormalTok{n.repeats <-}\StringTok{ }\DecValTok{2} \CommentTok{# sample per group}
\NormalTok{groups <-}\StringTok{ }\KeywordTok{rep}\NormalTok{(}\DecValTok{1}\OperatorTok{:}\NormalTok{n.groups, }\DataTypeTok{each=}\NormalTok{n.repeats) }\OperatorTok\StringTok{ }\NormalTok{as.factor}
\NormalTok{n <-}\StringTok{ }\KeywordTok{length}\NormalTok{(groups)}
\NormalTok{z0 <-}\StringTok{ }\KeywordTok{rnorm}\NormalTok{(n.groups,}\DecValTok{0}\NormalTok{,}\DecValTok{10}\NormalTok{) }\CommentTok{# generate group effects}
\NormalTok{(z <-}\StringTok{ }\NormalTok{z0[}\KeywordTok{as.numeric}\NormalTok{(groups)]) }\CommentTok{# generate and inspect random group effects}
\end{Highlighting}
\end{Shaded}

\begin{verbatim}
## [1]   8.901364   8.901364  -4.318889  -4.318889   9.708611   9.708611
## [7] -10.693773 -10.693773
\end{verbatim}

\begin{Shaded}
\begin{Highlighting}[]
\NormalTok{epsilon <-}\StringTok{ }\KeywordTok{rnorm}\NormalTok{(n,}\DecValTok{0}\NormalTok{,}\DecValTok{1}\NormalTok{) }\CommentTok{# generate measurement error}

\CommentTok{# Generate data}
\NormalTok{beta0 <-}\StringTok{ }\DecValTok{2} \CommentTok{# set global mean}
\NormalTok{y <-}\StringTok{ }\NormalTok{beta0 }\OperatorTok{+}\StringTok{ }\NormalTok{z }\OperatorTok{+}\StringTok{ }\NormalTok{epsilon }\CommentTok{# generate synthetic sample}
\end{Highlighting}
\end{Shaded}

We can now fit the linear and mixed models.

\begin{Shaded}
\begin{Highlighting}[]
\NormalTok{lm.}\DecValTok{5}\NormalTok{ <-}\StringTok{ }\KeywordTok{lm}\NormalTok{(y}\OperatorTok{~}\DecValTok{1}\NormalTok{)  }\CommentTok{# fit a linear model assuming independence}
\KeywordTok{library}\NormalTok{(lme4)}
\NormalTok{lme.}\DecValTok{5}\NormalTok{ <-}\StringTok{ }\KeywordTok{lmer}\NormalTok{(y}\OperatorTok{~}\DecValTok{1}\OperatorTok{|}\NormalTok{groups) }\CommentTok{# fit a mixed-model that deals with the group dependence}
\end{Highlighting}
\end{Shaded}

The summary of the linear model

\begin{Shaded}
\begin{Highlighting}[]
\NormalTok{summary.lm.}\DecValTok{5}\NormalTok{ <-}\StringTok{ }\KeywordTok{summary}\NormalTok{(lm.}\DecValTok{5}\NormalTok{)}
\NormalTok{summary.lm.}\DecValTok{5}
\end{Highlighting}
\end{Shaded}

\begin{verbatim}
## 
## Call:
## lm(formula = y ~ 1)
## 
## Residuals:
##     Min      1Q  Median      3Q     Max 
## -13.949  -7.275   1.629   8.668  10.005 
## 
## Coefficients:
##             Estimate Std. Error t value Pr(>|t|)
## (Intercept)    3.317      3.500   0.948    0.375
## 
## Residual standard error: 9.898 on 7 degrees of freedom
\end{verbatim}

The summary of the mixed-model

\begin{Shaded}
\begin{Highlighting}[]
\NormalTok{summary.lme.}\DecValTok{5}\NormalTok{ <-}\StringTok{ }\KeywordTok{summary}\NormalTok{(lme.}\DecValTok{5}\NormalTok{)}
\NormalTok{summary.lme.}\DecValTok{5}
\end{Highlighting}
\end{Shaded}

\begin{verbatim}
## Linear mixed model fit by REML ['lmerMod']
## Formula: y ~ 1 | groups
## 
## REML criterion at convergence: 41
## 
## Scaled residuals: 
##      Min       1Q   Median       3Q      Max 
## -1.15395 -0.50048  0.04306  0.55891  0.99797 
## 
## Random effects:
##  Groups   Name        Variance Std.Dev.
##  groups   (Intercept) 111.962  10.581  
##  Residual               2.012   1.418  
## Number of obs: 8, groups:  groups, 4
## 
## Fixed effects:
##             Estimate Std. Error t value
## (Intercept)    3.317      5.314   0.624
\end{verbatim}

Look at the standard error of the global mean, i.e., the intercept: for
\texttt{lm} it is 3.4996374, and for \texttt{lme} it is 5.3143284. Why
this difference? Because \texttt{lm} treats the group effect\footnote{A.k.a.
  the \emph{cluster effect}.} as a fixed while the mixed model treats
the group effect as a source of noise/uncertainty. Clearly, inference
using \texttt{lm} underestimates our uncertainty in the estimated
population mean (\(\beta_0\)).

Now let's adopt the paired t-test view, which removes the group mean, so
that it implicitly ignores the between-group variability. Which is the
model compatible with this view?

\begin{Shaded}
\begin{Highlighting}[]
\NormalTok{diffs <-}\StringTok{ }\KeywordTok{tapply}\NormalTok{(y, groups, diff) }
\NormalTok{diffs }\CommentTok{# Q:what is this estimating? A: epsilon+epsilon.}
\end{Highlighting}
\end{Shaded}

\begin{verbatim}
##         1         2         3         4 
## -1.411024 -1.598983 -1.493730  3.052394
\end{verbatim}

\begin{Shaded}
\begin{Highlighting}[]
\KeywordTok{sd}\NormalTok{(diffs) }\CommentTok{# }
\end{Highlighting}
\end{Shaded}

\begin{verbatim}
## [1] 2.278119
\end{verbatim}

So we see that a paired t-test infers only against the within-group
variability. Q:Is this a good think? A: depends\ldots{}

\subsection{A Single Random Effect}\label{a-single-random-effect}

We will use the \texttt{Dyestuff} data from the \textbf{lme4} package,
which encodes the yield, in grams, of a coloring solution
(\texttt{dyestuff}), produced in 6 batches using 5 different
preparations.

\begin{Shaded}
\begin{Highlighting}[]
\KeywordTok{data}\NormalTok{(Dyestuff, }\DataTypeTok{package=}\StringTok{'lme4'}\NormalTok{)}
\KeywordTok{attach}\NormalTok{(Dyestuff)}
\KeywordTok{head}\NormalTok{(Dyestuff)}
\end{Highlighting}
\end{Shaded}

\begin{verbatim}
##   Batch Yield
## 1     A  1545
## 2     A  1440
## 3     A  1440
## 4     A  1520
## 5     A  1580
## 6     B  1540
\end{verbatim}

And visually

\begin{Shaded}
\begin{Highlighting}[]
\NormalTok{lattice}\OperatorTok{::}\KeywordTok{dotplot}\NormalTok{(Yield}\OperatorTok{~}\NormalTok{Batch)}
\end{Highlighting}
\end{Shaded}

\includegraphics[width=0.5\linewidth]{Rcourse_files/figure-latex/unnamed-chunk-199-1}

If we want to do inference on the (global) mean yield, we need to
account for the two sources of variability: the within-batch
variability, and the between-batch variability We thus fit a mixed
model, with an intercept and random batch effect.

\begin{Shaded}
\begin{Highlighting}[]
\NormalTok{lme.}\DecValTok{1}\NormalTok{<-}\StringTok{ }\KeywordTok{lmer}\NormalTok{( Yield }\OperatorTok{~}\StringTok{ }\DecValTok{1}  \OperatorTok{|}\StringTok{ }\NormalTok{Batch  , Dyestuff )}
\KeywordTok{summary}\NormalTok{(lme.}\DecValTok{1}\NormalTok{)}
\end{Highlighting}
\end{Shaded}

\begin{verbatim}
## Linear mixed model fit by REML ['lmerMod']
## Formula: Yield ~ 1 | Batch
##    Data: Dyestuff
## 
## REML criterion at convergence: 319.7
## 
## Scaled residuals: 
##     Min      1Q  Median      3Q     Max 
## -1.4117 -0.7634  0.1418  0.7792  1.8296 
## 
## Random effects:
##  Groups   Name        Variance Std.Dev.
##  Batch    (Intercept) 1764     42.00   
##  Residual             2451     49.51   
## Number of obs: 30, groups:  Batch, 6
## 
## Fixed effects:
##             Estimate Std. Error t value
## (Intercept)  1527.50      19.38    78.8
\end{verbatim}

Things to note:

\begin{itemize}
\tightlist
\item
  The syntax \texttt{Yield\ \textasciitilde{}\ 1\ \ \textbar{}\ Batch}
  tells R to fit a model with a global intercept (\texttt{1}) and a
  random Batch effect (\texttt{\textbar{}Batch}). More on that later.
\item
  As usual, \texttt{summary} is content aware and has a different
  behavior for \texttt{lme} class objects.
\item
  The output distinguishes between random effects (\(u\)), a source of
  variability, and fixed effect (\(\beta\)), which we want to study. The
  mean of the random effect is not reported because it is unassumingly
  0.
\item
  Were we not interested in the variance components, and only in the
  coefficients or predictions, an (almost) equivalent \texttt{lm}
  formulation is \texttt{lm(Yield\ \textasciitilde{}\ Batch)}.
\end{itemize}

Some utility functions let us query the \texttt{lme} object. The
function \texttt{coef} will work, but will return a cumbersome output.
Better use \texttt{fixef} to extract the fixed effects, and
\texttt{ranef} to extract the random effects. The model matrix (of the
fixed effects alone), can be extracted with \texttt{model.matrix}, and
predictions made with \texttt{predict}. Note, however, that predictions
with mixed-effect models are better treated as prediction problems as in
the Supervised Learning Chapter \ref{supervised}, but are a very
delicate matter.

\begin{Shaded}
\begin{Highlighting}[]
\KeywordTok{detach}\NormalTok{(Dyestuff)}
\end{Highlighting}
\end{Shaded}

\subsection{Multiple Random Effects}\label{multiple-random-effects}

Let's make things more interesting by allowing more than one random
effect. One-way ANOVA can be thought of as the fixed-effects counterpart
of the single random effect.

In the \texttt{Penicillin} data, we measured the diameter of spread of
an organism, along the plate used (a to x), and penicillin type (A to
F). We will now try to infer on the diameter of typical organism, and
compute its variability over plates and Penicillin types.

\begin{Shaded}
\begin{Highlighting}[]
\KeywordTok{head}\NormalTok{(Penicillin)}
\end{Highlighting}
\end{Shaded}

\begin{verbatim}
##   diameter plate sample
## 1       27     a      A
## 2       23     a      B
## 3       26     a      C
## 4       23     a      D
## 5       23     a      E
## 6       21     a      F
\end{verbatim}

One sample per combination:

\begin{Shaded}
\begin{Highlighting}[]
\KeywordTok{attach}\NormalTok{(Penicillin)}
\KeywordTok{table}\NormalTok{(sample, plate) }\CommentTok{# how many observations per plate & type?}
\end{Highlighting}
\end{Shaded}

\begin{verbatim}
##       plate
## sample a b c d e f g h i j k l m n o p q r s t u v w x
##      A 1 1 1 1 1 1 1 1 1 1 1 1 1 1 1 1 1 1 1 1 1 1 1 1
##      B 1 1 1 1 1 1 1 1 1 1 1 1 1 1 1 1 1 1 1 1 1 1 1 1
##      C 1 1 1 1 1 1 1 1 1 1 1 1 1 1 1 1 1 1 1 1 1 1 1 1
##      D 1 1 1 1 1 1 1 1 1 1 1 1 1 1 1 1 1 1 1 1 1 1 1 1
##      E 1 1 1 1 1 1 1 1 1 1 1 1 1 1 1 1 1 1 1 1 1 1 1 1
##      F 1 1 1 1 1 1 1 1 1 1 1 1 1 1 1 1 1 1 1 1 1 1 1 1
\end{verbatim}

And visually:

\includegraphics[width=0.5\linewidth]{Rcourse_files/figure-latex/unnamed-chunk-203-1}

Let's fit a mixed-effects model with a random plate effect, and a random
sample effect:

\begin{Shaded}
\begin{Highlighting}[]
\NormalTok{lme.}\DecValTok{2}\NormalTok{ <-}\StringTok{ }\KeywordTok{lmer}\NormalTok{ ( diameter }\OperatorTok{~}\StringTok{  }\DecValTok{1}  \OperatorTok{+}\StringTok{ }\NormalTok{(}\DecValTok{1}\OperatorTok{|}\NormalTok{plate )}\OperatorTok{+}\NormalTok{(}\DecValTok{1}\OperatorTok{|}\NormalTok{sample) , Penicillin )}
\KeywordTok{fixef}\NormalTok{(lme.}\DecValTok{2}\NormalTok{) }\CommentTok{# Fixed effects}
\end{Highlighting}
\end{Shaded}

\begin{verbatim}
## (Intercept) 
##    22.97222
\end{verbatim}

\begin{Shaded}
\begin{Highlighting}[]
\KeywordTok{ranef}\NormalTok{(lme.}\DecValTok{2}\NormalTok{) }\CommentTok{# Random effects}
\end{Highlighting}
\end{Shaded}

\begin{verbatim}
## $plate
##   (Intercept)
## a  0.80454389
## b  0.80454389
## c  0.18167120
## d  0.33738937
## e  0.02595303
## f -0.44120149
## g -1.37551052
## h  0.80454389
## i -0.75263783
## j -0.75263783
## k  0.96026206
## l  0.49310755
## m  1.42741658
## n  0.49310755
## o  0.96026206
## p  0.02595303
## q -0.28548332
## r -0.28548332
## s -1.37551052
## t  0.96026206
## u -0.90835601
## v -0.28548332
## w -0.59691966
## x -1.21979235
## 
## $sample
##   (Intercept)
## A  2.18705819
## B -1.01047625
## C  1.93789966
## D -0.09689498
## E -0.01384214
## F -3.00374447
## 
## with conditional variances for "plate" "sample"
\end{verbatim}

Things to note:

\begin{itemize}
\tightlist
\item
  The syntax
  \texttt{1+\ (1\textbar{}\ plate\ )\ +\ (1\textbar{}\ sample\ )} fits a
  global intercept (mean), a random plate effect, and a random sample
  effect.
\item
  Were we not interested in the variance components, an (almost)
  equivalent \texttt{lm} formulation is
  \texttt{lm(diameter\ \textasciitilde{}\ plate\ +\ sample)}.
\item
  The output of \texttt{ranef} is somewhat controversial. Think about
  it: Why would we want to plot the estimates of a random variable?
\end{itemize}

Since we have two random effects, we may compute the variability of the
global mean (the only fixed effect) as we did before. Perhaps more
interestingly, we can compute the variability in the response, for a
particular plate or sample type.

\begin{Shaded}
\begin{Highlighting}[]
\NormalTok{random.effect.lme2 <-}\StringTok{ }\KeywordTok{ranef}\NormalTok{(lme.}\DecValTok{2}\NormalTok{, }\DataTypeTok{condVar =} \OtherTok{TRUE}\NormalTok{) }
\NormalTok{qrr2 <-}\StringTok{ }\NormalTok{lattice}\OperatorTok{::}\KeywordTok{dotplot}\NormalTok{(random.effect.lme2, }\DataTypeTok{strip =} \OtherTok{FALSE}\NormalTok{)}
\end{Highlighting}
\end{Shaded}

Variability in response for each plate, over various sample types:

\begin{Shaded}
\begin{Highlighting}[]
\KeywordTok{print}\NormalTok{(qrr2[[}\DecValTok{1}\NormalTok{]]) }
\end{Highlighting}
\end{Shaded}

\includegraphics[width=0.5\linewidth]{Rcourse_files/figure-latex/unnamed-chunk-206-1}

Variability in response for each sample type, over the various plates:

\begin{Shaded}
\begin{Highlighting}[]
\KeywordTok{print}\NormalTok{(qrr2[[}\DecValTok{2}\NormalTok{]])  }
\end{Highlighting}
\end{Shaded}

\includegraphics[width=0.5\linewidth]{Rcourse_files/figure-latex/unnamed-chunk-207-1}

Things to note:

\begin{itemize}
\tightlist
\item
  The \texttt{condVar} argument of the \texttt{ranef} function tells R
  to compute the variability in response conditional on each random
  effect at a time.
\item
  The \texttt{dotplot} function, from the \textbf{lattice} package, is
  only there for the fancy plotting.
\end{itemize}

We used the penicillin example to demonstrate the incorporation of two
random-effects. We could have, however, compared between penicillin
types. For this matter, penicillin types are fixed effects to infer on,
and not part of the uncertainty in the mean diameter. The appropriate
model is the following:

\begin{Shaded}
\begin{Highlighting}[]
\NormalTok{lme.}\FloatTok{2.2}\NormalTok{ <-}\StringTok{ }\KeywordTok{lmer}\NormalTok{( diameter }\OperatorTok{~}\StringTok{  }\DecValTok{1}  \OperatorTok{+}\StringTok{ }\NormalTok{sample }\OperatorTok{+}\StringTok{ }\NormalTok{(}\DecValTok{1}\OperatorTok{|}\NormalTok{plate) , Penicillin )}
\end{Highlighting}
\end{Shaded}

I may now ask myself: does the \texttt{sample}, i.e.~penicillin, have
any effect? This is what the ANOVA table typically gives us. The next
table can be thought of as a ``repeated measures ANOVA'':

\begin{Shaded}
\begin{Highlighting}[]
\KeywordTok{anova}\NormalTok{(lme.}\FloatTok{2.2}\NormalTok{)}
\end{Highlighting}
\end{Shaded}

\begin{verbatim}
## Analysis of Variance Table
##        Df Sum Sq Mean Sq F value
## sample  5 449.22  89.844  297.09
\end{verbatim}

Ugh! No p-values. Why is this? Because Doug Bates, the author of
\textbf{lme4} makes a
\href{https://stat.ethz.ch/pipermail/r-help/2006-May/094765.html}{strong
argument} against current methods of computing p-values in mixed models.
If you insist on an p-value, you may recur to other packages that
provide that, at your own caution:

\begin{Shaded}
\begin{Highlighting}[]
\NormalTok{car}\OperatorTok{::}\KeywordTok{Anova}\NormalTok{(lme.}\FloatTok{2.2}\NormalTok{) }
\end{Highlighting}
\end{Shaded}

\begin{verbatim}
## Analysis of Deviance Table (Type II Wald chisquare tests)
## 
## Response: diameter
##         Chisq Df Pr(>Chisq)    
## sample 1485.5  5  < 2.2e-16 ***
## ---
## Signif. codes:  0 '***' 0.001 '**' 0.01 '*' 0.05 '.' 0.1 ' ' 1
\end{verbatim}

\ldots{} and yes; the penicillin type has a significant effect on the
diameter.

\subsection{A Full Mixed-Model}\label{a-full-mixed-model}

In the \texttt{sleepstudy} data, we recorded the reaction times to a
series of tests (\texttt{Reaction}), after various subject
(\texttt{Subject}) underwent various amounts of sleep deprivation
(\texttt{Day}).

\includegraphics[width=0.5\linewidth]{Rcourse_files/figure-latex/unnamed-chunk-211-1}

We now want to estimate the (fixed) effect of the days of sleep
deprivation on response time, while allowing each subject to have
his/hers own effect. Put differently, we want to estimate a \emph{random
slope} for the effect of \texttt{day}. The fixed \texttt{Days} effect
can be thought of as the average slope over subjects.

\begin{Shaded}
\begin{Highlighting}[]
\NormalTok{lme.}\DecValTok{3}\NormalTok{ <-}\StringTok{ }\KeywordTok{lmer}\NormalTok{ ( Reaction }\OperatorTok{~}\StringTok{ }\NormalTok{Days }\OperatorTok{+}\StringTok{ }\NormalTok{( Days }\OperatorTok{|}\StringTok{ }\NormalTok{Subject ) , }\DataTypeTok{data=}\NormalTok{ sleepstudy )}
\end{Highlighting}
\end{Shaded}

Things to note:

\begin{itemize}
\tightlist
\item
  \texttt{\textasciitilde{}Days} specifies the fixed effect.
\item
  We used the \texttt{Days\textbar{}Subect} syntax to tell R we want to
  fit the model \texttt{\textasciitilde{}Days} within each subject.
\item
  Were we fitting the model for purposes of prediction only, an (almost)
  equivalent \texttt{lm} formulation is
  \texttt{lm(Reaction\textasciitilde{}Days*Subject)}.
\end{itemize}

The fixed day effect is:

\begin{Shaded}
\begin{Highlighting}[]
\KeywordTok{fixef}\NormalTok{(lme.}\DecValTok{3}\NormalTok{)}
\end{Highlighting}
\end{Shaded}

\begin{verbatim}
## (Intercept)        Days 
##   251.40510    10.46729
\end{verbatim}

The variability in the average response (intercept) and day effect is

\begin{Shaded}
\begin{Highlighting}[]
\KeywordTok{ranef}\NormalTok{(lme.}\DecValTok{3}\NormalTok{)}
\end{Highlighting}
\end{Shaded}

\begin{verbatim}
## $Subject
##     (Intercept)        Days
## 308   2.2575329   9.1992737
## 309 -40.3942719  -8.6205161
## 310 -38.9563542  -5.4495796
## 330  23.6888704  -4.8141448
## 331  22.2585409  -3.0696766
## 332   9.0387625  -0.2720535
## 333  16.8389833  -0.2233978
## 334  -7.2320462   1.0745075
## 335  -0.3326901 -10.7524799
## 337  34.8865253   8.6290208
## 349 -25.2080191   1.1730997
## 350 -13.0694180   6.6142185
## 351   4.5777099  -3.0152825
## 352  20.8614523   3.5364062
## 369   3.2750882   0.8722876
## 370 -25.6110745   4.8222518
## 371   0.8070591  -0.9881730
## 372  12.3133491   1.2842380
## 
## with conditional variances for "Subject"
\end{verbatim}

Did we really need the whole \texttt{lme} machinery to fit a
within-subject linear regression and then average over subjects? The
answer is yes. The assumptions on the distribution of random effect,
namely, that they are normally distributed, allows us to pool
information from one subject to another. In the words of John Tukey:
``we borrow strength over subjects''. Is this a good thing? If the
normality assumption is true, it certainly is. If, on the other hand,
you have a lot of samples per subject, and you don't need to ``borrow
strength'' from one subject to another, you can simply fit
within-subject linear models without the mixed-models machinery.

To demonstrate the ``strength borrowing'', here is a comparison of the
lme, versus the effects of fitting a linear model to each subject
separately.

\includegraphics[width=0.5\linewidth]{Rcourse_files/figure-latex/unnamed-chunk-214-1}

Here is a comparison of the random-day effect from \texttt{lme} versus a
subject-wise linear model. They are not the same.

\includegraphics[width=0.5\linewidth]{Rcourse_files/figure-latex/unnamed-chunk-215-1}

\begin{Shaded}
\begin{Highlighting}[]
\KeywordTok{detach}\NormalTok{(Penicillin)}
\end{Highlighting}
\end{Shaded}

\section{Serial Correlations}\label{serial}

As previously stated, a hierarchical model is a very convenient way to
state correlations. The hierarchical sampling scheme will always yield
correlations in blocks. What is the correlation does not have a block
structure? Like a smooth temporal decay for time-series, or a smooth
spatial decay for geospatial data?

One way to go about, is to find a dedicated package. For instance, in
the
\href{https://cran.r-project.org/web/views/SpatioTemporal.html}{Spatio-Temporal
Data} task view, or the
\href{https://cran.r-project.org/web/views/Environmetrics.html}{Ecological
and Environmental} task view. Fans of vector-auto-regression should have
a look at the \href{https://cran.r-project.org/package=vars}{vars}
package.

Instead, we will show how to solve this matter using the \textbf{nlme}
package. This is because \textbf{nlme} allows to specify both a
block-covariance structure using the mixed-models framework, and the
smooth parametric covariances we find in temporal and spatial data.

The \texttt{nlme::Ovary} data is panel data of number of ovarian
follicles in different mares (female horse), at various times.

with an AR(1) temporal correlation, alongside random-effects, we take an
example from the help of \texttt{nlme::corAR1}.

\begin{Shaded}
\begin{Highlighting}[]
\KeywordTok{library}\NormalTok{(nlme)}
\KeywordTok{head}\NormalTok{(nlme}\OperatorTok{::}\NormalTok{Ovary)}
\end{Highlighting}
\end{Shaded}

\begin{verbatim}
## Grouped Data: follicles ~ Time | Mare
##   Mare        Time follicles
## 1    1 -0.13636360        20
## 2    1 -0.09090910        15
## 3    1 -0.04545455        19
## 4    1  0.00000000        16
## 5    1  0.04545455        13
## 6    1  0.09090910        10
\end{verbatim}

\begin{Shaded}
\begin{Highlighting}[]
\NormalTok{fm1Ovar.lme <-}\StringTok{ }\NormalTok{nlme}\OperatorTok{::}\KeywordTok{lme}\NormalTok{(}\DataTypeTok{fixed=}\NormalTok{follicles }\OperatorTok{~}\StringTok{ }\KeywordTok{sin}\NormalTok{(}\DecValTok{2}\OperatorTok{*}\NormalTok{pi}\OperatorTok{*}\NormalTok{Time) }\OperatorTok{+}\StringTok{ }\KeywordTok{cos}\NormalTok{(}\DecValTok{2}\OperatorTok{*}\NormalTok{pi}\OperatorTok{*}\NormalTok{Time), }
                   \DataTypeTok{data =}\NormalTok{ Ovary, }
                   \DataTypeTok{random =} \KeywordTok{pdDiag}\NormalTok{(}\OperatorTok{~}\KeywordTok{sin}\NormalTok{(}\DecValTok{2}\OperatorTok{*}\NormalTok{pi}\OperatorTok{*}\NormalTok{Time)), }
                   \DataTypeTok{correlation=}\KeywordTok{corAR1}\NormalTok{() )}
\KeywordTok{summary}\NormalTok{(fm1Ovar.lme)}
\end{Highlighting}
\end{Shaded}

\begin{verbatim}
## Linear mixed-effects model fit by REML
##  Data: Ovary 
##        AIC     BIC   logLik
##   1563.448 1589.49 -774.724
## 
## Random effects:
##  Formula: ~sin(2 * pi * Time) | Mare
##  Structure: Diagonal
##         (Intercept) sin(2 * pi * Time) Residual
## StdDev:    2.858385           1.257977 3.507053
## 
## Correlation Structure: AR(1)
##  Formula: ~1 | Mare 
##  Parameter estimate(s):
##       Phi 
## 0.5721866 
## Fixed effects: follicles ~ sin(2 * pi * Time) + cos(2 * pi * Time) 
##                        Value Std.Error  DF   t-value p-value
## (Intercept)        12.188089 0.9436602 295 12.915760  0.0000
## sin(2 * pi * Time) -2.985297 0.6055968 295 -4.929513  0.0000
## cos(2 * pi * Time) -0.877762 0.4777821 295 -1.837159  0.0672
##  Correlation: 
##                    (Intr) s(*p*T
## sin(2 * pi * Time)  0.000       
## cos(2 * pi * Time) -0.123  0.000
## 
## Standardized Within-Group Residuals:
##         Min          Q1         Med          Q3         Max 
## -2.34910093 -0.58969626 -0.04577893  0.52931186  3.37167486 
## 
## Number of Observations: 308
## Number of Groups: 11
\end{verbatim}

Things to note:

\begin{itemize}
\tightlist
\item
  The fitting is done with the \texttt{nlme::lme} function, and not
  \texttt{lme4::lmer} (which does not allow for non blocked covariance
  models).
\item
  \texttt{sin(2*pi*Time)\ +\ cos(2*pi*Time)} is a fixed effect that
  captures seasonality.
\item
  The temporal covariance, is specified using the \texttt{correlations=}
  argument.
\item
  AR(1) was assumed by calling \texttt{correlation=corAR1()}. See
  \texttt{nlme::corClasses} for a list of supported correlation
  structures.
\item
  From the summary, we see that a \texttt{Mare} random effect has also
  been added. Where is it specified? It is implied by the
  \texttt{random=} argument. Read \texttt{?lme} for further details.
\end{itemize}

We can now inspect the contrivance implied by our model's specification:

\begin{Shaded}
\begin{Highlighting}[]
\NormalTok{the.cov <-}\StringTok{ }\NormalTok{mgcv}\OperatorTok{::}\KeywordTok{extract.lme.cov}\NormalTok{(fm1Ovar.lme, }\DataTypeTok{data =}\NormalTok{ Ovary) }
\NormalTok{lattice}\OperatorTok{::}\KeywordTok{levelplot}\NormalTok{(the.cov)}
\end{Highlighting}
\end{Shaded}

\includegraphics[width=0.5\linewidth]{Rcourse_files/figure-latex/unnamed-chunk-218-1}

\section{Extensions}\label{extensions-1}

\subsection{Cluster Robust Standard
Errors}\label{cluster-robust-standard-errors}

As previously stated, random effects are nothing more than a convenient
way to specify dependencies within a level of a random effect, i.e.,
within a group/cluster. This is also the motivation underlying
\emph{cluster robust} inference, which is immensely popular with
econometricians, but less so elsewhere. What is the difference between
the two?

Mixed models framework is a bona-fide generalization of cluster robust
inference. This author thus recommends using the \textbf{lme4} and
\textbf{nlme} packages for mixed models to deal with correlations within
cluster.

For a longer comparison between the two approaches, see
\href{https://m-clark.github.io/docs/clustered/}{Michael Clarck's
guide}.

\subsection{Linear Models for Panel
Data}\label{linear-models-for-panel-data}

\textbf{nlme} and \textbf{lme4} will probably provide you with all the
functionality you need for panel data. If, however, you are trained as
an econometrist, prefer the econometric parlance, and are not using
non-linead models, then the
\href{https://cran.r-project.org/package=plm}{plm} and
\href{https://www.jacob-long.com/post/panelr-intro/}{panelr} packages
are just for you. In particular, it allows for cluster-robust covariance
estimates, and
\href{https://en.wikipedia.org/wiki/Durbin\%E2\%80\%93Wu\%E2\%80\%93Hausman_test}{Durbin--Wu--Hausman
test} for random effects. The \textbf{plm}
\href{https://cran.r-project.org/web/packages/plm/vignettes/plm.pdf}{package
vignette} also has a comparison to the \textbf{nlme} package.

\subsection{Testing Hypotheses on
Correlations}\label{testing-hypotheses-on-correlations}

After working so hard to model the correlations in observation, we may
want to test if it was all required. Douglas Bates, the author of
\textbf{nlme} and \textbf{lme4} wrote a famous cautionary note,
\href{https://stat.ethz.ch/pipermail/r-help/2006-May/094765.html}{found
here}, on hypothesis testing in mixed models. Many practitioners,
however, do not adopt Doug's view. Many of the popular tests,
particularly the ones in the econometric literature, can be found in the
\textbf{plm} package (see Section 6 in the
\href{https://cran.r-project.org/web/packages/plm/vignettes/plm.pdf}{package
vignette}). These include tests for poolability, Hausman test, tests for
serial correlations, tests for cross-sectional dependence, and unit root
tests.

\section{Relation to Other
Estimators}\label{relation-to-other-estimators}

\subsection{Fixed Effects in the Econometric
Literature}\label{fixed-effects-in-the-econometric-literature}

Fixed effects in the statistical literature, as discussed herein, are
different than those in the econometric literature. See Section 7 of the
\textbf{plm}
\href{https://cran.r-project.org/web/packages/plm/vignettes/plm.pdf}{package
vignette} for a comparison.

\subsection{Relation to Generalized Least Squares
(GLS)}\label{relation-to-generalized-least-squares-gls}

GLS is the solution to a decorrelated least squares problem:
\[\hat{\beta}_{GLS}:=argmin_\beta\{(X'\beta-y)'\Sigma^{-1}(X'\beta-y)'\}.\]
This estimator can be viewed as a least squares estimator that accounts
for correlations in the data. It is also a maximum likelihood estimator
under a Gaussian error assumption. Viewed as the latter, then linear
mixed models under a Gaussian error assumption, collapses to a GLS
estimator.

\subsection{Relation to Conditional Gaussian
Fields}\label{relation-to-conditional-gaussian-fields}

In the geo-spatial literature, geo-located measurements are typically
assumed to be sampled from a \emph{Gaussian Random Field}. All the
models discussed in this chapter can be stated in terms of these random
fields. In the random field nomenclature, the fixed effects are known as
the \emph{drift}, or the \emph{mean field}, and the covariance in errors
is known as the \emph{correlation function}. In other fields of
literature the correlation function is known as a \emph{charachteristic
function}, \emph{radial basis functions}, or \emph{kernel}. Assuming
stationarity, these simplify to the \emph{power spectrum} via the
\emph{Wiener--Khinchin theorem}. The predictions of such models may be
found under the names of \emph{linear projection operators}, \emph{best
linear unbiased prediction}, \emph{Kriging}, \emph{radial basis function
interpolators}.

\subsection{Relation to Empirical Risk Minimization
(ERM)}\label{relation-to-empirical-risk-minimization-erm}

ERM is more general than mixed-models estimation since it allows loss
functions that are not the (log) likelihood. ERM is less general than
LMM, in that ERM (typically) does not account for correlations in the
data.

\subsection{Relation to M-Estimation}\label{relation-to-m-estimation}

M-estimation is term in the statistical literature for ERM.

\subsection{Relation to Generalize Estimating Equations
(GEE)}\label{relation-to-generalize-estimating-equations-gee}

The first order condition of the LMM problem returns a set of
(non-linear) estimating equations. In this sense, GEE can be seen as
more general than LMM in that the GEE need not be the derivative of the
(log) likelihood.

\subsection{Relation to MANOVA}\label{manova}

Multivariate analysis of variance (MANOVA) deals with the estimation of
effect on \textbf{vector valued} outcomes. Put differently: in ANOVA the
response, \(y\), is univariate. In MANOVA, the outcome is multivariate.
MANOVA is useful when there are correlations among the entries of \(y\).
Otherwise- one may simply solve many ANOVA problems, instead of a single
MANOVA.

Now assume that the outcome of a MANOVA is measurements of an individual
at several time periods. The measurements are clearly correlated, so
that MANOVA may be useful. But one may also treat the subject as a
random effect, with a univariate response. We thus see that this
seemingly MANOVA problem can be solved with the mixed models framework.

What MANOVA problems cannot be solved with mixed models? There may be
cases where the covariance of the multivariate outcome, \(y\), is very
complicated. If the covariance in \(y\) may not be stated using a
combination of random and fixed effects, then the covariance has to be
stated explicitly. It is also possible to consider mixed-models with
multivariate outcomes, i.e., a \emph{mixed MANOVA}, or \emph{hirarchial
MANOVA}. The R functions we present herein permit this.

\subsection{Relation to Seemingly Unrelated Equations
(SUR)}\label{relation-to-seemingly-unrelated-equations-sur}

SUR is the econometric term for MANOVA.

\section{Bibliographic Notes}\label{bibliographic-notes-6}

Most of the examples in this chapter are from the documentation of the
\textbf{lme4} package \citep{lme4}. For a general and very applied
treatment, see \citet{pinero2000mixed}. As usual, a hands on view can be
found in \citet{venables2013modern}, and also in an excellent blog post
by
\href{http://rpsychologist.com/r-guide-longitudinal-lme-lmer}{Kristoffer
Magnusson} For a more theoretical view see \citet{weiss2005modeling} or
\citet{searle2009variance}. Sometimes it is unclear if an effect is
random or fixed; on the difference between the two types of inference
see the classics: \citet{eisenhart1947assumptions},
\citet{kempthorne1975fixed}, and the more recent
\citet{rosset2018fixed}. For more on predictions in linear mixed models
see \citet{robinson1991blup}, \citet{rabinowicz2018assessing}, and
references therein. See
\href{https://m-clark.github.io/docs/clustered/}{Michael Clarck's} guide
for various ways of dealing with correlations within groups. For the
geo-spatial view and terminology of correlated data, see
\citet{christakos2000modern}, \citet{diggle1998model},
\citet{allard2013j}, and \citet{cressie2015statistics}.

\section{Practice Yourself}\label{practice-yourself-4}

\begin{enumerate}
\def\labelenumi{\arabic{enumi}.}
\item
  Computing the variance of the sample mean given dependent
  correlations. How does it depend on the covariance between
  observations? When is the sample most informative on the population
  mean?
\item
  Return to the \texttt{Penicillin} data set. Instead of fitting an LME
  model, fit an LM model with \texttt{lm}. I.e., treat all random
  effects as fixed.

  \begin{enumerate}
  \def\labelenumii{\alph{enumii}.}
  \tightlist
  \item
    Compare the effect estimates.
  \item
    Compare the standard errors.
  \item
    Compare the predictions of the two models.
  \end{enumerate}
\item
  {[}Very Advanced!{]} Return to the \texttt{Penicillin} data and use
  the \texttt{gls} function to fit a generalized linear model,
  equivalent to the LME model in our text.
\item
  Read about the ``oats'' dataset using \texttt{?\ MASS::oats}.Inspect
  the dependency of the yield (Y) in the Varieties (V) and the Nitrogen
  treatment (N).

  \begin{enumerate}
  \def\labelenumii{\arabic{enumii}.}
  \tightlist
  \item
    Fit a linear model, does the effect of the treatment significant?
    The interaction between the Varieties and Nitrogen is significant?
  \item
    An expert told you that could be a variance between the different
    blocks (B) which can bias the analysis. fit a LMM for the data.
  \item
    Do you think the blocks should be taken into account as ``random
    effect'' or ``fixed effect''?
  \end{enumerate}
\item
  Return to the temporal correlation in Section \ref{serial}, and
  replace the AR(1) covariance, with an ARMA covariance. Visualize the
  data's covariance matrix, and compare the fitted values.
\end{enumerate}

See DataCamps'
\href{https://www.datacamp.com/courses/hierarchical-and-mixed-effects-models}{Hierarchical
and Mixed Effects Models} for more self practice.

\chapter{Multivariate Data Analysis}\label{multivariate}

The term ``multivariate data analysis'' is so broad and so overloaded,
that we start by clarifying what is discussed and what is not discussed
in this chapter. Broadly speaking, we will discuss statistical
\emph{inference}, and leave more ``exploratory flavored'' matters like
clustering, and visualization, to the Unsupervised Learning Chapter
\ref{unsupervised}.

We start with an example.

\BeginKnitrBlock{example}
\protect\hypertarget{exm:icu}{}{\label{exm:icu} }Consider the problem of a
patient monitored in the intensive care unit. At every minute the
monitor takes \(p\) physiological measurements: blood pressure, body
temperature, etc. The total number of minutes in our data is \(n\), so
that in total, we have \(n \times p\) measurements, arranged in a
matrix. We also know the typical measurements for this patient when
healthy: \(\mu_0\).
\EndKnitrBlock{example}

Formally, let \(y\) be single (random) measurement of a \(p\)-variate
random vector. Denote \(\mu:=E[y]\). Here is the set of problems we will
discuss, in order of their statistical difficulty.

\begin{itemize}
\item
  \textbf{Signal detection}: a.k.a. \emph{multivariate hypothesis
  testing}, i.e., testing if \(\mu\) equals \(\mu_0\) and for
  \(\mu_0=0\) in particular. In our example: ``are the current
  measurement different than a typical one?''
\item
  \textbf{Signal counting}: Counting the number of elements in \(\mu\)
  that differ from \(\mu_0\), and for \(\mu_0=0\) in particular. In our
  example: ``how many measurements differ than their typical values?''
\item
  \textbf{Signal identification}: a.k.a. \emph{multiple testing}, i.e.,
  testing which of the elements in \(\mu\) differ from \(\mu_0\) and for
  \(\mu_0=0\) in particular. In the ANOVA literature, this is known as a
  \textbf{post-hoc} analysis. In our example: ``which measurements
  differ than their typical values?''
\item
  \textbf{Signal estimation}: Estimating the magnitudes of the departure
  of \(\mu\) from \(\mu_0\), and for \(\mu_0=0\) in particular. If
  estimation follows a \emph{signal detection} or \emph{signal
  identification} stage, this is known as a \emph{selective estimation}
  problem. In our example: ``what is the value of the measurements that
  differ than their typical values?''
\item
  \textbf{Multivariate Regression}: a.k.a. \emph{MANOVA} in statistical
  literature, and \emph{structured learning} in the machine learning
  literature. In our example: ``what factors affect the physiological
  measurements?''
\end{itemize}

\BeginKnitrBlock{example}
\protect\hypertarget{exm:brain-imaging}{}{\label{exm:brain-imaging}
}Consider the problem of detecting regions of cognitive function in the
brain using fMRI. Each measurement is the activation level at each
location in a brain's region. If the region has a cognitive function,
the mean activation differs than \(\mu_0=0\) when the region is evoked.
\EndKnitrBlock{example}

\BeginKnitrBlock{example}
\protect\hypertarget{exm:genetics}{}{\label{exm:genetics} }Consider the
problem of detecting cancer encoding regions in the genome. Each
measurement is the vector of the genetic configuration of an individual.
A cancer encoding region will have a different (multivariate)
distribution between sick and healthy. In particular, \(\mu\) of sick
will differ from \(\mu\) of healthy.
\EndKnitrBlock{example}

\BeginKnitrBlock{example}
\protect\hypertarget{exm:regression}{}{\label{exm:regression} }Consider the
problem of the simplest multiple regression. The estimated coefficient,
\(\hat \beta\) are a random vector. Regression theory tells us that its
covariance is \((X'X)^{-1}\sigma^2\), and null mean of \(\beta\). We
thus see that inference on the vector of regression coefficients, is
nothing more than a multivaraite inference problem.
\EndKnitrBlock{example}

\BeginKnitrBlock{remark}
\iffalse{} {Remark. } \fi{}In the above, ``signal'' is defined in terms
of \(\mu\). It is possible that the signal is not in the location,
\(\mu\), but rather in the covariance, \(\Sigma\). We do not discuss
these problems here, and refer the reader to \citet{nadler2008finite}.
\EndKnitrBlock{remark}

Another possible question is: does a multivariate analysis gives us
something we cannot get from a mass-univariate analysis (i.e., a
multivariate analysis on each variable separately). In Example
\ref{exm:icu} we could have just performed multiple univariate tests,
and sign an alarm when any of the univariate detectors was triggered.
The reason we want a multivariate detector, and not multiple univariate
detectors is that it is possible that each measurement alone is
borderline, but together, the signal accumulates. In our ICU example is
may mean that the pulse is borderline, the body temperature is
borderline, etc. Analyzed simultaneously, it is clear that the patient
is in distress.

The next figure\footnote{My thanks to Efrat Vilneski for the figure.}
illustrates the idea that some bi-variate measurements may seem ordinary
univariately, while very anomalous when examined bi-variately.

\BeginKnitrBlock{remark}
\iffalse{} {Remark. } \fi{}The following figure may also be used to
demonstrate the difference between Euclidean Distance and Mahalanobis
Distance.
\EndKnitrBlock{remark}

\includegraphics[width=0.5\linewidth]{Rcourse_files/figure-latex/unnamed-chunk-221-1}

\section{Signal Detection}\label{signal-detection}

Signal detection deals with the detection of the departure of \(\mu\)
from some \(\mu_0\), and especially, \(\mu_0=0\). This problem can be
thought of as the multivariate counterpart of the univariate hypothesis
t-test.

\subsection{Hotelling's T2 Test}\label{hotellings-t2-test}

The most fundamental approach to signal detection is a mere
generalization of the t-test, known as \emph{Hotelling's \(T^2\) test}.

Recall the univariate t-statistic of a data vector \(x\) of length
\(n\):

\begin{align}
  t^2(x):= \frac{(\bar{x}-\mu_0)^2}{Var[\bar{x}]}= (\bar{x}-\mu_0)Var[\bar{x}]^{-1}(\bar{x}-\mu_0),
  \label{eq:t-test}
\end{align}

where \(Var[\bar{x}]=S^2(x)/n\), and \(S^2(x)\) is the unbiased variance
estimator \(S^2(x):=(n-1)^{-1}\sum (x_i-\bar x)^2\).

Generalizing Eq\eqref{eq:t-test} to the multivariate case: \(\mu_0\) is a
\(p\)-vector, \(\bar x\) is a \(p\)-vector, and \(Var[\bar x]\) is a
\(p \times p\) matrix of the covariance between the \(p\) coordinated of
\(\bar x\). When operating with vectors, the squaring becomes a
quadratic form, and the division becomes a matrix inverse. We thus have

\begin{align}
  T^2(x):= (\bar{x}-\mu_0)' Var[\bar{x}]^{-1} (\bar{x}-\mu_0),
  \label{eq:hotelling-test}
\end{align}

which is the definition of Hotelling's \(T^2\) test statistic. We
typically denote the covariance between coordinates in \(x\) with
\(\hat \Sigma(x)\), so that
\(\widehat \Sigma_{k,l}:=\widehat {Cov}[x_k,x_l]=(n-1)^{-1} \sum (x_{k,i}-\bar x_k)(x_{l,i}-\bar x_l)\).
Using the \(\Sigma\) notation, Eq.\eqref{eq:hotelling-test} becomes

\begin{align}
  T^2(x):= n (\bar{x}-\mu_0)' \hat \Sigma(x)^{-1} (\bar{x}-\mu_0),
\end{align}

which is the standard notation of Hotelling's test statistic.

For inference, we need the null distribution of Hotelling's test
statistic. For this we introduce some vocabulary\footnote{This
  vocabulary is not standard in the literature, so when you read a text,
  you need to verify yourself what the author means.}:

\begin{enumerate}
\def\labelenumi{\arabic{enumi}.}
\tightlist
\item
  \textbf{Low Dimension}: We call a problem \emph{low dimensional} if
  \(n \gg p\), i.e. \(p/n \approx 0\). This means there are many
  observations per estimated parameter.
\item
  \textbf{High Dimension}: We call a problem \emph{high dimensional} if
  \(p/n \to c\), where \(c\in (0,1)\). This means there are more
  observations than parameters, but not many.
\item
  \textbf{Very High Dimension}: We call a problem \emph{very high
  dimensional} if \(p/n \to c\), where \(1<c<\infty\). This means there
  are less observations than parameter.
\end{enumerate}

Hotelling's \(T^2\) test can only be used in the low dimensional regime.
For some intuition on this statement, think of taking \(n=20\)
measurements of \(p=100\) physiological variables. We seemingly have
\(20\) observations, but there are \(100\) unknown quantities in
\(\mu\). Would you trust your conclusion that \(\bar x\) is different
than \(\mu_0\) based on merely \(20\) observations.

The above criticism is formalized in \citet{bai1996effect}. For modern
applications, Hotelling's \(T^2\) is not recommended, since many modern
alternatives have been made available. See \citet{rosenblatt2016better}
and references for a review.

\subsection{Various Types of Signal to
Detect}\label{various-types-of-signal-to-detect}

In the previous, we assumed that the signal is a departure of \(\mu\)
from some \(\mu_0\). For vactor-valued data \(y\), that is distributed
\(F\), we may define ``signal'' as any departure from some \(F_0\). This
is the multivaraite counterpart of goodness-of-fit (GOF) tests.

Even when restricting ``signal'' to departures of \(\mu\) from
\(\mu_0\), we may try to detect various types of signal:

\begin{enumerate}
\def\labelenumi{\arabic{enumi}.}
\tightlist
\item
  \textbf{Dense Signal}: when the departure is in all coordinates of
  \(\mu\).
\item
  \textbf{Sparse Signal}: when the departure is in a subset of
  coordinates of \(\mu\).
\end{enumerate}

A manufactoring motivation is consistent with a dense signal: if a
manufacturing process has failed, we expect a change in many
measurements (i.e.~coordinates of \(\mu\)). A brain-imaging motivation
is consistent with a dense signal: if a region encodes cognitive
function, we expect a change in many brain locations (i.e.~coordinates
of \(\mu\).) A genetic motivation is consistent with a sparse signal: if
susceptibility of disease is genetic, only a small subset of locations
in the genome will encode it.

Hotelling's \(T^2\) statistic is designed for dense signal. The
following is a simple statistic designed for sparse signal.

\subsection{Simes' Test}\label{simes-test}

Hotelling's \(T^2\) statistic has currently two limitations: It is
designed for dense signals, and it requires estimating the covariance,
which is a very difficult problem.

An algorithm, that is sensitive to sparse signal and allows
statistically valid detection under a wide range of covariances (even if
we don't know the covariance) is known as \emph{Simes' Test}. The
statistic is defined vie the following algorithm:

\begin{enumerate}
\def\labelenumi{\arabic{enumi}.}
\tightlist
\item
  Compute \(p\) variable-wise p-values: \(p_1,\dots,p_j\).
\item
  Denote \(p_{(1)},\dots,p_{(j)}\) the sorted p-values.
\item
  Simes' statistic is \(p_{Simes}:=min_j\{p_{(j)} \times p/j\}\).
\item
  Reject the ``no signal'' null hypothesis at significance \(\alpha\) if
  \(p_{Simes}<\alpha\).
\end{enumerate}

\subsection{Signal Detection with R}\label{signal-detection-with-r}

Let's generate some data with no signal.

\begin{Shaded}
\begin{Highlighting}[]
\KeywordTok{library}\NormalTok{(mvtnorm)}
\NormalTok{n <-}\StringTok{ }\DecValTok{100} \CommentTok{# observations}
\NormalTok{p <-}\StringTok{ }\DecValTok{18} \CommentTok{# parameter dimension}
\NormalTok{mu <-}\StringTok{ }\KeywordTok{rep}\NormalTok{(}\DecValTok{0}\NormalTok{,p) }\CommentTok{# no signal}
\NormalTok{x <-}\StringTok{ }\KeywordTok{rmvnorm}\NormalTok{(}\DataTypeTok{n =}\NormalTok{ n, }\DataTypeTok{mean =}\NormalTok{ mu)}
\KeywordTok{dim}\NormalTok{(x)}
\end{Highlighting}
\end{Shaded}

\begin{verbatim}
## [1] 100  18
\end{verbatim}

\begin{Shaded}
\begin{Highlighting}[]
\NormalTok{lattice}\OperatorTok{::}\KeywordTok{levelplot}\NormalTok{(x)}
\end{Highlighting}
\end{Shaded}

\includegraphics[width=0.5\linewidth]{Rcourse_files/figure-latex/unnamed-chunk-222-1}

Now make our own Hotelling function.

\begin{Shaded}
\begin{Highlighting}[]
\NormalTok{hotellingOneSample <-}\StringTok{ }\ControlFlowTok{function}\NormalTok{(x, }\DataTypeTok{mu0=}\KeywordTok{rep}\NormalTok{(}\DecValTok{0}\NormalTok{,}\KeywordTok{ncol}\NormalTok{(x)))\{}
\NormalTok{  n <-}\StringTok{ }\KeywordTok{nrow}\NormalTok{(x)}
\NormalTok{  p <-}\StringTok{ }\KeywordTok{ncol}\NormalTok{(x)}
  \KeywordTok{stopifnot}\NormalTok{(n }\OperatorTok{>}\StringTok{ }\DecValTok{5}\OperatorTok{*}\StringTok{ }\NormalTok{p)}
\NormalTok{  bar.x <-}\StringTok{ }\KeywordTok{colMeans}\NormalTok{(x)}
\NormalTok{  Sigma <-}\StringTok{ }\KeywordTok{var}\NormalTok{(x)}
\NormalTok{  Sigma.inv <-}\StringTok{ }\KeywordTok{solve}\NormalTok{(Sigma)}
\NormalTok{  T2 <-}\StringTok{ }\NormalTok{n }\OperatorTok{*}\StringTok{ }\NormalTok{(bar.x}\OperatorTok{-}\NormalTok{mu0) }\OperatorTok\StringTok{ }\NormalTok{Sigma.inv }\OperatorTok\StringTok{ }\NormalTok{(bar.x}\OperatorTok{-}\NormalTok{mu0)}
\NormalTok{  p.value <-}\StringTok{ }\KeywordTok{pchisq}\NormalTok{(}\DataTypeTok{q =}\NormalTok{ T2, }\DataTypeTok{df =}\NormalTok{ p, }\DataTypeTok{lower.tail =} \OtherTok{FALSE}\NormalTok{)}
  \KeywordTok{return}\NormalTok{(}\KeywordTok{list}\NormalTok{(}\DataTypeTok{statistic=}\NormalTok{T2, }\DataTypeTok{pvalue=}\NormalTok{p.value))}
\NormalTok{\}}
\KeywordTok{hotellingOneSample}\NormalTok{(x)}
\end{Highlighting}
\end{Shaded}

\begin{verbatim}
## $statistic
##          [,1]
## [1,] 17.22438
## 
## $pvalue
##           [,1]
## [1,] 0.5077323
\end{verbatim}

Things to note:

\begin{itemize}
\tightlist
\item
  \texttt{stopifnot(n\ \textgreater{}\ 5\ *\ p)} is a little
  verification to check that the problem is indeed low dimensional.
  Otherwise, the \(\chi^2\) approximation cannot be trusted.
\item
  \texttt{solve} returns a matrix inverse.
\item
  \texttt{\%*\%} is the matrix product operator (see also
  \texttt{crossprod()}).
\item
  A function may return only a single object, so we wrap the statistic
  and its p-value in a \texttt{list} object.
\end{itemize}

Just for verification, we compare our home made Hotelling's test, to the
implementation in the \textbf{rrcov} package. The statistic is clearly
OK, but our \(\chi^2\) approximation of the distribution leaves room to
desire. Personally, I would never trust a Hotelling test if \(n\) is not
much greater than \(p\), in which case I would use a high-dimensional
adaptation (see Bibliography).

\begin{Shaded}
\begin{Highlighting}[]
\NormalTok{rrcov}\OperatorTok{::}\KeywordTok{T2.test}\NormalTok{(x)}
\end{Highlighting}
\end{Shaded}

\begin{verbatim}
## 
##  One-sample Hotelling test
## 
## data:  x
## T2 = 17.22400, F = 0.79259, df1 = 18, df2 = 82, p-value = 0.703
## alternative hypothesis: true mean vector is not equal to (0, 0, 0, 0, 0, 0, 0, 0, 0, 0, 0, 0, 0, 0, 0, 0, 0, 0)' 
## 
## sample estimates:
##                      [,1]       [,2]      [,3]       [,4]      [,5]
## mean x-vector -0.01746212 0.03776332 0.1006145 -0.2083005 0.1026982
##                      [,6]         [,7]       [,8]       [,9]       [,10]
## mean x-vector -0.05220043 -0.009497987 -0.1139856 0.02851701 -0.03089953
##                     [,11]      [,12]      [,13]      [,14]      [,15]
## mean x-vector -0.02457798 -0.1270753 0.04717076 0.01683591 0.03085023
##                   [,16]       [,17]     [,18]
## mean x-vector 0.1499485 -0.07630663 0.1004852
\end{verbatim}

Let's do the same with Simes':

\begin{Shaded}
\begin{Highlighting}[]
\NormalTok{Simes <-}\StringTok{ }\ControlFlowTok{function}\NormalTok{(x)\{}
\NormalTok{  p.vals <-}\StringTok{ }\KeywordTok{apply}\NormalTok{(x, }\DecValTok{2}\NormalTok{, }\ControlFlowTok{function}\NormalTok{(z) }\KeywordTok{t.test}\NormalTok{(z)}\OperatorTok{$}\NormalTok{p.value) }\CommentTok{# Compute variable-wise pvalues}
\NormalTok{  p <-}\StringTok{ }\KeywordTok{ncol}\NormalTok{(x)}
\NormalTok{  p.Simes <-}\StringTok{ }\NormalTok{p }\OperatorTok{*}\StringTok{ }\KeywordTok{min}\NormalTok{(}\KeywordTok{sort}\NormalTok{(p.vals)}\OperatorTok{/}\KeywordTok{seq_along}\NormalTok{(p.vals)) }\CommentTok{# Compute the Simes statistic}
  \KeywordTok{return}\NormalTok{(}\KeywordTok{c}\NormalTok{(}\DataTypeTok{pvalue=}\NormalTok{p.Simes))}
\NormalTok{\}}
\KeywordTok{Simes}\NormalTok{(x)}
\end{Highlighting}
\end{Shaded}

\begin{verbatim}
##    pvalue 
## 0.6398998
\end{verbatim}

And now we verify that both tests can indeed detect signal when present.
Are p-values small enough to reject the ``no signal'' null hypothesis?

\begin{Shaded}
\begin{Highlighting}[]
\NormalTok{mu <-}\StringTok{ }\KeywordTok{rep}\NormalTok{(}\DataTypeTok{x =} \DecValTok{10}\OperatorTok{/}\NormalTok{p,}\DataTypeTok{times=}\NormalTok{p) }\CommentTok{# inject signal}
\NormalTok{x <-}\StringTok{ }\KeywordTok{rmvnorm}\NormalTok{(}\DataTypeTok{n =}\NormalTok{ n, }\DataTypeTok{mean =}\NormalTok{ mu)}
\KeywordTok{hotellingOneSample}\NormalTok{(x)}
\end{Highlighting}
\end{Shaded}

\begin{verbatim}
## $statistic
##          [,1]
## [1,] 686.8046
## 
## $pvalue
##               [,1]
## [1,] 3.575926e-134
\end{verbatim}

\begin{Shaded}
\begin{Highlighting}[]
\KeywordTok{Simes}\NormalTok{(x)}
\end{Highlighting}
\end{Shaded}

\begin{verbatim}
##       pvalue 
## 2.765312e-10
\end{verbatim}

\ldots{} yes. All p-values are very small, so that all statistics can
detect the non-null distribution.

\section{Signal Counting}\label{signal-counting}

There are many ways to approach the \emph{signal counting} problem. For
the purposes of this book, however, we will not discuss them directly,
and solve the signal counting problem as a signal identification
problem: if we know \textbf{where} \(\mu\) departs from \(\mu_0\), we
only need to count coordinates to solve the signal counting problem.

\BeginKnitrBlock{remark}
\iffalse{} {Remark. } \fi{}In the sparsity or multiple-testing
literature, what we call ``signal counting'' is known as ``adapting to
sparsit'', or ``adaptivity''.
\EndKnitrBlock{remark}

\section{Signal Identification}\label{identification}

The problem of \emph{signal identification} is also known as
\emph{selective testing}, or more commonly as \emph{multiple testing}.

In the ANOVA literature, an identification stage will typically follow a
detection stage. These are known as the \emph{omnibus F test}, and
\emph{post-hoc} tests, respectively. In the multiple testing literature
there will typically be no preliminary detection stage. It is typically
assumed that signal is present, and the only question is ``where?''

The first question when approaching a multiple testing problem is ``what
is an error''? Is an error declaring a coordinate in \(\mu\) to be
different than \(\mu_0\) when it is actually not? Is an error an overly
high proportion of falsely identified coordinates? The former is known
as the \emph{family wise error rate} (FWER), and the latter as the
\emph{false discovery rate} (FDR).

\BeginKnitrBlock{remark}
\iffalse{} {Remark. } \fi{}These types of errors have many names in many
communities. See the Wikipedia entry on
\href{https://en.wikipedia.org/wiki/Receiver_operating_characteristic}{ROC}
for a table of the (endless) possible error measures.
\EndKnitrBlock{remark}

\subsection{Signal Identification in
R}\label{signal-identification-in-r}

One (of many) ways to do signal identification involves the
\texttt{stats::p.adjust} function. The function takes as inputs a
\(p\)-vector of the variable-wise \textbf{p-values}. Why do we start
with variable-wise p-values, and not the full data set?

\begin{enumerate}
\def\labelenumi{\alph{enumi}.}
\tightlist
\item
  Because we want to make inference variable-wise, so it is natural to
  start with variable-wise statistics.
\item
  Because we want to avoid dealing with covariances if possible.
  Computing variable-wise p-values does not require estimating
  covariances.
\item
  So that the identification problem is decoupled from the variable-wise
  inference problem, and may be applied much more generally than in the
  setup we presented.
\end{enumerate}

We start be generating some high-dimensional multivariate data and
computing the coordinate-wise (i.e.~hypothesis-wise) p-value.

\begin{Shaded}
\begin{Highlighting}[]
\KeywordTok{library}\NormalTok{(mvtnorm)}
\NormalTok{n <-}\StringTok{ }\FloatTok{1e1}
\NormalTok{p <-}\StringTok{ }\FloatTok{1e2}
\NormalTok{mu <-}\StringTok{ }\KeywordTok{rep}\NormalTok{(}\DecValTok{0}\NormalTok{,p)}
\NormalTok{x <-}\StringTok{ }\KeywordTok{rmvnorm}\NormalTok{(}\DataTypeTok{n =}\NormalTok{ n, }\DataTypeTok{mean =}\NormalTok{ mu)}
\KeywordTok{dim}\NormalTok{(x)}
\end{Highlighting}
\end{Shaded}

\begin{verbatim}
## [1]  10 100
\end{verbatim}

\begin{Shaded}
\begin{Highlighting}[]
\NormalTok{lattice}\OperatorTok{::}\KeywordTok{levelplot}\NormalTok{(x)}
\end{Highlighting}
\end{Shaded}

\includegraphics[width=0.5\linewidth]{Rcourse_files/figure-latex/unnamed-chunk-229-1}

We now compute the pvalues of each coordinate. We use a coordinate-wise
t-test. Why a t-test? Because for the purpose of demonstration we want a
simple test. In reality, you may use any test that returns valid
p-values.

\begin{Shaded}
\begin{Highlighting}[]
\NormalTok{t.pval <-}\StringTok{ }\ControlFlowTok{function}\NormalTok{(y) }\KeywordTok{t.test}\NormalTok{(y)}\OperatorTok{$}\NormalTok{p.value}
\NormalTok{p.values <-}\StringTok{ }\KeywordTok{apply}\NormalTok{(}\DataTypeTok{X =}\NormalTok{ x, }\DataTypeTok{MARGIN =} \DecValTok{2}\NormalTok{, }\DataTypeTok{FUN =}\NormalTok{ t.pval) }
\KeywordTok{plot}\NormalTok{(p.values, }\DataTypeTok{type=}\StringTok{'h'}\NormalTok{)}
\end{Highlighting}
\end{Shaded}

\includegraphics[width=0.5\linewidth]{Rcourse_files/figure-latex/unnamed-chunk-230-1}

Things to note:

\begin{itemize}
\tightlist
\item
  \texttt{t.pval} is a function that merely returns the p-value of a
  t.test.
\item
  We used the \texttt{apply} function to apply the same function to each
  column of \texttt{x}.
\item
  \texttt{MARGIN=2} tells \texttt{apply} to compute over columns and not
  rows.
\item
  The output, \texttt{p.values}, is a vector of 100 p-values.
\end{itemize}

We are now ready to do the identification, i.e., find which coordinate
of \(\mu\) is different than \(\mu_0=0\). The workflow for
identification has the same structure, regardless of the desired error
guarantees:

\begin{enumerate}
\def\labelenumi{\arabic{enumi}.}
\tightlist
\item
  Compute an \texttt{adjusted\ p-value}.
\item
  Compare the adjusted p-value to the desired error level.
\end{enumerate}

If we want \(FWER \leq 0.05\), meaning that we allow a \(5\%\)
probability of making any mistake, we will use the
\texttt{method="holm"} argument of \texttt{p.adjust}.

\begin{Shaded}
\begin{Highlighting}[]
\NormalTok{alpha <-}\StringTok{ }\FloatTok{0.05}
\NormalTok{p.values.holm <-}\StringTok{ }\KeywordTok{p.adjust}\NormalTok{(p.values, }\DataTypeTok{method =} \StringTok{'holm'}\NormalTok{ )}
\KeywordTok{which}\NormalTok{(p.values.holm }\OperatorTok{<}\StringTok{ }\NormalTok{alpha)}
\end{Highlighting}
\end{Shaded}

\begin{verbatim}
## integer(0)
\end{verbatim}

If we want \(FDR \leq 0.05\), meaning that we allow the proportion of
false discoveries to be no larger than \(5\%\), we use the
\texttt{method="BH"} argument of \texttt{p.adjust}.

\begin{Shaded}
\begin{Highlighting}[]
\NormalTok{alpha <-}\StringTok{ }\FloatTok{0.05}
\NormalTok{p.values.BH <-}\StringTok{ }\KeywordTok{p.adjust}\NormalTok{(p.values, }\DataTypeTok{method =} \StringTok{'BH'}\NormalTok{ )}
\KeywordTok{which}\NormalTok{(p.values.BH }\OperatorTok{<}\StringTok{ }\NormalTok{alpha)}
\end{Highlighting}
\end{Shaded}

\begin{verbatim}
## integer(0)
\end{verbatim}

We now inject some strong signal in \(\mu\) just to see that the process
works. We will artificially inject signal in the first 10 coordinates.

\begin{Shaded}
\begin{Highlighting}[]
\NormalTok{mu[}\DecValTok{1}\OperatorTok{:}\DecValTok{10}\NormalTok{] <-}\StringTok{ }\DecValTok{2} \CommentTok{# inject signal in first 10 variables}
\NormalTok{x <-}\StringTok{ }\KeywordTok{rmvnorm}\NormalTok{(}\DataTypeTok{n =}\NormalTok{ n, }\DataTypeTok{mean =}\NormalTok{ mu) }\CommentTok{# generate data}
\NormalTok{p.values <-}\StringTok{ }\KeywordTok{apply}\NormalTok{(}\DataTypeTok{X =}\NormalTok{ x, }\DataTypeTok{MARGIN =} \DecValTok{2}\NormalTok{, }\DataTypeTok{FUN =}\NormalTok{ t.pval) }
\NormalTok{p.values.BH <-}\StringTok{ }\KeywordTok{p.adjust}\NormalTok{(p.values, }\DataTypeTok{method =} \StringTok{'BH'}\NormalTok{ )}
\KeywordTok{which}\NormalTok{(p.values.BH }\OperatorTok{<}\StringTok{ }\NormalTok{alpha)}
\end{Highlighting}
\end{Shaded}

\begin{verbatim}
##  [1]  1  2  3  4  5  6  7  9 10 55
\end{verbatim}

Indeed- we are now able to detect that the first coordinates carry
signal, because their respective coordinate-wise null hypotheses have
been rejected.

\section{Signal Estimation (*)}\label{signal-estimation}

The estimation of the elements of \(\mu\) is a seemingly straightforward
task. This is not the case, however, if we estimate only the elements
that were selected because they were significant (or any other
data-dependent criterion). Clearly, estimating only significant entries
will introduce a bias in the estimation. In the statistical literature,
this is known as \emph{selection bias}. Selection bias also occurs when
you perform inference on regression coefficients after some model
selection, say, with a lasso, or a forward search\footnote{You might
  find this shocking, but it does mean that you cannot trust the
  \texttt{summary} table of a model that was selected from a multitude
  of models.}.

Selective inference is a complicated and active research topic so we
will not offer any off-the-shelf solution to the matter. The curious
reader is invited to read \citet{rosenblatt2014selective},
\citet{javanmard2014confidence}, or
\href{http://www.stat.berkeley.edu/~wfithian/}{Will Fithian's} PhD
thesis \citep{fithian2015topics} for more on the topic.

\section{Bibliographic Notes}\label{bibliographic-notes-7}

For a general introduction to multivariate data analysis see
\citet{anderson2004introduction}. For an R oriented introduction, see
\citet{everitt2011introduction}. For more on the difficulties with high
dimensional problems, see \citet{bai1996effect}. For some cutting edge
solutions for testing in high-dimension, see
\citet{rosenblatt2016better} and references therein. Simes' test is not
very well known. It is introduced in \citet{simes1986improved}, and
proven to control the type I error of detection under a PRDS type of
dependence in \citet{benjamini2001control}. For more on multiple
testing, and signal identification, see \citet{efron2012large}. For more
on the choice of your error rate see \citet{rosenblatt2013practitioner}.
For an excellent review on graphical models see
\citet{kalisch2014causal}. Everything you need on graphical models,
Bayesian belief networks, and structure learning in R, is collected in
the \href{https://cran.r-project.org/web/views/gR.html}{Task View}.

\section{Practice Yourself}\label{practice-yourself-5}

\begin{enumerate}
\def\labelenumi{\arabic{enumi}.}
\item
  Generate multivariate data with:

\begin{Shaded}
\begin{Highlighting}[]
\KeywordTok{set.seed}\NormalTok{(}\DecValTok{3}\NormalTok{)}
\NormalTok{mean<-}\KeywordTok{rexp}\NormalTok{(}\DecValTok{50}\NormalTok{,}\DecValTok{6}\NormalTok{)}
\NormalTok{multi<-}\StringTok{  }\KeywordTok{rmvnorm}\NormalTok{(}\DataTypeTok{n =} \DecValTok{100}\NormalTok{, }\DataTypeTok{mean =}\NormalTok{ mean) }
\end{Highlighting}
\end{Shaded}

  \begin{enumerate}
  \def\labelenumii{\alph{enumii}.}
  \tightlist
  \item
    Use Hotelling's test to determine if \(\mu\) equals \(\mu_0=0\). Can
    you detect the signal?
  \item
    Perform t.test on each variable and extract the p-value. Try to
    identify visually the variables which depart from \(\mu_0\).
  \item
    Use \texttt{p.adjust} to identify in which variables there are any
    departures from \(\mu_0=0\). Allow 5\% probability of making any
    false identification.
  \item
    Use \texttt{p.adjust} to identify in which variables there are any
    departures from \(\mu_0=0\). Allow a 5\% proportion of errors within
    identifications.
  \end{enumerate}
\item
  Generate multivariate data from two groups:
  \texttt{rmvnorm(n\ =\ 100,\ mean\ =\ rep(0,10))} for the first, and
  \texttt{rmvnorm(n\ =\ 100,\ mean\ =\ rep(0.1,10))} for the second.

  \begin{enumerate}
  \def\labelenumii{\alph{enumii}.}
  \tightlist
  \item
    Do we agree the groups differ?
  \item
    Implement the two-group Hotelling test described in Wikipedia:
    (\url{https://en.wikipedia.org/wiki/Hotelling\%27s_T-squared_distribution\#Two-sample_statistic}).
  \item
    Verify that you are able to detect that the groups differ.
  \item
    Perform a two-group t-test on each coordinate. On which coordinates
    can you detect signal while controlling the FWER? On which while
    controlling the FDR? Use \texttt{p.adjust}.
  \end{enumerate}
\item
  Return to the previous problem, but set \texttt{n=9}. Verify that you
  cannot compute your Hotelling statistic.
\end{enumerate}

\chapter{Supervised Learning}\label{supervised}

Machine learning is very similar to statistics, but it is certainly not
the same. As the name suggests, in machine learning we want machines to
learn. This means that we want to replace hard-coded expert algorithm,
with data-driven self-learned algorithm.

There are many learning setups, that depend on what information is
available to the machine. The most common setup, discussed in this
chapter, is \emph{supervised learning}. The name takes from the fact
that by giving the machine data samples with known inputs (a.k.a.
features) and desired outputs (a.k.a. labels), the human is effectively
supervising the learning. If we think of the inputs as predictors, and
outcomes as predicted, it is no wonder that supervised learning is very
similar to statistical prediction. When asked ``are these the same?'' I
like to give the example of internet fraud. If you take a sample of
fraud ``attacks'', a statistical formulation of the problem is highly
unlikely. This is because fraud events are not randomly drawn from some
distribution, but rather, arrive from an adversary learning the defenses
and adapting to it. This instance of supervised learning is more similar
to game theory than statistics.

Other types of machine learning problems include
\citep{sammut2011encyclopedia}:

\begin{itemize}
\item
  \textbf{Unsupervised Learning}: Where we merely analyze the
  inputs/features, but no desirable outcome is available to the learning
  machine. See Chapter \ref{unsupervised}.
\item
  \textbf{Semi Supervised Learning}: Where only part of the samples are
  labeled. A.k.a. \emph{co-training}, \emph{learning from labeled and
  unlabeled data}, \emph{transductive learning}.
\item
  \textbf{Active Learning}: Where the machine is allowed to query the
  user for labels. Very similar to \emph{adaptive design of
  experiments}.
\item
  \textbf{Learning on a Budget}: A version of active learning where
  querying for labels induces variable costs.
\item
  \textbf{Weak Learning}: A version of supervised learning where the
  labels are given not by an expert, but rather by some heuristic rule.
  Example: mass-labeling cyber attacks by a rule based software, instead
  of a manual inspection.
\item
  \textbf{Reinforcement Learning}:\\
  Similar to active learning, in that the machine may query for labels.
  Different from active learning, in that the machine does not receive
  labels, but \emph{rewards}.
\item
  \textbf{Structure Learning}: An instance of supervised learning where
  we predict objects with structure such as dependent vectors, graphs,
  images, tensors, etc.
\item
  \textbf{Online Learning}: An instance of supervised learning, where we
  need to make predictions where data inputs as a stream.
\item
  \textbf{Transduction}: An instance of supervised learning where we
  need to make predictions for a new set of predictors, but which are
  known at the time of learning. Can be thought of as semi-supervised
  \emph{extrapolation}.
\item
  \textbf{Covariate shift}: An instance of supervised learning where we
  need to make predictions for a set of predictors that ha a different
  distribution than the data generating source.
\item
  \textbf{Targeted Learning}: A form of supervised learning, designed at
  causal inference for decision making.
\item
  \textbf{Co-training}: An instance of supervised learning where we
  solve several problems, and exploit some assumed relation between the
  problems.
\item
  \textbf{Manifold learning}: An instance of unsupervised learning,
  where the goal is to reduce the dimension of the data by embedding it
  into a lower dimensional manifold. A.k.a. \emph{support estimation}.
\item
  \textbf{Similarity Learning}: Where we try to learn how to measure
  similarity between objects (like faces, texts, images, etc.).
\item
  \textbf{Metric Learning}: Like \emph{similarity learning}, only that
  the similarity has to obey the definition of a \emph{metric}.
\item
  \textbf{Learning to learn}: Deals with the carriage of ``experience''
  from one learning problem to another. A.k.a. \emph{cummulative
  learning}, \emph{knowledge transfer}, and \emph{meta learning}.
\end{itemize}

\section{Problem Setup}\label{problem-setup-3}

We now present the \emph{empirical risk minimization} (ERM) approach to
supervised learning, a.k.a. \emph{M-estimation} in the statistical
literature.

\BeginKnitrBlock{remark}
\iffalse{} {Remark. } \fi{}We do not discuss purely algorithmic
approaches such as K-nearest neighbour and \emph{kernel smoothing} due
to space constraints. For a broader review of supervised learning, see
the Bibliographic Notes.
\EndKnitrBlock{remark}

\BeginKnitrBlock{example}[Rental Prices]
\protect\hypertarget{exm:rental-prices}{}{\label{exm:rental-prices}
\iffalse (Rental Prices) \fi{} }Consider the problem of predicting if a
mail is spam or not based on its attributes: length, number of
exclamation marks, number of recipients, etc.
\EndKnitrBlock{example}

Given \(n\) samples with inputs \(x\) from some space \(\mathcal{X}\)
and desired outcome, \(y\), from some space \(\mathcal{Y}\). In our
example, \(y\) is the spam/no-spam label, and \(x\) is a vector of the
mail's attributes. Samples, \((x,y)\) have some distribution we denote
\(P\). We want to learn a function that maps inputs to outputs, i.e.,
that classifies to spam given. This function is called a
\emph{hypothesis}, or \emph{predictor}, denoted \(f\), that belongs to a
hypothesis class \(\mathcal{F}\) such that
\(f:\mathcal{X} \to \mathcal{Y}\). We also choose some other function
that fines us for erroneous prediction. This function is called the
\emph{loss}, and we denote it by
\(l:\mathcal{Y}\times \mathcal{Y} \to \mathbb{R}^+\).

\BeginKnitrBlock{remark}
\iffalse{} {Remark. } \fi{}The \emph{hypothesis} in machine learning is
only vaguely related the \emph{hypothesis} in statistical testing, which
is quite confusing.
\EndKnitrBlock{remark}

\BeginKnitrBlock{remark}
\iffalse{} {Remark. } \fi{}The \emph{hypothesis} in machine learning is
not a bona-fide \emph{statistical model} since we don't assume it is the
data generating process, but rather some function which we choose for
its good predictive performance.
\EndKnitrBlock{remark}

The fundamental task in supervised (statistical) learning is to recover
a hypothesis that minimizes the average loss in the sample, and not in
the population. This is know as the \emph{risk minimization problem}.

\BeginKnitrBlock{definition}[Risk Function]
\protect\hypertarget{def:unnamed-chunk-236}{}{\label{def:unnamed-chunk-236}
\iffalse (Risk Function) \fi{} }The \emph{risk function}, a.k.a.
\emph{generalization error}, or \emph{test error}, is the population
average loss of a predictor \(f\):

\begin{align}
  R(f):=\mathbb{E}_P[l(f(x),y)].
\end{align}
\EndKnitrBlock{definition}

The best predictor, is the risk minimizer:

\begin{align}
  f^* := argmin_f \{R(f)\}.
  \label{eq:risk}  
\end{align}

Another fundamental problem is that we do not know the distribution of
all possible inputs and outputs, \(P\). We typically only have a sample
of \((x_i,y_i), i=1,\dots,n\). We thus state the \emph{empirical}
counterpart of \eqref{eq:risk}, which consists of minimizing the average
loss. This is known as the \emph{empirical risk miminization} problem
(ERM).

\BeginKnitrBlock{definition}[Empirical Risk]
\protect\hypertarget{def:unnamed-chunk-237}{}{\label{def:unnamed-chunk-237}
\iffalse (Empirical Risk) \fi{} }The \emph{empirical risk function},
a.k.a. \emph{in-sample error}, or \emph{train error}, is the sample
average loss of a predictor \(f\):

\begin{align}
  R_n(f):= 1/n \sum_i l(f(x_i),y_i).
\end{align}
\EndKnitrBlock{definition}

A good candidate proxy for \(f^*\) is its empirical counterpart,
\(\hat f\), known as the \emph{empirical risk minimizer}:

\begin{align}
  \hat f := argmin_f \{ R_n(f) \}.
  \label{eq:erm}  
\end{align}

To make things more explicit:

\begin{itemize}
\tightlist
\item
  \(f\) may be a linear function of the attributes, so that it may be
  indexed simply with its coefficient vector \(\beta\).
\item
  \(l\) may be a squared error loss: \(l(f(x),y):=(f(x)-y)^2\).
\end{itemize}

Under these conditions, the best predictor \(f^* \in \mathcal{F}\) from
problem \eqref{eq:risk} is to

\begin{align}
  f^* := argmin_\beta \{ \mathbb{E}_{P(x,y)}[(x'\beta-y)^2] \}.
\end{align}

When using a linear hypothesis with squared loss, we see that the
empirical risk minimization problem collapses to an ordinary
least-squares problem:

\begin{align}
  \hat f := argmin_\beta \{1/n \sum_i (x_i'\beta - y_i)^2 \}.
\end{align}

When data samples are assumingly independent, then maximum likelihood
estimation is also an instance of ERM, when using the (negative) log
likelihood as the loss function.

If we don't assume any structure on the hypothesis, \(f\), then
\(\hat f\) from \eqref{eq:erm} will interpolate the data, and \(\hat f\)
will be a very bad predictor. We say, it will \emph{overfit} the
observed data, and will have bad performance on new data.

We have several ways to avoid overfitting:

\begin{enumerate}
\def\labelenumi{\arabic{enumi}.}
\tightlist
\item
  Restrict the hypothesis class \(\mathcal{F}\) (such as linear
  functions).
\item
  Penalize for the complexity of \(f\). The penalty denoted by
  \(\Vert f \Vert\).
\item
  Unbiased risk estimation: \(R_n(f)\) is not an unbiased estimator of
  \(R(f)\). Why? Think of estimating the mean with the sample
  minimum\ldots{} Because \(R_n(f)\) is downward biased, we may add some
  correction term, or compute \(R_n(f)\) on different data than the one
  used to recover \(\hat f\).
\end{enumerate}

Almost all ERM algorithms consist of some combination of all the three
methods above.

\subsection{Common Hypothesis Classes}\label{common-hypothesis-classes}

Some common hypothesis classes, \(\mathcal{F}\), with restricted
complexity, are:

\begin{enumerate}
\def\labelenumi{\arabic{enumi}.}
\item
  \textbf{Linear hypotheses}: such as linear models, GLMs, and (linear)
  support vector machines (SVM).
\item
  \textbf{Neural networks}: a.k.a. \emph{feed-forward} neural nets,
  \emph{artificial} neural nets, and the celebrated class of \emph{deep}
  neural nets.
\item
  \textbf{Tree}: a.k.a. \emph{decision rules}, is a class of hypotheses
  which can be stated as ``if-then'' rules.
\item
  \textbf{Reproducing Kernel Hilbert Space}: a.k.a. RKHS, is a subset of
  ``the space of all functions\footnote{It is even a subset of the
    Hilbert space, itself a subset of the space of all functions.}''
  that is both large enough to capture very complicated relations, but
  small enough so that it is less prone to overfitting, and also
  surprisingly simple to compute with.
\end{enumerate}

\subsection{Common Complexity
Penalties}\label{common-complexity-penalties}

The most common complexity penalty applies to classes that have a finite
dimensional parametric representation, such as the class of linear
predictors, parametrized via its coefficients \(\beta\). In such classes
we may penalize for the norm of the parameters. Common penalties
include:

\begin{enumerate}
\def\labelenumi{\arabic{enumi}.}
\tightlist
\item
  \textbf{Ridge penalty}: penalizing the \(l_2\) norm of the parameter.
  I.e. \(\Vert f \Vert=\Vert \beta \Vert_2^2=\sum_j \beta_j^2\).
\item
  \textbf{LASSO penalty}: penalizing the \(l_1\) norm of the parameter.
  I.e., \(\Vert f \Vert=\Vert \beta \Vert_1=\sum_j |\beta_j|\)
\item
  \textbf{Elastic net}: a combination of the lasso and ridge penalty.
  I.e.
  ,\(\Vert f \Vert= \alpha \Vert \beta \Vert_2^2 + (1-\alpha) \Vert \beta \Vert_1\).
\item
  \textbf{Function Norms}: If the hypothesis class \(\mathcal{F}\) does
  not admit a finite dimensional representation, the penalty is no
  longer a function of the parameters of the function. We may, however,
  penalize not the parametric representation of the function, but rather
  the function itself \(\Vert f \Vert=\sqrt{\int f(t)^2 dt}\).
\end{enumerate}

\subsection{Unbiased Risk Estimation}\label{unbiased-risk-estimation}

The fundamental problem of overfitting, is that the empirical risk,
\(R_n(\hat f)\), is downward biased to the population risk,
\(R(\hat f)\). We can remove this bias in two ways: (a) purely
algorithmic \emph{resampling} approaches, and (b) theory driven
estimators.

\begin{enumerate}
\def\labelenumi{\arabic{enumi}.}
\item
  \textbf{Train-Validate-Test}: The simplest form of algorithmic
  validation is to split the data. A \emph{train} set to
  train/estimate/learn \(\hat f\). A \emph{validation} set to compute
  the out-of-sample expected loss, \(R(\hat f)\), and pick the best
  performing predictor. A \emph{test} sample to compute the
  out-of-sample performance of the selected hypothesis. This is a very
  simple approach, but it is very ``data inefficient'', thus motivating
  the next method.
\item
  \textbf{V-Fold Cross Validation}: By far the most popular algorithmic
  unbiased risk estimator; in \emph{V-fold CV} we ``fold'' the data into
  \(V\) non-overlapping sets. For each of the \(V\) sets, we learn
  \(\hat f\) with the non-selected fold, and assess \(R(\hat f)\)) on
  the selected fold. We then aggregate results over the \(V\) folds,
  typically by averaging.
\item
  \textbf{AIC}: Akaike's information criterion (AIC) is a theory driven
  correction of the empirical risk, so that it is unbiased to the true
  risk. It is appropriate when using the likelihood loss.
\item
  \textbf{Cp}: Mallow's Cp is an instance of AIC for likelihood loss
  under normal noise.
\end{enumerate}

Other theory driven unbiased risk estimators include the \emph{Bayesian
Information Criterion} (BIC, aka SBC, aka SBIC), the \emph{Minimum
Description Length} (MDL), \emph{Vapnic's Structural Risk Minimization}
(SRM), the \emph{Deviance Information Criterion} (DIC), and the
\emph{Hannan-Quinn Information Criterion} (HQC).

Other resampling based unbiased risk estimators include resampling
\textbf{without replacement} algorithms like \emph{delete-d cross
validation} with its many variations, and \textbf{resampling with
replacement}, like the \emph{bootstrap}, with its many variations.

\subsection{Collecting the Pieces}\label{collecting-the-pieces}

An ERM problem with regularization will look like

\begin{align}
  \hat f := argmin_{f \in \mathcal{F}} \{ R_n(f)  + \lambda \Vert f \Vert \}.
  \label{eq:erm-regularized}  
\end{align}

Collecting ideas from the above sections, a typical supervised learning
pipeline will include: choosing the hypothesis class, choosing the
penalty function and level, unbiased risk estimator. We emphasize that
choosing the penalty function, \(\Vert f \Vert\) is not enough, and we
need to choose how ``hard'' to apply it. This if known as the
\emph{regularization level}, denoted by \(\lambda\) in
Eq.\eqref{eq:erm-regularized}.

Examples of such combos include:

\begin{enumerate}
\def\labelenumi{\arabic{enumi}.}
\tightlist
\item
  Linear regression, no penalty, train-validate test.
\item
  Linear regression, no penalty, AIC.
\item
  Linear regression, \(l_2\) penalty, V-fold CV. This combo is typically
  known as \emph{ridge regression}.
\item
  Linear regression, \(l_1\) penalty, V-fold CV. This combo is typically
  known as \emph{LASSO regression}.
\item
  Linear regression, \(l_1\) and \(l_2\) penalty, V-fold CV. This combo
  is typically known as \emph{elastic net regression}.
\item
  Logistic regression, \(l_2\) penalty, V-fold CV.
\item
  SVM classification, \(l_2\) penalty, V-fold CV.
\item
  Deep network, no penalty, V-fold CV.
\item
  Unrestricted, \(\Vert \partial^2 f \Vert_2\), V-fold CV. This combo is
  typically known as a \emph{smoothing spline}.
\end{enumerate}

For fans of statistical hypothesis testing we will also emphasize:
Testing and prediction are related, but are not the same:

\begin{itemize}
\tightlist
\item
  In the current chapter, we do not claim our models, \(f\), are
  generative. I.e., we do not claim that there is some causal relation
  between \(x\) and \(y\). We only claim that \(x\) predicts \(y\).
\item
  It is possible that we will want to ignore a significant predictor,
  and add a non-significant one \citep{foster2004variable}.
\item
  Some authors will use hypothesis testing as an initial screening for
  candidate predictors. This is a useful heuristic, but that is all it
  is-- a heuristic. It may also fail miserably if predictors are
  linearly dependent (a.k.a. multicollinear).
\end{itemize}

\section{Supervised Learning in R}\label{supervised-learning-in-r}

At this point, we have a rich enough language to do supervised learning
with R.

In these examples, I will use two data sets from the
\textbf{ElemStatLearn} package, that accompanies the seminal book by
\citet{friedman2001elements}. I use the \texttt{spam} data for
categorical predictions, and \texttt{prostate} for continuous
predictions. In \texttt{spam} we will try to decide if a mail is spam or
not. In \texttt{prostate} we will try to predict the size of a cancerous
tumor. You can now call \texttt{?prostate} and \texttt{?spam} to learn
more about these data sets.

Some boring pre-processing.

\begin{Shaded}
\begin{Highlighting}[]
\CommentTok{# Preparing prostate data}
\KeywordTok{data}\NormalTok{(}\StringTok{"prostate"}\NormalTok{, }\DataTypeTok{package =} \StringTok{'ElemStatLearn'}\NormalTok{)}
\NormalTok{prostate <-}\StringTok{ }\NormalTok{data.table}\OperatorTok{::}\KeywordTok{data.table}\NormalTok{(prostate)}
\NormalTok{prostate.train <-}\StringTok{ }\NormalTok{prostate[train}\OperatorTok{==}\OtherTok{TRUE}\NormalTok{, }\OperatorTok{-}\StringTok{"train"}\NormalTok{]}
\NormalTok{prostate.test <-}\StringTok{ }\NormalTok{prostate[train}\OperatorTok{!=}\OtherTok{TRUE}\NormalTok{, }\OperatorTok{-}\StringTok{"train"}\NormalTok{]}
\NormalTok{y.train <-}\StringTok{ }\NormalTok{prostate.train}\OperatorTok{$}\NormalTok{lcavol}
\NormalTok{X.train <-}\StringTok{ }\KeywordTok{as.matrix}\NormalTok{(prostate.train[, }\OperatorTok{-}\StringTok{'lcavol'}\NormalTok{] )}
\NormalTok{y.test <-}\StringTok{ }\NormalTok{prostate.test}\OperatorTok{$}\NormalTok{lcavol }
\NormalTok{X.test <-}\StringTok{ }\KeywordTok{as.matrix}\NormalTok{(prostate.test[, }\OperatorTok{-}\StringTok{'lcavol'}\NormalTok{] )}

\CommentTok{# Preparing spam data:}
\KeywordTok{data}\NormalTok{(}\StringTok{"spam"}\NormalTok{, }\DataTypeTok{package =} \StringTok{'ElemStatLearn'}\NormalTok{)}
\NormalTok{n <-}\StringTok{ }\KeywordTok{nrow}\NormalTok{(spam)}
\NormalTok{train.prop <-}\StringTok{ }\FloatTok{0.66}
\NormalTok{train.ind <-}\StringTok{ }\KeywordTok{sample}\NormalTok{(}\DataTypeTok{x =} \KeywordTok{c}\NormalTok{(}\OtherTok{TRUE}\NormalTok{,}\OtherTok{FALSE}\NormalTok{), }
                    \DataTypeTok{size =}\NormalTok{ n, }
                    \DataTypeTok{prob =} \KeywordTok{c}\NormalTok{(train.prop,}\DecValTok{1}\OperatorTok{-}\NormalTok{train.prop), }
                    \DataTypeTok{replace=}\OtherTok{TRUE}\NormalTok{)}
\NormalTok{spam.train <-}\StringTok{ }\NormalTok{spam[train.ind,]}
\NormalTok{spam.test <-}\StringTok{ }\NormalTok{spam[}\OperatorTok{!}\NormalTok{train.ind,]}

\NormalTok{y.train.spam <-}\StringTok{ }\NormalTok{spam.train}\OperatorTok{$}\NormalTok{spam}
\NormalTok{X.train.spam <-}\StringTok{ }\KeywordTok{as.matrix}\NormalTok{(spam.train[,}\KeywordTok{names}\NormalTok{(spam.train)}\OperatorTok{!=}\StringTok{'spam'}\NormalTok{] ) }
\NormalTok{y.test.spam <-}\StringTok{ }\NormalTok{spam.test}\OperatorTok{$}\NormalTok{spam}
\NormalTok{X.test.spam <-}\StringTok{  }\KeywordTok{as.matrix}\NormalTok{(spam.test[,}\KeywordTok{names}\NormalTok{(spam.test)}\OperatorTok{!=}\StringTok{'spam'}\NormalTok{]) }

\NormalTok{spam.dummy <-}\StringTok{ }\NormalTok{spam}
\NormalTok{spam.dummy}\OperatorTok{$}\NormalTok{spam <-}\StringTok{ }\KeywordTok{as.numeric}\NormalTok{(spam}\OperatorTok{$}\NormalTok{spam}\OperatorTok{==}\StringTok{'spam'}\NormalTok{) }
\NormalTok{spam.train.dummy <-}\StringTok{ }\NormalTok{spam.dummy[train.ind,]}
\NormalTok{spam.test.dummy <-}\StringTok{ }\NormalTok{spam.dummy[}\OperatorTok{!}\NormalTok{train.ind,]}
\end{Highlighting}
\end{Shaded}

We also define some utility functions that we will require down the
road.

\begin{Shaded}
\begin{Highlighting}[]
\NormalTok{l2 <-}\StringTok{ }\ControlFlowTok{function}\NormalTok{(x) x}\OperatorTok{^}\DecValTok{2} \OperatorTok\StringTok{ }\NormalTok{sum }\OperatorTok\StringTok{ }\NormalTok{sqrt }
\NormalTok{l1 <-}\StringTok{ }\ControlFlowTok{function}\NormalTok{(x) }\KeywordTok{abs}\NormalTok{(x) }\OperatorTok\StringTok{ }\NormalTok{sum  }
\NormalTok{MSE <-}\StringTok{ }\ControlFlowTok{function}\NormalTok{(x) x}\OperatorTok{^}\DecValTok{2} \OperatorTok\StringTok{ }\NormalTok{mean }
\NormalTok{missclassification <-}\StringTok{ }\ControlFlowTok{function}\NormalTok{(tab) }\KeywordTok{sum}\NormalTok{(tab[}\KeywordTok{c}\NormalTok{(}\DecValTok{2}\NormalTok{,}\DecValTok{3}\NormalTok{)])}\OperatorTok{/}\KeywordTok{sum}\NormalTok{(tab)}
\end{Highlighting}
\end{Shaded}

\subsection{Linear Models with Least Squares Loss}\label{least-squares}

The simplest approach to supervised learning, is simply with OLS: a
linear predictor, squared error loss, and train-test risk estimator.
Notice the better in-sample MSE than the out-of-sample. That is
overfitting in action.

\begin{Shaded}
\begin{Highlighting}[]
\NormalTok{ols.}\DecValTok{1}\NormalTok{ <-}\StringTok{ }\KeywordTok{lm}\NormalTok{(lcavol}\OperatorTok{~}\NormalTok{. ,}\DataTypeTok{data =}\NormalTok{ prostate.train)}
\CommentTok{# Train error:}
\KeywordTok{MSE}\NormalTok{( }\KeywordTok{predict}\NormalTok{(ols.}\DecValTok{1}\NormalTok{)}\OperatorTok{-}\NormalTok{prostate.train}\OperatorTok{$}\NormalTok{lcavol) }
\end{Highlighting}
\end{Shaded}

\begin{verbatim}
## [1] 0.4383709
\end{verbatim}

\begin{Shaded}
\begin{Highlighting}[]
\CommentTok{# Test error:}
\KeywordTok{MSE}\NormalTok{( }\KeywordTok{predict}\NormalTok{(ols.}\DecValTok{1}\NormalTok{, }\DataTypeTok{newdata=}\NormalTok{prostate.test)}\OperatorTok{-}\StringTok{ }\NormalTok{prostate.test}\OperatorTok{$}\NormalTok{lcavol)}
\end{Highlighting}
\end{Shaded}

\begin{verbatim}
## [1] 0.5084068
\end{verbatim}

Things to note:

\begin{itemize}
\tightlist
\item
  I use the \texttt{newdata} argument of the \texttt{predict} function
  to make the out-of-sample predictions required to compute the
  test-error.
\item
  The test error is larger than the train error. That is overfitting in
  action.
\end{itemize}

We now implement a V-fold CV, instead of our train-test approach. The
assignment of each observation to each fold is encoded in
\texttt{fold.assignment}. The following code is extremely inefficient,
but easy to read.

\begin{Shaded}
\begin{Highlighting}[]
\NormalTok{folds <-}\StringTok{ }\DecValTok{10}
\NormalTok{fold.assignment <-}\StringTok{ }\KeywordTok{sample}\NormalTok{(}\DecValTok{1}\OperatorTok{:}\NormalTok{folds, }\KeywordTok{nrow}\NormalTok{(prostate), }\DataTypeTok{replace =} \OtherTok{TRUE}\NormalTok{)}
\NormalTok{errors <-}\StringTok{ }\OtherTok{NULL}

\ControlFlowTok{for}\NormalTok{ (k }\ControlFlowTok{in} \DecValTok{1}\OperatorTok{:}\NormalTok{folds)\{}
\NormalTok{  prostate.cross.train <-}\StringTok{ }\NormalTok{prostate[fold.assignment}\OperatorTok{!=}\NormalTok{k,] }\CommentTok{# train subset}
\NormalTok{  prostate.cross.test <-}\StringTok{  }\NormalTok{prostate[fold.assignment}\OperatorTok{==}\NormalTok{k,] }\CommentTok{# test subset}
\NormalTok{  .ols <-}\StringTok{ }\KeywordTok{lm}\NormalTok{(lcavol}\OperatorTok{~}\NormalTok{. ,}\DataTypeTok{data =}\NormalTok{ prostate.cross.train) }\CommentTok{# train}
\NormalTok{  .predictions <-}\StringTok{ }\KeywordTok{predict}\NormalTok{(.ols, }\DataTypeTok{newdata=}\NormalTok{prostate.cross.test)}
\NormalTok{  .errors <-}\StringTok{  }\NormalTok{.predictions}\OperatorTok{-}\NormalTok{prostate.cross.test}\OperatorTok{$}\NormalTok{lcavol }\CommentTok{# save prediction errors in the fold}
\NormalTok{  errors <-}\StringTok{ }\KeywordTok{c}\NormalTok{(errors, .errors) }\CommentTok{# aggregate error over folds.}
\NormalTok{\}}

\CommentTok{# Cross validated prediction error:}
\KeywordTok{MSE}\NormalTok{(errors)}
\end{Highlighting}
\end{Shaded}

\begin{verbatim}
## [1] 0.5742128
\end{verbatim}

Let's try all possible variable subsets, and choose the best performer
with respect to the Cp criterion, which is an unbiased risk estimator.
This is done with \texttt{leaps::regsubsets}. We see that the best
performer has 3 predictors.

\begin{Shaded}
\begin{Highlighting}[]
\NormalTok{regfit.full <-}\StringTok{ }\NormalTok{prostate.train }\OperatorTok\StringTok{ }
\StringTok{  }\NormalTok{leaps}\OperatorTok{::}\KeywordTok{regsubsets}\NormalTok{(lcavol}\OperatorTok{~}\NormalTok{.,}\DataTypeTok{data =}\NormalTok{ ., }\DataTypeTok{method =} \StringTok{'exhaustive'}\NormalTok{) }\CommentTok{# best subset selection}
\KeywordTok{plot}\NormalTok{(regfit.full, }\DataTypeTok{scale =} \StringTok{"Cp"}\NormalTok{)}
\end{Highlighting}
\end{Shaded}

\includegraphics[width=0.5\linewidth]{Rcourse_files/figure-latex/all subset-1}

Things to note:

\begin{itemize}
\tightlist
\item
  The plot shows us which is the variable combination which is the best,
  i.e., has the smallest Cp.
\item
  Scanning over all variable subsets is impossible when the number of
  variables is large.
\end{itemize}

Instead of the Cp criterion, we now compute the train and test errors
for all the possible predictor subsets\footnote{Example taken from
  \url{https://lagunita.stanford.edu/c4x/HumanitiesScience/StatLearning/asset/ch6.html}}.
In the resulting plot we can see overfitting in action.

\begin{Shaded}
\begin{Highlighting}[]
\NormalTok{model.n <-}\StringTok{ }\NormalTok{regfit.full }\OperatorTok\StringTok{ }\NormalTok{summary }\OperatorTok\StringTok{ }\NormalTok{length}
\NormalTok{X.train.named <-}\StringTok{ }\KeywordTok{model.matrix}\NormalTok{(lcavol }\OperatorTok{~}\StringTok{ }\NormalTok{., }\DataTypeTok{data =}\NormalTok{ prostate.train ) }
\NormalTok{X.test.named <-}\StringTok{ }\KeywordTok{model.matrix}\NormalTok{(lcavol }\OperatorTok{~}\StringTok{ }\NormalTok{., }\DataTypeTok{data =}\NormalTok{ prostate.test ) }

\NormalTok{val.errors <-}\StringTok{ }\KeywordTok{rep}\NormalTok{(}\OtherTok{NA}\NormalTok{, model.n)}
\NormalTok{train.errors <-}\StringTok{ }\KeywordTok{rep}\NormalTok{(}\OtherTok{NA}\NormalTok{, model.n)}
\ControlFlowTok{for}\NormalTok{ (i }\ControlFlowTok{in} \DecValTok{1}\OperatorTok{:}\NormalTok{model.n) \{}
\NormalTok{    coefi <-}\StringTok{ }\KeywordTok{coef}\NormalTok{(regfit.full, }\DataTypeTok{id =}\NormalTok{ i) }\CommentTok{# exctract coefficients of i'th model}
    
\NormalTok{    pred <-}\StringTok{  }\NormalTok{X.train.named[, }\KeywordTok{names}\NormalTok{(coefi)] }\OperatorTok\StringTok{ }\NormalTok{coefi }\CommentTok{# make in-sample predictions}
\NormalTok{    train.errors[i] <-}\StringTok{ }\KeywordTok{MSE}\NormalTok{(y.train }\OperatorTok{-}\StringTok{ }\NormalTok{pred) }\CommentTok{# train errors}

\NormalTok{    pred <-}\StringTok{  }\NormalTok{X.test.named[, }\KeywordTok{names}\NormalTok{(coefi)] }\OperatorTok\StringTok{ }\NormalTok{coefi }\CommentTok{# make out-of-sample predictions}
\NormalTok{    val.errors[i] <-}\StringTok{ }\KeywordTok{MSE}\NormalTok{(y.test }\OperatorTok{-}\StringTok{ }\NormalTok{pred) }\CommentTok{# test errors}
\NormalTok{\}}
\end{Highlighting}
\end{Shaded}

Plotting results.

\begin{Shaded}
\begin{Highlighting}[]
\KeywordTok{plot}\NormalTok{(train.errors, }\DataTypeTok{ylab =} \StringTok{"MSE"}\NormalTok{, }\DataTypeTok{pch =} \DecValTok{19}\NormalTok{, }\DataTypeTok{type =} \StringTok{"o"}\NormalTok{)}
\KeywordTok{points}\NormalTok{(val.errors, }\DataTypeTok{pch =} \DecValTok{19}\NormalTok{, }\DataTypeTok{type =} \StringTok{"b"}\NormalTok{, }\DataTypeTok{col=}\StringTok{"blue"}\NormalTok{)}
\KeywordTok{legend}\NormalTok{(}\StringTok{"topright"}\NormalTok{, }
       \DataTypeTok{legend =} \KeywordTok{c}\NormalTok{(}\StringTok{"Training"}\NormalTok{, }\StringTok{"Validation"}\NormalTok{), }
       \DataTypeTok{col =} \KeywordTok{c}\NormalTok{(}\StringTok{"black"}\NormalTok{, }\StringTok{"blue"}\NormalTok{), }
       \DataTypeTok{pch =} \DecValTok{19}\NormalTok{)}
\end{Highlighting}
\end{Shaded}

\includegraphics[width=0.5\linewidth]{Rcourse_files/figure-latex/unnamed-chunk-239-1}

Checking all possible models is computationally very hard. \emph{Forward
selection} is a greedy approach that adds one variable at a time.

\begin{Shaded}
\begin{Highlighting}[]
\NormalTok{ols.}\DecValTok{0}\NormalTok{ <-}\StringTok{ }\KeywordTok{lm}\NormalTok{(lcavol}\OperatorTok{~}\DecValTok{1}\NormalTok{ ,}\DataTypeTok{data =}\NormalTok{ prostate.train)}
\NormalTok{model.scope <-}\StringTok{ }\KeywordTok{list}\NormalTok{(}\DataTypeTok{upper=}\NormalTok{ols.}\DecValTok{1}\NormalTok{, }\DataTypeTok{lower=}\NormalTok{ols.}\DecValTok{0}\NormalTok{)}
\KeywordTok{step}\NormalTok{(ols.}\DecValTok{0}\NormalTok{, }\DataTypeTok{scope=}\NormalTok{model.scope, }\DataTypeTok{direction=}\StringTok{'forward'}\NormalTok{, }\DataTypeTok{trace =} \OtherTok{TRUE}\NormalTok{)}
\end{Highlighting}
\end{Shaded}

\begin{verbatim}
## Start:  AIC=30.1
## lcavol ~ 1
## 
##           Df Sum of Sq     RSS     AIC
## + lpsa     1    54.776  47.130 -19.570
## + lcp      1    48.805  53.101 -11.578
## + svi      1    35.829  66.077   3.071
## + pgg45    1    23.789  78.117  14.285
## + gleason  1    18.529  83.377  18.651
## + lweight  1     9.186  92.720  25.768
## + age      1     8.354  93.552  26.366
## <none>                 101.906  30.097
## + lbph     1     0.407 101.499  31.829
## 
## Step:  AIC=-19.57
## lcavol ~ lpsa
## 
##           Df Sum of Sq    RSS     AIC
## + lcp      1   14.8895 32.240 -43.009
## + svi      1    5.0373 42.093 -25.143
## + gleason  1    3.5500 43.580 -22.817
## + pgg45    1    3.0503 44.080 -22.053
## + lbph     1    1.8389 45.291 -20.236
## + age      1    1.5329 45.597 -19.785
## <none>                 47.130 -19.570
## + lweight  1    0.4106 46.719 -18.156
## 
## Step:  AIC=-43.01
## lcavol ~ lpsa + lcp
## 
##           Df Sum of Sq    RSS     AIC
## <none>                 32.240 -43.009
## + age      1   0.92315 31.317 -42.955
## + pgg45    1   0.29594 31.944 -41.627
## + gleason  1   0.21500 32.025 -41.457
## + lbph     1   0.13904 32.101 -41.298
## + lweight  1   0.05504 32.185 -41.123
## + svi      1   0.02069 32.220 -41.052
\end{verbatim}

\begin{verbatim}
## 
## Call:
## lm(formula = lcavol ~ lpsa + lcp, data = prostate.train)
## 
## Coefficients:
## (Intercept)         lpsa          lcp  
##     0.08798      0.53369      0.38879
\end{verbatim}

Things to note:

\begin{itemize}
\tightlist
\item
  By default \texttt{step} add variables according to the
  \href{https://en.wikipedia.org/wiki/Akaike_information_criterion}{AIC}
  criterion, which is a theory-driven unbiased risk estimator.
\item
  We need to tell \texttt{step} which is the smallest and largest models
  to consider using the \texttt{scope} argument.
\item
  \texttt{direction=\textquotesingle{}forward\textquotesingle{}} is used
  to ``grow'' from a small model. For ``shrinking'' a large model, use
  \texttt{direction=\textquotesingle{}backward\textquotesingle{}}, or
  the default
  \texttt{direction=\textquotesingle{}stepwise\textquotesingle{}}.
\end{itemize}

We now learn a linear predictor on the \texttt{spam} data using, a least
squares loss, and train-test risk estimator.

\begin{Shaded}
\begin{Highlighting}[]
\CommentTok{# Train the predictor}
\NormalTok{ols.}\DecValTok{2}\NormalTok{ <-}\StringTok{ }\KeywordTok{lm}\NormalTok{(spam}\OperatorTok{~}\NormalTok{., }\DataTypeTok{data =}\NormalTok{ spam.train.dummy) }

\CommentTok{# make in-sample predictions}
\NormalTok{.predictions.train <-}\StringTok{ }\KeywordTok{predict}\NormalTok{(ols.}\DecValTok{2}\NormalTok{) }\OperatorTok{>}\StringTok{ }\FloatTok{0.5} 
\CommentTok{# inspect the confusion matrix}
\NormalTok{(confusion.train <-}\StringTok{ }\KeywordTok{table}\NormalTok{(}\DataTypeTok{prediction=}\NormalTok{.predictions.train, }\DataTypeTok{truth=}\NormalTok{spam.train.dummy}\OperatorTok{$}\NormalTok{spam)) }
\end{Highlighting}
\end{Shaded}

\begin{verbatim}
##           truth
## prediction    0    1
##      FALSE 1778  227
##      TRUE    66  980
\end{verbatim}

\begin{Shaded}
\begin{Highlighting}[]
\CommentTok{# compute the train (in sample) misclassification}
\KeywordTok{missclassification}\NormalTok{(confusion.train) }
\end{Highlighting}
\end{Shaded}

\begin{verbatim}
## [1] 0.09603409
\end{verbatim}

\begin{Shaded}
\begin{Highlighting}[]
\CommentTok{# make out-of-sample prediction}
\NormalTok{.predictions.test <-}\StringTok{ }\KeywordTok{predict}\NormalTok{(ols.}\DecValTok{2}\NormalTok{, }\DataTypeTok{newdata =}\NormalTok{ spam.test.dummy) }\OperatorTok{>}\StringTok{ }\FloatTok{0.5} 
\CommentTok{# inspect the confusion matrix}
\NormalTok{(confusion.test <-}\StringTok{ }\KeywordTok{table}\NormalTok{(}\DataTypeTok{prediction=}\NormalTok{.predictions.test, }\DataTypeTok{truth=}\NormalTok{spam.test.dummy}\OperatorTok{$}\NormalTok{spam))}
\end{Highlighting}
\end{Shaded}

\begin{verbatim}
##           truth
## prediction   0   1
##      FALSE 884 139
##      TRUE   60 467
\end{verbatim}

\begin{Shaded}
\begin{Highlighting}[]
\CommentTok{# compute the train (in sample) misclassification}
\KeywordTok{missclassification}\NormalTok{(confusion.test)}
\end{Highlighting}
\end{Shaded}

\begin{verbatim}
## [1] 0.1283871
\end{verbatim}

Things to note:

\begin{itemize}
\tightlist
\item
  I can use \texttt{lm} for categorical outcomes. \texttt{lm} will
  simply dummy-code the outcome.
\item
  A linear predictor trained on 0's and 1's will predict numbers. Think
  of these numbers as the probability of 1, and my prediction is the
  most probable class: \texttt{predicts()\textgreater{}0.5}.
\item
  The train error is smaller than the test error. This is overfitting in
  action.
\end{itemize}

The \texttt{glmnet} package is an excellent package that provides ridge,
LASSO, and elastic net regularization, for all GLMs, so for linear
models in particular.

\begin{Shaded}
\begin{Highlighting}[]
\KeywordTok{suppressMessages}\NormalTok{(}\KeywordTok{library}\NormalTok{(glmnet))}

\NormalTok{means <-}\StringTok{ }\KeywordTok{apply}\NormalTok{(X.train, }\DecValTok{2}\NormalTok{, mean)}
\NormalTok{sds <-}\StringTok{ }\KeywordTok{apply}\NormalTok{(X.train, }\DecValTok{2}\NormalTok{, sd)}
\NormalTok{X.train.scaled <-}\StringTok{ }\NormalTok{X.train }\OperatorTok\StringTok{ }\KeywordTok{sweep}\NormalTok{(}\DataTypeTok{MARGIN =} \DecValTok{2}\NormalTok{, }\DataTypeTok{STATS =}\NormalTok{ means, }\DataTypeTok{FUN =} \StringTok{`}\DataTypeTok{-}\StringTok{`}\NormalTok{) }\OperatorTok\StringTok{ }
\StringTok{  }\KeywordTok{sweep}\NormalTok{(}\DataTypeTok{MARGIN =} \DecValTok{2}\NormalTok{, }\DataTypeTok{STATS =}\NormalTok{ sds, }\DataTypeTok{FUN =} \StringTok{`}\DataTypeTok{/}\StringTok{`}\NormalTok{)}

\NormalTok{ridge.}\DecValTok{2}\NormalTok{ <-}\StringTok{ }\KeywordTok{glmnet}\NormalTok{(}\DataTypeTok{x=}\NormalTok{X.train.scaled, }\DataTypeTok{y=}\NormalTok{y.train, }\DataTypeTok{family =} \StringTok{'gaussian'}\NormalTok{, }\DataTypeTok{alpha =} \DecValTok{0}\NormalTok{)}

\CommentTok{# Train error:}
\KeywordTok{MSE}\NormalTok{( }\KeywordTok{predict}\NormalTok{(ridge.}\DecValTok{2}\NormalTok{, }\DataTypeTok{newx =}\NormalTok{X.train.scaled)}\OperatorTok{-}\StringTok{ }\NormalTok{y.train)}
\end{Highlighting}
\end{Shaded}

\begin{verbatim}
## [1] 1.006028
\end{verbatim}

\begin{Shaded}
\begin{Highlighting}[]
\CommentTok{# Test error:}
\NormalTok{X.test.scaled <-}\StringTok{ }\NormalTok{X.test }\OperatorTok\StringTok{ }\KeywordTok{sweep}\NormalTok{(}\DataTypeTok{MARGIN =} \DecValTok{2}\NormalTok{, }\DataTypeTok{STATS =}\NormalTok{ means, }\DataTypeTok{FUN =} \StringTok{`}\DataTypeTok{-}\StringTok{`}\NormalTok{) }\OperatorTok\StringTok{ }
\StringTok{  }\KeywordTok{sweep}\NormalTok{(}\DataTypeTok{MARGIN =} \DecValTok{2}\NormalTok{, }\DataTypeTok{STATS =}\NormalTok{ sds, }\DataTypeTok{FUN =} \StringTok{`}\DataTypeTok{/}\StringTok{`}\NormalTok{)}
\KeywordTok{MSE}\NormalTok{(}\KeywordTok{predict}\NormalTok{(ridge.}\DecValTok{2}\NormalTok{, }\DataTypeTok{newx =}\NormalTok{ X.test.scaled)}\OperatorTok{-}\StringTok{ }\NormalTok{y.test)}
\end{Highlighting}
\end{Shaded}

\begin{verbatim}
## [1] 0.7678264
\end{verbatim}

Things to note:

\begin{itemize}
\tightlist
\item
  The \texttt{alpha=0} parameters tells R to do ridge regression.
  Setting \(alpha=1\) will do LASSO, and any other value, with return an
  elastic net with appropriate weights.
\item
  The \texttt{family=\textquotesingle{}gaussian\textquotesingle{}}
  argument tells R to fit a linear model, with least squares loss.
\item
  Features for regularized predictors should be z-scored before
  learning.
\item
  We use the \texttt{sweep} function to z-score the predictors: we learn
  the z-scoring from the train set, and apply it to both the train and
  the test.
\item
  The test error is \textbf{smaller} than the train error. This may
  happen because risk estimators are random. Their variance may mask the
  overfitting.
\end{itemize}

We now use the LASSO penalty.

\begin{Shaded}
\begin{Highlighting}[]
\NormalTok{lasso.}\DecValTok{1}\NormalTok{ <-}\StringTok{ }\KeywordTok{glmnet}\NormalTok{(}\DataTypeTok{x=}\NormalTok{X.train.scaled, }\DataTypeTok{y=}\NormalTok{y.train, , }\DataTypeTok{family=}\StringTok{'gaussian'}\NormalTok{, }\DataTypeTok{alpha =} \DecValTok{1}\NormalTok{)}

\CommentTok{# Train error:}
\KeywordTok{MSE}\NormalTok{( }\KeywordTok{predict}\NormalTok{(lasso.}\DecValTok{1}\NormalTok{, }\DataTypeTok{newx =}\NormalTok{X.train.scaled)}\OperatorTok{-}\StringTok{ }\NormalTok{y.train)}
\end{Highlighting}
\end{Shaded}

\begin{verbatim}
## [1] 0.5525279
\end{verbatim}

\begin{Shaded}
\begin{Highlighting}[]
\CommentTok{# Test error:}
\KeywordTok{MSE}\NormalTok{( }\KeywordTok{predict}\NormalTok{(lasso.}\DecValTok{1}\NormalTok{, }\DataTypeTok{newx =}\NormalTok{ X.test.scaled)}\OperatorTok{-}\StringTok{ }\NormalTok{y.test)}
\end{Highlighting}
\end{Shaded}

\begin{verbatim}
## [1] 0.5211263
\end{verbatim}

We now use \texttt{glmnet} for classification.

\begin{Shaded}
\begin{Highlighting}[]
\NormalTok{means.spam <-}\StringTok{ }\KeywordTok{apply}\NormalTok{(X.train.spam, }\DecValTok{2}\NormalTok{, mean)}
\NormalTok{sds.spam <-}\StringTok{ }\KeywordTok{apply}\NormalTok{(X.train.spam, }\DecValTok{2}\NormalTok{, sd)}
\NormalTok{X.train.spam.scaled <-}\StringTok{ }\NormalTok{X.train.spam }\OperatorTok\StringTok{ }\KeywordTok{sweep}\NormalTok{(}\DataTypeTok{MARGIN =} \DecValTok{2}\NormalTok{, }\DataTypeTok{STATS =}\NormalTok{ means.spam, }\DataTypeTok{FUN =} \StringTok{`}\DataTypeTok{-}\StringTok{`}\NormalTok{) }\OperatorTok\StringTok{ }
\StringTok{  }\KeywordTok{sweep}\NormalTok{(}\DataTypeTok{MARGIN =} \DecValTok{2}\NormalTok{, }\DataTypeTok{STATS =}\NormalTok{ sds.spam, }\DataTypeTok{FUN =} \StringTok{`}\DataTypeTok{/}\StringTok{`}\NormalTok{) }\OperatorTok\StringTok{ }\NormalTok{as.matrix}

\NormalTok{logistic.}\DecValTok{2}\NormalTok{ <-}\StringTok{ }\KeywordTok{cv.glmnet}\NormalTok{(}\DataTypeTok{x=}\NormalTok{X.train.spam.scaled, }\DataTypeTok{y=}\NormalTok{y.train.spam, }\DataTypeTok{family =} \StringTok{"binomial"}\NormalTok{, }\DataTypeTok{alpha =} \DecValTok{0}\NormalTok{)}
\end{Highlighting}
\end{Shaded}

Things to note:

\begin{itemize}
\tightlist
\item
  We used \texttt{cv.glmnet} to do an automatic search for the optimal
  level of regularization (the \texttt{lambda} argument in
  \texttt{glmnet}) using V-fold CV.
\item
  Just like the \texttt{glm} function,
  \texttt{\textquotesingle{}family=\textquotesingle{}binomial\textquotesingle{}}
  is used for logistic regression.
\item
  We z-scored features so that they all have the same scale.
\item
  We set \texttt{alpha=0} for an \(l_2\) penalization of the
  coefficients of the logistic regression.
\end{itemize}

\begin{Shaded}
\begin{Highlighting}[]
\CommentTok{# Train confusion matrix:}
\NormalTok{.predictions.train <-}\StringTok{ }\KeywordTok{predict}\NormalTok{(logistic.}\DecValTok{2}\NormalTok{, }\DataTypeTok{newx =}\NormalTok{ X.train.spam.scaled, }\DataTypeTok{type =} \StringTok{'class'}\NormalTok{) }
\NormalTok{(confusion.train <-}\StringTok{ }\KeywordTok{table}\NormalTok{(}\DataTypeTok{prediction=}\NormalTok{.predictions.train, }\DataTypeTok{truth=}\NormalTok{spam.train}\OperatorTok{$}\NormalTok{spam))}
\end{Highlighting}
\end{Shaded}

\begin{verbatim}
##           truth
## prediction email spam
##      email  1778  167
##      spam     66 1040
\end{verbatim}

\begin{Shaded}
\begin{Highlighting}[]
\CommentTok{# Train misclassification error}
\KeywordTok{missclassification}\NormalTok{(confusion.train)}
\end{Highlighting}
\end{Shaded}

\begin{verbatim}
## [1] 0.0763684
\end{verbatim}

\begin{Shaded}
\begin{Highlighting}[]
\CommentTok{# Test confusion matrix:}
\NormalTok{X.test.spam.scaled <-}\StringTok{ }\NormalTok{X.test.spam }\OperatorTok\StringTok{ }\KeywordTok{sweep}\NormalTok{(}\DataTypeTok{MARGIN =} \DecValTok{2}\NormalTok{, }\DataTypeTok{STATS =}\NormalTok{ means.spam, }\DataTypeTok{FUN =} \StringTok{`}\DataTypeTok{-}\StringTok{`}\NormalTok{) }\OperatorTok\StringTok{ }
\StringTok{  }\KeywordTok{sweep}\NormalTok{(}\DataTypeTok{MARGIN =} \DecValTok{2}\NormalTok{, }\DataTypeTok{STATS =}\NormalTok{ sds.spam, }\DataTypeTok{FUN =} \StringTok{`}\DataTypeTok{/}\StringTok{`}\NormalTok{) }\OperatorTok\StringTok{ }\NormalTok{as.matrix}

\NormalTok{.predictions.test <-}\StringTok{ }\KeywordTok{predict}\NormalTok{(logistic.}\DecValTok{2}\NormalTok{, }\DataTypeTok{newx =}\NormalTok{ X.test.spam.scaled, }\DataTypeTok{type=}\StringTok{'class'}\NormalTok{) }
\NormalTok{(confusion.test <-}\StringTok{ }\KeywordTok{table}\NormalTok{(}\DataTypeTok{prediction=}\NormalTok{.predictions.test, }\DataTypeTok{truth=}\NormalTok{y.test.spam))}
\end{Highlighting}
\end{Shaded}

\begin{verbatim}
##           truth
## prediction email spam
##      email   885  110
##      spam     59  496
\end{verbatim}

\begin{Shaded}
\begin{Highlighting}[]
\CommentTok{# Test misclassification error:}
\KeywordTok{missclassification}\NormalTok{(confusion.test)}
\end{Highlighting}
\end{Shaded}

\begin{verbatim}
## [1] 0.1090323
\end{verbatim}

\subsection{SVM}\label{svm}

A support vector machine (SVM) is a linear hypothesis class with a
particular loss function known as a
\href{https://en.wikipedia.org/wiki/Hinge_loss}{hinge loss}. We learn an
SVM with the \texttt{svm} function from the \textbf{e1071} package,
which is merely a wrapper for the
\href{https://www.csie.ntu.edu.tw/~cjlin/libsvm/}{libsvm} C library; the
most popular implementation of SVM today.

\begin{Shaded}
\begin{Highlighting}[]
\KeywordTok{library}\NormalTok{(e1071)}
\NormalTok{svm.}\DecValTok{1}\NormalTok{ <-}\StringTok{ }\KeywordTok{svm}\NormalTok{(spam}\OperatorTok{~}\NormalTok{., }\DataTypeTok{data =}\NormalTok{ spam.train, }\DataTypeTok{kernel=}\StringTok{'linear'}\NormalTok{)}

\CommentTok{# Train confusion matrix:}
\NormalTok{.predictions.train <-}\StringTok{ }\KeywordTok{predict}\NormalTok{(svm.}\DecValTok{1}\NormalTok{) }
\NormalTok{(confusion.train <-}\StringTok{ }\KeywordTok{table}\NormalTok{(}\DataTypeTok{prediction=}\NormalTok{.predictions.train, }\DataTypeTok{truth=}\NormalTok{spam.train}\OperatorTok{$}\NormalTok{spam))}
\end{Highlighting}
\end{Shaded}

\begin{verbatim}
##           truth
## prediction email spam
##      email  1774  106
##      spam     70 1101
\end{verbatim}

\begin{Shaded}
\begin{Highlighting}[]
\KeywordTok{missclassification}\NormalTok{(confusion.train)}
\end{Highlighting}
\end{Shaded}

\begin{verbatim}
## [1] 0.057686
\end{verbatim}

\begin{Shaded}
\begin{Highlighting}[]
\CommentTok{# Test confusion matrix:}
\NormalTok{.predictions.test <-}\StringTok{ }\KeywordTok{predict}\NormalTok{(svm.}\DecValTok{1}\NormalTok{, }\DataTypeTok{newdata =}\NormalTok{ spam.test) }
\NormalTok{(confusion.test <-}\StringTok{ }\KeywordTok{table}\NormalTok{(}\DataTypeTok{prediction=}\NormalTok{.predictions.test, }\DataTypeTok{truth=}\NormalTok{spam.test}\OperatorTok{$}\NormalTok{spam))}
\end{Highlighting}
\end{Shaded}

\begin{verbatim}
##           truth
## prediction email spam
##      email   876   75
##      spam     68  531
\end{verbatim}

\begin{Shaded}
\begin{Highlighting}[]
\KeywordTok{missclassification}\NormalTok{(confusion.test)}
\end{Highlighting}
\end{Shaded}

\begin{verbatim}
## [1] 0.09225806
\end{verbatim}

We can also use SVM for regression.

\begin{Shaded}
\begin{Highlighting}[]
\NormalTok{svm.}\DecValTok{2}\NormalTok{ <-}\StringTok{ }\KeywordTok{svm}\NormalTok{(lcavol}\OperatorTok{~}\NormalTok{., }\DataTypeTok{data =}\NormalTok{ prostate.train, }\DataTypeTok{kernel=}\StringTok{'linear'}\NormalTok{)}

\CommentTok{# Train error:}
\KeywordTok{MSE}\NormalTok{( }\KeywordTok{predict}\NormalTok{(svm.}\DecValTok{2}\NormalTok{)}\OperatorTok{-}\StringTok{ }\NormalTok{prostate.train}\OperatorTok{$}\NormalTok{lcavol)}
\end{Highlighting}
\end{Shaded}

\begin{verbatim}
## [1] 0.4488577
\end{verbatim}

\begin{Shaded}
\begin{Highlighting}[]
\CommentTok{# Test error:}
\KeywordTok{MSE}\NormalTok{( }\KeywordTok{predict}\NormalTok{(svm.}\DecValTok{2}\NormalTok{, }\DataTypeTok{newdata =}\NormalTok{ prostate.test)}\OperatorTok{-}\StringTok{ }\NormalTok{prostate.test}\OperatorTok{$}\NormalTok{lcavol)}
\end{Highlighting}
\end{Shaded}

\begin{verbatim}
## [1] 0.5547759
\end{verbatim}

Things to note:

\begin{itemize}
\tightlist
\item
  The use of \texttt{kernel=\textquotesingle{}linear\textquotesingle{}}
  forces the predictor to be linear. Various hypothesis classes may be
  used by changing the \texttt{kernel} argument.
\end{itemize}

\subsection{Neural Nets}\label{neural-nets}

Neural nets (non deep) can be fitted, for example, with the
\texttt{nnet} function in the \textbf{nnet} package. We start with a
nnet regression.

\begin{Shaded}
\begin{Highlighting}[]
\KeywordTok{library}\NormalTok{(nnet)}
\NormalTok{nnet.}\DecValTok{1}\NormalTok{ <-}\StringTok{ }\KeywordTok{nnet}\NormalTok{(lcavol}\OperatorTok{~}\NormalTok{., }\DataTypeTok{size=}\DecValTok{20}\NormalTok{, }\DataTypeTok{data=}\NormalTok{prostate.train, }\DataTypeTok{rang =} \FloatTok{0.1}\NormalTok{, }\DataTypeTok{decay =} \FloatTok{5e-4}\NormalTok{, }\DataTypeTok{maxit =} \DecValTok{1000}\NormalTok{, }\DataTypeTok{trace=}\OtherTok{FALSE}\NormalTok{)}

\CommentTok{# Train error:}
\KeywordTok{MSE}\NormalTok{( }\KeywordTok{predict}\NormalTok{(nnet.}\DecValTok{1}\NormalTok{)}\OperatorTok{-}\StringTok{ }\NormalTok{prostate.train}\OperatorTok{$}\NormalTok{lcavol)}
\end{Highlighting}
\end{Shaded}

\begin{verbatim}
## [1] 1.177099
\end{verbatim}

\begin{Shaded}
\begin{Highlighting}[]
\CommentTok{# Test error:}
\KeywordTok{MSE}\NormalTok{( }\KeywordTok{predict}\NormalTok{(nnet.}\DecValTok{1}\NormalTok{, }\DataTypeTok{newdata =}\NormalTok{ prostate.test)}\OperatorTok{-}\StringTok{ }\NormalTok{prostate.test}\OperatorTok{$}\NormalTok{lcavol)}
\end{Highlighting}
\end{Shaded}

\begin{verbatim}
## [1] 1.21175
\end{verbatim}

And nnet classification.

\begin{Shaded}
\begin{Highlighting}[]
\NormalTok{nnet.}\DecValTok{2}\NormalTok{ <-}\StringTok{ }\KeywordTok{nnet}\NormalTok{(spam}\OperatorTok{~}\NormalTok{., }\DataTypeTok{size=}\DecValTok{5}\NormalTok{, }\DataTypeTok{data=}\NormalTok{spam.train, }\DataTypeTok{rang =} \FloatTok{0.1}\NormalTok{, }\DataTypeTok{decay =} \FloatTok{5e-4}\NormalTok{, }\DataTypeTok{maxit =} \DecValTok{1000}\NormalTok{, }\DataTypeTok{trace=}\OtherTok{FALSE}\NormalTok{)}

\CommentTok{# Train confusion matrix:}
\NormalTok{.predictions.train <-}\StringTok{ }\KeywordTok{predict}\NormalTok{(nnet.}\DecValTok{2}\NormalTok{, }\DataTypeTok{type=}\StringTok{'class'}\NormalTok{) }
\NormalTok{(confusion.train <-}\StringTok{ }\KeywordTok{table}\NormalTok{(}\DataTypeTok{prediction=}\NormalTok{.predictions.train, }\DataTypeTok{truth=}\NormalTok{spam.train}\OperatorTok{$}\NormalTok{spam))}
\end{Highlighting}
\end{Shaded}

\begin{verbatim}
##           truth
## prediction email spam
##      email  1806   59
##      spam     38 1148
\end{verbatim}

\begin{Shaded}
\begin{Highlighting}[]
\KeywordTok{missclassification}\NormalTok{(confusion.train)}
\end{Highlighting}
\end{Shaded}

\begin{verbatim}
## [1] 0.03179285
\end{verbatim}

\begin{Shaded}
\begin{Highlighting}[]
\CommentTok{# Test confusion matrix:}
\NormalTok{.predictions.test <-}\StringTok{ }\KeywordTok{predict}\NormalTok{(nnet.}\DecValTok{2}\NormalTok{, }\DataTypeTok{newdata =}\NormalTok{ spam.test, }\DataTypeTok{type=}\StringTok{'class'}\NormalTok{) }
\NormalTok{(confusion.test <-}\StringTok{ }\KeywordTok{table}\NormalTok{(}\DataTypeTok{prediction=}\NormalTok{.predictions.test, }\DataTypeTok{truth=}\NormalTok{spam.test}\OperatorTok{$}\NormalTok{spam))}
\end{Highlighting}
\end{Shaded}

\begin{verbatim}
##           truth
## prediction email spam
##      email   897   64
##      spam     47  542
\end{verbatim}

\begin{Shaded}
\begin{Highlighting}[]
\KeywordTok{missclassification}\NormalTok{(confusion.test)}
\end{Highlighting}
\end{Shaded}

\begin{verbatim}
## [1] 0.0716129
\end{verbatim}

\subsubsection{Deep Neural Nets}\label{deep-neural-nets}

Deep-Neural-Networks are undoubtedly the ``hottest'' topic in
machine-learning and artificial intelligence. This real is too vast to
be covered in this text. We merely refer the reader to the
\href{https://cran.r-project.org/package=tensorflow}{tensorflow} package
documentation as a starting point.

\subsection{Classification and Regression Trees (CART)}\label{trees}

A CART, is not a linear hypothesis class. It partitions the feature
space \(\mathcal{X}\), thus creating a set of if-then rules for
prediction or classification. It is thus particularly useful when you
believe that the predicted classes may change abruptly with small
changes in \(x\).

\subsubsection{The rpart Package}\label{rpart}

This view clarifies the name of the function \texttt{rpart}, which
\emph{recursively partitions} the feature space.

We start with a regression tree.

\begin{Shaded}
\begin{Highlighting}[]
\KeywordTok{library}\NormalTok{(rpart)}
\NormalTok{tree.}\DecValTok{1}\NormalTok{ <-}\StringTok{ }\KeywordTok{rpart}\NormalTok{(lcavol}\OperatorTok{~}\NormalTok{., }\DataTypeTok{data=}\NormalTok{prostate.train)}

\CommentTok{# Train error:}
\KeywordTok{MSE}\NormalTok{( }\KeywordTok{predict}\NormalTok{(tree.}\DecValTok{1}\NormalTok{)}\OperatorTok{-}\StringTok{ }\NormalTok{prostate.train}\OperatorTok{$}\NormalTok{lcavol)}
\end{Highlighting}
\end{Shaded}

\begin{verbatim}
## [1] 0.4909568
\end{verbatim}

\begin{Shaded}
\begin{Highlighting}[]
\CommentTok{# Test error:}
\KeywordTok{MSE}\NormalTok{( }\KeywordTok{predict}\NormalTok{(tree.}\DecValTok{1}\NormalTok{, }\DataTypeTok{newdata =}\NormalTok{ prostate.test)}\OperatorTok{-}\StringTok{ }\NormalTok{prostate.test}\OperatorTok{$}\NormalTok{lcavol)}
\end{Highlighting}
\end{Shaded}

\begin{verbatim}
## [1] 0.5623316
\end{verbatim}

We can use the \textbf{rpart.plot} package to visualize and interpret
the predictor.

\begin{Shaded}
\begin{Highlighting}[]
\NormalTok{rpart.plot}\OperatorTok{::}\KeywordTok{rpart.plot}\NormalTok{(tree.}\DecValTok{1}\NormalTok{)}
\end{Highlighting}
\end{Shaded}

\includegraphics[width=0.5\linewidth]{Rcourse_files/figure-latex/unnamed-chunk-244-1}

Trees are very prone to overfitting. To avoid this, we reduce a tree's
complexity by \emph{pruning} it. This is done with the
\texttt{rpart::prune} function (not demonstrated herein).

We now fit a classification tree.

\begin{Shaded}
\begin{Highlighting}[]
\NormalTok{tree.}\DecValTok{2}\NormalTok{ <-}\StringTok{ }\KeywordTok{rpart}\NormalTok{(spam}\OperatorTok{~}\NormalTok{., }\DataTypeTok{data=}\NormalTok{spam.train)}

\CommentTok{# Train confusion matrix:}
\NormalTok{.predictions.train <-}\StringTok{ }\KeywordTok{predict}\NormalTok{(tree.}\DecValTok{2}\NormalTok{, }\DataTypeTok{type=}\StringTok{'class'}\NormalTok{) }
\NormalTok{(confusion.train <-}\StringTok{ }\KeywordTok{table}\NormalTok{(}\DataTypeTok{prediction=}\NormalTok{.predictions.train, }\DataTypeTok{truth=}\NormalTok{spam.train}\OperatorTok{$}\NormalTok{spam))}
\end{Highlighting}
\end{Shaded}

\begin{verbatim}
##           truth
## prediction email spam
##      email  1785  217
##      spam     59  990
\end{verbatim}

\begin{Shaded}
\begin{Highlighting}[]
\KeywordTok{missclassification}\NormalTok{(confusion.train)}
\end{Highlighting}
\end{Shaded}

\begin{verbatim}
## [1] 0.09046214
\end{verbatim}

\begin{Shaded}
\begin{Highlighting}[]
\CommentTok{# Test confusion matrix:}
\NormalTok{.predictions.test <-}\StringTok{ }\KeywordTok{predict}\NormalTok{(tree.}\DecValTok{2}\NormalTok{, }\DataTypeTok{newdata =}\NormalTok{ spam.test, }\DataTypeTok{type=}\StringTok{'class'}\NormalTok{) }
\NormalTok{(confusion.test <-}\StringTok{ }\KeywordTok{table}\NormalTok{(}\DataTypeTok{prediction=}\NormalTok{.predictions.test, }\DataTypeTok{truth=}\NormalTok{spam.test}\OperatorTok{$}\NormalTok{spam))}
\end{Highlighting}
\end{Shaded}

\begin{verbatim}
##           truth
## prediction email spam
##      email   906  125
##      spam     38  481
\end{verbatim}

\begin{Shaded}
\begin{Highlighting}[]
\KeywordTok{missclassification}\NormalTok{(confusion.test)}
\end{Highlighting}
\end{Shaded}

\begin{verbatim}
## [1] 0.1051613
\end{verbatim}

\subsubsection{The caret Package}\label{caret}

In the \textbf{rpart} package {[}\ref{rpart}{]} we grow a tree with one
function, and then prune it with another.\\
The \textbf{caret} implementation of trees does both with a single
function. We demonstrate the package in the context of trees, but it is
actually a very convenient wrapper for many learning algorithms;
\href{http://topepo.github.io/caret/available-models.html\#}{237(!)}
learning algorithms to be precise.

\begin{Shaded}
\begin{Highlighting}[]
\KeywordTok{library}\NormalTok{(caret)}
\CommentTok{# Control some training parameters}
\NormalTok{train.control <-}\StringTok{ }\KeywordTok{trainControl}\NormalTok{(}\DataTypeTok{method =} \StringTok{"cv"}\NormalTok{,}
                           \DataTypeTok{number =} \DecValTok{10}\NormalTok{)}

\NormalTok{tree.}\DecValTok{3}\NormalTok{ <-}\StringTok{ }\KeywordTok{train}\NormalTok{(lcavol}\OperatorTok{~}\NormalTok{., }\DataTypeTok{data=}\NormalTok{prostate.train, }
                \DataTypeTok{method=}\StringTok{'rpart'}\NormalTok{, }
                \DataTypeTok{trControl=}\NormalTok{train.control)}
\NormalTok{tree.}\DecValTok{3}
\end{Highlighting}
\end{Shaded}

\begin{verbatim}
## CART 
## 
## 67 samples
##  8 predictor
## 
## No pre-processing
## Resampling: Cross-Validated (10 fold) 
## Summary of sample sizes: 61, 60, 59, 60, 60, 61, ... 
## Resampling results across tuning parameters:
## 
##   cp          RMSE       Rsquared   MAE      
##   0.04682924  0.9118374  0.5026786  0.7570798
##   0.14815712  0.9899308  0.4690557  0.7972803
##   0.44497285  1.1912870  0.3264172  1.0008574
## 
## RMSE was used to select the optimal model using the smallest value.
## The final value used for the model was cp = 0.04682924.
\end{verbatim}

\begin{Shaded}
\begin{Highlighting}[]
\CommentTok{# Train error:}
\KeywordTok{MSE}\NormalTok{( }\KeywordTok{predict}\NormalTok{(tree.}\DecValTok{3}\NormalTok{)}\OperatorTok{-}\StringTok{ }\NormalTok{prostate.train}\OperatorTok{$}\NormalTok{lcavol)}
\end{Highlighting}
\end{Shaded}

\begin{verbatim}
## [1] 0.6188435
\end{verbatim}

\begin{Shaded}
\begin{Highlighting}[]
\CommentTok{# Test error:}
\KeywordTok{MSE}\NormalTok{( }\KeywordTok{predict}\NormalTok{(tree.}\DecValTok{3}\NormalTok{, }\DataTypeTok{newdata =}\NormalTok{ prostate.test)}\OperatorTok{-}\StringTok{ }\NormalTok{prostate.test}\OperatorTok{$}\NormalTok{lcavol)}
\end{Highlighting}
\end{Shaded}

\begin{verbatim}
## [1] 0.545632
\end{verbatim}

Things to note:

\begin{itemize}
\tightlist
\item
  A tree was trained because of the
  \texttt{method=\textquotesingle{}rpart\textquotesingle{}} argument.
  Many other predictive models are available. See
  \href{http://topepo.github.io/caret/available-models.html}{here}.
\item
  The pruning of the tree was done automatically by the
  \texttt{caret::train()} function.
\item
  The method of pruning is controlled by a control object, generated
  with the \texttt{caret::trainControl()} function. In our case,
  \texttt{method\ =\ "cv"} for cross-validation, and
  \texttt{number\ =\ 10} for 10-folds.
\item
  The train error is larger than the test error. This is possible
  because the tree is not an ERM on the train data. Rather, it is an ERM
  on the variations of the data generated by the cross-validation
  process.
\end{itemize}

\subsection{K-nearest neighbour (KNN)}\label{k-nearest-neighbour-knn}

KNN is not an ERM problem. In the KNN algorithm, a prediction at some
\(x\) is made based on the \(y\) is it neighbors. This means that:

\begin{itemize}
\tightlist
\item
  KNN is an
  \href{https://en.wikipedia.org/wiki/Instance-based_learning}{Instance
  Based} learning algorithm where we do not learn the values of some
  parametric function, but rather, need the original sample to make
  predictions. This has many implications when dealing with ``BigData''.
\item
  It may only be applied in spaces with known/defined metric. It is thus
  harder to apply in the presence of missing values, or in
  ``string-spaces'', ``genome-spaces'', etc. where no canonical metric
  exists.
\end{itemize}

KNN is so fundamental that we show how to fit such a hypothesis class,
even if it not an ERM algorithm. Is KNN any good? I have never seen a
learning problem where KNN beats other methods. Others claim
differently.

\begin{Shaded}
\begin{Highlighting}[]
\KeywordTok{library}\NormalTok{(class)}
\NormalTok{knn.}\DecValTok{1}\NormalTok{ <-}\StringTok{ }\KeywordTok{knn}\NormalTok{(}\DataTypeTok{train =}\NormalTok{ X.train.spam.scaled, }\DataTypeTok{test =}\NormalTok{ X.test.spam.scaled, }\DataTypeTok{cl =}\NormalTok{y.train.spam, }\DataTypeTok{k =} \DecValTok{1}\NormalTok{)}

\CommentTok{# Test confusion matrix:}
\NormalTok{.predictions.test <-}\StringTok{ }\NormalTok{knn.}\DecValTok{1} 
\NormalTok{(confusion.test <-}\StringTok{ }\KeywordTok{table}\NormalTok{(}\DataTypeTok{prediction=}\NormalTok{.predictions.test, }\DataTypeTok{truth=}\NormalTok{spam.test}\OperatorTok{$}\NormalTok{spam))}
\end{Highlighting}
\end{Shaded}

\begin{verbatim}
##           truth
## prediction email spam
##      email   856   86
##      spam     88  520
\end{verbatim}

\begin{Shaded}
\begin{Highlighting}[]
\KeywordTok{missclassification}\NormalTok{(confusion.test)}
\end{Highlighting}
\end{Shaded}

\begin{verbatim}
## [1] 0.1122581
\end{verbatim}

\subsection{Linear Discriminant Analysis
(LDA)}\label{linear-discriminant-analysis-lda}

LDA is equivalent to least squares classification \ref{least-squares}.
This means that we actually did LDA when we used \texttt{lm} for binary
classification (feel free to compare the confusion matrices). There are,
however, some dedicated functions to fit it which we now introduce.

\begin{Shaded}
\begin{Highlighting}[]
\KeywordTok{library}\NormalTok{(MASS)}
\NormalTok{lda.}\DecValTok{1}\NormalTok{ <-}\StringTok{ }\KeywordTok{lda}\NormalTok{(spam}\OperatorTok{~}\NormalTok{., spam.train)}

\CommentTok{# Train confusion matrix:}
\NormalTok{.predictions.train <-}\StringTok{ }\KeywordTok{predict}\NormalTok{(lda.}\DecValTok{1}\NormalTok{)}\OperatorTok{$}\NormalTok{class}
\NormalTok{(confusion.train <-}\StringTok{ }\KeywordTok{table}\NormalTok{(}\DataTypeTok{prediction=}\NormalTok{.predictions.train, }\DataTypeTok{truth=}\NormalTok{spam.train}\OperatorTok{$}\NormalTok{spam))}
\end{Highlighting}
\end{Shaded}

\begin{verbatim}
##           truth
## prediction email spam
##      email  1776  227
##      spam     68  980
\end{verbatim}

\begin{Shaded}
\begin{Highlighting}[]
\KeywordTok{missclassification}\NormalTok{(confusion.train)}
\end{Highlighting}
\end{Shaded}

\begin{verbatim}
## [1] 0.09668961
\end{verbatim}

\begin{Shaded}
\begin{Highlighting}[]
\CommentTok{# Test confusion matrix:}
\NormalTok{.predictions.test <-}\StringTok{ }\KeywordTok{predict}\NormalTok{(lda.}\DecValTok{1}\NormalTok{, }\DataTypeTok{newdata =}\NormalTok{ spam.test)}\OperatorTok{$}\NormalTok{class}
\NormalTok{(confusion.test <-}\StringTok{ }\KeywordTok{table}\NormalTok{(}\DataTypeTok{prediction=}\NormalTok{.predictions.test, }\DataTypeTok{truth=}\NormalTok{spam.test}\OperatorTok{$}\NormalTok{spam))}
\end{Highlighting}
\end{Shaded}

\begin{verbatim}
##           truth
## prediction email spam
##      email   884  138
##      spam     60  468
\end{verbatim}

\begin{Shaded}
\begin{Highlighting}[]
\KeywordTok{missclassification}\NormalTok{(confusion.test)}
\end{Highlighting}
\end{Shaded}

\begin{verbatim}
## [1] 0.1277419
\end{verbatim}

\subsection{Naive Bayes}\label{naive-bayes}

Naive-Bayes can be thought of LDA, i.e.~linear regression, where
predictors are assume to be uncorrelated. Predictions may be very good
and certainly very fast, even if this assumption is not true.

\begin{Shaded}
\begin{Highlighting}[]
\KeywordTok{library}\NormalTok{(e1071)}
\NormalTok{nb.}\DecValTok{1}\NormalTok{ <-}\StringTok{ }\KeywordTok{naiveBayes}\NormalTok{(spam}\OperatorTok{~}\NormalTok{., }\DataTypeTok{data =}\NormalTok{ spam.train)}

\CommentTok{# Train confusion matrix:}
\NormalTok{.predictions.train <-}\StringTok{ }\KeywordTok{predict}\NormalTok{(nb.}\DecValTok{1}\NormalTok{, }\DataTypeTok{newdata =}\NormalTok{ spam.train)}
\NormalTok{(confusion.train <-}\StringTok{ }\KeywordTok{table}\NormalTok{(}\DataTypeTok{prediction=}\NormalTok{.predictions.train, }\DataTypeTok{truth=}\NormalTok{spam.train}\OperatorTok{$}\NormalTok{spam))}
\end{Highlighting}
\end{Shaded}

\begin{verbatim}
##           truth
## prediction email spam
##      email  1025   55
##      spam    819 1152
\end{verbatim}

\begin{Shaded}
\begin{Highlighting}[]
\KeywordTok{missclassification}\NormalTok{(confusion.train)}
\end{Highlighting}
\end{Shaded}

\begin{verbatim}
## [1] 0.2864635
\end{verbatim}

\begin{Shaded}
\begin{Highlighting}[]
\CommentTok{# Test confusion matrix:}
\NormalTok{.predictions.test <-}\StringTok{ }\KeywordTok{predict}\NormalTok{(nb.}\DecValTok{1}\NormalTok{, }\DataTypeTok{newdata =}\NormalTok{ spam.test)}
\NormalTok{(confusion.test <-}\StringTok{ }\KeywordTok{table}\NormalTok{(}\DataTypeTok{prediction=}\NormalTok{.predictions.test, }\DataTypeTok{truth=}\NormalTok{spam.test}\OperatorTok{$}\NormalTok{spam))}
\end{Highlighting}
\end{Shaded}

\begin{verbatim}
##           truth
## prediction email spam
##      email   484   42
##      spam    460  564
\end{verbatim}

\begin{Shaded}
\begin{Highlighting}[]
\KeywordTok{missclassification}\NormalTok{(confusion.test)}
\end{Highlighting}
\end{Shaded}

\begin{verbatim}
## [1] 0.323871
\end{verbatim}

\subsection{Random Forrest}\label{random-forrest}

A Random Forrest is one of the most popular supervised learning
algorithms. It it an extremely successful algorithm, with very few
tuning parameters, and easily parallelizable (thus salable to massive
datasets).

\begin{Shaded}
\begin{Highlighting}[]
\CommentTok{# Control some training parameters}
\NormalTok{train.control <-}\StringTok{ }\KeywordTok{trainControl}\NormalTok{(}\DataTypeTok{method =} \StringTok{"cv"}\NormalTok{, }\DataTypeTok{number =} \DecValTok{10}\NormalTok{)}
\NormalTok{rf.}\DecValTok{1}\NormalTok{ <-}\StringTok{ }\NormalTok{caret}\OperatorTok{::}\KeywordTok{train}\NormalTok{(lcavol}\OperatorTok{~}\NormalTok{., }\DataTypeTok{data=}\NormalTok{prostate.train, }
                \DataTypeTok{method=}\StringTok{'rf'}\NormalTok{, }
                \DataTypeTok{trControl=}\NormalTok{train.control)}
\NormalTok{rf.}\DecValTok{1}
\end{Highlighting}
\end{Shaded}

\begin{verbatim}
## Random Forest 
## 
## 67 samples
##  8 predictor
## 
## No pre-processing
## Resampling: Cross-Validated (10 fold) 
## Summary of sample sizes: 62, 59, 60, 60, 59, 61, ... 
## Resampling results across tuning parameters:
## 
##   mtry  RMSE       Rsquared   MAE      
##   2     0.7885535  0.6520820  0.6684168
##   5     0.7782809  0.6687843  0.6550590
##   8     0.7894338  0.6665277  0.6626417
## 
## RMSE was used to select the optimal model using the smallest value.
## The final value used for the model was mtry = 5.
\end{verbatim}

\begin{Shaded}
\begin{Highlighting}[]
\CommentTok{# Train error:}
\KeywordTok{MSE}\NormalTok{( }\KeywordTok{predict}\NormalTok{(rf.}\DecValTok{1}\NormalTok{)}\OperatorTok{-}\StringTok{ }\NormalTok{prostate.train}\OperatorTok{$}\NormalTok{lcavol)}
\end{Highlighting}
\end{Shaded}

\begin{verbatim}
## [1] 0.1340291
\end{verbatim}

\begin{Shaded}
\begin{Highlighting}[]
\CommentTok{# Test error:}
\KeywordTok{MSE}\NormalTok{( }\KeywordTok{predict}\NormalTok{(rf.}\DecValTok{1}\NormalTok{, }\DataTypeTok{newdata =}\NormalTok{ prostate.test)}\OperatorTok{-}\StringTok{ }\NormalTok{prostate.test}\OperatorTok{$}\NormalTok{lcavol)}
\end{Highlighting}
\end{Shaded}

\begin{verbatim}
## [1] 0.5147782
\end{verbatim}

Some of the many many many packages that learn random-forests include:
\href{https://cran.r-project.org/package=randomForest}{randomForest},
\href{https://cran.r-project.org/package=ranger}{ranger}.

\subsection{Boosting}\label{boosting}

The fundamental idea behind \textbf{Boosting} is to construct a
predictor, as the sum of several ``weak'' predictors. These weak
predictors, are not trained on the same data. Instead, each predictor is
trained on the residuals of the previous. Think of it this way: The
first predictor targets the strongest signal. The second targets what
the first did not predict. Etc. At some point, the residuals cannot be
predicted anymore, and the learning will stabilize. Boosting is
typically, but not necessarily, implemented as a sum of trees
(@(trees)).

\subsubsection{The gbm Package}\label{the-gbm-package}

TODO

\subsubsection{The xgboost Package}\label{the-xgboost-package}

TODO

\section{Bibliographic Notes}\label{bibliographic-notes-8}

The ultimate reference on (statistical) machine learning is
\citet{friedman2001elements}. For a softer introduction, see
\citet{james2013introduction}. A statistician will also like
\citet{ripley2007pattern}. For a very algorithmic view, see the seminal
\citet{leskovec2014mining} or \citet{conway2012machine}. For a much more
theoretical reference, see \citet{mohri2012foundations},
\citet{vapnik2013nature}, \citet{shalev2014understanding}. Terminology
taken from \citet{sammut2011encyclopedia}. For an R oriented view see
\citet{lantz2013machine}. For review of other R sources for machine
learning see
\href{http://modernstatisticalworkflow.blogspot.com/2018/01/some-good-introductory-machine-learning.html}{Jim
Savege's post}, or the official
\href{https://cran.r-project.org/web/views/MachineLearning.html}{Task
View}. For a review of resampling based unbiased risk estimation
(i.e.~cross validation) see the exceptional review of
\citet{arlot2010survey}. If you want to know about Deep-Nets in R see
\href{https://www.datacamp.com/community/tutorials/keras-r-deep-learning}{here}.

\section{Practice Yourself}\label{practice-yourself-6}

\begin{enumerate}
\def\labelenumi{\arabic{enumi}.}
\tightlist
\item
  In \ref{practice-glm} we fit a GLM for the \texttt{MASS::epil} data
  (Poisson family). We assume that the number of seizures (\(y\))
  depending on the age of the patient (\texttt{age}) and the treatment
  (\texttt{trt}).

  \begin{enumerate}
  \def\labelenumii{\arabic{enumii}.}
  \tightlist
  \item
    What was the MSE of the model?
  \item
    Now, try the same with a ridge penalty using \texttt{glmnet}
    (\texttt{alpha=0}).
  \item
    Do the same with a LASSO penalty (\texttt{alpha=1}).
  \item
    Compare the test MSE of the three models. Which is the best ?
  \end{enumerate}
\item
  Read about the \texttt{Glass} dataset using
  \texttt{data(Glass,\ package="mlbench")} and \texttt{?Glass}.

  \begin{enumerate}
  \def\labelenumii{\arabic{enumii}.}
  \tightlist
  \item
    Divide the dataset to train set and test set.
  \item
    Apply the various predictors from this chapter, and compare them
    using the proportion of missclassified.
  \end{enumerate}
\end{enumerate}

See DataCamp's
\href{https://www.datacamp.com/courses/supervised-learning-in-r-classification}{Supervised
Learning in R: Classification}, and
\href{https://www.datacamp.com/courses/supervised-learning-in-r-regression}{Supervised
Learning in R: Regression} for more self practice.

\hypertarget{plotting}{\chapter{Plotting}\label{plotting}}

Whether you are doing EDA, or preparing your results for publication,
you need plots. R has many plotting mechanisms, allowing the user a
tremendous amount of flexibility, while abstracting away a lot of the
tedious details. To be concrete, many of the plots in R are simply
impossible to produce with Excel, SPSS, or SAS, and would take a
tremendous amount of work to produce with Python, Java and lower level
programming languages.

In this text, we will focus on two plotting packages. The basic
\textbf{graphics} package, distributed with the base R distribution, and
the \textbf{ggplot2} package.

Before going into the details of the plotting packages, we start with
some philosophy. The \textbf{graphics} package originates from the
mainframe days. Computers had no graphical interface, and the output of
the plot was immediately sent to a printer. Once a plot has been
produced with the \textbf{graphics} package, just like a printed output,
it cannot be queried nor changed, except for further additions.

The philosophy of R is that \textbf{everyting is an object}. The
\textbf{graphics} package does not adhere to this philosophy, and indeed
it was soon augmented with the \textbf{grid} package \citep{Rlanguage},
that treats plots as objects. \textbf{grid} is a low level graphics
interface, and users may be more familiar with the \textbf{lattice}
package built upon it \citep{lattice}.

\textbf{lattice} is very powerful, but soon enough, it was overtaken in
popularity by the \textbf{ggplot2} package \citep{ggplot2}.
\textbf{ggplot2} was the PhD project of \href{http://hadley.nz/}{Hadley
Wickham}, a name to remember\ldots{} Two fundamental ideas underlay
\textbf{ggplot2}: (i) everything is an object, and (ii), plots can be
described by a simple grammar, i.e., a language to describe the building
blocks of the plot. The grammar in \textbf{ggplot2} are is the one
stated by \citet{wilkinson2006grammar}. The objects and grammar of
\textbf{ggplot2} have later evolved to allow more complicated plotting
and in particular, interactive plotting.

Interactive plotting is a very important feature for EDA, and reporting.
The major leap in interactive plotting was made possible by the
advancement of web technologies, such as JavaScript and
\href{https://en.wikipedia.org/wiki/D3.js}{D3.JS}. Why is this? Because
an interactive plot, or report, can be seen as a web-site. Building upon
the capabilities of JavaScript and your web browser to provide the
interactivity, greatly facilitates the development of such plots, as the
programmer can rely on the web-browsers capabilities for interactivity.

\section{The graphics System}\label{the-graphics-system}

The R code from the Basics Chapter \ref{basics} is a demonstration of
the \textbf{graphics} package and plotting system. We make a quick
review of the basics.

\subsection{Using Existing Plotting
Functions}\label{using-existing-plotting-functions}

\subsubsection{Scatter Plot}\label{scatter-plot}

A simple scatter plot.

\begin{Shaded}
\begin{Highlighting}[]
\KeywordTok{attach}\NormalTok{(trees)}
\KeywordTok{plot}\NormalTok{(Girth }\OperatorTok{~}\StringTok{ }\NormalTok{Height)}
\end{Highlighting}
\end{Shaded}

\includegraphics[width=0.5\linewidth]{Rcourse_files/figure-latex/unnamed-chunk-247-1}

Various types of plots.

\begin{Shaded}
\begin{Highlighting}[]
\NormalTok{par.old <-}\StringTok{ }\KeywordTok{par}\NormalTok{(}\DataTypeTok{no.readonly =} \OtherTok{TRUE}\NormalTok{)}
\KeywordTok{par}\NormalTok{(}\DataTypeTok{mfrow=}\KeywordTok{c}\NormalTok{(}\DecValTok{2}\NormalTok{,}\DecValTok{3}\NormalTok{))}
\KeywordTok{plot}\NormalTok{(Girth, }\DataTypeTok{type=}\StringTok{'h'}\NormalTok{, }\DataTypeTok{main=}\StringTok{"type='h'"}\NormalTok{) }
\KeywordTok{plot}\NormalTok{(Girth, }\DataTypeTok{type=}\StringTok{'o'}\NormalTok{, }\DataTypeTok{main=}\StringTok{"type='o'"}\NormalTok{) }
\KeywordTok{plot}\NormalTok{(Girth, }\DataTypeTok{type=}\StringTok{'l'}\NormalTok{, }\DataTypeTok{main=}\StringTok{"type='l'"}\NormalTok{)}
\KeywordTok{plot}\NormalTok{(Girth, }\DataTypeTok{type=}\StringTok{'s'}\NormalTok{, }\DataTypeTok{main=}\StringTok{"type='s'"}\NormalTok{)}
\KeywordTok{plot}\NormalTok{(Girth, }\DataTypeTok{type=}\StringTok{'b'}\NormalTok{, }\DataTypeTok{main=}\StringTok{"type='b'"}\NormalTok{)}
\KeywordTok{plot}\NormalTok{(Girth, }\DataTypeTok{type=}\StringTok{'p'}\NormalTok{, }\DataTypeTok{main=}\StringTok{"type='p'"}\NormalTok{)}
\end{Highlighting}
\end{Shaded}

\includegraphics[width=0.5\linewidth]{Rcourse_files/figure-latex/unnamed-chunk-248-1}

\begin{Shaded}
\begin{Highlighting}[]
\KeywordTok{par}\NormalTok{(par.old)}
\end{Highlighting}
\end{Shaded}

Things to note:

\begin{itemize}
\tightlist
\item
  The \texttt{par} command controls the plotting parameters.
  \texttt{mfrow=c(2,3)} is used to produce a matrix of plots with 2 rows
  and 3 columns.
\item
  The \texttt{par.old} object saves the original plotting setting. It is
  restored after plotting using \texttt{par(par.old)}.
\item
  The \texttt{type} argument controls the type of plot.
\item
  The \texttt{main} argument controls the title.
\item
  See \texttt{?plot} and \texttt{?par} for more options.
\end{itemize}

Control the plotting characters with the \texttt{pch} argument, and size
with the \texttt{cex} argument.

\begin{Shaded}
\begin{Highlighting}[]
\KeywordTok{plot}\NormalTok{(Girth, }\DataTypeTok{pch=}\StringTok{'+'}\NormalTok{, }\DataTypeTok{cex=}\DecValTok{3}\NormalTok{)}
\end{Highlighting}
\end{Shaded}

\includegraphics[width=0.5\linewidth]{Rcourse_files/figure-latex/unnamed-chunk-249-1}

Control the line's type with \texttt{lty} argument, and width with
\texttt{lwd}.

\begin{Shaded}
\begin{Highlighting}[]
\KeywordTok{par}\NormalTok{(}\DataTypeTok{mfrow=}\KeywordTok{c}\NormalTok{(}\DecValTok{2}\NormalTok{,}\DecValTok{3}\NormalTok{))}
\KeywordTok{plot}\NormalTok{(Girth, }\DataTypeTok{type=}\StringTok{'l'}\NormalTok{, }\DataTypeTok{lty=}\DecValTok{1}\NormalTok{, }\DataTypeTok{lwd=}\DecValTok{2}\NormalTok{)}
\KeywordTok{plot}\NormalTok{(Girth, }\DataTypeTok{type=}\StringTok{'l'}\NormalTok{, }\DataTypeTok{lty=}\DecValTok{2}\NormalTok{, }\DataTypeTok{lwd=}\DecValTok{2}\NormalTok{)}
\KeywordTok{plot}\NormalTok{(Girth, }\DataTypeTok{type=}\StringTok{'l'}\NormalTok{, }\DataTypeTok{lty=}\DecValTok{3}\NormalTok{, }\DataTypeTok{lwd=}\DecValTok{2}\NormalTok{)}
\KeywordTok{plot}\NormalTok{(Girth, }\DataTypeTok{type=}\StringTok{'l'}\NormalTok{, }\DataTypeTok{lty=}\DecValTok{4}\NormalTok{, }\DataTypeTok{lwd=}\DecValTok{2}\NormalTok{)}
\KeywordTok{plot}\NormalTok{(Girth, }\DataTypeTok{type=}\StringTok{'l'}\NormalTok{, }\DataTypeTok{lty=}\DecValTok{5}\NormalTok{, }\DataTypeTok{lwd=}\DecValTok{2}\NormalTok{)}
\KeywordTok{plot}\NormalTok{(Girth, }\DataTypeTok{type=}\StringTok{'l'}\NormalTok{, }\DataTypeTok{lty=}\DecValTok{6}\NormalTok{, }\DataTypeTok{lwd=}\DecValTok{2}\NormalTok{)}
\end{Highlighting}
\end{Shaded}

\includegraphics[width=0.5\linewidth]{Rcourse_files/figure-latex/unnamed-chunk-250-1}

Add line by slope and intercept with \texttt{abline}.

\begin{Shaded}
\begin{Highlighting}[]
\KeywordTok{plot}\NormalTok{(Girth)}
\KeywordTok{abline}\NormalTok{(}\DataTypeTok{v=}\DecValTok{14}\NormalTok{, }\DataTypeTok{col=}\StringTok{'red'}\NormalTok{) }\CommentTok{# vertical line at 14.}
\KeywordTok{abline}\NormalTok{(}\DataTypeTok{h=}\DecValTok{9}\NormalTok{, }\DataTypeTok{lty=}\DecValTok{4}\NormalTok{,}\DataTypeTok{lwd=}\DecValTok{4}\NormalTok{, }\DataTypeTok{col=}\StringTok{'pink'}\NormalTok{) }\CommentTok{# horizontal line at 9.}
\KeywordTok{abline}\NormalTok{(}\DataTypeTok{a =} \DecValTok{0}\NormalTok{, }\DataTypeTok{b=}\DecValTok{1}\NormalTok{) }\CommentTok{# linear line with intercept a=0, and slope b=1.}
\end{Highlighting}
\end{Shaded}

\includegraphics[width=0.5\linewidth]{Rcourse_files/figure-latex/unnamed-chunk-251-1}

\begin{Shaded}
\begin{Highlighting}[]
\KeywordTok{plot}\NormalTok{(Girth)}
\KeywordTok{points}\NormalTok{(}\DataTypeTok{x=}\DecValTok{1}\OperatorTok{:}\DecValTok{30}\NormalTok{, }\DataTypeTok{y=}\KeywordTok{rep}\NormalTok{(}\DecValTok{12}\NormalTok{,}\DecValTok{30}\NormalTok{), }\DataTypeTok{cex=}\FloatTok{0.5}\NormalTok{, }\DataTypeTok{col=}\StringTok{'darkblue'}\NormalTok{)}
\KeywordTok{lines}\NormalTok{(}\DataTypeTok{x=}\KeywordTok{rep}\NormalTok{(}\KeywordTok{c}\NormalTok{(}\DecValTok{5}\NormalTok{,}\DecValTok{10}\NormalTok{), }\DecValTok{7}\NormalTok{), }\DataTypeTok{y=}\DecValTok{7}\OperatorTok{:}\DecValTok{20}\NormalTok{, }\DataTypeTok{lty=}\DecValTok{2}\NormalTok{ )}
\KeywordTok{lines}\NormalTok{(}\DataTypeTok{x=}\KeywordTok{rep}\NormalTok{(}\KeywordTok{c}\NormalTok{(}\DecValTok{5}\NormalTok{,}\DecValTok{10}\NormalTok{), }\DecValTok{7}\NormalTok{)}\OperatorTok{+}\DecValTok{2}\NormalTok{, }\DataTypeTok{y=}\DecValTok{7}\OperatorTok{:}\DecValTok{20}\NormalTok{, }\DataTypeTok{lty=}\DecValTok{2}\NormalTok{ )}
\KeywordTok{lines}\NormalTok{(}\DataTypeTok{x=}\KeywordTok{rep}\NormalTok{(}\KeywordTok{c}\NormalTok{(}\DecValTok{5}\NormalTok{,}\DecValTok{10}\NormalTok{), }\DecValTok{7}\NormalTok{)}\OperatorTok{+}\DecValTok{4}\NormalTok{, }\DataTypeTok{y=}\DecValTok{7}\OperatorTok{:}\DecValTok{20}\NormalTok{, }\DataTypeTok{lty=}\DecValTok{2}\NormalTok{ , }\DataTypeTok{col=}\StringTok{'darkgreen'}\NormalTok{)}
\KeywordTok{lines}\NormalTok{(}\DataTypeTok{x=}\KeywordTok{rep}\NormalTok{(}\KeywordTok{c}\NormalTok{(}\DecValTok{5}\NormalTok{,}\DecValTok{10}\NormalTok{), }\DecValTok{7}\NormalTok{)}\OperatorTok{+}\DecValTok{6}\NormalTok{, }\DataTypeTok{y=}\DecValTok{7}\OperatorTok{:}\DecValTok{20}\NormalTok{, }\DataTypeTok{lty=}\DecValTok{4}\NormalTok{ , }\DataTypeTok{col=}\StringTok{'brown'}\NormalTok{, }\DataTypeTok{lwd=}\DecValTok{4}\NormalTok{)}
\end{Highlighting}
\end{Shaded}

\includegraphics[width=0.5\linewidth]{Rcourse_files/figure-latex/unnamed-chunk-252-1}

Things to note:

\begin{itemize}
\tightlist
\item
  \texttt{points} adds points on an existing plot.
\item
  \texttt{lines} adds lines on an existing plot.
\item
  \texttt{col} controls the color of the element. It takes names or
  numbers as argument.
\item
  \texttt{cex} controls the scale of the element. Defaults to
  \texttt{cex=1}.
\end{itemize}

Add other elements.

\begin{Shaded}
\begin{Highlighting}[]
\KeywordTok{plot}\NormalTok{(Girth)}
\KeywordTok{segments}\NormalTok{(}\DataTypeTok{x0=}\KeywordTok{rep}\NormalTok{(}\KeywordTok{c}\NormalTok{(}\DecValTok{5}\NormalTok{,}\DecValTok{10}\NormalTok{), }\DecValTok{7}\NormalTok{), }\DataTypeTok{y0=}\DecValTok{7}\OperatorTok{:}\DecValTok{20}\NormalTok{, }\DataTypeTok{x1=}\KeywordTok{rep}\NormalTok{(}\KeywordTok{c}\NormalTok{(}\DecValTok{5}\NormalTok{,}\DecValTok{10}\NormalTok{), }\DecValTok{7}\NormalTok{)}\OperatorTok{+}\DecValTok{2}\NormalTok{, }\DataTypeTok{y1=}\NormalTok{(}\DecValTok{7}\OperatorTok{:}\DecValTok{20}\NormalTok{)}\OperatorTok{+}\DecValTok{2}\NormalTok{ ) }\CommentTok{# line segments}
\KeywordTok{arrows}\NormalTok{(}\DataTypeTok{x0=}\DecValTok{13}\NormalTok{,}\DataTypeTok{y0=}\DecValTok{16}\NormalTok{,}\DataTypeTok{x1=}\DecValTok{16}\NormalTok{,}\DataTypeTok{y1=}\DecValTok{17}\NormalTok{) }\CommentTok{# arrows}
\KeywordTok{rect}\NormalTok{(}\DataTypeTok{xleft=}\DecValTok{10}\NormalTok{, }\DataTypeTok{ybottom=}\DecValTok{12}\NormalTok{,  }\DataTypeTok{xright=}\DecValTok{12}\NormalTok{, }\DataTypeTok{ytop=}\DecValTok{16}\NormalTok{) }\CommentTok{# rectangle}
\KeywordTok{polygon}\NormalTok{(}\DataTypeTok{x=}\KeywordTok{c}\NormalTok{(}\DecValTok{10}\NormalTok{,}\DecValTok{11}\NormalTok{,}\DecValTok{12}\NormalTok{,}\FloatTok{11.5}\NormalTok{,}\FloatTok{10.5}\NormalTok{), }\DataTypeTok{y=}\KeywordTok{c}\NormalTok{(}\DecValTok{9}\NormalTok{,}\FloatTok{9.5}\NormalTok{,}\DecValTok{10}\NormalTok{,}\FloatTok{10.5}\NormalTok{,}\FloatTok{9.8}\NormalTok{), }\DataTypeTok{col=}\StringTok{'grey'}\NormalTok{) }\CommentTok{# polygon}
\KeywordTok{title}\NormalTok{(}\DataTypeTok{main=}\StringTok{'This plot makes no sense'}\NormalTok{, }\DataTypeTok{sub=}\StringTok{'Or does it?'}\NormalTok{) }
\KeywordTok{mtext}\NormalTok{(}\StringTok{'Printing in the margins'}\NormalTok{, }\DataTypeTok{side=}\DecValTok{2}\NormalTok{) }\CommentTok{# math text}
\KeywordTok{mtext}\NormalTok{(}\KeywordTok{expression}\NormalTok{(alpha}\OperatorTok{==}\KeywordTok{log}\NormalTok{(f[i])), }\DataTypeTok{side=}\DecValTok{4}\NormalTok{)}
\end{Highlighting}
\end{Shaded}

\includegraphics[width=0.5\linewidth]{Rcourse_files/figure-latex/unnamed-chunk-253-1}

Things to note:

\begin{itemize}
\tightlist
\item
  The following functions add the elements they are named after:
  \texttt{segments}, \texttt{arrows}, \texttt{rect}, \texttt{polygon},
  \texttt{title}.
\item
  \texttt{mtext} adds mathematical text, which needs to be wrapped in
  \texttt{expression()}. For more information for mathematical
  annotation see \texttt{?plotmath}.
\end{itemize}

Add a legend.

\begin{Shaded}
\begin{Highlighting}[]
\KeywordTok{plot}\NormalTok{(Girth, }\DataTypeTok{pch=}\StringTok{'G'}\NormalTok{,}\DataTypeTok{ylim=}\KeywordTok{c}\NormalTok{(}\DecValTok{8}\NormalTok{,}\DecValTok{77}\NormalTok{), }\DataTypeTok{xlab=}\StringTok{'Tree number'}\NormalTok{, }\DataTypeTok{ylab=}\StringTok{''}\NormalTok{, }\DataTypeTok{type=}\StringTok{'b'}\NormalTok{, }\DataTypeTok{col=}\StringTok{'blue'}\NormalTok{)}
\KeywordTok{points}\NormalTok{(Volume, }\DataTypeTok{pch=}\StringTok{'V'}\NormalTok{, }\DataTypeTok{type=}\StringTok{'b'}\NormalTok{, }\DataTypeTok{col=}\StringTok{'red'}\NormalTok{)}
\KeywordTok{legend}\NormalTok{(}\DataTypeTok{x=}\DecValTok{2}\NormalTok{, }\DataTypeTok{y=}\DecValTok{70}\NormalTok{, }\DataTypeTok{legend=}\KeywordTok{c}\NormalTok{(}\StringTok{'Girth'}\NormalTok{, }\StringTok{'Volume'}\NormalTok{), }\DataTypeTok{pch=}\KeywordTok{c}\NormalTok{(}\StringTok{'G'}\NormalTok{,}\StringTok{'V'}\NormalTok{), }\DataTypeTok{col=}\KeywordTok{c}\NormalTok{(}\StringTok{'blue'}\NormalTok{,}\StringTok{'red'}\NormalTok{), }\DataTypeTok{bg=}\StringTok{'grey'}\NormalTok{)}
\end{Highlighting}
\end{Shaded}

\includegraphics[width=0.5\linewidth]{Rcourse_files/figure-latex/unnamed-chunk-254-1}

Adjusting Axes with \texttt{xlim} and \texttt{ylim}.

\begin{Shaded}
\begin{Highlighting}[]
\KeywordTok{plot}\NormalTok{(Girth, }\DataTypeTok{xlim=}\KeywordTok{c}\NormalTok{(}\DecValTok{0}\NormalTok{,}\DecValTok{15}\NormalTok{), }\DataTypeTok{ylim=}\KeywordTok{c}\NormalTok{(}\DecValTok{8}\NormalTok{,}\DecValTok{12}\NormalTok{))}
\end{Highlighting}
\end{Shaded}

\includegraphics[width=0.5\linewidth]{Rcourse_files/figure-latex/unnamed-chunk-255-1}

Use \texttt{layout} for complicated plot layouts.

\begin{Shaded}
\begin{Highlighting}[]
\NormalTok{A<-}\KeywordTok{matrix}\NormalTok{(}\KeywordTok{c}\NormalTok{(}\DecValTok{1}\NormalTok{,}\DecValTok{1}\NormalTok{,}\DecValTok{2}\NormalTok{,}\DecValTok{3}\NormalTok{,}\DecValTok{4}\NormalTok{,}\DecValTok{4}\NormalTok{,}\DecValTok{5}\NormalTok{,}\DecValTok{6}\NormalTok{), }\DataTypeTok{byrow=}\OtherTok{TRUE}\NormalTok{, }\DataTypeTok{ncol=}\DecValTok{2}\NormalTok{)}
\KeywordTok{layout}\NormalTok{(A,}\DataTypeTok{heights=}\KeywordTok{c}\NormalTok{(}\DecValTok{1}\OperatorTok{/}\DecValTok{14}\NormalTok{,}\DecValTok{6}\OperatorTok{/}\DecValTok{14}\NormalTok{,}\DecValTok{1}\OperatorTok{/}\DecValTok{14}\NormalTok{,}\DecValTok{6}\OperatorTok{/}\DecValTok{14}\NormalTok{))}

\NormalTok{oma.saved <-}\StringTok{ }\KeywordTok{par}\NormalTok{(}\StringTok{"oma"}\NormalTok{)}
\KeywordTok{par}\NormalTok{(}\DataTypeTok{oma =} \KeywordTok{rep.int}\NormalTok{(}\DecValTok{0}\NormalTok{, }\DecValTok{4}\NormalTok{))}
\KeywordTok{par}\NormalTok{(}\DataTypeTok{oma =}\NormalTok{ oma.saved)}
\NormalTok{o.par <-}\StringTok{ }\KeywordTok{par}\NormalTok{(}\DataTypeTok{mar =} \KeywordTok{rep.int}\NormalTok{(}\DecValTok{0}\NormalTok{, }\DecValTok{4}\NormalTok{))}
\ControlFlowTok{for}\NormalTok{ (i }\ControlFlowTok{in} \KeywordTok{seq_len}\NormalTok{(}\DecValTok{6}\NormalTok{)) \{}
    \KeywordTok{plot.new}\NormalTok{()}
    \KeywordTok{box}\NormalTok{()}
    \KeywordTok{text}\NormalTok{(}\FloatTok{0.5}\NormalTok{, }\FloatTok{0.5}\NormalTok{, }\KeywordTok{paste}\NormalTok{(}\StringTok{'Box no.'}\NormalTok{,i), }\DataTypeTok{cex=}\DecValTok{3}\NormalTok{)}
\NormalTok{\}}
\end{Highlighting}
\end{Shaded}

\includegraphics[width=0.5\linewidth]{Rcourse_files/figure-latex/unnamed-chunk-256-1}

Always detach.

\begin{Shaded}
\begin{Highlighting}[]
\KeywordTok{detach}\NormalTok{(trees)}
\end{Highlighting}
\end{Shaded}

\subsection{Exporting a Plot}\label{exporting-a-plot}

The pipeline for exporting graphics is similar to the export of data.
Instead of the \texttt{write.table} or \texttt{save} functions, we will
use the \texttt{pdf}, \texttt{tiff}, \texttt{png}, functions. Depending
on the type of desired output.

Check and set the working directory.

\begin{Shaded}
\begin{Highlighting}[]
\KeywordTok{getwd}\NormalTok{()}
\KeywordTok{setwd}\NormalTok{(}\StringTok{"/tmp/"}\NormalTok{)}
\end{Highlighting}
\end{Shaded}

Export tiff.

\begin{Shaded}
\begin{Highlighting}[]
\KeywordTok{tiff}\NormalTok{(}\DataTypeTok{filename=}\StringTok{'graphicExample.tiff'}\NormalTok{)}
\KeywordTok{plot}\NormalTok{(}\KeywordTok{rnorm}\NormalTok{(}\DecValTok{100}\NormalTok{))}
\KeywordTok{dev.off}\NormalTok{()}
\end{Highlighting}
\end{Shaded}

Things to note:

\begin{itemize}
\tightlist
\item
  The \texttt{tiff} function tells R to open a .tiff file, and write the
  output of a plot.
\item
  Only a single (the last) plot is saved.
\item
  \texttt{dev.off} to close the tiff device, and return the plotting to
  the R console (or RStudio).
\end{itemize}

If you want to produce several plots, you can use a counter in the
file's name. The counter uses the
\href{https://en.wikipedia.org/wiki/Printf_format_string}{printf} format
string.

\begin{Shaded}
\begin{Highlighting}[]
\KeywordTok{tiff}\NormalTok{(}\DataTypeTok{filename=}\StringTok{'graphicExample%d.tiff'}\NormalTok{) }\CommentTok{#Creates a sequence of files}
\KeywordTok{plot}\NormalTok{(}\KeywordTok{rnorm}\NormalTok{(}\DecValTok{100}\NormalTok{))}
\KeywordTok{boxplot}\NormalTok{(}\KeywordTok{rnorm}\NormalTok{(}\DecValTok{100}\NormalTok{))}
\KeywordTok{hist}\NormalTok{(}\KeywordTok{rnorm}\NormalTok{(}\DecValTok{100}\NormalTok{))}
\KeywordTok{dev.off}\NormalTok{()}
\end{Highlighting}
\end{Shaded}

To see the list of all open devices use \texttt{dev.list()}. To close
\textbf{all} device, (not only the last one), use
\texttt{graphics.off()}.

See \texttt{?pdf} and \texttt{?jpeg} for more info.

\subsection{Fancy graphics Examples}\label{fancy}

\subsubsection{Line Graph}\label{line-graph}

\begin{Shaded}
\begin{Highlighting}[]
\NormalTok{x =}\StringTok{ }\DecValTok{1995}\OperatorTok{:}\DecValTok{2005}
\NormalTok{y =}\StringTok{ }\KeywordTok{c}\NormalTok{(}\FloatTok{81.1}\NormalTok{, }\FloatTok{83.1}\NormalTok{, }\FloatTok{84.3}\NormalTok{, }\FloatTok{85.2}\NormalTok{, }\FloatTok{85.4}\NormalTok{, }\FloatTok{86.5}\NormalTok{, }\FloatTok{88.3}\NormalTok{, }\FloatTok{88.6}\NormalTok{, }\FloatTok{90.8}\NormalTok{, }\FloatTok{91.1}\NormalTok{, }\FloatTok{91.3}\NormalTok{)}
\KeywordTok{plot.new}\NormalTok{()}
\KeywordTok{plot.window}\NormalTok{(}\DataTypeTok{xlim =} \KeywordTok{range}\NormalTok{(x), }\DataTypeTok{ylim =} \KeywordTok{range}\NormalTok{(y))}
\KeywordTok{abline}\NormalTok{(}\DataTypeTok{h =} \OperatorTok{-}\DecValTok{4}\OperatorTok{:}\DecValTok{4}\NormalTok{, }\DataTypeTok{v =} \OperatorTok{-}\DecValTok{4}\OperatorTok{:}\DecValTok{4}\NormalTok{, }\DataTypeTok{col =} \StringTok{"lightgrey"}\NormalTok{)}
\KeywordTok{lines}\NormalTok{(x, y, }\DataTypeTok{lwd =} \DecValTok{2}\NormalTok{)}
\KeywordTok{title}\NormalTok{(}\DataTypeTok{main =} \StringTok{"A Line Graph Example"}\NormalTok{,}
        \DataTypeTok{xlab =} \StringTok{"Time"}\NormalTok{,}
        \DataTypeTok{ylab =} \StringTok{"Quality of R Graphics"}\NormalTok{)}
\KeywordTok{axis}\NormalTok{(}\DecValTok{1}\NormalTok{)}
\KeywordTok{axis}\NormalTok{(}\DecValTok{2}\NormalTok{)}
\KeywordTok{box}\NormalTok{()}
\end{Highlighting}
\end{Shaded}

\includegraphics[width=0.5\linewidth]{Rcourse_files/figure-latex/unnamed-chunk-261-1}

Things to note:

\begin{itemize}
\tightlist
\item
  \texttt{plot.new} creates a new, empty, plotting device.
\item
  \texttt{plot.window} determines the limits of the plotting region.
\item
  \texttt{axis} adds the axes, and \texttt{box} the framing box.
\item
  The rest of the elements, you already know.
\end{itemize}

\subsubsection{Rosette}\label{rosette}

\begin{Shaded}
\begin{Highlighting}[]
\NormalTok{n =}\StringTok{ }\DecValTok{17}
\NormalTok{theta =}\StringTok{ }\KeywordTok{seq}\NormalTok{(}\DecValTok{0}\NormalTok{, }\DecValTok{2} \OperatorTok{*}\StringTok{ }\NormalTok{pi, }\DataTypeTok{length =}\NormalTok{ n }\OperatorTok{+}\StringTok{ }\DecValTok{1}\NormalTok{)[}\DecValTok{1}\OperatorTok{:}\NormalTok{n]}
\NormalTok{x =}\StringTok{ }\KeywordTok{sin}\NormalTok{(theta)}
\NormalTok{y =}\StringTok{ }\KeywordTok{cos}\NormalTok{(theta)}
\NormalTok{v1 =}\StringTok{ }\KeywordTok{rep}\NormalTok{(}\DecValTok{1}\OperatorTok{:}\NormalTok{n, n)}
\NormalTok{v2 =}\StringTok{ }\KeywordTok{rep}\NormalTok{(}\DecValTok{1}\OperatorTok{:}\NormalTok{n, }\KeywordTok{rep}\NormalTok{(n, n))}
\KeywordTok{plot.new}\NormalTok{()}
\KeywordTok{plot.window}\NormalTok{(}\DataTypeTok{xlim =} \KeywordTok{c}\NormalTok{(}\OperatorTok{-}\DecValTok{1}\NormalTok{, }\DecValTok{1}\NormalTok{), }\DataTypeTok{ylim =} \KeywordTok{c}\NormalTok{(}\OperatorTok{-}\DecValTok{1}\NormalTok{, }\DecValTok{1}\NormalTok{), }\DataTypeTok{asp =} \DecValTok{1}\NormalTok{)}
\KeywordTok{segments}\NormalTok{(x[v1], y[v1], x[v2], y[v2])}
\KeywordTok{box}\NormalTok{()}
\end{Highlighting}
\end{Shaded}

\includegraphics[width=0.5\linewidth]{Rcourse_files/figure-latex/unnamed-chunk-262-1}

\subsubsection{Arrows}\label{arrows}

\begin{Shaded}
\begin{Highlighting}[]
\KeywordTok{plot.new}\NormalTok{()}
\KeywordTok{plot.window}\NormalTok{(}\DataTypeTok{xlim =} \KeywordTok{c}\NormalTok{(}\DecValTok{0}\NormalTok{, }\DecValTok{1}\NormalTok{), }\DataTypeTok{ylim =} \KeywordTok{c}\NormalTok{(}\DecValTok{0}\NormalTok{, }\DecValTok{1}\NormalTok{))}
\KeywordTok{arrows}\NormalTok{(.}\DecValTok{05}\NormalTok{, .}\DecValTok{075}\NormalTok{, .}\DecValTok{45}\NormalTok{, .}\DecValTok{9}\NormalTok{, }\DataTypeTok{code =} \DecValTok{1}\NormalTok{)}
\KeywordTok{arrows}\NormalTok{(.}\DecValTok{55}\NormalTok{, .}\DecValTok{9}\NormalTok{, .}\DecValTok{95}\NormalTok{, .}\DecValTok{075}\NormalTok{, }\DataTypeTok{code =} \DecValTok{2}\NormalTok{)}
\KeywordTok{arrows}\NormalTok{(.}\DecValTok{1}\NormalTok{, }\DecValTok{0}\NormalTok{, .}\DecValTok{9}\NormalTok{, }\DecValTok{0}\NormalTok{, }\DataTypeTok{code =} \DecValTok{3}\NormalTok{)}
\KeywordTok{text}\NormalTok{(.}\DecValTok{5}\NormalTok{, }\DecValTok{1}\NormalTok{, }\StringTok{"A"}\NormalTok{, }\DataTypeTok{cex =} \FloatTok{1.5}\NormalTok{)}
\KeywordTok{text}\NormalTok{(}\DecValTok{0}\NormalTok{, }\DecValTok{0}\NormalTok{, }\StringTok{"B"}\NormalTok{, }\DataTypeTok{cex =} \FloatTok{1.5}\NormalTok{)}
\KeywordTok{text}\NormalTok{(}\DecValTok{1}\NormalTok{, }\DecValTok{0}\NormalTok{, }\StringTok{"C"}\NormalTok{, }\DataTypeTok{cex =} \FloatTok{1.5}\NormalTok{)}
\end{Highlighting}
\end{Shaded}

\includegraphics[width=0.5\linewidth]{Rcourse_files/figure-latex/unnamed-chunk-263-1}

\subsubsection{Arrows as error bars}\label{arrows-as-error-bars}

\begin{Shaded}
\begin{Highlighting}[]
\NormalTok{x =}\StringTok{ }\DecValTok{1}\OperatorTok{:}\DecValTok{10}
\NormalTok{y =}\StringTok{ }\KeywordTok{runif}\NormalTok{(}\DecValTok{10}\NormalTok{) }\OperatorTok{+}\StringTok{ }\KeywordTok{rep}\NormalTok{(}\KeywordTok{c}\NormalTok{(}\DecValTok{5}\NormalTok{, }\FloatTok{6.5}\NormalTok{), }\KeywordTok{c}\NormalTok{(}\DecValTok{5}\NormalTok{, }\DecValTok{5}\NormalTok{))}
\NormalTok{yl =}\StringTok{ }\NormalTok{y }\OperatorTok{-}\StringTok{ }\FloatTok{0.25} \OperatorTok{-}\StringTok{ }\KeywordTok{runif}\NormalTok{(}\DecValTok{10}\NormalTok{)}\OperatorTok{/}\DecValTok{3}
\NormalTok{yu =}\StringTok{ }\NormalTok{y }\OperatorTok{+}\StringTok{ }\FloatTok{0.25} \OperatorTok{+}\StringTok{ }\KeywordTok{runif}\NormalTok{(}\DecValTok{10}\NormalTok{)}\OperatorTok{/}\DecValTok{3}
\KeywordTok{plot.new}\NormalTok{()}
\KeywordTok{plot.window}\NormalTok{(}\DataTypeTok{xlim =} \KeywordTok{c}\NormalTok{(}\FloatTok{0.5}\NormalTok{, }\FloatTok{10.5}\NormalTok{), }\DataTypeTok{ylim =} \KeywordTok{range}\NormalTok{(yl, yu))}
\KeywordTok{arrows}\NormalTok{(x, yl, x, yu, }\DataTypeTok{code =} \DecValTok{3}\NormalTok{, }\DataTypeTok{angle =} \DecValTok{90}\NormalTok{, }\DataTypeTok{length =}\NormalTok{ .}\DecValTok{125}\NormalTok{)}
\KeywordTok{points}\NormalTok{(x, y, }\DataTypeTok{pch =} \DecValTok{19}\NormalTok{, }\DataTypeTok{cex =} \FloatTok{1.5}\NormalTok{)}
\KeywordTok{axis}\NormalTok{(}\DecValTok{1}\NormalTok{, }\DataTypeTok{at =} \DecValTok{1}\OperatorTok{:}\DecValTok{10}\NormalTok{, }\DataTypeTok{labels =}\NormalTok{ LETTERS[}\DecValTok{1}\OperatorTok{:}\DecValTok{10}\NormalTok{])}
\KeywordTok{axis}\NormalTok{(}\DecValTok{2}\NormalTok{, }\DataTypeTok{las =} \DecValTok{1}\NormalTok{)}
\KeywordTok{box}\NormalTok{()}
\end{Highlighting}
\end{Shaded}

\includegraphics[width=0.5\linewidth]{Rcourse_files/figure-latex/unnamed-chunk-264-1}

\subsubsection{Histogram}\label{histogram}

A histogram is nothing but a bunch of rectangle elements.

\begin{Shaded}
\begin{Highlighting}[]
\KeywordTok{plot.new}\NormalTok{()}
\KeywordTok{plot.window}\NormalTok{(}\DataTypeTok{xlim =} \KeywordTok{c}\NormalTok{(}\DecValTok{0}\NormalTok{, }\DecValTok{5}\NormalTok{), }\DataTypeTok{ylim =} \KeywordTok{c}\NormalTok{(}\DecValTok{0}\NormalTok{, }\DecValTok{10}\NormalTok{))}
\KeywordTok{rect}\NormalTok{(}\DecValTok{0}\OperatorTok{:}\DecValTok{4}\NormalTok{, }\DecValTok{0}\NormalTok{, }\DecValTok{1}\OperatorTok{:}\DecValTok{5}\NormalTok{, }\KeywordTok{c}\NormalTok{(}\DecValTok{7}\NormalTok{, }\DecValTok{8}\NormalTok{, }\DecValTok{4}\NormalTok{, }\DecValTok{3}\NormalTok{), }\DataTypeTok{col =} \StringTok{"lightblue"}\NormalTok{)}
\KeywordTok{axis}\NormalTok{(}\DecValTok{1}\NormalTok{)}
\KeywordTok{axis}\NormalTok{(}\DecValTok{2}\NormalTok{, }\DataTypeTok{las =} \DecValTok{1}\NormalTok{)}
\end{Highlighting}
\end{Shaded}

\includegraphics[width=0.5\linewidth]{Rcourse_files/figure-latex/unnamed-chunk-265-1}

\paragraph{Spiral Squares}\label{spiral-squares}

\begin{Shaded}
\begin{Highlighting}[]
\KeywordTok{plot.new}\NormalTok{()}
\KeywordTok{plot.window}\NormalTok{(}\DataTypeTok{xlim =} \KeywordTok{c}\NormalTok{(}\OperatorTok{-}\DecValTok{1}\NormalTok{, }\DecValTok{1}\NormalTok{), }\DataTypeTok{ylim =} \KeywordTok{c}\NormalTok{(}\OperatorTok{-}\DecValTok{1}\NormalTok{, }\DecValTok{1}\NormalTok{), }\DataTypeTok{asp =} \DecValTok{1}\NormalTok{)}
\NormalTok{x =}\StringTok{ }\KeywordTok{c}\NormalTok{(}\OperatorTok{-}\DecValTok{1}\NormalTok{, }\DecValTok{1}\NormalTok{, }\DecValTok{1}\NormalTok{, }\OperatorTok{-}\DecValTok{1}\NormalTok{)}
\NormalTok{y =}\StringTok{ }\KeywordTok{c}\NormalTok{( }\DecValTok{1}\NormalTok{, }\DecValTok{1}\NormalTok{, }\OperatorTok{-}\DecValTok{1}\NormalTok{, }\OperatorTok{-}\DecValTok{1}\NormalTok{)}
\KeywordTok{polygon}\NormalTok{(x, y, }\DataTypeTok{col =} \StringTok{"cornsilk"}\NormalTok{)}
\NormalTok{vertex1 =}\StringTok{ }\KeywordTok{c}\NormalTok{(}\DecValTok{1}\NormalTok{, }\DecValTok{2}\NormalTok{, }\DecValTok{3}\NormalTok{, }\DecValTok{4}\NormalTok{)}
\NormalTok{vertex2 =}\StringTok{ }\KeywordTok{c}\NormalTok{(}\DecValTok{2}\NormalTok{, }\DecValTok{3}\NormalTok{, }\DecValTok{4}\NormalTok{, }\DecValTok{1}\NormalTok{)}
\ControlFlowTok{for}\NormalTok{(i }\ControlFlowTok{in} \DecValTok{1}\OperatorTok{:}\DecValTok{50}\NormalTok{) \{}
\NormalTok{    x =}\StringTok{ }\FloatTok{0.9} \OperatorTok{*}\StringTok{ }\NormalTok{x[vertex1] }\OperatorTok{+}\StringTok{ }\FloatTok{0.1} \OperatorTok{*}\StringTok{ }\NormalTok{x[vertex2]}
\NormalTok{    y =}\StringTok{ }\FloatTok{0.9} \OperatorTok{*}\StringTok{ }\NormalTok{y[vertex1] }\OperatorTok{+}\StringTok{ }\FloatTok{0.1} \OperatorTok{*}\StringTok{ }\NormalTok{y[vertex2]}
    \KeywordTok{polygon}\NormalTok{(x, y, }\DataTypeTok{col =} \StringTok{"cornsilk"}\NormalTok{)}
\NormalTok{\}}
\end{Highlighting}
\end{Shaded}

\includegraphics[width=0.5\linewidth]{Rcourse_files/figure-latex/unnamed-chunk-266-1}

\subsubsection{Circles}\label{circles}

Circles are just dense polygons.

\begin{Shaded}
\begin{Highlighting}[]
\NormalTok{R =}\StringTok{ }\DecValTok{1}
\NormalTok{xc =}\StringTok{ }\DecValTok{0}
\NormalTok{yc =}\StringTok{ }\DecValTok{0}
\NormalTok{n =}\StringTok{ }\DecValTok{72}
\NormalTok{t =}\StringTok{ }\KeywordTok{seq}\NormalTok{(}\DecValTok{0}\NormalTok{, }\DecValTok{2} \OperatorTok{*}\StringTok{ }\NormalTok{pi, }\DataTypeTok{length =}\NormalTok{ n)[}\DecValTok{1}\OperatorTok{:}\NormalTok{(n}\OperatorTok{-}\DecValTok{1}\NormalTok{)]}
\NormalTok{x =}\StringTok{ }\NormalTok{xc }\OperatorTok{+}\StringTok{ }\NormalTok{R }\OperatorTok{*}\StringTok{ }\KeywordTok{cos}\NormalTok{(t)}
\NormalTok{y =}\StringTok{ }\NormalTok{yc }\OperatorTok{+}\StringTok{ }\NormalTok{R }\OperatorTok{*}\StringTok{ }\KeywordTok{sin}\NormalTok{(t)}
\KeywordTok{plot.new}\NormalTok{()}
\KeywordTok{plot.window}\NormalTok{(}\DataTypeTok{xlim =} \KeywordTok{range}\NormalTok{(x), }\DataTypeTok{ylim =} \KeywordTok{range}\NormalTok{(y), }\DataTypeTok{asp =} \DecValTok{1}\NormalTok{)}
\KeywordTok{polygon}\NormalTok{(x, y, }\DataTypeTok{col =} \StringTok{"lightblue"}\NormalTok{, }\DataTypeTok{border =} \StringTok{"navyblue"}\NormalTok{)}
\end{Highlighting}
\end{Shaded}

\includegraphics[width=0.5\linewidth]{Rcourse_files/figure-latex/unnamed-chunk-267-1}

\subsubsection{Spiral}\label{spiral}

\begin{Shaded}
\begin{Highlighting}[]
\NormalTok{k =}\StringTok{ }\DecValTok{5}
\NormalTok{n =}\StringTok{ }\NormalTok{k }\OperatorTok{*}\StringTok{ }\DecValTok{72}
\NormalTok{theta =}\StringTok{ }\KeywordTok{seq}\NormalTok{(}\DecValTok{0}\NormalTok{, k }\OperatorTok{*}\StringTok{ }\DecValTok{2} \OperatorTok{*}\StringTok{ }\NormalTok{pi, }\DataTypeTok{length =}\NormalTok{ n)}
\NormalTok{R =}\StringTok{ }\NormalTok{.}\DecValTok{98}\OperatorTok{^}\NormalTok{(}\DecValTok{1}\OperatorTok{:}\NormalTok{n }\OperatorTok{-}\StringTok{ }\DecValTok{1}\NormalTok{)}
\NormalTok{x =}\StringTok{ }\NormalTok{R }\OperatorTok{*}\StringTok{ }\KeywordTok{cos}\NormalTok{(theta)}
\NormalTok{y =}\StringTok{ }\NormalTok{R }\OperatorTok{*}\StringTok{ }\KeywordTok{sin}\NormalTok{(theta)}
\KeywordTok{plot.new}\NormalTok{()}
\KeywordTok{plot.window}\NormalTok{(}\DataTypeTok{xlim =} \KeywordTok{range}\NormalTok{(x), }\DataTypeTok{ylim =} \KeywordTok{range}\NormalTok{(y), }\DataTypeTok{asp =} \DecValTok{1}\NormalTok{)}
\KeywordTok{lines}\NormalTok{(x, y)}
\end{Highlighting}
\end{Shaded}

\includegraphics[width=0.5\linewidth]{Rcourse_files/figure-latex/unnamed-chunk-268-1}

\section{The ggplot2 System}\label{the-ggplot2-system}

The philosophy of \textbf{ggplot2} is very different from the
\textbf{graphics} device. Recall, in \textbf{ggplot2}, a plot is a
object. It can be queried, it can be changed, and among other things, it
can be plotted.

\textbf{ggplot2} provides a convenience function for many plots:
\texttt{qplot}. We take a non-typical approach by ignoring
\texttt{qplot}, and presenting the fundamental building blocks. Once the
building blocks have been understood, mastering \texttt{qplot} will be
easy.

The following is taken from
\href{http://www.ats.ucla.edu/stat/r/seminars/ggplot2_intro/ggplot2_intro.htm}{UCLA's
idre}.

A \textbf{ggplot2} object will have the following elements:

\begin{itemize}
\tightlist
\item
  \textbf{Data} the data frame holding the data to be plotted.
\item
  \textbf{Aes} defines the mapping between variables to their
  visualization.
\item
  \textbf{Geoms} are the objects/shapes you add as layers to your graph.
\item
  \textbf{Stats} are statistical transformations when you are not
  plotting the raw data, such as the mean or confidence intervals.
\item
  \textbf{Faceting} splits the data into subsets to create multiple
  variations of the same graph (paneling).
\end{itemize}

The \texttt{nlme::Milk} dataset has the protein level of various cows,
at various times, with various diets.

\begin{Shaded}
\begin{Highlighting}[]
\KeywordTok{library}\NormalTok{(nlme)}
\KeywordTok{data}\NormalTok{(Milk)}
\KeywordTok{head}\NormalTok{(Milk)}
\end{Highlighting}
\end{Shaded}

\begin{verbatim}
## Grouped Data: protein ~ Time | Cow
##   protein Time Cow   Diet
## 1    3.63    1 B01 barley
## 2    3.57    2 B01 barley
## 3    3.47    3 B01 barley
## 4    3.65    4 B01 barley
## 5    3.89    5 B01 barley
## 6    3.73    6 B01 barley
\end{verbatim}

\begin{Shaded}
\begin{Highlighting}[]
\KeywordTok{library}\NormalTok{(ggplot2)}
\KeywordTok{ggplot}\NormalTok{(}\DataTypeTok{data =}\NormalTok{ Milk, }\KeywordTok{aes}\NormalTok{(}\DataTypeTok{x=}\NormalTok{Time, }\DataTypeTok{y=}\NormalTok{protein)) }\OperatorTok{+}
\StringTok{  }\KeywordTok{geom_point}\NormalTok{()}
\end{Highlighting}
\end{Shaded}

\includegraphics[width=0.5\linewidth]{Rcourse_files/figure-latex/unnamed-chunk-270-1}

Things to note:

\begin{itemize}
\tightlist
\item
  The \texttt{ggplot} function is the constructor of the
  \textbf{ggplot2} object. If the object is not assigned, it is plotted.
\item
  The \texttt{aes} argument tells R that the \texttt{Time} variable in
  the \texttt{Milk} data is the x axis, and protein is y.
\item
  The \texttt{geom\_point} defines the \textbf{Geom}, i.e., it tells R
  to plot the points as they are (and not lines, histograms, etc.).
\item
  The \textbf{ggplot2} object is build by compounding its various
  elements separated by the \texttt{+} operator.
\item
  All the variables that we will need are assumed to be in the
  \texttt{Milk} data frame. This means that (a) the data needs to be a
  data frame (not a matrix for instance), and (b) we will not be able to
  use variables that are not in the \texttt{Milk} data frame.
\end{itemize}

Let's add some color.

\begin{Shaded}
\begin{Highlighting}[]
\KeywordTok{ggplot}\NormalTok{(}\DataTypeTok{data =}\NormalTok{ Milk, }\KeywordTok{aes}\NormalTok{(}\DataTypeTok{x=}\NormalTok{Time, }\DataTypeTok{y=}\NormalTok{protein)) }\OperatorTok{+}
\StringTok{  }\KeywordTok{geom_point}\NormalTok{(}\KeywordTok{aes}\NormalTok{(}\DataTypeTok{color=}\NormalTok{Diet))}
\end{Highlighting}
\end{Shaded}

\includegraphics[width=0.5\linewidth]{Rcourse_files/figure-latex/unnamed-chunk-271-1}

The \texttt{color} argument tells R to use the variable \texttt{Diet} as
the coloring. A legend is added by default. If we wanted a fixed color,
and not a variable dependent color, \texttt{color} would have been put
outside the \texttt{aes} function.

\begin{Shaded}
\begin{Highlighting}[]
\KeywordTok{ggplot}\NormalTok{(}\DataTypeTok{data =}\NormalTok{ Milk, }\KeywordTok{aes}\NormalTok{(}\DataTypeTok{x=}\NormalTok{Time, }\DataTypeTok{y=}\NormalTok{protein)) }\OperatorTok{+}
\StringTok{  }\KeywordTok{geom_point}\NormalTok{(}\DataTypeTok{color=}\StringTok{"green"}\NormalTok{)}
\end{Highlighting}
\end{Shaded}

\includegraphics[width=0.5\linewidth]{Rcourse_files/figure-latex/unnamed-chunk-272-1}

Let's save the \textbf{ggplot2} object so we can reuse it. Notice it is
not plotted.

\begin{Shaded}
\begin{Highlighting}[]
\NormalTok{p <-}\StringTok{ }\KeywordTok{ggplot}\NormalTok{(}\DataTypeTok{data =}\NormalTok{ Milk, }\KeywordTok{aes}\NormalTok{(}\DataTypeTok{x=}\NormalTok{Time, }\DataTypeTok{y=}\NormalTok{protein)) }\OperatorTok{+}
\StringTok{  }\KeywordTok{geom_point}\NormalTok{()}
\end{Highlighting}
\end{Shaded}

We can change\^{}\{In the Object-Oriented Programming lingo, this is
known as
\href{https://en.wikipedia.org/wiki/Immutable_object}{mutating}\}
existing plots using the \texttt{+} operator. Here, we add a smoothing
line to the plot \texttt{p}.

\begin{Shaded}
\begin{Highlighting}[]
\NormalTok{p }\OperatorTok{+}\StringTok{ }\KeywordTok{geom_smooth}\NormalTok{(}\DataTypeTok{method =} \StringTok{'gam'}\NormalTok{)}
\end{Highlighting}
\end{Shaded}

\includegraphics[width=0.5\linewidth]{Rcourse_files/figure-latex/unnamed-chunk-274-1}

Things to note:

\begin{itemize}
\tightlist
\item
  The smoothing line is a layer added with the \texttt{geom\_smooth()}
  function.
\item
  Lacking arguments of its own, the new layer will inherit the
  \texttt{aes} of the original object, x and y variables in particular.
\end{itemize}

To split the plot along some variable, we use faceting, done with the
\texttt{facet\_wrap} function.

\begin{Shaded}
\begin{Highlighting}[]
\NormalTok{p }\OperatorTok{+}\StringTok{ }\KeywordTok{facet_wrap}\NormalTok{(}\OperatorTok{~}\NormalTok{Diet)}
\end{Highlighting}
\end{Shaded}

\includegraphics[width=0.5\linewidth]{Rcourse_files/figure-latex/unnamed-chunk-275-1}

Instead of faceting, we can add a layer of the mean of each
\texttt{Diet} subgroup, connected by lines.

\begin{Shaded}
\begin{Highlighting}[]
\NormalTok{p }\OperatorTok{+}\StringTok{ }\KeywordTok{stat_summary}\NormalTok{(}\KeywordTok{aes}\NormalTok{(}\DataTypeTok{color=}\NormalTok{Diet), }\DataTypeTok{fun.y=}\StringTok{"mean"}\NormalTok{, }\DataTypeTok{geom=}\StringTok{"line"}\NormalTok{)}
\end{Highlighting}
\end{Shaded}

\includegraphics[width=0.5\linewidth]{Rcourse_files/figure-latex/unnamed-chunk-276-1}

Things to note:

\begin{itemize}
\tightlist
\item
  \texttt{stat\_summary} adds a statistical summary.
\item
  The summary is applied along \texttt{Diet} subgroups, because of the
  \texttt{color=Diet} aesthetic, which has already split the data.
\item
  The summary to be applied is the mean, because of
  \texttt{fun.y="mean"}.
\item
  The group means are connected by lines, because of the
  \texttt{geom="line"} argument.
\end{itemize}

What layers can be added using the \textbf{geoms} family of functions?

\begin{itemize}
\tightlist
\item
  \texttt{geom\_bar}: bars with bases on the x-axis.
\item
  \texttt{geom\_boxplot}: boxes-and-whiskers.
\item
  \texttt{geom\_errorbar}: T-shaped error bars.
\item
  \texttt{geom\_histogram}: histogram.
\item
  \texttt{geom\_line}: lines.
\item
  \texttt{geom\_point}: points (scatterplot).
\item
  \texttt{geom\_ribbon}: bands spanning y-values across a range of
  x-values.
\item
  \texttt{geom\_smooth}: smoothed conditional means (e.g.~loess smooth).
\end{itemize}

To demonstrate the layers added with the \texttt{geoms\_*} functions, we
start with a histogram.

\begin{Shaded}
\begin{Highlighting}[]
\NormalTok{pro <-}\StringTok{ }\KeywordTok{ggplot}\NormalTok{(Milk, }\KeywordTok{aes}\NormalTok{(}\DataTypeTok{x=}\NormalTok{protein))}
\NormalTok{pro }\OperatorTok{+}\StringTok{ }\KeywordTok{geom_histogram}\NormalTok{(}\DataTypeTok{bins=}\DecValTok{30}\NormalTok{)}
\end{Highlighting}
\end{Shaded}

\includegraphics[width=0.5\linewidth]{Rcourse_files/figure-latex/unnamed-chunk-277-1}

A bar plot.

\begin{Shaded}
\begin{Highlighting}[]
\KeywordTok{ggplot}\NormalTok{(Milk, }\KeywordTok{aes}\NormalTok{(}\DataTypeTok{x=}\NormalTok{Diet)) }\OperatorTok{+}
\StringTok{  }\KeywordTok{geom_bar}\NormalTok{()}
\end{Highlighting}
\end{Shaded}

\includegraphics[width=0.5\linewidth]{Rcourse_files/figure-latex/unnamed-chunk-278-1}

A scatter plot.

\begin{Shaded}
\begin{Highlighting}[]
\NormalTok{tp <-}\StringTok{ }\KeywordTok{ggplot}\NormalTok{(Milk, }\KeywordTok{aes}\NormalTok{(}\DataTypeTok{x=}\NormalTok{Time, }\DataTypeTok{y=}\NormalTok{protein))}
\NormalTok{tp }\OperatorTok{+}\StringTok{ }\KeywordTok{geom_point}\NormalTok{()}
\end{Highlighting}
\end{Shaded}

\includegraphics[width=0.5\linewidth]{Rcourse_files/figure-latex/unnamed-chunk-279-1}

A smooth regression plot, reusing the \texttt{tp} object.

\begin{Shaded}
\begin{Highlighting}[]
\NormalTok{tp }\OperatorTok{+}\StringTok{ }\KeywordTok{geom_smooth}\NormalTok{(}\DataTypeTok{method=}\StringTok{'gam'}\NormalTok{)}
\end{Highlighting}
\end{Shaded}

\includegraphics[width=0.5\linewidth]{Rcourse_files/figure-latex/unnamed-chunk-280-1}

And now, a simple line plot, reusing the \texttt{tp} object, and
connecting lines along \texttt{Cow}.

\begin{Shaded}
\begin{Highlighting}[]
\NormalTok{tp }\OperatorTok{+}\StringTok{ }\KeywordTok{geom_line}\NormalTok{(}\KeywordTok{aes}\NormalTok{(}\DataTypeTok{group=}\NormalTok{Cow))}
\end{Highlighting}
\end{Shaded}

\includegraphics[width=0.5\linewidth]{Rcourse_files/figure-latex/unnamed-chunk-281-1}

The line plot is completely incomprehensible. Better look at boxplots
along time (even if omitting the \texttt{Cow} information).

\begin{Shaded}
\begin{Highlighting}[]
\NormalTok{tp }\OperatorTok{+}\StringTok{ }\KeywordTok{geom_boxplot}\NormalTok{(}\KeywordTok{aes}\NormalTok{(}\DataTypeTok{group=}\NormalTok{Time))}
\end{Highlighting}
\end{Shaded}

\includegraphics[width=0.5\linewidth]{Rcourse_files/figure-latex/unnamed-chunk-282-1}

We can do some statistics for each subgroup. The following will compute
the mean and standard errors of \texttt{protein} at each time point.

\begin{Shaded}
\begin{Highlighting}[]
\KeywordTok{ggplot}\NormalTok{(Milk, }\KeywordTok{aes}\NormalTok{(}\DataTypeTok{x=}\NormalTok{Time, }\DataTypeTok{y=}\NormalTok{protein)) }\OperatorTok{+}
\StringTok{  }\KeywordTok{stat_summary}\NormalTok{(}\DataTypeTok{fun.data =} \StringTok{'mean_se'}\NormalTok{)}
\end{Highlighting}
\end{Shaded}

\includegraphics[width=0.5\linewidth]{Rcourse_files/figure-latex/unnamed-chunk-283-1}

Some popular statistical summaries, have gained their own functions:

\begin{itemize}
\tightlist
\item
  \texttt{mean\_cl\_boot}: mean and bootstrapped confidence interval
  (default 95\%).
\item
  \texttt{mean\_cl\_normal}: mean and Gaussian (t-distribution based)
  confidence interval (default 95\%).
\item
  \texttt{mean\_dsl}: mean plus or minus standard deviation times some
  constant (default constant=2).
\item
  \texttt{median\_hilow}: median and outer quantiles (default outer
  quantiles = 0.025 and 0.975).
\end{itemize}

For less popular statistical summaries, we may specify the statistical
function in \texttt{stat\_summary}. The median is a first example.

\begin{Shaded}
\begin{Highlighting}[]
\KeywordTok{ggplot}\NormalTok{(Milk, }\KeywordTok{aes}\NormalTok{(}\DataTypeTok{x=}\NormalTok{Time, }\DataTypeTok{y=}\NormalTok{protein)) }\OperatorTok{+}
\StringTok{  }\KeywordTok{stat_summary}\NormalTok{(}\DataTypeTok{fun.y=}\StringTok{"median"}\NormalTok{, }\DataTypeTok{geom=}\StringTok{"point"}\NormalTok{)}
\end{Highlighting}
\end{Shaded}

\includegraphics[width=0.5\linewidth]{Rcourse_files/figure-latex/unnamed-chunk-284-1}

We can also define our own statistical summaries.

\begin{Shaded}
\begin{Highlighting}[]
\NormalTok{medianlog <-}\StringTok{ }\ControlFlowTok{function}\NormalTok{(y) \{}\KeywordTok{median}\NormalTok{(}\KeywordTok{log}\NormalTok{(y))\}}
\KeywordTok{ggplot}\NormalTok{(Milk, }\KeywordTok{aes}\NormalTok{(}\DataTypeTok{x=}\NormalTok{Time, }\DataTypeTok{y=}\NormalTok{protein)) }\OperatorTok{+}
\StringTok{  }\KeywordTok{stat_summary}\NormalTok{(}\DataTypeTok{fun.y=}\StringTok{"medianlog"}\NormalTok{, }\DataTypeTok{geom=}\StringTok{"line"}\NormalTok{)}
\end{Highlighting}
\end{Shaded}

\includegraphics[width=0.5\linewidth]{Rcourse_files/figure-latex/unnamed-chunk-285-1}

\textbf{Faceting} allows to split the plotting along some variable.
\texttt{face\_wrap} tells R to compute the number of columns and rows of
plots automatically.

\begin{Shaded}
\begin{Highlighting}[]
\KeywordTok{ggplot}\NormalTok{(Milk, }\KeywordTok{aes}\NormalTok{(}\DataTypeTok{x=}\NormalTok{protein, }\DataTypeTok{color=}\NormalTok{Diet)) }\OperatorTok{+}
\StringTok{  }\KeywordTok{geom_density}\NormalTok{() }\OperatorTok{+}
\StringTok{  }\KeywordTok{facet_wrap}\NormalTok{(}\OperatorTok{~}\NormalTok{Time)}
\end{Highlighting}
\end{Shaded}

\includegraphics[width=0.5\linewidth]{Rcourse_files/figure-latex/unnamed-chunk-286-1}

\texttt{facet\_grid} forces the plot to appear allow rows or columns,
using the \texttt{\textasciitilde{}} syntax.

\begin{Shaded}
\begin{Highlighting}[]
\KeywordTok{ggplot}\NormalTok{(Milk, }\KeywordTok{aes}\NormalTok{(}\DataTypeTok{x=}\NormalTok{Time, }\DataTypeTok{y=}\NormalTok{protein)) }\OperatorTok{+}
\StringTok{  }\KeywordTok{geom_point}\NormalTok{() }\OperatorTok{+}
\StringTok{  }\KeywordTok{facet_grid}\NormalTok{(Diet}\OperatorTok{~}\NormalTok{.) }\CommentTok{# `.~Diet` to split along columns and not rows.}
\end{Highlighting}
\end{Shaded}

\includegraphics[width=0.5\linewidth]{Rcourse_files/figure-latex/unnamed-chunk-287-1}

To control the looks of the plot, \textbf{ggplot2} uses \textbf{themes}.

\begin{Shaded}
\begin{Highlighting}[]
\KeywordTok{ggplot}\NormalTok{(Milk, }\KeywordTok{aes}\NormalTok{(}\DataTypeTok{x=}\NormalTok{Time, }\DataTypeTok{y=}\NormalTok{protein)) }\OperatorTok{+}
\StringTok{  }\KeywordTok{geom_point}\NormalTok{() }\OperatorTok{+}
\StringTok{  }\KeywordTok{theme}\NormalTok{(}\DataTypeTok{panel.background=}\KeywordTok{element_rect}\NormalTok{(}\DataTypeTok{fill=}\StringTok{"lightblue"}\NormalTok{))}
\end{Highlighting}
\end{Shaded}

\includegraphics[width=0.5\linewidth]{Rcourse_files/figure-latex/unnamed-chunk-288-1}

\begin{Shaded}
\begin{Highlighting}[]
\KeywordTok{ggplot}\NormalTok{(Milk, }\KeywordTok{aes}\NormalTok{(}\DataTypeTok{x=}\NormalTok{Time, }\DataTypeTok{y=}\NormalTok{protein)) }\OperatorTok{+}
\StringTok{  }\KeywordTok{geom_point}\NormalTok{() }\OperatorTok{+}
\StringTok{  }\KeywordTok{theme}\NormalTok{(}\DataTypeTok{panel.background=}\KeywordTok{element_blank}\NormalTok{(),}
        \DataTypeTok{axis.title.x=}\KeywordTok{element_blank}\NormalTok{())}
\end{Highlighting}
\end{Shaded}

\includegraphics[width=0.5\linewidth]{Rcourse_files/figure-latex/unnamed-chunk-289-1}

Saving plots can be done using \texttt{ggplot2::ggsave}, or with
\texttt{pdf} like the \textbf{graphics} plots:

\begin{Shaded}
\begin{Highlighting}[]
\KeywordTok{pdf}\NormalTok{(}\DataTypeTok{file =} \StringTok{'myplot.pdf'}\NormalTok{)}
\KeywordTok{print}\NormalTok{(tp) }\CommentTok{# You will need an explicit print command!}
\KeywordTok{dev.off}\NormalTok{()}
\end{Highlighting}
\end{Shaded}

\BeginKnitrBlock{remark}
\iffalse{} {Remark. } \fi{}If you are exporting a PDF for publication,
you will probably need to embed your fonts in the PDF. In this case, use
\texttt{cairo\_pdf()} instead of \texttt{pdf()}.
\EndKnitrBlock{remark}

Finally, what every user of \textbf{ggplot2} constantly uses, is the
(excellent!) online documentation at \url{http://docs.ggplot2.org}.

\subsection{Extensions of the ggplot2
System}\label{extensions-of-the-ggplot2-system}

Because \textbf{ggplot2} plots are R objects, they can be used for
computations and altered. Many authors, have thus extended the basic
\textbf{ggplot2} functionality. A list of \textbf{ggplot2} extensions is
curated by Daniel Emaasit at
\href{http://www.ggplot2-exts.org/gallery/}{http://www.ggplot2-exts.org}.
The RStudio team has its own list of recommended packages at
\href{https://github.com/rstudio/RStartHere}{RStartHere}.

\section{Interactive Graphics}\label{interactive-graphics}

As already mentioned, the recent and dramatic advancement in interactive
visualization was made possible by the advances in web technologies, and
the \href{https://d3js.org/}{D3.JS} JavaScript library in particular.
This is because it allows developers to rely on existing libraries
designed for web browsing, instead of re-implementing interactive
visualizations. These libraries are more visually pleasing, and
computationally efficient, than anything they could have developed
themselves.

The \href{http://www.htmlwidgets.org/}{htmlwidgets} package does not
provide visualization, but rather, it facilitates the creation of new
interactive visualizations. This is because it handles all the technical
details that are required to use R output within JavaScript
visualization libraries.

For a list of interactive visualization tools that rely on
\textbf{htmlwidgets} see \href{http://gallery.htmlwidgets.org/}{their
(amazing) gallery}, and the
\href{https://github.com/rstudio/RStartHere}{RStartsHere} page. In the
following sections, we discuss a selected subset.

\subsection{Plotly}\label{plotly}

You can create nice interactive graphs using \texttt{plotly::plot\_ly}:

\begin{Shaded}
\begin{Highlighting}[]
\KeywordTok{library}\NormalTok{(plotly)}
\KeywordTok{set.seed}\NormalTok{(}\DecValTok{100}\NormalTok{)}
\NormalTok{d <-}\StringTok{ }\NormalTok{diamonds[}\KeywordTok{sample}\NormalTok{(}\KeywordTok{nrow}\NormalTok{(diamonds), }\DecValTok{1000}\NormalTok{), ]}
\end{Highlighting}
\end{Shaded}

\begin{Shaded}
\begin{Highlighting}[]
\KeywordTok{plot_ly}\NormalTok{(}\DataTypeTok{data =}\NormalTok{ d, }\DataTypeTok{x =} \OperatorTok{~}\NormalTok{carat, }\DataTypeTok{y =} \OperatorTok{~}\NormalTok{price, }\DataTypeTok{color =} \OperatorTok{~}\NormalTok{carat, }\DataTypeTok{size =} \OperatorTok{~}\NormalTok{carat, }\DataTypeTok{text =} \OperatorTok{~}\KeywordTok{paste}\NormalTok{(}\StringTok{"Clarity: "}\NormalTok{, clarity))}
\end{Highlighting}
\end{Shaded}

More conveniently, any \textbf{ggplot2} graph can be made interactive
using \texttt{plotly::ggplotly}:

\begin{Shaded}
\begin{Highlighting}[]
\NormalTok{p <-}\StringTok{ }\KeywordTok{ggplot}\NormalTok{(}\DataTypeTok{data =}\NormalTok{ d, }\KeywordTok{aes}\NormalTok{(}\DataTypeTok{x =}\NormalTok{ carat, }\DataTypeTok{y =}\NormalTok{ price)) }\OperatorTok{+}
\StringTok{  }\KeywordTok{geom_smooth}\NormalTok{(}\KeywordTok{aes}\NormalTok{(}\DataTypeTok{colour =}\NormalTok{ cut, }\DataTypeTok{fill =}\NormalTok{ cut), }\DataTypeTok{method =} \StringTok{'loess'}\NormalTok{) }\OperatorTok{+}\StringTok{ }
\StringTok{  }\KeywordTok{facet_wrap}\NormalTok{(}\OperatorTok{~}\StringTok{ }\NormalTok{cut) }\CommentTok{# make ggplot}
\KeywordTok{ggplotly}\NormalTok{(p) }\CommentTok{# from ggplot to plotly}
\end{Highlighting}
\end{Shaded}

\hypertarget{htmlwidget-5b2dfe07b90b480ea8e6}{}

How about exporting \textbf{plotly} objects? Well, a \textbf{plotly}
object is nothing more than a little web site: an HTML file. When
showing a \textbf{plotly} figure, RStudio merely servers you as a web
browser. You could, alternatively, export this HTML file to send your
colleagues as an email attachment, or embed it in a web site. To export
these, use the \texttt{plotly::export} or the
\texttt{htmlwidgets::saveWidget} functions.

For more on \textbf{plotly} see \url{https://plot.ly/r/}.

\section{Other R Interfaces to JavaScript
Plotting}\label{other-r-interfaces-to-javascript-plotting}

Plotly is not the only interactive plotting framework in R that relies o
JavaScript for interactivity. Here are some more interactive and
beautiful charting libraries.

\begin{itemize}
\item
  \href{https://www.highcharts.com/}{Highcharts}, like Plotly
  {[}\ref{plotly}{]}, is a popular collection of JavaScript plotting
  libraries, with great emphasis on aesthetics. The package
  \href{https://cran.r-project.org/package=highcharter}{highcharter} is
  an R wrapper for dispatching plots to highcharts. For a demo of the
  capabilities of Highcarts, see
  \href{https://www.highcharts.com/demo}{here}.
\item
  \href{http://hafen.github.io/rbokeh/}{Rbokeh} is a R wrapper for the
  popular \href{https://bokeh.pydata.org/en/latest/}{Bokeh} JavaScript
  charting libraries.
\item
  \href{https://rstudio.github.io/r2d3/}{r2d3}: a R wrapper to the
  \href{https://d3js.org/}{D3} plotting libraries.
\item
  \href{https://hafen.github.io/trelliscopejs/\#trelliscope}{trelliscope}:
  for beautiful, interactive, plotting of
  \href{https://www.juiceanalytics.com/writing/better-know-visualization-small-multiples}{small
  multiples}; think of it as interactive faceting.
\end{itemize}

\section{Bibliographic Notes}\label{bibliographic-notes-9}

For the \textbf{graphics} package, see \citet{Rlanguage}. For
\textbf{ggplot2} see \citet{ggplot2}. For the theory underlying
\textbf{ggplot2}, i.e.~the Grammar of Graphics, see
\citet{wilkinson2006grammar}. A
\href{https://www.youtube.com/watch?v=9Objw9Tvhb4\&feature=youtu.be}{video}
by one of my heroes, \href{http://www.bcaffo.com/}{Brian Caffo},
discussing \textbf{graphics} vs. \textbf{ggplot2}.

\section{Practice Yourself}\label{practice-yourself-7}

\begin{enumerate}
\def\labelenumi{\arabic{enumi}.}
\item
  Go to the Fancy Graphics Section \ref{fancy}. Try parsing the commands
  in your head.
\item
  Recall the \texttt{medianlog} example and replace the
  \texttt{medianlog} function with a
  \href{https://en.wikipedia.org/wiki/Harmonic_mean}{harmonic mean}.

\begin{Shaded}
\begin{Highlighting}[]
\NormalTok{medianlog <-}\StringTok{ }\ControlFlowTok{function}\NormalTok{(y) \{}\KeywordTok{median}\NormalTok{(}\KeywordTok{log}\NormalTok{(y))\}}
\KeywordTok{ggplot}\NormalTok{(Milk, }\KeywordTok{aes}\NormalTok{(}\DataTypeTok{x=}\NormalTok{Time, }\DataTypeTok{y=}\NormalTok{protein)) }\OperatorTok{+}
\StringTok{  }\KeywordTok{stat_summary}\NormalTok{(}\DataTypeTok{fun.y=}\StringTok{"medianlog"}\NormalTok{, }\DataTypeTok{geom=}\StringTok{"line"}\NormalTok{)}
\end{Highlighting}
\end{Shaded}

  \includegraphics[width=0.5\linewidth]{Rcourse_files/figure-latex/unnamed-chunk-295-1}
  ```
\item
  Write a function that creates a boxplot from scratch. See how I built
  a line graph in Section \ref{fancy}.
\item
  Export my plotly example using the RStudio interface and send it to
  yourself by email.
\end{enumerate}

ggplot2:

\begin{enumerate}
\def\labelenumi{\arabic{enumi}.}
\tightlist
\item
  Read about the ``oats'' dataset using \texttt{?\ MASS::oats}.

  \begin{enumerate}
  \def\labelenumii{\arabic{enumii}.}
  \tightlist
  \item
    Inspect, visually, the dependency of the yield (Y) in the Varieties
    (V) and the Nitrogen treatment (N).
  \item
    Compute the mean and the standard error of the yield for every value
    of Varieties and Nitrogen treatment.
  \item
    Change the axis labels to be informative with \texttt{labs} function
    and give a title to the plot with \texttt{ggtitle} function.
  \end{enumerate}
\item
  Read about the ``mtcars'' data set using \texttt{?\ mtcars}.

  \begin{enumerate}
  \def\labelenumii{\arabic{enumii}.}
  \tightlist
  \item
    Inspect, visually, the dependency of the Fuel consumption (mpg) in
    the weight (wt)
  \item
    Inspect, visually, the assumption that the Fuel consumption also
    depends on the number of cylinders.
  \item
    Is there an interaction between the number of cylinders to the
    weight (i.e.~the slope of the regression line is different between
    the number of cylinders)? Use \texttt{geom\_smooth}.
  \end{enumerate}
\end{enumerate}

See DataCamp's
\href{https://www.datacamp.com/courses/data-visualization-with-ggplot2-1}{Data
Visualization with ggplot2} for more self practice.

\chapter{Reports}\label{report}

If you have ever written a report, you are probably familiar with the
process of preparing your figures in some software, say R, and then
copy-pasting into your text editor, say MS Word. While very popular,
this process is both tedious, and plain painful if your data has changed
and you need to update the report. Wouldn't it be nice if you could
produce figures and numbers from within the text of the report, and
everything else would be automated? It turns out it is possible. There
are actually several systems in R that allow this. We start with a brief
review.

\begin{enumerate}
\def\labelenumi{\arabic{enumi}.}
\item
  \textbf{Sweave}: \emph{LaTeX} is a markup language that compiles to
  \emph{Tex} programs that compile, in turn, to documents (typically PS
  or PDFs). If you never heard of it, it may be because you were born
  the the MS Windows+MS Word era. You should know, however, that
  \emph{LaTeX} was there much earlier, when computers were mainframes
  with text-only graphic devices. You should also know that \emph{LaTeX}
  is still very popular (in some communities) due to its very rich
  markup syntax, and beautiful output. \emph{Sweave}
  \citep{leisch2002sweave} is a compiler for \emph{LaTeX} that allows
  you do insert R commands in the \emph{LaTeX} source file, and get the
  result as part of the outputted PDF. It's name suggests just that: it
  allows to weave S\footnote{Recall, S was the original software from
    which R evolved.} output into the document, thus, Sweave.
\item
  \textbf{knitr}: \emph{Markdown} is a text editing syntax that, unlike
  \emph{LaTeX}, is aimed to be human-readable, but also compilable by a
  machine. If you ever tried to read HTML or \emph{LaTeX} source files,
  you may understand why human-readability is a desirable property.
  There are many \emph{markdown} compilers. One of the most popular is
  Pandoc, written by the Berkeley philosopher(!) Jon MacFarlane. The
  availability of Pandoc gave \href{https://yihui.name/}{Yihui Xie}, a
  name to remember, the idea that it is time for Sweave to evolve. Yihui
  thus wrote \textbf{knitr} \citep{xie2015dynamic}, which allows to
  write human readable text in \emph{Rmarkdown}, a superset of
  \emph{markdown}, compile it with R and the compile it with Pandoc.
  Because Pandoc can compile to PDF, but also to HTML, and DOCX, among
  others, this means that you can write in Rmarkdown, and get output in
  almost all text formats out there.
\item
  \textbf{bookdown}: \textbf{Bookdown} \citep{xie2016bookdown} is an
  evolution of \textbf{knitr}, also written by Yihui Xie, now working
  for RStudio. The text you are now reading was actually written in
  \textbf{bookdown}. It deals with the particular needs of writing large
  documents, and cross referencing in particular (which is very
  challenging if you want the text to be human readable).
\item
  \textbf{Shiny}: Shiny is essentially a framework for quick
  web-development. It includes (i) an abstraction layer that specifies
  the layout of a web-site which is our report, (ii) the command to
  start a web server to deliver the site. For more on Shiny see
  \citet{shiny}.
\end{enumerate}

\section{knitr}\label{knitr}

\subsection{Installation}\label{installation}

To run \textbf{knitr} you will need to install the package.

\begin{Shaded}
\begin{Highlighting}[]
\KeywordTok{install.packages}\NormalTok{(}\StringTok{'knitr'}\NormalTok{)}
\end{Highlighting}
\end{Shaded}

It is also recommended that you use it within RStudio
(version\textgreater{}0.96), where you can easily create a new
\texttt{.Rmd} file.

\subsection{Pandoc Markdown}\label{pandoc-markdown}

Because \textbf{knitr} builds upon \emph{Pandoc markdown}, here is a
simple example of markdown text, to be used in a \texttt{.Rmd} file,
which can be created using the \emph{File-\textgreater{} New File
-\textgreater{} R Markdown} menu of RStudio.

Underscores or asterisks for \texttt{\_italics1\_} and
\texttt{*italics2*} return \emph{italics1} and \emph{italics2}. Double
underscores or asterisks for \texttt{\_\_bold1\_\_} and
\texttt{**bold2**} return \textbf{bold1} and \textbf{bold2}. Subscripts
are enclosed in tildes,
\texttt{like\textasciitilde{}this\textasciitilde{}}
(like\textsubscript{this}), and superscripts are enclosed in carets
\texttt{like\^{}this\^{}} (like\textsuperscript{this}).

For links use \texttt{{[}text{]}(link)}, like
\texttt{{[}my\ site{]}(www.john-ros.com)}. An image is the same as a
link, starting with an exclamation, like this
\texttt{!{[}image\ caption{]}(image\ path)}.

An itemized list simply starts with hyphens preceeded by a blank line
(don't forget that!):

\begin{verbatim}

- bullet
- bullet
    - second level bullet
    - second level bullet
\end{verbatim}

Compiles into:

\begin{itemize}
\tightlist
\item
  bullet
\item
  bullet

  \begin{itemize}
  \tightlist
  \item
    second level bullet
  \item
    second level bullet
  \end{itemize}
\end{itemize}

An enumerated list starts with an arbitrary number:

\begin{verbatim}
1. number
1. number
    1. second level number
    1. second level number
\end{verbatim}

Compiles into:

\begin{enumerate}
\def\labelenumi{\arabic{enumi}.}
\tightlist
\item
  number
\item
  number

  \begin{enumerate}
  \def\labelenumii{\arabic{enumii}.}
  \tightlist
  \item
    second level number
  \item
    second level number
  \end{enumerate}
\end{enumerate}

For more on markdown see
\url{https://bookdown.org/yihui/bookdown/markdown-syntax.html}.

\subsection{Rmarkdown}\label{rmarkdown}

\emph{Rmarkdown}, is an extension of \emph{markdown} due to RStudio,
that allows to incorporate R expressions in the text, that will be
evaluated at the time of compilation, and the output automatically
inserted in the outputted text. The output can be a \texttt{.PDF},
\texttt{.DOCX}, \texttt{.HTML} or others, thanks to the power of
\emph{Pandoc}.

The start of a code chunk is indicated by three backticks and the end of
a code chunk is indicated by three backticks. Here is an example.

\begin{verbatim}
```{r  eval=FALSE}
rnorm(10)
```
\end{verbatim}

This chunk will compile to the following output (after setting
\texttt{eval=FALSE} to \texttt{eval=TRUE}):

\begin{Shaded}
\begin{Highlighting}[]
\KeywordTok{rnorm}\NormalTok{(}\DecValTok{10}\NormalTok{)}
\end{Highlighting}
\end{Shaded}

\begin{verbatim}
##  [1] -1.4462875  0.3158558 -0.3427475 -1.9313531  0.2428210 -0.3627679
##  [7]  2.4327289  0.5920912 -0.5762008  0.4066282
\end{verbatim}

Things to note:

\begin{itemize}
\tightlist
\item
  The evaluated expression is added in a chunk of highlighted text,
  before the R output.
\item
  The output is prefixed with \texttt{\#\#}.
\item
  The \texttt{eval=} argument is not required, since it is set to
  \texttt{eval=TRUE} by default. It does demonstrate how to set the
  options of the code chunk.
\end{itemize}

In the same way, we may add a plot:

\begin{verbatim}
```{r  eval=FALSE}
plot(rnorm(10))
```
\end{verbatim}

which compiles into

\begin{Shaded}
\begin{Highlighting}[]
\KeywordTok{plot}\NormalTok{(}\KeywordTok{rnorm}\NormalTok{(}\DecValTok{10}\NormalTok{))}
\end{Highlighting}
\end{Shaded}

\includegraphics[width=0.5\linewidth]{Rcourse_files/figure-latex/unnamed-chunk-300-1}

Some useful code chunk options include:

\begin{itemize}
\tightlist
\item
  \texttt{eval=FALSE}: to return code only, without output.
\item
  \texttt{echo=FALSE}: to return output, without code.
\item
  \texttt{cache=}: to save results so that future compilations are
  faster.
\item
  \texttt{results=\textquotesingle{}hide\textquotesingle{}}: to plot
  figures, without text output.
\item
  \texttt{collapse=TRUE}: if you want the whole output after the whole
  code, and not interleaved.
\item
  \texttt{warning=FALSE}: to supress watning. The same for
  \texttt{message=FALSE}, and \texttt{error=FALSE}.
\end{itemize}

You can also call r expressions inline. This is done with a single tick
and the \texttt{r} argument. For instance:

\begin{quote}
\texttt{\textasciigrave{}r\ rnorm(1)\textasciigrave{}} is a random
Gaussian
\end{quote}

will output

\begin{quote}
0.3378953 is a random Gaussian.
\end{quote}

\subsection{BibTex}\label{bibtex}

BibTex is both a file format and a compiler. The bibtex compiler links
documents to a reference database stored in the \texttt{.bib} file
format.

Bibtex is typically associated with Tex and LaTex typesetting, but it
also operates within the markdown pipeline.

Just store your references in a \texttt{.bib} file, add a
\texttt{bibliography:\ yourFile.bib} in the YML preamble of your
Rmarkdown file, and call your references from the Rmarkdown text using
\texttt{@referencekey}. Rmarkdow will take care of creating the
bibliography, and linking to it from the text.

\subsection{Compiling}\label{compiling}

Once you have your \texttt{.Rmd} file written in RMarkdown,
\textbf{knitr} will take care of the compilation for you. You can call
the \texttt{knitr::knitr} function directly from some \texttt{.R} file,
or more conveniently, use the RStudio (0.96) Knit button above the text
editing window. The location of the output file will be presented in the
console.

\section{bookdown}\label{bookdown}

As previously stated, \textbf{bookdown} is an extension of
\textbf{knitr} intended for documents more complicated than simple
reports-- such as books. Just like \textbf{knitr}, the writing is done
in \textbf{RMarkdown}. Being an extension of \textbf{knitr},
\textbf{bookdown} does allow some markdowns that are not supported by
other compilers. In particular, it has a more powerful cross referencing
system.

\section{Shiny}\label{shiny}

\textbf{Shiny} \citep{shiny} is different than the previous systems,
because it sets up an interactive web-site, and not a static file. The
power of Shiny is that the layout of the web-site, and the settings of
the web-server, is made with several simple R commands, with no need for
web-programming. Once you have your app up and running, you can setup
your own Shiny server on the web, or publish it via
\href{https://www.shinyapps.io/}{Shinyapps.io}. The freemium versions of
the service can deal with a small amount of traffic. If you expect a lot
of traffic, you will probably need the paid versions.

\subsection{Installation}\label{installation-1}

To setup your first Shiny app, you will need the \textbf{shiny} package.
You will probably want RStudio, which facilitates the process.

\begin{Shaded}
\begin{Highlighting}[]
\KeywordTok{install.packages}\NormalTok{(}\StringTok{'shiny'}\NormalTok{)}
\end{Highlighting}
\end{Shaded}

Once installed, you can run an example app to get the feel of it.

\begin{Shaded}
\begin{Highlighting}[]
\KeywordTok{library}\NormalTok{(shiny)}
\KeywordTok{runExample}\NormalTok{(}\StringTok{"01_hello"}\NormalTok{)}
\end{Highlighting}
\end{Shaded}

Remember to press the \textbf{Stop} button in RStudio to stop the
web-server, and get back to RStudio.

\subsection{The Basics of Shiny}\label{the-basics-of-shiny}

Every Shiny app has two main building blocks.

\begin{enumerate}
\def\labelenumi{\arabic{enumi}.}
\tightlist
\item
  A user interface, specified via the \texttt{ui.R} file in the app's
  directory.
\item
  A server side, specified via the \texttt{server.R} file, in the app's
  directory.
\end{enumerate}

You can run the app via the \textbf{RunApp} button in the RStudio
interface, of by calling the app's directory with the \texttt{shinyApp}
or \texttt{runApp} functions-- the former designed for single-app
projects, and the latter, for multiple app projects.

\begin{Shaded}
\begin{Highlighting}[]
\NormalTok{shiny}\OperatorTok{::}\KeywordTok{runApp}\NormalTok{(}\StringTok{"my_app"}\NormalTok{) }\CommentTok{# my_app is the app's directory.}
\end{Highlighting}
\end{Shaded}

The site's layout, is specified in the \texttt{ui.R} file using one of
the \emph{layout functions}. For instance, the function
\texttt{sidebarLayout}, as the name suggest, will create a sidebar. More
layouts are detailed in the
\href{http://shiny.rstudio.com/articles/layout-guide.html}{layout
guide}.

The active elements in the UI, that control your report, are known as
\emph{widgets}. Each widget will have a unique \texttt{inputId} so that
it's values can be sent from the UI to the server. More about widgets,
in the
\href{http://shiny.rstudio.com/gallery/widget-gallery.html}{widget
gallery}.

The \texttt{inputId} on the UI are mapped to \texttt{input} arguments on
the server side. The value of the \texttt{mytext} \texttt{inputId} can
be queried by the server using \texttt{input\$mytext}. These are called
\emph{reactive values}. The way the server ``listens'' to the UI, is
governed by a set of functions that must wrap the \texttt{input} object.
These are the \texttt{observe}, \texttt{reactive}, and
\texttt{reactive*} class of functions.

With \texttt{observe} the server will get triggered when any of the
reactive values change. With \texttt{observeEvent} the server will only
be triggered by specified reactive values. Using \texttt{observe} is
easier, and \texttt{observeEvent} is more prudent programming.

A \texttt{reactive} function is a function that gets triggered when a
reactive element changes. It is defined on the server side, and reside
within an \texttt{observe} function.

We now analyze the \texttt{1\_Hello} app using these ideas. Here is the
\texttt{ui.R} file.

\begin{Shaded}
\begin{Highlighting}[]
\KeywordTok{library}\NormalTok{(shiny)}

\KeywordTok{shinyUI}\NormalTok{(}\KeywordTok{fluidPage}\NormalTok{(}

  \KeywordTok{titlePanel}\NormalTok{(}\StringTok{"Hello Shiny!"}\NormalTok{),}

  \KeywordTok{sidebarLayout}\NormalTok{(}
    \KeywordTok{sidebarPanel}\NormalTok{(}
      \KeywordTok{sliderInput}\NormalTok{(}\DataTypeTok{inputId =} \StringTok{"bins"}\NormalTok{,}
                  \DataTypeTok{label =} \StringTok{"Number of bins:"}\NormalTok{, }
                  \DataTypeTok{min =} \DecValTok{1}\NormalTok{,}
                  \DataTypeTok{max =} \DecValTok{50}\NormalTok{,}
                  \DataTypeTok{value =} \DecValTok{30}\NormalTok{)}
\NormalTok{    ),}

    \KeywordTok{mainPanel}\NormalTok{(}
      \KeywordTok{plotOutput}\NormalTok{(}\DataTypeTok{outputId =} \StringTok{"distPlot"}\NormalTok{)}
\NormalTok{    )}
\NormalTok{  )}
\NormalTok{))}
\end{Highlighting}
\end{Shaded}

Here is the \texttt{server.R} file:

\begin{Shaded}
\begin{Highlighting}[]
\KeywordTok{library}\NormalTok{(shiny)}

\KeywordTok{shinyServer}\NormalTok{(}\ControlFlowTok{function}\NormalTok{(input, output) \{}

\NormalTok{  output}\OperatorTok{$}\NormalTok{distPlot <-}\StringTok{ }\KeywordTok{renderPlot}\NormalTok{(\{}
\NormalTok{    x    <-}\StringTok{ }\NormalTok{faithful[, }\DecValTok{2}\NormalTok{]  }\CommentTok{# Old Faithful Geyser data}
\NormalTok{    bins <-}\StringTok{ }\KeywordTok{seq}\NormalTok{(}\KeywordTok{min}\NormalTok{(x), }\KeywordTok{max}\NormalTok{(x), }\DataTypeTok{length.out =}\NormalTok{ input}\OperatorTok{$}\NormalTok{bins }\OperatorTok{+}\StringTok{ }\DecValTok{1}\NormalTok{)}

    \KeywordTok{hist}\NormalTok{(x, }\DataTypeTok{breaks =}\NormalTok{ bins, }\DataTypeTok{col =} \StringTok{'darkgray'}\NormalTok{, }\DataTypeTok{border =} \StringTok{'white'}\NormalTok{)}
\NormalTok{  \})}
\NormalTok{\})}
\end{Highlighting}
\end{Shaded}

Things to note:

\begin{itemize}
\tightlist
\item
  \texttt{ShinyUI} is a (deprecated) wrapper for the UI.
\item
  \texttt{fluidPage} ensures that the proportions of the elements adapt
  to the window side, thus, are fluid.
\item
  The building blocks of the layout are a title, and the body. The title
  is governed by \texttt{titlePanel}, and the body is governed by
  \texttt{sidebarLayout}. The \texttt{sidebarLayout} includes the
  \texttt{sidebarPanel} to control the sidebar, and the
  \texttt{mainPanel} for the main panel.
\item
  \texttt{sliderInput} calls a widget with a slider. Its
  \texttt{inputId} is \texttt{bins}, which is later used by the server
  within the \texttt{renderPlot} reactive function.
\item
  \texttt{plotOutput} specifies that the content of the
  \texttt{mainPanel} is a plot (\texttt{textOutput} for text). This
  expectation is satisfied on the server side with the
  \texttt{renderPlot} function (\texttt{renderText}).
\item
  \texttt{shinyServer} is a (deprecated) wrapper function for the
  server.
\item
  The server runs a function with an \texttt{input} and an
  \texttt{output}. The elements of \texttt{input} are the
  \texttt{inputId}s from the UI. The elements of the \texttt{output}
  will be called by the UI using their \texttt{outputId}.
\end{itemize}

This is the output.

Here is another example, taken from the RStudio
\href{https://github.com/rstudio/shiny-examples/tree/master/006-tabsets}{Shiny
examples}.

\texttt{ui.R}:

\begin{Shaded}
\begin{Highlighting}[]
\KeywordTok{library}\NormalTok{(shiny)}

\KeywordTok{fluidPage}\NormalTok{(}
    
  \KeywordTok{titlePanel}\NormalTok{(}\StringTok{"Tabsets"}\NormalTok{),}
  
  \KeywordTok{sidebarLayout}\NormalTok{(}
    \KeywordTok{sidebarPanel}\NormalTok{(}
      \KeywordTok{radioButtons}\NormalTok{(}\DataTypeTok{inputId =} \StringTok{"dist"}\NormalTok{, }
                   \DataTypeTok{label =} \StringTok{"Distribution type:"}\NormalTok{,}
                   \KeywordTok{c}\NormalTok{(}\StringTok{"Normal"}\NormalTok{ =}\StringTok{ "norm"}\NormalTok{,}
                     \StringTok{"Uniform"}\NormalTok{ =}\StringTok{ "unif"}\NormalTok{,}
                     \StringTok{"Log-normal"}\NormalTok{ =}\StringTok{ "lnorm"}\NormalTok{,}
                     \StringTok{"Exponential"}\NormalTok{ =}\StringTok{ "exp"}\NormalTok{)),}
      \KeywordTok{br}\NormalTok{(), }\CommentTok{# add a break in the HTML page.}
      
      \KeywordTok{sliderInput}\NormalTok{(}\DataTypeTok{inputId =} \StringTok{"n"}\NormalTok{, }
                  \DataTypeTok{label =} \StringTok{"Number of observations:"}\NormalTok{, }
                   \DataTypeTok{value =} \DecValTok{500}\NormalTok{,}
                   \DataTypeTok{min =} \DecValTok{1}\NormalTok{, }
                   \DataTypeTok{max =} \DecValTok{1000}\NormalTok{)}
\NormalTok{    ),}
    
    \KeywordTok{mainPanel}\NormalTok{(}
      \KeywordTok{tabsetPanel}\NormalTok{(}\DataTypeTok{type =} \StringTok{"tabs"}\NormalTok{, }
        \KeywordTok{tabPanel}\NormalTok{(}\DataTypeTok{title =} \StringTok{"Plot"}\NormalTok{, }\KeywordTok{plotOutput}\NormalTok{(}\DataTypeTok{outputId =} \StringTok{"plot"}\NormalTok{)), }
        \KeywordTok{tabPanel}\NormalTok{(}\DataTypeTok{title =} \StringTok{"Summary"}\NormalTok{, }\KeywordTok{verbatimTextOutput}\NormalTok{(}\DataTypeTok{outputId =} \StringTok{"summary"}\NormalTok{)), }
        \KeywordTok{tabPanel}\NormalTok{(}\DataTypeTok{title =} \StringTok{"Table"}\NormalTok{, }\KeywordTok{tableOutput}\NormalTok{(}\DataTypeTok{outputId =} \StringTok{"table"}\NormalTok{))}
\NormalTok{      )}
\NormalTok{    )}
\NormalTok{  )}
\NormalTok{)}
\end{Highlighting}
\end{Shaded}

\texttt{server.R}:

\begin{Shaded}
\begin{Highlighting}[]
\KeywordTok{library}\NormalTok{(shiny)}

\CommentTok{# Define server logic for random distribution application}
\ControlFlowTok{function}\NormalTok{(input, output) \{}
  
\NormalTok{  data <-}\StringTok{ }\KeywordTok{reactive}\NormalTok{(\{}
\NormalTok{    dist <-}\StringTok{ }\ControlFlowTok{switch}\NormalTok{(input}\OperatorTok{$}\NormalTok{dist,}
                   \DataTypeTok{norm =}\NormalTok{ rnorm,}
                   \DataTypeTok{unif =}\NormalTok{ runif,}
                   \DataTypeTok{lnorm =}\NormalTok{ rlnorm,}
                   \DataTypeTok{exp =}\NormalTok{ rexp,}
\NormalTok{                   rnorm)}
    
    \KeywordTok{dist}\NormalTok{(input}\OperatorTok{$}\NormalTok{n)}
\NormalTok{  \})}
  
\NormalTok{  output}\OperatorTok{$}\NormalTok{plot <-}\StringTok{ }\KeywordTok{renderPlot}\NormalTok{(\{}
\NormalTok{    dist <-}\StringTok{ }\NormalTok{input}\OperatorTok{$}\NormalTok{dist}
\NormalTok{    n <-}\StringTok{ }\NormalTok{input}\OperatorTok{$}\NormalTok{n}
    
    \KeywordTok{hist}\NormalTok{(}\KeywordTok{data}\NormalTok{(), }\DataTypeTok{main=}\KeywordTok{paste}\NormalTok{(}\StringTok{'r'}\NormalTok{, dist, }\StringTok{'('}\NormalTok{, n, }\StringTok{')'}\NormalTok{, }\DataTypeTok{sep=}\StringTok{''}\NormalTok{))}
\NormalTok{  \})}
  
\NormalTok{  output}\OperatorTok{$}\NormalTok{summary <-}\StringTok{ }\KeywordTok{renderPrint}\NormalTok{(\{}
    \KeywordTok{summary}\NormalTok{(}\KeywordTok{data}\NormalTok{())}
\NormalTok{  \})}
  
\NormalTok{  output}\OperatorTok{$}\NormalTok{table <-}\StringTok{ }\KeywordTok{renderTable}\NormalTok{(\{}
    \KeywordTok{data.frame}\NormalTok{(}\DataTypeTok{x=}\KeywordTok{data}\NormalTok{())}
\NormalTok{  \})}
  
\NormalTok{\}}
\end{Highlighting}
\end{Shaded}

Things to note:

\begin{itemize}
\tightlist
\item
  We reused the \texttt{sidebarLayout}.
\item
  As the name suggests, \texttt{radioButtons} is a widget that produces
  radio buttons, above the \texttt{sliderInput} widget. Note the
  different \texttt{inputId}s.
\item
  Different widgets are separated in \texttt{sidebarPanel} by commas.
\item
  \texttt{br()} produces extra vertical spacing (break).
\item
  \texttt{tabsetPanel} produces tabs in the main output panel.
  \texttt{tabPanel} governs the content of each panel. Notice the use of
  various output functions
  (\texttt{plotOutput},\texttt{verbatimTextOutput},
  \texttt{tableOutput}) with corresponding \texttt{outputId}s.
\item
  In \texttt{server.R} we see the usual \texttt{function(input,output)}.
\item
  The \texttt{reactive} function tells the server the trigger the
  function whenever \texttt{input} changes.
\item
  The \texttt{output} object is constructed outside the
  \texttt{reactive} function. See how the elements of \texttt{output}
  correspond to the \texttt{outputId}s in the UI.
\end{itemize}

This is the output:

\subsection{Beyond the Basics}\label{beyond-the-basics}

Now that we have seen the basics, we may consider extensions to the
basic report.

\subsubsection{Widgets}\label{widgets}

\begin{itemize}
\tightlist
\item
  \texttt{actionButton} Action Button.
\item
  \texttt{checkboxGroupInput} A group of check boxes.
\item
  \texttt{checkboxInput} A single check box.
\item
  \texttt{dateInput} A calendar to aid date selection.
\item
  \texttt{dateRangeInput} A pair of calendars for selecting a date
  range.
\item
  \texttt{fileInput} A file upload control wizard.
\item
  \texttt{helpText} Help text that can be added to an input form.
\item
  \texttt{numericInput} A field to enter numbers.
\item
  \texttt{radioButtons} A set of radio buttons.
\item
  \texttt{selectInput} A box with choices to select from.
\item
  \texttt{sliderInput} A slider bar.
\item
  \texttt{submitButton} A submit button.
\item
  \texttt{textInput} A field to enter text.
\end{itemize}

See examples
\href{https://shiny.rstudio.com/gallery/widget-gallery.html}{here}.

\subsubsection{Output Elements}\label{output-elements}

The \texttt{ui.R} output types.

\begin{itemize}
\tightlist
\item
  \texttt{htmlOutput} raw HTML.
\item
  \texttt{imageOutput} image.
\item
  \texttt{plotOutput} plot.
\item
  \texttt{tableOutput} table.
\item
  \texttt{textOutput} text.
\item
  \texttt{uiOutput} raw HTML.
\item
  \texttt{verbatimTextOutput} text.
\end{itemize}

The corresponding \texttt{server.R} renderers.

\begin{itemize}
\tightlist
\item
  \texttt{renderImage} images (saved as a link to a source file).
\item
  \texttt{renderPlot} plots.
\item
  \texttt{renderPrint} any printed output.
\item
  \texttt{renderTable} data frame, matrix, other table like structures.
\item
  \texttt{renderText} character strings.
\item
  \texttt{renderUI} a Shiny tag object or HTML.
\end{itemize}

Your Shiny app can use any R object. The things to remember:

\begin{itemize}
\tightlist
\item
  The working directory of the app is the location of \texttt{server.R}.
\item
  The code before \texttt{shinyServer} is run only once.
\item
  The code inside `\texttt{shinyServer} is run whenever a reactive is
  triggered, and may thus slow things.
\end{itemize}

To keep learning, see the RStudio's
\href{http://shiny.rstudio.com/tutorial/}{tutorial}, and the
Biblipgraphic notes herein.

\subsection{shinydashboard}\label{shinydashboard}

A template for Shiny to give it s modern look.

\section{flexdashboard}\label{flexdashboard}

If you want to quickly write an interactive dashboard, which is simple
enough to be a static HTML file and does not need an HTML server, then
Shiny may be an overkill. With \textbf{flexdashboard} you can write your
dashboard a single .Rmd file, which will generate an interactive
dashboard as a static HTML file.

See {[}\url{http://rmarkdown.rstudio.com/flexdashboard/}{]} for more
info.

\section{Bibliographic Notes}\label{bibliographic-notes-10}

For RMarkdown see \href{http://rmarkdown.rstudio.com/}{here}. For
everything on \textbf{knitr} see
\href{https://yihui.name/knitr/}{Yihui's blog}, or the book
\citet{xie2015dynamic}. For a \textbf{bookdown} manual, see
\citet{xie2016bookdown}. For a Shiny manual, see \citet{shiny}, the
\href{http://shiny.rstudio.com/tutorial/}{RStudio tutorial}, or
\href{http://zevross.com/blog/2016/04/19/r-powered-web-applications-with-shiny-a-tutorial-and-cheat-sheet-with-40-example-apps/}{Zev
Ross's} excellent guide. Video tutorials are available
\href{https://www.rstudio.com/resources/webinars/shiny-developer-conference/}{here}.

\section{Practice Yourself}\label{practice-yourself-8}

\begin{enumerate}
\def\labelenumi{\arabic{enumi}.}
\item
  Generate a report using \textbf{knitr} with your name as title, and a
  scatter plot of two random variables in the body. Save it as PDF,
  DOCX, and HTML.
\item
  Recall that this book is written in \textbf{bookdown}, which is a
  superset of \textbf{knitr}. Go to the source .Rmd file of the first
  chapter, and parse it in your head:
  (\url{https://raw.githubusercontent.com/johnros/Rcourse/master/02-r-basics.Rmd})
\end{enumerate}

\bibliography{bib.bib}

\end{document}
